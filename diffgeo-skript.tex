\documentclass[12pt,a4paper]{article}
%
\date{14. Februar 2011}
%
\usepackage[latin1]{inputenc}
%\usepackage[T1]{fontenc}
\usepackage[ngerman]{babel}

\usepackage{graphicx}
\usepackage{amssymb}
\usepackage{latexsym}
%\usepackage{fullpage}
\usepackage{amscd}
\usepackage{amsmath}

\usepackage[
% Hyperlink-Optionen
hyperfootnotes=true, % Fu\ss{}noten verlinken
hyperindex=true, % Seitenzahlen als Link
%backref=true, % R\"ucklinks von bibitem,
% erfordert Leerzeile nach jedem bibitem
breaklinks=true, % Zeilenumbruch bei Links
% Farben
colorlinks=true, % true = farbige Links, false = Rahmen
linkcolor=red, % Farbe f\"ur Links auf gleicher Seite
%pagecolor=gray, % Farbe f\"ur Links auf anderen Seiten
citecolor=gray, % Farbe f\"ur Links auf Zitatstellen
filecolor=magenta,
urlcolor=red, % Farbe f\"ur Links auf URLs
anchorcolor=blue, % Farbe f\"ur Anker (?)
menucolor=red, % Farbe f\"ur Acrobatmen\"ueintr\"age
% Fehlermeldungen reduzieren
plainpages=false,
% PDF-Metainformationen
unicode=true, % Mit Unicode kodierte Texte
baseurl={http://url},
pdfauthor={Prof. Uwe Semmelmann},
pdfproducer={pdflatex},
% Adobe Reader: Anzeigeoptionen
bookmarks=true, % Acrobat-Lesezeichen einf\"ugen
bookmarksnumbered=true, % Kapitelnummern f\"ur Lesezeichen
bookmarksopen=true, % Acrobat-Lesezeichen links ge\"offnet
bookmarksopenlevel=2, % Ge\"offnete Hierarchiestufe
%bookmarkstype=Richter_LaTeX_Tipps.toc,
% Lesezeichentyp. Fehlen aber.
pdfview=FitH, % Zoom beim Link-Traversieren
% FitH=Seitenbreite anpassen
% weitere: Fit, FitH, FitV, FitR, FitBH
pdfstartview=FitH, % Zoom beim Öffnen
pdfstartpage={1}, % Seite, die angezeigt wird
pdfpagelayout=OneColumn % Anzeigemodus, hier 1 Seite fortlaufend
% SinglePage, OneColumn,
% TwoColumnLeft, TwoColumnRight
%pdfpagemode=None, % Optionen: UseOutlines = Bookmarks links
% UseThumbs, FullScreen
]{hyperref}

%
\setlength{\textheight}{23.5cm}
\setlength{\textwidth}{15.5cm}
\setlength{\topmargin}{-1cm}
\setlength{\oddsidemargin}{0.3cm}
\setlength{\evensidemargin}{.5cm}
\parindent=0pt % paragraph indentation
\thispagestyle{empty}
%
\def\R{\mathbb{R}}
\def\C{\mathbb{C}}
\def\H{\mathbb{H}}
\def\N{\mathbb{N}}
\def\Z{\mathbb{Z}}
\def\Q{\mathbb{Q}}
\def\K{\mathbb{K}}
%
\def\O{\mathrm{O}}
\def\so{\mathfrak{so}}
\def\sp{\mathfrak{sp}}
\def\su{\mathfrak{su}}
\def\u{\mathfrak{u}}
\def\sl{\mathfrak{sl}}
\def\SO{\mathrm{SO}}
\def\SL{\mathrm{SL}}
\def\Spin{\mathrm{Spin}}
\def\SU{\mathrm{SU}}
\def\Sp{\mathrm{Sp}}
\def\GL{\mathrm{GL}}
\def\U{\mathrm{U}}
\def\G{\mathrm{G}}
\def\Ad{\mathrm{Ad}}
\def\ad{\mathrm{ad}}
%
\def\Cas{\mathrm{Cas}}
\def\End{\mathrm{End}\,}
\def\Hom{\mathrm{Hom}\,}
\def\Hol{\mathrm{Hol}}
\def\hol{\mathfrak{hol}}
\def\Iso{\mathrm{Iso}}
\def\id{\mathrm{id}}
\def\im{\mathrm{im}\,}
\def\rg{\mathrm{rg}\,}
\def\vol{\mathrm{vol}}
\def\pr{\mathrm{pr}}
\def\Sym{\mathrm{Sym}}
\def\Im{\mathrm{Im}}
\def\Lie{\mathrm{Lie}}
\def\Der{\mathrm{Der}}
\def\Tr{\mathrm{Tr}}
\def\Diff{\mathrm{Diff}}
%
\def\#{\sharp}
\def\b{\flat}
\def\D{\Delta}
\def\e{\varepsilon}
\def\G{\mathbf{G}}
\def\g{\mathfrak{g}}
\def\k{\mathfrak{k}}
\def\m{\mathfrak{m}}
%\def\K{\mathrm{KI}}
\def\L{\Lambda}
\def\l{\lambda}
\def\p{\mathfrak{p}}
\def\h{\mathfrak{h}}
\def\S{\mathrm{Sym}}
\def\W{\mathcal{W}\,}
\def\tt{\mathfrak{t}\,}
\def\diag{\mathrm{diag}\,}
\def\grad{\mathrm{grad}\,}
%
\def\Aut{\mathrm{Aut}}
\def\ins{\,\lrcorner\,\,}
\setlength{\parindent}{0pt}
\def\bea{\begin{eqnarray*}}
\def\eea{\end{eqnarray*}}
\def\Ric{\mathrm{Ric}}
\def\tr{\mathrm{tr}}
\def\la{\langle}
\def\ra{\rangle}
\def\Ul{\mathcal{U}\,}
\def\E{\mathcal{E}\,}
\def\spin{\mathfrak{spin}}
\def\gl{\mathfrak{gl}}
\def\ad{\mathrm{ad}}
\def\Re{\mathrm{Re}}
\def\sgn{\mathrm{sgn}}
\def\Zent{\mathrm{Zent}}
\def\Eig{\mathrm{Eig}}
\def\Zl{\mathrm{Zent}\,\Ul (\g)}
\def\Abb{\mathrm{Abb}}
\def\Id{\mathrm{Id}}
\def\sign{\mathrm{sign}}
\def\ord{\mathrm{ord}}
\def\supp{\mathrm{supp}\,}
%
\def\al{i\,}
\def\W{B}
%
\def\<#1,#2>{\langle\,#1,\,#2\,\rangle}
%
\renewcommand{\theequation}{\thesection.\arabic{equation}}
%
\newtheorem{Lemma}{Lemma}[section]
\newtheorem{Proposition}[Lemma]{Proposition}
\newtheorem{Satz}[Lemma]{Satz}
\newtheorem{Theorem}[Lemma]{Theorem}
\newtheorem{Folgerung}[Lemma]{Folgerung}
\newtheorem{Definition}[Lemma]{Definition}
\newtheorem{Remark}[Lemma]{Remark}
%
\def\proof{\noindent\textbf{Beweis:}\quad}
\def\summary#1{(\textsl{#1})\hfill\break}
\def\nosummary{\hfill\break}
\def\pfill{\par\vskip3mm plus1mm minus1mm\noindent}
\def\qed{\quad\hfill\ensuremath{\Box}}
%
\newcommand{\firstline}[2]{\makebox[#1][l]{$\displaystyle{#2}$}&&}
%
\begin{document}

%\begin{figure}[tbp]
%		\includegraphics[width=7cm]{logo1_10.jpg}
%\end{figure}

\title{Vorlesungsskript Differentialgeometrie }
\maketitle


\section{Topologische Mannigfaltigkeiten}

Topologische Mannigfaltigkeiten sollen als spezielle topologische R\"aume definiert werden. Daf\"ur sind zun\"achst einige
Grundbegriffe der Topologie zu wiederholen.

\bigskip

Sei  $M$ eine Menge. Ein Mengensystem $\mathcal O = \mathcal O_M \subset \mathcal P(M)$ heisst {\it Topologie} auf $M$, falls
\begin{enumerate}
\item \quad $\emptyset, M \in \mathcal O$
\item \quad $U, V \in \mathcal O $  dann ist auch $\; U \cap V \in \mathcal O$
\item \quad $U_i \in \mathcal O, i \in I$  dann ist auch $\;\bigcup_{i\in I} U_i \in \mathcal O$
\end{enumerate}

\bigskip

Ein Paar $(M, \mathcal O)$ heisst {\it topologischer Raum}. Eine Teilmenge heisst
{\it offen}, falls $U \in \mathcal O$. Eine Teilmenge $A \subset M$ heisst {\it abgeschlossen}, falls
$M \setminus A$ offen.

\medskip

{\bf Beispiele:} (1) Die offenen Mengen der {\it Standardtopologie} auf $\R^n$ sind die Mengen $U \subset \R^n$, f\"ur die
zu jedem Punkt $x\in U$ eine kleine Kugel $B_r(x)$ existiert, die ganz in $U$ liegt. (2) Die {\it diskrete Topologie} auf
einer Menge $M$ ist definiert durch $\mathcal O = \mathcal P(M)$, d.h. jede Teilmenge von $M$ ist offen. (3) Sei
$(M, \mathcal O_M)$ ein topologischer Raum und $X\subset M$ eine Teilmenge. Dann sind die offenen Mengen der
{\it Teilraumtopologie} (auch {\it induzierte} Topologie) auf $X$ genau die Schnitte von $X$ mit offenen Mengen in $M$.

\bigskip

Ein Mengensystem $\mathcal B \subset \mathcal P(M)$ heisst {\it Basis der Topologie} $\mathcal O_M$, falls die offenen Mengen
aus $\mathcal O_M$ genau die Vereinigungen der Mengen aus $\mathcal B$ sind. Insbesondere ist also $\mathcal B \subset
\mathcal O_M$.

\medskip

{\bf Beispiel:} Eine (abz\"ahlbare) Basis der Standardtopologie auf $\R^n$ bilden die Kugeln $B_r(x)$ mit rationalem Radius $r$
um Punkte $x\in \R^n$ mit rationalen Koordinaten.



\bigskip

Eine Abbildung $f: (M, \mathcal O_M) \rightarrow (N, \mathcal O_N)$ heisst {\it stetig}, falls $f^{-1}(V) \in \mathcal O_M$
f\"ur alle $V \in \mathcal O_N$. Eine bijektive Abbildung $f:M \rightarrow N$ heisst {\it Hom\"oomorphismus} falls $f$ und
$f^{-1}$ stetig sind. In diesem Fall nennt man die R\"aume $M$ und $N$ zueinander {\it hom\"oomorph}.


\bigskip


Ein topologischer Raum $M$ heisst {\it lokal hom\"oomorph} zu $\R^n$ (auch {\it lokal euklidisch}), falls f\"ur alle $p\in M$
eine offene Menge $U\subset M$ mit $p\in U$, eine offene Menge $V \subset \R^n$ und ein Hom\"oomorphismus $\phi :U\rightarrow V$
existieren.

 \bigskip

Ein topologischer Raum $M$ heisst {\it zusammenh\"angend}, falls kein $U\subset M, U \neq \emptyset, M$ existiert, dass zugleich offen
und abgeschlossen ist. \"Aquivalent ist, dass $M$ nicht als nicht-triviale Vereinigung zweier disjunkter offener Mengen dargestellt
werden kann. Eine Teilmenge $X\subset M$ heisst zusammenh\"angend, falls $X$ mit der Teilraumtopologie zusammenh\"angend ist.

\medskip

Ein topologischer Raum $M$ heisst {\it wegzusammenh\"angend}, falls sich je zwei Punkte in $M$ durch einen stetigen Weg in $M$
verbinden lassen. D.h. f\"ur je zwei Punkte $p,q \in M$ existiert eine stetige Abbildung $c:[0,1]\rightarrow M$ mit $c(0)=p$
und $c(1)=q$.

\begin{Lemma}
Wegzusammenh\"angende R\"aume sind zusammenh\"angend. Zusammenh\"angende, lokal euklidische R\"aume sind auch wegzusammenh\"angend.
\end{Lemma}

%\bigskip

Ein topologischer Raum $M$ heisst {\it hausdorff} (auch $T_2$), falls f\"ur alle $p,q \in M, p\neq q$ offene Mengen $U, V \subset M$
existieren mit $p\in U, q\in V$ und $U \cap V = \emptyset$.

\medskip

{\bf Bemerkung} Aus lokal euklidisch folgt {\it nicht} hausdorff.

\bigskip

\begin{Definition}
Eine $n$-dimensionale {\em topologische Mannigfaltigkeit } ist ein topologischer Raum $M$ mit:
\begin{enumerate}
\item \quad $M$ ist lokal hom\"oomorph zu $\R^n$
\item \quad $M$ ist hausdorff
\item \quad $M$ besitzt eine abz\"ahlbare Basis der Topologie
\end{enumerate}

Die lokalen Hom\"oomorphismen $\phi: U \subset M \rightarrow V \subset \R^n$ heissen {\it Karten}
oder lokale Koordinatensysteme von $M$.
\end{Definition}

\medskip

Genaugenommen muss man noch \"uberpr\"ufen, dass die Dimension wohldefiniert ist, also nicht vom Punkt abh\"angig ist.
Das folgt aus dem Satz von Brouwer bzw. aus dem Satz \"uber die Invarianz des Gebietes.

\medskip

{\bf Satz: (Brouwer)} {\it Sei $U\subset \R^n $ offen und $f:U \rightarrow \R^n$ stetig und injektiv. Dann ist $f(U)\subset \R^n$
offen. }

\medskip

{\bf Folgerung:}
{\it Seien $U \subset \R^n$ und $V\subset \R^m$ offene Mengen. Sind $U$ und $V$ zueinander hom\"oomorph, so folgt
$n=m$.}

\medskip

\proof
Man nimmt $n\neq m$ an und o.B.d.A. auch $m<n$. Sei $i:\R^m\rightarrow \R^n$ die kanonische Einbettung
$i(x_1,\ldots, x_m)= (x_1,\ldots, x_m,0,\ldots,0)$. Dann ist $i$ stetig, injektiv und $i(\R^m)$ enth\"alt
keine in $\R^n$ offene Menge.

\medskip

Sei nun $\phi : U \rightarrow V$ der laut Voraussetzung existierende Hom\"oomorphismus. Dann ist auch
$f:= i|_V \circ \phi : U \rightarrow \R^n$ stetig und injektiv. Aus dem Satz von Brouwer folgt nun, dass
$f(U) = i(V) \subset i(\R^n)$ offen in $\R^n$. Das ist ein Widerspruch und es muss also $m=n$ gelten.
\qed

\bigskip

Im Folgenden sollen drei Beispiele topologischer Mannigfaltigkeiten vorgestellt werden.

\begin{enumerate}
\item Der $n$-dimensionale {\it euklidische Raum} $M=\R^n$. Hier hat man eine Karte, die durch die Identit\"at
gegeben ist. Analog ist auch jede offene Teilmenge $M\subset \R^n$ eine $n$-dimensionale Mannigfaltigkeit.

\item Die $n$-dimensionale {\it Sph\"are}: $S^n := \{x\in \R^{n+1} | \, x_1^2 + \ldots + x_{n+1}^2 = 1\}$. Als Topologie
auf $S^n \subset \R^{n+1}$ w\"ahlt man die von der Standardtopologie auf $\R^{n+1}$ induzierte Topologie. Dann ist klar,
dass der topologische Raum $S^n$ hausdorff ist und eine abz\"ahlbare Basis der Topologie besitzt. Auf $S^n$ kann
man mittels der stereographischen Projektion zwei Karten definieren, die zeigen, dass $S^n$ lokal hom\"oomorph
zu $\R^n$ ist.

Sei $U_1 = S^n \setminus (0,\ldots,0,1)$ und $U_2 = S^n \setminus (0,\ldots,0,-1)$. Dann definiert man
Hom\"oomorphismen $g_i: U_i \rightarrow \R^n, i = 1,2$ durch
$$
g_1(x) = \frac{1}{1-x_{n+1}}\,(x_1,\ldots, x_n), \qquad g_2(x) = \frac{1}{1+x_{n+1}}\,(x_1,\ldots, x_n) \ .
$$

\item Der $n$-dimensionale {\it hyperbolische Raum}: $H^n :=\{ x \in \R^{n+1} |\, x_0^2 -x_1^2 -
\ldots - x_n^2 = 1, x_0 >0\}$. Hier reicht eine Karte, d.h. $H^n$ ist hom\"oomorph zu $\R^n$. Man definiert
$\phi : H^n \subset \R^{n+1}\rightarrow \R^n$ durch $\phi(x) = (x_1,\ldots, x_n)$. Die (stetige)
Umkehrabbildung ist gegeben durch
$$
\psi : \R^n \rightarrow H^n \subset \R^{n+1}, \qquad \psi(y) = (\sqrt{1+y_1^2 + \ldots + y_n^2},y_1, \ldots , y_n)
$$
\end{enumerate}

\bigskip

{\bf Bemerkung:} Der Doppelkegel $M = \{(x,y,z) \in \R^3 | z^2 = x^2 + y^2\}$ ist keine topologische
Mannigfaltigkeit.

\bigskip

\begin{Satz}
Eine 0-dimensionale Mannigfaltigkeit ist eine abz\"ahlbare Menge von Punkten mit der diskreten Topologie.
\end{Satz}
\proof
Sei $M$ eine 0-dimensionale Mannigfaltigkeit und $p\in M$. Dann existiert eine offene Menge $U\subset M$ mit
$p\in U$, die hom\"oomorph zu $\R^0= \{0\}$ ist. Daraus folgt $U = \{p\}$, d.h. die Punkte in $M$ sind
offene Mengen und $M$ tr\"agt damit die diskrete Topologie.
\qed

\medskip

{\bf Bemerkung:} Eine zusammenh\"angende 1-dimensionale Mannigfaltigkeit ist hom\"oomorph zu $S^1$ oder $\R^1$.

\bigskip

\begin{Satz} Sei $M$ eine n-dimensionale und $N$ eine m-dimensionale Mannigfaltigkeit. Dann ist $M\times N$
eine Mannigfaltigkeit der Dimension $n+m$.
\end{Satz}

\medskip

{\bf Beispiel:} Sei $M=N=S^1$, dann ist $T^2= S^1 \times S^1$ der 2-dimensionale {\it Torus}. Allgemeiner
hat man den n-dimensionalen Torus $T^n = S^1 \times \ldots \times S^1$.


\section{Differenzierbare Mannigfaltigkeiten}

Es soll nun f\"ur topologische Mannigfaltigkeiten der Begriff einer differenzierbaren Abbildung definiert werden.
Dazu muss zun\"achst auf der topologischen Mannigfaltigkeit eine differenzierbare Struktur eingef\"uhrt werden. Sei
$M$ eine topologische Mannigfaltigkeit und sei $f:M \rightarrow \R$ eine stetige Abbildung. Seien $(U,x), (V,y)$
Karten um $p\in M$. Dann gilt:
$$
f \circ y^{-1} = (f\circ x^{-1}) \; (x\circ y^{-1}) \ .
$$
Man sieht, dass f\"ur die Definition der Differenzierbarkeit von $f$ mit Hilfe von Karten, die Differenzierbarkeit
der Kartenwechsel $x\circ y^{-1}$ ben\"otigt wird.

\medskip

\begin{Definition}
Zwei Karten $(U,x), (V,y)$ heissen {\em vertr\"aglich} falls die Abbildung
$$
x\circ y^{-1} : y(U \cap V) \rightarrow x(U\cap V)
$$
ein Diffeomorphismus von offenen Mengen in $\R^n$ ist. Die Abbildung $x\circ y^{-1}$ heisst {\em Kartenwechsel}.
Eine Menge von Karten $\mathcal A = \{(U_\alpha, x_\alpha)\}$ heisst {\em Atlas} von $M$, falls die Karten $M$
\"uberdecken und je zwei Karten vertr\"aglich sind.
Ein Atlas $\mathcal A$ heisst {\em maximaler Atlas}, falls jede Karte, die mit allen Karten in $\mathcal A$
vertr\"aglich ist, schon in $\mathcal A$ liegt.
\end{Definition}

{\bf Bemerkung}: Sei $\mathcal A$ ein Atlas f\"ur $M$, dann definiert man
$$
\mathcal A_{max} := \mbox{Menge aller Karten von $M$, die mit allen Karten aus} \quad \mathcal A \quad \mbox{vertr\"aglich sind}.
$$

\begin{Lemma}

\begin{enumerate}
\item \quad $ \mathcal A \subset \mathcal A_{max}$
\item \quad $\mathcal A _{max}$ ist ein Atlas
\item \quad $\mathcal A _{max}$ ist maximaler Atlas
\item \quad  Jeder Atlas ist in genau einem maximalen Atlas enthalten
\end{enumerate}
\end{Lemma}

\begin{Definition}
Eine n-dimensionale {\em differenzierbare Mannigfaltigkeit} ist eine n-dimensionale topologische
Mannigfaltigkeit zusammen mit einem maximalen Atlas.
\end{Definition}

{\bf Bemerkung:}
Differenzierbare Mannigfaltigkeiten nennt man auch $\mathcal C^\infty$- oder {\it glatte} Mannigfaltigkeiten. Ein
maximaler Atlas definiert eine {\it differenzierbare} Struktur. Es gibt topologische Mannigfaltigkeiten, die keine
differenzierbare Strukturen zulassen. Auf einer topologischen Mannigfaltigkeit kann es "unterschiedliche" differenzierbare
Strukturen geben.

\medskip

{\bf Notation:} Im Weiteren soll mit "Mannigfaltigkeit" immer "differenzierbare Mannigfaltigkeit" gemeint sein.

\bigskip

{\bf Beispiele:} (1) Jede offene Teilmenge $U\subset \R^n$ ist trivialerweise eine n-dimensionale
differenzierbare Mannigfaltigkeit. Als Atlas nimmt man hier $\mathcal A = \{(U, \id)\}$.
(2) Die Sph\"are $S^n$ ist eine $n$-dimensionale Mannigfaltigkeit. Seien $(U_i, g_i), i=1,2$
die oben eingef\"uhrten Karten. Zu \"uberpr\"ufen bleibt, ob der Kartenwechsel $g_1 \circ g_2^{-1}$
differenzierbar ist. F\"ur $x\in g_2(U_1\cap U_2)= g_2(S^n\setminus \{NP, SP\})= \R^n\setminus \{0\}\subset \R^n$ gilt:
$$
g_1 \circ g_2^{-1}(x) = g_1(\frac{1-\|x\|^2}{1+\|x\|^2},\frac{2x}{1+\|x\|^2})=
\frac{1}{1-\frac{1-\|x\|^2}{1+\|x\|^2}}\cdot\frac{2x}{1+\|x\|^2}
=
\frac{2x}{2\|x\|^2} = \frac{x}{\|x\|^2}
$$
Damit sind die Karten $(U_1,g_1)$ und  $(U_2,g_2)$ vertr\"aglich. Da sie auch $S^n$ \"uberdecken erh\"alt man den
Atlas $\mathcal A = \{(U_i, g_i), i=1,2 \}$.

\bigskip

\begin{Definition}
Seien $M, N$ differenzierbare Mannigfaltigkeiten. Eine stetige Abbildung $f:M\rightarrow N$ heisst
{\em differenzierbar}, wenn es um jeden Punkt $p\in M$ Karten $(U,x)$ um $p$ und $(V,y)$ um $f(p)$
gibt, so dass auf einer Umgebung $W \subset x(f^{-1}(V)\cap U)$ von $x(p)$ die Abbildung
$$
y \circ f \circ x^{-1} : W \subset x(f^{-1}(V)\cap U) \rightarrow y(V)
$$
differenzierbar ist. Ein {\em Diffeomorphismus} ist ein Hom\"oomorphismus $f$, f\"ur den $f$ und
$f^{-1}$ differenzierbar sind. In diesem Fall nennt man die Mannigfaltigkeiten $M$ und $N$
{\em diffeomorph}.
\end{Definition}

\medskip

{\bf Bemerkungen:}
(1) Genauer kann man noch von Differenzierbarkeit in einem Punkt und k-facher Differenzierbarkeit sprechen.
Im Weiteren soll "differenzierbar" immer "beliebig oft differenzierbar" bedeuten. Solche Abbildungen nennt
man dann {\it glatt} oder $\mathcal C^\infty$. (2) Ist $f$ in $p$ bzgl. eines Paares von Karten differenzierbar,
so auch f\"ur alle anderen Karten um $p$ bzw. $f(p)$.

\bigskip

{\bf Beispiele:}
\begin{enumerate}
\item Die antipodale Abbildung $f: S^n\rightarrow S^n, f(y)=-y$ ist differenzierbar. (\"UA)
\item   Auf $\R$ seien die Atlanten $\mathcal A_1 = \{x=\id:\R \rightarrow \R\}$ und
$\mathcal A_2 = \{\tilde x :\R \rightarrow \R, \tilde x(t) = t^3\}$ fixiert. Dann sind
$x$  und $\tilde x$ keine vertr\"aglichen Karten, da $x\circ \tilde x^{-1}(t) = \sqrt[3]{t}$
nicht differenzierbar ist. Damit ist $\id : (\R, \mathcal A_{1,max})\rightarrow (\R, \mathcal A_{2,max})$
differenzierbar, aber kein Diffeomorphismus. Allerdings liefert die Abbildung $f(t) = \sqrt[3]{t}$
einen Diffeomorphismus $f:(\R, \mathcal A_{1,max})\rightarrow (\R, \mathcal A_{2,max})$.
\end{enumerate}

\bigskip

{\bf Bemerkungen:}
\begin{enumerate}
\item F\"ur $n\neq 4$ ist jede differenzierbare Struktur auf $\R^n$ diffeomorph zur Standardstruktur
$\mathcal A_{max}$ zu $\mathcal A = \{(\R^n, \id)\}$.
\item Auf $\R^4$ gibt es \"uberabz\"ahlbar viele differenzierbare Strukturen, die paarweise nicht diffeomorph
sind (= exotische Strukturen).
\item Jede topologische Mannigfaltigkeit in Dimension 1, 2 und 3 besitzt genau eine differenzierbare
Struktur.
\item Es gibt topologische Mannigfaltigkeiten in Dimension 4, die keine differenzierbare Struktur zulassen.
\item Es gibt f\"ur $n\geq7$ Mannigfaltigkeiten, die hom\"oomorph zu $S^n$ aber nicht diffeomorph sind.
Diese nennt man exotische Sph\"aren. In jeder Dimension $n, n\ge 7$ gibt es h\"ochstens endlich viele exotische Sph\"aren.
Die Existenz einer exotischen $S^4$ ist ungekl\"art.
\end{enumerate}


\bigskip

Im Folgenden soll noch ein wichtiges Beispiel differenzierbarer Mannigfaltigkeiten vor\-gestellt werden:
die projektiven R\"aume. Diese sind definiert als Quotientenr\"aume.

\medskip

Sei $X$ ein topologischer Raum, sei $\sim$ eine \"Aquivalenzrelation auf $X$, dann bezeichne
$X/_\sim$ die Menge der \"Aquivalenzklassen. Auf dieser l\"asst sich eine Topologie definieren.
Daf\"ur bezeichne $\pi : X \rightarrow X/_\sim, \, x \mapsto [x]$ die kanonische Projektion. Dann ist
eine Menge $U \subset X/_\sim$ genau dann offen, wenn $\pi^{-1}(U)\subset X$ offen ist. Die
Quotiententopologie wird damit die gr\"o\ss te (feinste) Topologie auf $X/_\sim$, f\"ur die die kanonische
Projektion stetig ist. Im Allgemeinen ist $X/_\sim$ keine Mannigfaltigkeit, noch nicht einmal hausdorff.
Die Eigenschaften kompakt und zusammenh\"angend \"ubertragen sich von $X$ auf $X/_\sim$.

\medskip

Auf $S^n$ betrachtet man die \"Aquivalenzrelation $x \sim -x$. Dann ist der {\it reell-projektive Raum}
definiert als
$$
\R  {\mathrm P}^n = S^n/_\sim
$$
mit der Quotientenraumtopologie. Man kann $\R  {\mathrm P}^n$ auch mit dem Raum aller Geraden in $\R^{n+1}$
identifizieren. Daf\"ur bildet man die \"Aquivalenzklasse $[x]$ auf die Gerade durch $x$ ab.

\begin{Lemma}
Der reell-projektive Raum $\R  {\mathrm P}^n$ ist eine n-dimensionale kompakte, zusammenh\"angende Mannigfaltigkeit.
\end{Lemma}

%\medskip

Etwas allgemeiner kann man zu jedem $(n+1)$-dimensionalen Vektorraum $V$  \"uber einem K\"orper $\K$ einen projektiven Raum
definieren:
$$
\K {\mathrm P}^n := \mbox{Menge aller 1-dimensionaler Unterr\"aume in}\quad V \ .
$$
Diese R\"aume sind Quotientenr\"aume von $V$ bzgl. der \"Aquivalenzrelation: $v \sim w$ genau dann,
wenn $v$ und $w$ linear abh\"angig sind.

\bigskip

{\bf Bemerkung:} Es gilt $\R {\mathrm P}^1 \cong S^1, \C {\mathrm P}^1 \cong S^2, \H {\mathrm P}^1 \cong S^4 $. F\"ur $n\ge 2$ ist $\R {\mathrm P}^n \neq S^n$.




\section{Der Satz vom regul\"aren Wert}

In diesem Abschnitt soll die Konstruktion differenzierbarer Mannigfaltigkeiten als Urbild regul\"arer Werte
beschrieben werden. Sei $f:M\rightarrow N$ eine differenzierbare Abbildung.

\begin{Definition}
Seien $(U,x)$ bzw. $(V,y)$ Karten um $p$ bzw. $f(p)$. Dann ist der {\em Rang} von $f$ in $p$ definiert als
der Rang der Jacobi-Matrix von $y\circ f \circ x^{-1}$, d.h.
$$
\rg_p(f) = \mathrm{Rang} \left( \frac{\partial (y\,f\,x^{-1})_i}{\partial x_j}\right) = J_p(y\,f\,x^{-1})\ ,
$$
wobei $J_p(r)$ im Weiteren die Jacobi-Matrix einer Abbildung $r$ im Punkt $p$ bezeichnet.
\end{Definition}


{\bf Bemerkung:} (1) Ist $r$ ein Diffeomorphismus, dann ist $J_p(r)$ invertierbar. (2) Der Rang der Jacobi-Matrix
h\"angt nicht von den Karten ab, d.h. $\rg_p(f)$ ist wohldefiniert.

\medskip

Beweis von (2): Seien $(U,x)$ und $(\tilde U, \tilde x)$ Karten um $p$ und $(V,y)$ und $(\tilde V, \tilde y)$
Karten um $f(p)$. Dann gilt
$$
\tilde y \, f \, \tilde x^{-1} \;= \; (\tilde y \, y^{-1}) \circ (y\,f\, x^{-1}) \circ (\tilde x \, x^{-1})^{-1} \ .
$$
Nun sind die Kartenwechsel $\tilde y \, y^{-1}$ bzw. $\tilde x \, x^{-1}$ Diffeomorphismen und deren Jacobi-Matrizen
damit Isomorphismen, die den Rang einer linearen Abbildung also nicht \"andern. Daraus und aus der Kettenregel folgt nun
$$
\rg J(\tilde y\,f\,\tilde x^{-1}) \;=\; \rg \left( J(\tilde y \, y^{-1}) \circ J(y\,f\, x^{-1}) \circ J(\tilde x \, x^{-1})^{-1} \right)
\;=\; \rg J(y\,f\,x^{-1})\ .
$$

Aus der Analysis ist bekannt, dass der Rang das lokale Verhalten von Abbildungen bestimmt. Die entsprechenden S\"atze
\"ubertragen sich nun direkt auf Mannigfaltigkeiten.

\medskip

{\bf Umkehrsatz:}{\it  \, Sei $f:M\rightarrow N$ eine differenzierbare Abbildung zwischen zwei Mannigfaltigkeiten der
Dimension $n$. Sei $p\in M$ ein Punkt mit $\rg_p(f)=n$. Dann existiert eine Umgebung von $p$, auf der $f$ ein
Diffeomorphismus ist.}

\medskip

Zum Beweis: ist der Rang von $f$ in $p$ gleich $n$, dann ist die Jacobi-Matrix von $y f x^{-1}$ invertierbar und
damit ist $y f x^{-1}$ ein lokaler Diffeomorphismus um $x(p)$, was sich entsprechend auf die Mannigfaltigkeit
\"ubertr\"agt.

\medskip

Den letzten Satz kann man etwas verallgemeinern, wenn man von der Jacobi-Matrix nur noch verlangt, dass sie
surjektiv sein soll. Dazu definiert man zun\"achst

\begin{Definition}
Sei $f:M\rightarrow N$ eine differenzierbare Abbildung. Dann hei\ss t ein Punkt $p \in M$ {\em regul\"ar}, falls
$\rg_p(f) = \dim N$, d.h. falls $J(yfx^{-1})$ surjektiv in $p$ ist.
\end{Definition}

{\bf Bemerkung:}
(1) Ist $p$ regul\"ar, dann folgt $\dim M \ge \dim N$ (2) Ist $\dim M < \dim N$, dann sind alle Punkte in
$M$ {\it kritisch}, d.h. nicht regul\"ar.

\bigskip

{\bf Satz vom regul\"aren Punkt:}{\it \, Sei $f:M\rightarrow N$ differenzierbar und sei $p\in M$ regul\"ar. Dann
existieren Karten $(U,x)$ um $p$ und $(V,y)$ um $f(p)$ mit $f(U)\subset V$ und
$$
y\,f\,x^{-1}: (x_1,\ldots,x_{n+s}) \mapsto (x_{1+s},\ldots,x_{n+s})
$$
wobei $\dim M = n+s$ und $\dim N = n$, d.h. in lokalen Koordinaten stimmt $f$  mit der kanonischen
Projektion von $\R^{n+s}$ auf $\R^n$ \"uberein.}

\bigskip

\begin{Definition}
Eine Teilmenge $M_0\subset M$, einer n-dimensionalen Mannigfaltigkeit $M$ nennt man k-dimensionale
{\em Untermannigfaltigkeit}, wenn es um jeden Punkt von $M_0$ eine Karte $(U,x)$ von $M$ gibt mit
$$
x(U \cap M_0) = \R^k \cap x(U) \ .
$$
Die Differenz $\dim M - \dim M_0$ nennt man die {\em Kodimension} von $M_0$ in $M$.
\end{Definition}

\medskip

{\bf Bemerkungen:} (1) Jede Untermannigfaltigkeit ist wieder eine Mannigfaltigkeit. Untermannigfaltigkeiten
von $M$ der Kodimension 0 sind genau die offenen Mengen in $M$. Untermannigfaltigkeiten der Dimension 0 sind
genau die diskreten Teilmengen in $M$. (2) Eine Teilmenge $M \subset \R^n$ ist genau dann eine Untermannigfaltigkeit
der Dimension k von $\R^n$, wenn $M$ lokal diffeomorph zu einer offenen Menge in $\R^k$ ist. (3) Der Satz von
Whitney besagt, dass jede n-dimensionale Mannigfaltigkeit diffeomorph zu einer Untermannigfaltigkeit von $\R^{2n}$
ist. Genauer kann man noch sagen:

\medskip

\begin{enumerate}
\item Falls $n$ eine 2er-Potenz ist, l\"a\ss t sich $\R  {\mathrm P}^n$ nicht in $\R^{2n-1}$ einbetten, z.B. ist $\R  {\mathrm P}^2$ nicht
diffeomorph zu einer Untermannigfaltigkeit von $\R^3$.

\item Ist $n$ keine 2er-Potenz, dann existiert f\"ur jede n-dimensionale Mannigfaltigkeit eine Einbettung nach $\R^{2n-1}$.
(Haeflinger-Hirsch-Wall)

\item Orientierbare Fl\"achen lassen sich in $\R^3$ einbetten. Nicht-orientierbare Fl\"achen nur nach $\R^4$.

\item Jede kompakte, orientierbare n-dimensionale Mannigfaltigkeit besitzt eine Einbettung nach $\R^{2n-1}$.
\end{enumerate}


\bigskip

\begin{Definition}
Sei $f:M\rightarrow N$ eine differenzierbare Abbildung. Ein Punkt $q\in N$ hei\ss t
{\em regul\"arer Wert} von $f$, falls jedes $p \in f^{-1}(q)$ ein regul\"arer Punkt
von $f$ ist.
\end{Definition}



\bigskip

{\bf Satz vom regul\"aren Wert:}{\it \, Ist $q\in N$ ein regul\"arer Wert einer differenzierbaren Abbildung
$f:M\rightarrow N$, dann ist $f^{-1}(q)$ eine Untermannigfaltigkeit von $M$ und es gilt
$$
\dim  f^{-1}(q) = \dim M - \dim N \ .
$$ }
\proof
Sei $p \in f^{-1}(q)$ ein regul\"arer Punkt. Dann w\"ahlt man Karten $(U,x)$ um $p$ und $(V,y)$ um $q$, wie
sie nach dem Satz vom regul\"aren Punkt existieren. O.B.d.A. kann man auch noch $y(q)=0$ voraussetzen. Dann gilt
$p \in U \cap f^{-1}(q)$ g.d.w. $yfx^{-1} \,x(p)=0$. Da man die Karten so gew\"ahlt hat, dass $f$ lokal mit der
kanonischen Projektion \"ubereinstimmt, ist dies \"aquivalent zu $x(p)_i = 0$ f\"ur $i=1+s,\ldots, n+s$, d.h.
zu $x(p) \in \R^s \cap x(U)$. Insgesamt erh\"alt man die gesuchte Untermannigfaltigkeitsgleichung
$x(U \cap f^{-1}(q))= \R^s \cap x(U)$.
\qed

\bigskip

Der Satz vom regul\"aren Wert kann benutzt werden, um Mannigfaltigkeitsstrukturen nachzuweisen. Daf\"ur sollen
zwei Beispiele gegeben werden.

\medskip

(1) Die n-dimensionale Sph\"are

\medskip

F\"ur $M=\R^{n+1}$ und $N=\R$
betrachtet man die Abbildung $f:\R^{n+1}\rightarrow \R$, gegeben durch $f(x) = \|x\|^2 = \sum_i x_i^2$.
Als Jacobi-Matrix in $x=(x_0, \ldots, x_n)$ findet man $J(f)= (2x_0,\ldots, 2x_n)$. Damit folgt
$\rg_p(f)\neq 0$ f\"ur alle Punkte $p\neq 0$, d.h. 1 ist ein regul\"arer Wert f\"ur $f$ und
$f^{-1}(1)= S^n$ damit eine Untermannigfaltigkeit von $\R^{n+1}$.

\bigskip

(2) Die orthogonale Gruppe $\O(n) = \{A \in M(n,\R) \,|\, A^T\cdot A = E\}$

\medskip

Sei $M=M(n,\R)= \R^{n^2}$ und sei $N= \Sym(n,\R) = \{ A \in M(n,\R) \,|\, A^T=A\}= \R^{\frac12n(n+1)}$.
D.h. $N$ ist die Menge der symmetrischen Matrizen.

\begin{Lemma}
F\"ur die Abbildung $f:M(n,\R)\rightarrow \Sym(n,\R), A \mapsto A^TA$ ist die Einheitsmatrix $E$ ein regul\"arer
Wert. Damit ist die orthogonale Gruppe $\O(n)=f^{-1}(E)$ eine Mannigfaltigkeit der Dimension $\frac12n(n-1)$.
\end{Lemma}
\proof
Zu zeigen ist, dass die Jacobi-Matrix der Abbildung $f$ in jedem Punkt $p\in f^{-1}(E)=\O(n)$ surjektiv ist.
Daf\"ur nutzt man die Definition der Jacobi-Matrix als Abbildung (Richtungsableitung) und berechnet
f\"ur $v\in \R^{n^2}$:
$$
\begin{array}{rl}
J(f)_p v &=\left. \frac{d}{dt}\right|_{t=0} f(p + t v) \\[1ex]
         & =   \left. \frac{d}{dt}\right|_{t=0} (p + tv)^T(p+tv)\\[1ex]
         &= \left.\frac{d}{dt}\right|_{t=0} (p^Tp + t p^Tv + tv^Tp + t^2v^Tv)\\[1ex]
         &= p^Tv + v^Tp
\end{array}
$$
F\"ur $B\in \Sym(n,\R)$ und $p \in \O(n)=f^{-1}(E)$ definiert man $v:= \frac12 p B$ und berechnet
$$
p^Tv + v^Tp = \frac12 p^T p B + \frac12 B^Tp^Tp = \frac12 B + \frac12 B^T = B \ .
$$
D.h. die Jacobi-Matrix der oben definierten Abbildung $f$ ist surjektiv in allen Punkten von $f^{-1}(E)$ und
die Einheitsmatrix $E$ ist damit ein regul\"arer Wert von $f$.
Nach dem Satz vom regul\"aren Wert ist damit die orthogonale Gruppe $\O(n)$ als Urbild der Einheitsmatrix $E$ unter
$f$ eine Mannigfaltigkeit, n\"amlich eine Untermannigfaltigkeit von $\R^{n^2}$ der Dimension $\frac12 n(n-1)$.
\qed

\bigskip

{\bf Bemerkungen:}
(1) Die spezielle orthogonale Gruppe $\SO(n)$ ist eine offene Teilmenge von $O(n)$ und daher eine Untermannigfaltigkeit
der Kodimension 0. (2) Die Mannigfaltigkeit $\O(n)$ zerf\"allt in zwei Zusammenhangskomponenten. (3)
Die klassischen Matrizengruppen $\GL_n, \U_n, \SU_n, \Sp_n,\ldots$ sind alles Mannigfaltigkeiten. Zus\"atzlich
sind die Gruppenoperationen differenzierbare Abbildungen. Solche Gruppen nennt man {\it Lie-Gruppen}.

\bigskip


\section{Der Tangentialraum}

Sei $f:\R^n\rightarrow \R^m$ eine differenzierbare Abbildung. Dann ist eine lineare Approximation von $f$ in $x$
gegeben durch das Differential $df_x:\R^n\rightarrow \R^m$, d.h. die lineare Abbildung $df_x=J_x$. Nun m\"ochte man
Abbildungen zwischen Mannigfaltigkeiten approximieren. Dazu definiert man zu jedem Punkt $p$
der Mannigfaltigkeit $M$ einen abstrakten Vektorraum $T_pM$, den {\it Tangentialraum} in $p$, und dann definiert man zu
jeder Abbildung $f:M\rightarrow N$ eine lineare Abbildung $df_p : T_pM \rightarrow T_{f(p)}N$, das {\it Differential}
von $f$ in $p$.

\medskip

F\"ur Untermannigfaltigkeiten $M\subset \R^N$ hat man eine anschauliche Definition des Tangentialraumes,
als Menge der Tangentialvektoren an Kurven in $M$:
$$
T_pM :=\{ \dot \gamma (0) \, |\, \gamma : (-\varepsilon,\varepsilon)\rightarrow M \; \mbox{differenzierbar}, \gamma (0) = p\} \ .
$$
Im Allgemeinen hat man aber keine kanonische Einbettung und mu\ss{} daher den Tangentialraum $T_pM$ durch innere
Eigenschaften von $M$ definieren. Dazu definiert man auf der Menge der Kurven in $M$ eine \"Aquivalenzrelation:
zwei Kurven $\alpha, \beta : (-\varepsilon,\varepsilon)\rightarrow M$ mit $\alpha(0)=p=\beta(0)$ hei\ss en \"aquivalent,
$\alpha \sim \beta$, falls f\"ur eine und damit f\"ur jede Karte $(U,x)$ um $p$ gilt:
$$
\dot{(x\circ \alpha)}(0) = \dot{(x\circ \beta)}(0) \ .
$$

\begin{Definition}
{\em Tangentialvektoren} an $M$ im Punkt $p\in M$ sind \"Aquivalenzklassen bzgl. $\sim$ von Kurven in $M$ durch $p$. Der
{\em Tangentialraum} von $M$ in $p$ ist die Menge all dieser \"Aquivalenzklassen:
$$
T_pM = \{\gamma: (-\varepsilon,\varepsilon)\rightarrow M \; \mbox{differenzierbar}, \gamma (0) = p\}/_\sim \ .
$$
\end{Definition}

F\"ur die \"Aquivalenzklasse einer Kurve $\gamma$ in $M$ nutzt man folgende Schreibweisen:
$$
[\gamma] = \dot\gamma(0)= \left.\frac{d}{dt}\right|_{t=0} \gamma(t)\ .
$$

\bigskip

\begin{Lemma}
Sei $M$ eine n-dimensionale Mannigfaltigkeit, $p\in M$. Dann ist der Tangentialraum $T_pM$ ein n-dimensionaler
reeller Vektorraum.
\end{Lemma}
\proof
Sei $(U,x)$ eine Karte um $p$, dann definiert man:
$$
dx_p : T_pM \rightarrow \R^n,\qquad [\gamma] \mapsto \dot{(x\circ \gamma)}(0) \ .
$$
Diese Abbildung ist wohldefiniert nach Definition der \"Aquivalenzrelation $\sim$ bzw. des
Tangentialraumes $T_pM$. Man zeigt nun, dass $dx_p$ bijektiv ist. Injektivit\"at ist wiederum klar
nach der Definition. Um die Surjektivit\"at zu zeigen, definiert man f\"ur einen Vektor $v\in \R^n$
die Kurve $\gamma(t):= x^{-1}(x(p)+tv)$. Man w\"ahlt $\varepsilon$ so klein, dass $x(p)+tv$ f\"ur $|t|<\varepsilon
$ in $x(U)$ liegt. Dann folgt
$$
dx_p ([\gamma]) = \left.\frac{d}{dt}\right|_{t=0} x \circ x^{-1}(x(p)+tv) = \left.\frac{d}{dt}\right|_{t=0}
(x(p)+tv) = v \ .
$$
Die Vektorraumstruktur auf $T_pM$ wird nun so definiert, dass $dx_p$ eine lineare Abbildung wird. Also z.B.
f\"ur $v, w \in T_pM$ definiert man:
$$
v + w := (dx_p)^{-1}(dx_p(v) + dx_p(w)) \ .
$$
Zu zeigen bleibt noch, dass die Vektorraumstruktur auf $T_pM$ nicht von der gew\"ahlten Karte abh\"angt. Die Abbildung
$dx_p$ ist das Differential der Kartenabbildung $x$ im Punkt $p$ (siehe unten).
\qed

\medskip

{\bf Bemerkung:}
Die kanonische differenzierbare Struktur auf $\R^n$ ist definiert durch die Karte $(\R^n,x=\id)$. Daher liefert
$dx_p$ hier die kanonische Identifizierung
$$
\Phi : T_p\R^n \rightarrow \R^n, \quad [\alpha]\mapsto \dot \alpha(0) \ ,
$$
die nach Definition ein linearer Isomorphismus ist.
In der umgekehrten Richtung ordnet man einem Vektor $v\in \R^n$ die Klasse $[\alpha]$ zu mit
$\alpha(t) = p+tv$. Dabei ist $\alpha$ eine Kurve durch $p$ mit $\dot\alpha(0)=v$.


\medskip


\begin{Definition}
Sei $f:M\rightarrow N$ eine differenzierbare Abbildung. Dann ist das {\em Differential} von $f$ in $p$ definiert
durch
$$
df_p : T_pM \rightarrow T_{f(p)}N, \quad [\gamma]\mapsto [f\circ \gamma]  \ .
$$
Man schreibt auch: $df_p (\dot\gamma(0)) = \dot{f\circ \gamma}(0)$.
\end{Definition}

\medskip

Die Eigenschaften des Differentials f\"ur Funktionen zwischen Euklidischen R\"aumen \"ubertragen sich jetzt
direkt auf Mannigfaltigkeiten.

\medskip

\begin{Lemma}
Das Differential ist wohldefiniert, d.h. unabh\"angig von der Wahl der Repr\"asentanten, linear
und erf\"ullt die Kettenregel.
\end{Lemma}
\proof
Seien $\alpha$ und $\tilde\alpha$ zwei \"aquivalente Kurven auf $M$ durch $p$, d.h es gilt
$\dot{x\alpha}(0)=\dot{x\tilde\alpha}(0)$. Dann folgt
$$
\begin{array}{rl}
dy_{f(p)} [f\circ \alpha] &= \left.\frac{d}{dt}\right|_{t=0} yf\alpha = \left.\frac{d}{dt}\right|_{t=0} (yfx^{-1})x\alpha
=d (yfx^{-1}) \dot{x\alpha}(0)=d (yfx^{-1}) \dot{x\tilde\alpha}(0)\\[1ex]
&=dy_{f(p)} [f\circ \tilde\alpha]
\end{array}
$$
Es war schon gezeigt worden, dass die Abbildung $dy$ bijektiv ist und somit folgt die Gleichung $[f\circ \alpha]=[f\circ \tilde \alpha]$,
d.h. die Definition des Differentials $df$ ist unabh\"angig vom gew\"ahlten Repr\"asentanten des Tangentialvektors.
Die Kettenregel folgt aus einer formalen Rechnung:
$$
dg \circ df [\alpha] = dg (df [\alpha]) = dg [f\circ \alpha] = [g \circ f \circ \alpha] = d(g\circ f)[\alpha]\ .
$$
Das Differential $dx$ der Kartenabbildung ist nach Definition linear. Hierbei nutzt man die kanonische Identifikation
$\Phi : T_p\R^n \cong \R^n, [\alpha]\mapsto \dot \alpha (0)$ und unterscheidet nicht zwischen $dx_p:T_pM\rightarrow T_{x(p)}\R^n$
und dem vorher eingef\"uhrten $dx_p:T_pM\rightarrow \R^n$. Schreibt man  nun $f= y^{-1}\circ (yfx^{-1})\circ x$ folgt
nach der Kettenregel $df = (dy)^{-1}\circ d(yfx^{-1})\circ dx$. Damit ist das Differential als Verkn\"upfung linearer
Abbildungen linear.
\qed

\bigskip

{\bf Bemerkungen:} (1) $(d\,\id_M)_p = \id_{T_pM}$. (2) Ist $f$ ein Diffeomorphismus, dann ist $df_p$ f\"ur alle $p$ ein
Isomorphismus. (3) Ist $df_p$ ein Isomorphismus, dann ist $f$ lokal um $p$ ein Diffeomorphismus.

\bigskip

Den Tangentialraum in einem Punkt $p$ kann man identifizieren mit dem Raum der Derivationen auf Funktionskeimen um $p$.
Das soll im Folgenden genauer erkl\"art werden. Dieser Zugang verallgemeinert den Begriff der Richtungsableitung
aus der Analysis auf $\R^n$.

\bigskip

Zwei Funktionen $f$ und $h$ definiert auf einer Umgebung $U$ von $p$ hei\ss en \"aquivalent, $f\sim h$, falls eine
Umgebung $V\subset U$ von $p$ existiert, so dass $f|_V = h|_V$ gilt.

\begin{Definition} Die \"Aquivalenzklassen bzgl. $\sim$ differenzierbarer Funktionen, definiert  auf Umgebungen von
$p\in M$ nennt man {\em Funktionskeime} in $p$. Man schreibt $\mathcal C^\infty_p(M)$ f\"ur den Raum der
Funktionskeime in $p$.
\end{Definition}

\medskip

{\bf Bemerkung:} Man kann Funktionskeime addieren und multiplizieren, z.B. definiert man $[f]\cdot [g] =[f\cdot g]$.
Damit wird $\mathcal C^\infty_p(M)$ eine
reelle Algebra, d.h. ein Vektorraum mit einer vertr\"aglichen Multiplikation.

\bigskip

\begin{Lemma}
Sei $M$ eine n-dimensionale Mannigfaltigkeit. Dann ist:
$
\mathcal C^\infty_p(M) \cong \mathcal C^\infty_0(\R^n) \ .
$
\end{Lemma}
\proof
Sei $(U,x)$ eine Karte um $p$ mit $x(p)=0$, dann wird die Isomorphie definiert durch die Abbildung
$[f]\mapsto [f\circ x^{-1}]$.
\qed

\bigskip

\begin{Definition} Eine {\em Derivation} auf $\mathcal C^\infty_p(M)$ ist eine lineare Abbildung
$v:\mathcal C^\infty_p(M)\rightarrow \R$ mit
$$
v([f][g])= v([f])g(p) + f(p)v([g]) \ .
$$
Die Menge der Derivationen auf $\mathcal C^\infty_p(M)$ wird mit $\mathcal D_p(M)$
bezeichnet.
\end{Definition}

\bigskip

{\bf Bemerkung:} (1) $\mathcal D_p(M)$ ist ein reeller Vektorraum
(2) $\mathcal D_p(M)\cong \mathcal D_0(\R^n)$. Dieser Isomorphismus ist
folgendermassen definiert: Sei $v$ eine
Derivation in $\mathcal D_0(\R^n)$, dann definiert die Abbildung
$f\mapsto v(f\circ x^{-1})$ eine Derivation in $\mathcal D^\infty_p(M)$

\medskip

{\bf Beispiel:}
Jeder Vektor $v=(v_1,\ldots, v_n)\in \R^n$ definiert durch die Richtungsableitung eine
Derivation auf $\mathcal C^\infty_p(\R^n)$:
$$
v([f]) = \left.\frac{d}{dt}\right|_{t=0}f(p+tv) =df_p(v) = \la v, \grad f \ra = \sum^n_{j=1}
v_j \left.\frac{\partial f}{\partial x_j}\right|_p \ .
$$
Dabei entsprechen also
die Vektoren $\,e_i\,$ der kanonischen Basis, den partiellen
Ableitungen $\frac{\partial }{\partial x_i}, i=1,\ldots, n$. Im folgenden Satz wird gezeigt,
dass auch umgekehrt jede Derivation von dieser Form ist, d.h. man eine Isomorphie
$\R^n\cong \mathcal D_p(\R^n)$ hat.


\bigskip

\begin{Satz}
Jede Derivation $\delta$ auf $\mathcal C^\infty_0(\R^n)$ schreibt sich, angewendet
auf $[f] \in \mathcal C^\infty_0(\R^n)$, als
$$
\delta ([f]) = \sum^n_{j=1}\, \delta([x_j])\, \left.\frac{\partial f}{\partial x_j}\right|_0  \ .
$$
Insbesondere hat der Raum der Derivationen auf $\mathcal C^\infty_0(\R^n)$ die Dimension $n$,
mit den Derivationen $\frac{\partial }{\partial x_i}, i=1,\ldots, n$, als Basis.
\end{Satz}
\proof
Die Aussage des Satzes folgt aus einem fundamentalen Lemma, das zuerst bewiesen werden muss.

\begin{Lemma}\label{analysis}
Sei $f$ eine differenzierbare Funktion auf einer konvexen Umgebung von $0\in \R^n$. Dann kann
$f$ geschrieben werden als:
$$
f(x) = f(0) + \sum^n_{j=1}\,x_j\,h_j(x) \ ,
$$
f\"ur gewisse differenzierbare Funktionen $h_j$ mit $h_j(0) = \left.\frac{\partial f}{\partial x_j} \right|_0$.
\end{Lemma}
Beweis des Lemmas: Aus dem Hauptsatz der Differential- und Integralrechnung und der Kettenregel folgt
$$
f(x) - f(0) = \int^1_0 \frac{d}{dt} \, f(tx) dt = \int^1_0 \sum^n_{j=1} \frac{\partial f}{\partial x_j}
(tx)\,x_j \, dt
= \sum^n_{j=1} x_j\, \int^1_0 \frac{\partial f}{\partial x_j} (tx)\, dt \ .
$$

Um den Beweis des Satzes abzuschliessen, muss man noch bemerken, dass Derivationen auf konstanten
Funktionen verschwinden. Es gen\"ugt dies f\"ur die konstante Funktion $f\equiv 1$ nachzuweisen:
Sei $\delta$ eine Derivation, dann folgt $\delta(1\cdot 1)=\delta(1) + \delta(1)$ und somit
$\delta(1)=0$. Sei nun $[f]$ ein Funktionskeim in $0\in \R^n$ und $\delta$ eine beliebige Derivation.
Schreibt man $f$ wie in dem oben angef\"uhrten Lemma und wendet $\delta$ darauf an, so folgt
$$
\delta([f]) = \sum^n_{j=1}\, \delta(x_j)\, \left.\frac{\partial f}{\partial x_i} \right|_0 \ .
$$
Die Abbildung $[f] \mapsto \left.\frac{\partial f}{\partial x_j} \right|_0$, also die partielle
Ableitung nach $x_j$ im Punkt $0$ ist eine Derivation, die wie \"ublich mit $\frac{\partial }{\partial x_j} $
bezeichnet wird und, wie gezeigt, den Raum der Derivationen aufspannen. Offensichtlich sind sie
auch linear unabh\"angig, bilden also eine Basis in $\mathcal D_0(\R^n)$.
\qed


\bigskip


\begin{Definition}
Sei $v\in T_pM$ und $f\in \mathcal C^\infty(M)$. Dann ist die {\em Lie-Ableitung} von $f$ in Richtung
$v$ definiert als:
$$
L_v(f) = v(f) = df_p(v) = \left.\frac{d}{dt}\right|_{t=0} f(\alpha(t)) \ ,
$$
wobei $\alpha : (-\varepsilon, \varepsilon)\rightarrow M$ eine glatte Kurve in $M$ ist mit
$\alpha(0)=p$ und $\dot \alpha(0)=v$. Auf Funktionskeimen definiert man die Lie-Ableitung
durch $L_v([f]) = [L_v(f)]$.
\end{Definition}




\bigskip

\begin{Satz}
Die Abbildung $\,\xi\mapsto L_\xi$ definiert eine lineare Isomorphie zwischen $T_pM$ und dem
Raum der Derivationen auf $\mathcal C^\infty_p(M)$, d.h.
$$
T_pM \cong \mathcal D_p(M) \ .
$$
\end{Satz}
\proof
Sei $(U,x)$ eine Karte um $p\in M$ mit $x(p)=0\in\R^n$, sei $\xi\in T_pM$ ein Tangentialvektor
mit $dx_p(\xi)=v= (v_1,\ldots, v_n)\in \R^n$. Dann folgt
$$
L_\xi(f) = df_p(\xi) = d (f  x^{-1} x )_p(\xi)= d(f x^{-1})_0\,dx_p(\xi) = v(f x^{-1})(0)
= \sum^n_{j=1}\, v_j \, \frac{\partial f x^{-1}}{\partial x_j}(0) \ .
$$
Aus dieser Formel folgt sofort, dass die Zuordnung $\xi \mapsto L_\xi$ injektiv ist. Denn gilt
$L_\xi(f)=0$ f\"ur alle $f$ dann auch f\"ur $f=x_j$, woraus dann $v_j=0$ und schlie\ss lich auch
$\xi=0$ folgt. Die Zuordnung ist bijektiv aus Dimensionsgr\"unden: $\dim T_M = n = \dim \mathcal D_0(\R^n)
= \mathcal D_p(M)$.
\qed

\bigskip

{\bf Bezeichnung:}
Sei $f \in \mathcal C^\infty(M)$ und sei $(U,x)$ eine Karte um $p\in M$. Dann schreibt man
$$
\frac{\partial f }{\partial x_j}(p) = \left.\frac{\partial }{\partial x_j}\right|_p(f)
= \frac{\partial f x^{-1}}{\partial x_j}(x(p)) \ .
$$
Genaugenommen betrachtet man $f$ auf der Kartenumgebung $U$ von $p$ bzw. geht zu dem Funktionskeim
$[f]$ in $p$ \"uber.

\bigskip

{\bf Bemerkungen:}
\begin{enumerate}
\item
Die Derivationen $\left.\frac{\partial }{\partial x_j}\right|_p, j=1,\ldots,  n$ bilden eine Basis
in $\mathcal D_p(M)\cong T_pM$, d.h. jeder Tangentialvektor $\xi \in T_pM$ schreibt sich eindeutig
als
$$
\xi = \sum^n_{j=1} \, v_j \, \left.\frac{\partial }{\partial x_j}\right|_p, \qquad v_j \in \R \ .
$$
\item
Da $\frac{\partial x_i}{\partial x_j}= \delta_{ij}$, bestimmen sich die Koeffizienten $v_j$ durch
$\;
v_j = L_\xi(x_j) = dx_j(\xi) \ .
$
Insbesondere gilt:
$$
\left.\frac{\partial }{\partial x_j}\right|_p = (dx_p)^{-1} (e_j) \ .
$$

\item
Die Derivationen $\left.\frac{\partial }{\partial x_j}\right|_p \in T_pM$ entsprechen den Kurven
$$
\alpha_j(t) = x^{-1} (x(p) + te_j) \ .
$$
\end{enumerate}

\section{Das Tangentialb\"undel}

\begin{Definition}
Das {\em Tangentialb\"undel} $TM$ einer Mannigfaltigkeit $M$ ist definiert als die disjunkte
Vereinigung aller Tangentialr\"aume:
$$
TM \;:=\; \bigcup_{p\in M} \, T_pM \ .
$$
\end{Definition}

\bigskip

\begin{Satz}
Sei $M$ eine $n$-dimensionale differenzierbare Mannigfaltigkeit. Dann tr\"agt $TM$ die Struktur einer
$2n$-dimensionalen differenzierbaren Mannigfaltigkeit. Die Fu\ss punktabbildung $\pi: TM \rightarrow M$
ist surjektiv und differenzierbar.
\end{Satz}
\proof
Man definiert die kanonische Projektion als die Fu\ss punktabbildung:
$$
\pi : TM \rightarrow M, \qquad \pi(\xi) = p \qquad \mbox{falls}\quad \xi \in T_pM \ .
$$
Sei $(U,x)$ eine Karte von $M$, dann definiert man
$$
\begin{array}{rl}
&\Phi : \pi^{-1}(U) \rightarrow x(U) \times \R^n \subset \R^{2n}\\[1ex]
&\Phi(\xi) = (x\circ \pi (\xi), dx_{\pi(\xi)}(\xi))
\end{array}
$$
Es ist klar, dass $\Phi$ eine bijektive Abbildung ist. Die Topologie auf $TM$ wird durch
die Forderung definiert, dass alle Abbildungen $\Phi$ Hom\"oomorphismen sind.

\medskip

Nun ist noch der Kartenwechsel zu berechnen. Daf\"ur seien $(U,x),\,(V,y)$ Karten um $p\in M$
mit den assoziierten Karten  $(\pi^{-1}(U), \Phi), (\pi^{-1}(V), \Psi)$. Dann gilt
$$
\begin{array}{rl}
\Psi \circ \Phi^{-1} : & x (U \cap V) \times \R^n \rightarrow y(U\cap V) \times \R^n\\[1ex]
& (q,v) \mapsto (y\circ x^{-1}(q), d(yx^{-1})_q v) \ .
\end{array}
$$
Denn f\"ur $\xi = \Phi^{-1}(q,v)$ folgt $\pi(\xi)=x^{-1}(q)$ und $v = dx_{\pi(\xi)}(\xi)$,
wobei $x(\pi(\xi))= q$. Wendet man darauf $\Psi$ an, so ergibt sich
$$
\Psi \circ \Phi^{-1}(q, v) =\Psi(\xi) = (y\circ \pi (\xi), dy_{\pi(\xi)}(\xi))
= (yx^{-1}(q), dy_{\pi(\xi)}\circ (dx_{\pi(\xi)})^{-1}v) \ ,
$$
der Kartenwechsel ist also differenzierbar und die Karten $(\pi^{-1}(U), \Phi)$
definieren eine differenzierbare Struktur auf $TM$.
\qed

\bigskip

Offensichtlich hat $TM$ eine abz\"ahlbare Basis der Topologie und $\pi: TM \rightarrow M$
ist differenzierbar, denn
$$
x\circ \pi \circ \Phi^{-1} : x(U) \times \R^n \rightarrow x(U), (x_1,\ldots, x_{2n})\mapsto (x_1,\ldots, x_n) \ .
$$
Es bleibt noch, die Hausdorff-Eigenschaft zu zeigen. Daf\"ur seien $(p,a)$ und $(q,b)$ zwei verschiedene Punkte
in $TM$, mit $a\in T_pM$ und $b\in T_qM$.

\medskip

1. Fall: Sei $p\neq q$. Dann existieren offene Umgebungen $U, V$ von $p,q$ mit $U\cap V= \emptyset$. Dann sind aber
auch $\pi^{-1}(U)$ und  $\pi^{-1}(V)$ disjunkte offene Umgebungen von $(p,a)$ und $(q,b)$.

\medskip

2. Fall: Sei $p=q, a\neq b$ in $T_pM$. Sei $(U,x)$ eine Karte um $p$ mit der Karte
$\Phi: \pi^{-1}(U)\rightarrow x(U)\times \R^n$ f\"ur $TM$ und $\Phi(a)=(x(p),v), \Phi(b)=(x(p),w)$.
Sind $V,W$ offene Mengen in $\R^n$ mit $v\in V, w\in W$ und $V\cap W = \emptyset$, dann sind
$\Phi^{-1}(x(U)\times V)$ und $\Phi^{-1}(x(U)\times W)$ disjunkte offene Umgebungen von $a,b$.


\subsection{Vektorb\"undel}

Das Tangentialb\"undel $TM$ ist ein Beispiel f\"ur ein reelles Vektorb\"undel. Diese sind
folgendermassen definiert.

\begin{Definition}
Seien $E$ und $B$ differenzierbare Mannigfaltigkeiten, $\pi:E\rightarrow B$ eine
differenzierbare Abbildung. Ein {\em Vektorb\"undel} vom Rang $n$ ist ein Tripel
$(E,\pi,B)$ mit:
\begin{enumerate}
\item
$\pi$ ist surjektiv
\item
Es existiert eine offene \"Uberdeckung $\{U_i\}_{i\in I}$ von $B$ und Diffeomorphismen
$$
h_i : \pi^{-1}(U_i) \rightarrow U_i \times \R^n
$$
mit:
\begin{itemize}
\item
$h_i(\pi^{-1}(x)) = \{x\} \times \R^n $
\item
$ h_i \circ h_j^{-1} : (U_i \cap U_j) \times \R^n \rightarrow (U_i \cap U_j) \times \R^n, \quad (x,v) \mapsto (x,g_{ij}(x)v)$
\newline f\"ur differenzierbare Abbildungen $\; g_{ij}: U_i\cap U_j \rightarrow GL(n,\R)$.
\end{itemize}
\end{enumerate}
\end{Definition}

\medskip

{\bf Bemerkungen:}
\begin{enumerate}
\item
Das Urbild $E_x=\pi^{-1}(x)$ hei\ss t {\it Faser} \"uber $x\in B$. Die Fasern $E_x$ sind f\"ur alle $x$
Vektorr\"aume isomorph zu $\R^n$.
\item
Die Mannigfaltigkeit $E$ nennt man {\it Totalraum}, $B$ die {\it Basis} und $h_i$ die {\it
lokalen Trivialisierungen} des Vektorb\"undels $\pi:E\rightarrow B$.
\end{enumerate}


\medskip


{\bf Beispiele:} (1) Das {\it Tangentialb\"undel} $\pi:TM\rightarrow M$ einer $n$-dimensionalen Mannigfaltigkeit
$M$ ist ein Vektorb\"undel vom Rang $n$. (2) Das {\it triviale B\"undel} \"uber $B$ ist definiert als
 $\;\pi: E=B\times \R^n\rightarrow B$,
wobei $\pi$ die Projektion auf den ersten Faktor ist.


\medskip


\begin{Definition}
Eine differenzierbare Abbildung $s:B\rightarrow E$ hei\ss t {\em Schnitt} des Vektorb\"undels, falls
$$
\pi \circ s = \id_B \ ,
$$
d.h. falls $s(x)\in E_x$ f\"ur alle $x\in B$. Den Raum der Schnitte in $E$ bezeichnet man mit $\Gamma(E)$.
\end{Definition}


\section{Vektorfelder}

\begin{Definition}
Ein {\em Vektorfeld} ist ein Schnitt des Tangentialb\"undels, d.h. eine differenzierbare
Abbildung $X:M\rightarrow TM$ mit $\pi\circ X = \id_M$, also $X_p=X(p) \in T_pM$ f\"ur
alle $p\in M$.
\end{Definition}

\medskip

{\bf Beispiel:} Sei $M=S^n$ dann schreibt sich der Tangentialraum in einem Punkt
$p\in S^n$ als $T_pS^n = p^\perp =  \{v \in \R^{n+1} | \, \la v, p \ra = 0\}$.
Ein Vektorfeld auf $S^n$ ist also eine glatte Abbildung $v:S^n\rightarrow \R^{n+1}$
mit $\la v(p), p \ra = 0$ f\"ur alle $p\in S^n$. Interessant ist die Frage, ob
es Vektorfelder ohne Nullstellen gibt, also ohne $p$ mit $v(p)=0$. Auf der $S^{2n+1}$
l\"a\ss t sich leicht ein Beispiel daf\"ur angeben:
$$
v(x_0, \ldots, x_{2n+1}) := (-x_1,x_0, \ldots, -x_{2n+1}, x_{2n}) \ .
$$
Der Satz vom Igel in allgemeiner Fassung besagt nun:

\medskip

{\bf Satz:} Jedes Vektorfeld auf $S^{2n}$ hat eine Nullstelle.

\bigskip

{\bf Bemerkung:}
\begin{enumerate}
\item
Sei $n+1= (2r+1)2^{c+4d}$ mit $0\le c \le 3$. Dann gibt es auf der $S^n$
genau $2^c+8d-1$ punktweise linear unabh\"angige Vektorfelder, die also
insbesondere keine Nullstellen haben.

\item
$S^n$ hat $n$ linear unabh\"angige Vektorfelder genau dann, wenn $n=1,3$ oder $7$.

\item
Gibt es auf einer $n$-dimensionalen Mannigfaltigkeit $M$ die maximal m\"ogliche
Anzahl von $n$ linear unabh\"angigen Vektorfeldern, so sagt man, dass $M$
parallelisierbar bzw. das Tangentialb\"undel trivial ist.

\item
Den unendlich-dimensionalen Vektorraum der Vektorfelder auf $M$ bezeichnet man mit
$\chi(M)$ bzw. $\Gamma(TM)$.
\end{enumerate}

\bigskip

Tangentialvektoren kann man identifizieren mit Derivationen auf Funktionskeimen. Analog
kann man Vektorfelder identifizieren mit Derivationen auf dem Raum $\mathcal C^\infty(M)$
der glatten Funktionen auf $M$. Die Identifikation wird wieder durch eine Lie-Ableitung
gegeben.

\begin{Definition}
Eine {\em Derivation} $\delta$ auf $\mathcal C^\infty(M)$ ist eine lineare Abbildung
$\delta : \mathcal C^\infty(M)\rightarrow \mathcal C^\infty(M)$ mit der Produktregel:
$$
\delta (f\cdot g) = \delta(f) \cdot g + f \cdot \delta(g) \ .
$$
Der Raum der Derivationen auf $M$ wird mit $\mathcal D(M)$ bezeichnet.
\end{Definition}

\bigskip

{\bf Beispiel:} Jedes Vektorfeld $X\in \chi(M)$ definiert durch die Lie-Ableitung
$L_X:\mathcal C^\infty(M)\rightarrow \mathcal C^\infty(M)$ eine Derivation auf $M$.
Dabei ist $L_X(f)(p) := L_{X_p}[f] = df_p(X_p)$.

\bigskip

\begin{Satz}
Die Abbildung $X\mapsto L_X$ von $\Gamma (TM)$ nach  $\mathcal D(M)$ ist
ein Isomorphismus unendlich-dimensionaler Vektorr\"aume.
\end{Satz}
\proof
Klar ist, dass $X\mapsto L_X$ eine lineare Abbildung ist, und dass die Lie-Ableitung
$L_X$ tats\"achlich eine Derivation auf $M$ definiert.

\medskip

Als erstes soll gezeigt werden, dass $X\mapsto L_X$ injektiv ist, d.h. f\"ur $X\in \chi(M)$ mit
$X\neq 0$ ist $L_X\neq 0$ zu zeigen. Daf\"ur sei $X\neq 0$ ein Vektorfeld auf $M$, d.h. es
existiert ein $p\in M$ mit $X_p=X(p) \neq 0$. Nun war aber schon gezeigt worden, dass die
Abbildung $X_p\in T_pM \mapsto L_{X_p}\in \mathcal D_p(M)$ injektiv ist. Damit existiert
eine Funktion $f$, definiert auf einer Umgebung $U$ von $p$ mit $L_{X_p}[f]= df_p(X_p)\neq 0$.

\medskip

Man kann nun $f$ zu einer auf ganz $M$ definierten Funktion fortsetzen. Daf\"ur w\"ahlt man
eine Funktion $\phi \in \mathcal C^\infty(M) $ mit $\supp (\phi )\subset U$ und $\left. \phi \right|_V\equiv 1$
auf einer Umgebung $V\subset U$ von $p$. Dann ist $f\cdot \phi$ die gew\"unschte Fortsetzung von
$f$, deren Lie-Ableitung nach $X$ nicht verschwindet:
$$
L_X(f \cdot \phi)(p) = d(f \cdot \phi)_p(X_p)= df_p(X_p) \neq 0 \ .
$$
Die Existenz der sogenannten {\it Abschneidefunktion} $\phi$ bleibt noch zu zeigen.

\medskip

Als zweites ist zu zeigen, dass $X\mapsto L_X$ surjektiv ist. Zun\"achst bemerkt man, dass sich mit
Hilfe von Abschneidefunktionen jeder Funktionskeim einer lokal definierten Funktion als Keim einer
global definierten Funktion auffassen l\"a\ss t. Sei nun $\delta$ eine Derivation auf $M$. Dann
definiert $\delta_p : f\mapsto \delta(f)(p)$ eine Derivation auf den Funktionskeimen $\mathcal C^\infty_p(M)$.
Wie schon gezeigt wurde, schreibt sich $\delta_p$ als
$
\delta_p =\sum^n_{j=1} \, \delta_p(x_j)\,\left. \frac{\partial}{\partial x_j}\right|_p \ .
$
Damit erh\"alt man ein Vektorfeld $X$ auf $M$: in jedem Punkt $p\in M$ definiert man $X_p$
als den Tangentialvektor, der zur Derivation $\delta_p$ geh\"ort. Es gilt dann $L_X= \delta$.
In lokalen  Koordinaten $(U,x)$ schreibt sich $X$ als
$$
\left. X\right|_U = \sum^n_{j=1}\, \delta(x_j)\, \frac{\partial}{\partial x_j} \ ,
$$
womit gezeigt ist, dass $X$ ein glattes Vektorfeld ist.

\medskip

Zur Konstruktion der Abschneidefunktion $\phi$:


\begin{Lemma}
Sei $R>0$, dann existiert eine glatte Funktioin $\phi : \R^n\rightarrow \R$ mit
$$
\left. \phi\right|_{B_0(\frac13R)} \equiv 1
\qquad \mbox{und}  \qquad
\left. \phi\right|_{\R^n \setminus B_0(\frac23R)} \equiv 0 \ .
$$
\end{Lemma}
%\proof
%Man definiert $\phi$ zun\"achst auf $\R$ und setzt dann, $\phi(x) := \phi (\|x\|^2)$
%f\"ur $x\in \R^n$. Sei
%\qed

\subsection{Die Lie-Algebra der Vektorfeldern}



Die Verkn\"upfung zweier Derivationen ist i.A. keine Derivation. Es gilt:
$$
\delta_1 \, \delta_2 (f\cdot g) = (\delta_1 \,\delta_2 f)\cdot g  + \delta_2 f \cdot \delta_1 g
+\delta_1 f \cdot \delta_2 g + f \cdot \delta_1\delta_2 g \ .
$$
Insbesondere hat man damit gezeigt:

\begin{Lemma}
Seien $\delta_1, \delta_2$ zwei Derivationen auf $\mathcal C^\infty$. Dann ist
$\delta_1 \, \delta_2 - \delta_2 \, \delta_1$ wieder eine Derivation.
\end{Lemma}

\begin{Definition}
Seien $X,Y$ Vektorfelder auf $M$. Dann ist $[X,Y]$ das eindeutig bestimmte Vektorfeld mit
$$
L_{[X,Y]} = L_X\circ L_Y  -  L_Y\circ L_X  \ .
$$
Man nennt $[X,Y]$ die {\em Lie-Klammer} bzw. den {\em Kommutator } der Vektorfelder
$X,Y$. Die {\em Lie-Ableitung} von $Y$ nach $X$ ist definiert als $L_XY:=[X,Y]$.
\end{Definition}

\bigskip

\begin{Lemma}
Die Vektorfelder $X,Y\in \chi(M)$ seien lokal gegeben als
$X = \sum a_i \frac{\partial}{\partial x_i}$ und $Y = \sum b_i \frac{\partial}{\partial x_i}$.
Dann gilt
$$
[X,Y] = \sum^n_{i,j=1} \left( a_j \frac{\partial b_i}{\partial x_j} - b_j \frac{\partial a_i}{\partial x_j}
\right) \frac{\partial}{\partial x_i} \ .
$$
\end{Lemma}
\proof
Man schreibt das Vektorfeld $[X,Y]$ lokal als $[X,Y]= \sum c_i \frac{\partial }{\partial x_i}$. Die Koeffizienten
$c_i$ bestimmen sich durch
$$
c_i = [X,Y](x_i)= L_X(L_Y(x_i))-L_Y(L_X(x_i)) = L_X(b_i) - L_Y(a_i)= \sum_j  a_j \frac{\partial b_i}{\partial x_j}
-  \sum_j b_j \frac{\partial a_i}{\partial x_j} \ .
$$
\qed

\bigskip

\begin{Satz}
Seien $X,Y, Z$ Vektorfelder auf $M$, $a,b$ reelle Zahlen und $f$ eine differenzierbare Funktion
auf $M$, dann gilt:
\begin{enumerate}
\item\label{a}
$[X, Y] = - [Y,X] \ ,\qquad $  (Schiefsymmetrie bzw. Antikommutativit\"at)
\item\label{b}
$[a X + b Y, Z] = a [X, Z] +  b [Y, Z] \ ,\qquad $ (Linearit\"at)
\item\label{c}
$[X,[Y,Z]]  + [Y,[Z,X]]  + [Z,[X,Y]] = 0 \ ,\qquad $  (Jacobi-Identit\"at)
\item
$[fX,Y] = f[X,Y] - Y(f) X,\quad [X,f Y] = f[X,Y] + X(f)Y \ .$
\end{enumerate}
\end{Satz}

\bigskip

{\bf Bemerkung:}
Eine reelle {\it Lie-Algebra} ist ein reeller Vektorraum $V$ mit einer Abbildung
$[\cdot, \cdot] : V\times V \rightarrow V$, die die Eigenschaften
$\ref{a}, \ref{b}, \ref{c}$ erf\"ullt. Die Lie-Klammer $[\cdot, \cdot]$
ist also eine bilineare schiefsymmetrische Abbildung, die die Jacobi-Identit\"at erf\"ullt.

\medskip

Der Raum $\chi(M)$ der Vektorfelder \"uber einer Mannigfaltigkeit $M$ zusammen mit dem Kommutator ist eine
unendlichdimensionale reelle Lie-Algebra.


\subsection{Das Bild von Vektorfeldern unter Diffeomorphismen}

\begin{Definition}
Sei $f:M\rightarrow N$ ein Diffeomorphismus, dann ist das Bild des Vektorfeldes $X\in \chi(M)$
unter $f$ ein Vektorfeld auf $N$ definiert durch
$$
(f_* X)_q := df_{f^{-1}(q)} (X_{f^{-1}(q)}) \ .
$$
Zwei Vektorfelder $X,Y$ nennt man $f$-verkn\"upft, falls es einen Diffeomorphismus
$f$ gibt mit $Y = f_* X$.
\end{Definition}


\begin{Satz}
Sei $f:M\rightarrow N$ ein Diffeomorphismus und $X\in \chi(M)$. Dann ist $f_* X$
ein glattes Vektorfeld auf $N$ und es gilt:
\begin{enumerate}
\item
$
L_{f_*X} (\phi ) = L_X(\phi \circ f) \circ f^{-1}
$
\item
$
f_* [X,Y] = [f_*X, f_*Y]
$
\end{enumerate}
\end{Satz}

\bigskip

\subsection{Vektorfelder und Differentialgleichungen}

\bigskip

Sei $X\in \chi(M)$ ein Vektorfeld auf $M$ und $(U,x)$ eine Karte um $p \in M$. Dann schreibt
sich $X$ eingeschr\"ankt auf $U$ als:
$$
X = \sum^n_{i=1}\, a_i \, \frac{\partial}{\partial x_i}
$$
f\"ur glatte Funktionen $a_i:U \rightarrow \R$. Man sucht nun  glatte Kurven
$c: (-\varepsilon, \varepsilon)\rightarrow M$ mit
\begin{enumerate}
\item
$\qquad c(0) = p$
\item\label{2}
$\qquad \dot c (t) = X_{c(t)}$ \qquad f\"ur alle $t\in (-\varepsilon, \varepsilon)$
\end{enumerate}

Der Tangentialvektor $\dot c (t)$ schreibt sich in der Karte $(U,x)$ als
$\dot c (t) = \sum^n_{i=1} \dot c_i (t) \frac{\partial}{\partial x_i}$ wobei
$c_i$ definiert ist durch $c_i:= x_i \circ c : \R\rightarrow \R$.
Gleichung (\ref{2}) ist damit \"aquivalent zu
$$
\dot c_i(t) = a_i(c(t)) =  (a_i \, x^{-1})\,(c_1(t),\ldots,c_n(t)) \qquad \mbox{f\"ur}\quad i=1,\ldots, n  \ .
$$
Zur Bestimmung der Kurve $c$ erh\"alt man also $n$ gew\"ohnliche Differentialgleichungen f\"ur die
Funktionen $c_i, i=1,\ldots, n$ mit den Anfangsbedingungen $c_i(0)=a_i(p)$.

\bigskip

\begin{Definition}
Eine Kurve $c: (-\varepsilon, \varepsilon)\rightarrow M$ hei\ss t {\em Integralkurve} des Vektorfeldes
$X\in \chi(M)$ durch $p$, falls $c(0)=p$ und $\dot c (t) = X_{c(t)}$ f\"ur alle $t\in (-\varepsilon, \varepsilon)$.
\end{Definition}

\bigskip

Aus dem Satz \"uber die lokale Existenz und Eindeutigkeit der L\"osungen gew\"ohnlicher Differentialgleichungen
bzw. dem Satz \"uber die differenzierbare Ab\"angigkeit von den Anfangswerten ergeben sich die folgenden
Eigenschaften der Integralkurven:

\begin{Satz}
F\"ur alle $p\in M$ existiert ein Intervall $I_p\subset \R$ um $0$
 und eine eindeutig bestimmte Kurve $c_p: I_p\rightarrow M$ mit
 $$
 c_p(0)= p \qquad \mbox{und} \qquad \dot c_p(t) = X_{c(t)}
 $$
 f\"ur alle $t\in I_p$, d.h. die Integralkurven von $X$ sind (lokal) eindeutig bestimmt und
 existieren lokal.
 \end{Satz}

 \medskip

 \begin{Satz}
 F\"ur alle $p\in M$ existiert eine offene Umgebung $U$ von $p$ und ein Intervall $I$ um Null, so dass
 f\"ur alle $q\in U$ die Kurve $c_q$ auf $I$ definiert ist.  Die Abbildung
 $$
 I \times U \rightarrow M, \qquad (t,q)\mapsto c_q(t)
 $$
 ist differenzierbar.
 \end{Satz}

 \bigskip

 \begin{Folgerung}
 Durch jeden Punkt der Mannigfaltigkeit $M$ geht genau eine Integralkurve von $X$, d.h. insbesondere,
 dass sich Integralkurven nicht schneiden.
 \end{Folgerung}

 \bigskip

 \begin{Definition}
 Die Abbildung $(t,q)\mapsto c_q(t)$ hei\ss t der {\em lokale Flu\ss{}} des Vektorfeldes $X$.
 Die Integralkurven von $X$ nennt man auch {\em Flu\ss linien} von $X$. Der Flu\ss{} definiert
 auch f\"ur alle hinreichend kleinen Parameter $t$ eine lokale Abbildung $\phi_t : U \subset M
 \rightarrow M$ durch  $\phi_t(q)= c_q(t)$.
 \end{Definition}

\bigskip

\begin{Satz}
Die Abbildungen $\phi_t$ sind lokale Diffeomorphismen und es gilt
$$
\phi_t \circ \phi_s = \phi_{t+s}
$$
f\"ur alle Parameter $t,s$, f\"ur die $\phi_t, \phi_s$ und $\phi_{t+s}$
definiert sind.
Man sagt: $\{\phi_t\}$ ist eine {\it 1-parametrige Gruppe} (lokaler)
Diffeomorphismen.
\end{Satz}
\proof
Aus der Definition der Flu\ss -Abbildung $\phi_t$ folgt:
$$
\phi_{t+s}(q) = c_q(t+s)= c_{c_q(s)}(t) = \phi_t(c_q(s)) = \phi_t \circ \phi_s (q)
$$
Insbesondere folgt aus dieser Gleichung $\id = \phi_0 = \phi_{t + (-t)} = \phi_t \circ \phi_{-t}$,
also $\phi_t^{-1}= \phi_t$. Damit ist $\phi_t$ invertierbar und die Umkehrabbildung ist wieder
differenzierbar, d.h. $\phi_t$ ist ein (lokaler) Diffeomorphismus.
\qed

\bigskip

{\bf Beispiel:}
Sei $M= \R$ und $X$ das Vektorfeld $X_x = x^2 \frac{d}{dx}$ f\"ur $x\in \R$. Die Integralkurve
von $X$ durch $x$ ist gegeben durch
$
c(t) = \frac{x}{1-tx}
$. Denn:
$$
\dot c (t) = x \, \frac{-1}{(1-tx)^2}\,(-x)= \frac{x^2}{(1-tx)^2}= c(t)^2 = X_{c(t)}
$$
wobei $T_x\R$ mit $\R$ identifiziert wird. Die Integralkurve von $X$ durch $x$ ist f\"ur
folgende Parameter $t$ definiert:
$$
t \in (-\infty, \frac1x) \qquad \mbox{falls}\quad x >0, \qquad
t \in (\frac1x, \infty) \qquad \mbox{falls}\quad x <0 \ .
$$
Der Flu\ss{} ist also definiert auf $I_x=(-\frac1{|x|},\frac1{|x|})$ und es
folgt, dass $I_x$ immer kleiner wird f\"ur $x\mapsto \pm\infty$.

\bigskip

\begin{Definition}
Ein Vektorfeld $X\in \chi(M)$ hat kompakten Tr\"ager, falls der Abschlu\ss{} der
Menge $\{p\in M \,|\, X_p\neq 0 \}$ kompakt ist.
\end{Definition}

\bigskip

\begin{Satz}
Ist $X$ ein Vektorfeld mit kompaktem Tr\"ager, so ist sein Flu\ss{} $\phi_t$
f\"ur alle $t\in \R$ definiert und jedes $\phi_t:M\rightarrow M$ ist ein
globaler Diffeomorphismus.
\end{Satz}

{\bf Bemerkung:} Die Voraussetzung des Satzes ist zum Beispiel f\"ur kompakte
Mannigfaltigkeiten erf\"ullt. Ist der lokale Flu\ss{} von $X$ f\"ur alle $t$
definiert, so nennt man $X$ vollst\"andig.

\bigskip

\proof
F\"ur alle $p\in M$ existieren offene Umgebungen $V$ von $p$ und ein Intervall
$I_q= (-\varepsilon, \varepsilon)$, so dass der Flu\ss{} $\varphi_t(q)$ f\"ur alle
$q\in V, t \in I_q$ definiert ist. Endlich viele dieser Mengen \"uberdecken den
Tr\"ager $K=\supp(X): \; V_1,\ldots, V_r$, d.h.
$$
X_q = 0 \qquad \mbox{f\"ur alle}\quad q \in M \setminus \bigcup^r_{i=1} V_i \ .
$$
Durch Punkte $q\in M$ mit  $X_q=0$ existiert immer die konstante Integralkurve
$c(t)=q$.

\medskip

Seien $(-\varepsilon_i, \varepsilon_i)$ die Definitionsbereiche der Fl\"usse auf $V_i$
und $\varepsilon:= \min \varepsilon_i$. Dann ist $\varphi_t(q)$ definiert f\"ur
alle $q\in M$ und $t\in (-\varepsilon, \varepsilon)$. Sei $T\in \R$ beliebig
und $|T| = k \,\frac{\varepsilon}{2}+r$, mit $k \in \N$ und $r\in (0,\frac{\varepsilon}{2})$.
Man definiert nun
$$
\varphi_T := (\varphi_{\frac{\varepsilon}{2}})^k\cdot \varphi_r \quad \mbox{f\"ur} \quad T>0
\qquad \mbox{und} \qquad
\varphi_T := (\varphi_{-\frac{\varepsilon}{2}})^k\cdot \varphi_r \quad \mbox{f\"ur} \quad T<0 \ .
$$
Mit dieser Definition wird $\{\varphi_t\}$ ein global definierter Flu\ss{} des Vektorfeldes $X$.
\qed

\bigskip

\subsection{Lie-Ableitung von Vektorfeldern}

Die Lie-Ableitung von Funktionen und Vektorfeldern kann man auch mit Hilfe des lokalen
Flu\ss es beschreiben. Sei $X$ ein glattes Vektorfeld auf $M$ mit dem lokalem Flu\ss{}
$\varphi_t : M \rightarrow M$. Dann ist $X$  nach Definition tangential zu den Flu\ss linien
$c_p(t) = \varphi_t(p)$, woraus folgt:
$$
\begin{array}{rl}
L_X(f)_p & = df_p(X_p) = \left. \frac{d}{dt}\right|_{t=0} f(\phi_t(p)) =
\lim_{t\rightarrow 0}\frac1t(f(\varphi_t(p))-f(p))\\[1ex]
&=
\lim_{t\rightarrow 0}\frac1t( (\varphi_t^* f)(p)-f(p)) \ ,
\end{array}
$$
wobei das  Zur\"uckziehen von Funktionen $f\in \mathcal C^\infty(M)$ mittels $\varphi_t$  durch
Verkn\"upfung definiert ist, d.h.
$(\varphi_t^*f)(p) = f(\varphi_t(p))$.
Analog, also unter Benutzung des lokalen Flu\ss es kann man nun auch die Lie-Ableitung anderer Objekte
beschreiben. F\"ur Vektorfelder definiert man das Zur\"uckziehen mittels der lokalen Diffeomorphismen
$\varphi_t$ durch
$$
(\varphi_t^*Y)_p = ((\varphi_{-t})_* Y)_p =
(d\varphi_{-t})_{\varphi_{t}(p)} Y_{\varphi_{t}(p)} \ .
$$
wobei $\varphi_t^{-1}=\varphi_{-t}$ und $d\varphi_{-t} : T_{\varphi_{t}(p)}M\rightarrow T_pM$.

\begin{Satz}
Seien $X,Y$ beliebige glatte Vektorfelder auf $M$. Dann gilt in $p\in M$:
$$
L_XY_p = [X,Y]_p \; = \; \lim_{t\rightarrow 0}\tfrac1t( (\varphi_t^*Y)_p - Y_p)
\; = \;
\lim_{t\rightarrow 0}\tfrac1t( Y_p - (\varphi_{t\,*} Y)_p )\ .
$$
\end{Satz}
\proof
Beim \"Ubergang von $\varphi_t^*$ zu $\varphi_{t\,*}$ geht $t$ in $-t$ \"uber, daher stimmen
die beiden Grenzwerte in der Formel f\"ur $L_XY$ offensichtlich \"uberein.

\medskip

Sei $f:M\rightarrow \R$ eine glatte Funktion. Dann definiert man
$f(t,p) := f(\varphi_t(p))-f(p)$. Es existiert eine glatte Funktion $g$ mit
$f(t,p) = t \, g(t,p)$. Insbesondere folgt:
$$
g(0,p) = \lim_{t\rightarrow 0} g(t,p) =
\left. \frac{d}{dt}\right|_{t=0} f(\varphi_t(p)) = X_p(f) = L_X f_p
$$
Die Existenz der Funktion $g$ ist ein Spezialfall des Lemmas~\ref{analysis}.
Unmittelbar aus der Definition des Differentials ergibt sich nun die
folgende Rechnung
$$
\begin{array}{rl}
(\varphi_t^*Y)_p(f)
&=
(d\varphi_{-t})_{\varphi_{t}(p)} Y_{\varphi_{t}(p)}(f) \\[1ex]
&=
Y_{\varphi_{t}(p)}(f\circ \varphi_{-t})\\[1ex]
&=
Y_{\varphi_{t}(p)}(f(-t, \cdot) + f)\\[1ex]
&=
-t\,Y_{\varphi_{t}(p)}(g(t,\cdot)) + Y_{\varphi_{t}(p)}(f)\\[1ex]
&=
Y(f)_{\varphi_t(p)}-t\,Y_{\varphi_{t}(p)}(g(t,\cdot))
\end{array}
$$
F\"ur den Grenzwert folgt also:
$$
\begin{array}{rl}
\lim_{t\rightarrow 0} \frac1t ((\varphi_t^*Y)_p - Y_p)(f)
&=
\lim_{t\rightarrow 0}\frac1t ( Y(f)_{\varphi_t(p)}- Y(f)_p  ) -
\lim_{t\rightarrow 0} Y_{\varphi_t(p)}(g(t,\cdot))\\[1ex]
&=
\left. \frac{d}{dt}\right|_{t=0} Y(f)(\varphi_t(p)) - Y_p(X(f))\\[1ex]
&=
X_p(Y(f)) - Y_p(X(f))\\[1ex]
&=
[X,Y]_p(f) \ .
\end{array}
$$
\qed

\subsection{Erg\"anzungen zu Vektorfeldern}

\bigskip

\begin{Satz}\label{karte}
Sei $X$ ein Vektorfeld auf $M$ mit $X_p\neq 0$ f\"ur ein $p\in M$.
Dann existiert eine Karte $(U,y)$ um $p$ mit
$$
X = \left. \frac{\partial}{\partial y_1}\right.
\qquad \mbox{auf}\quad U \ .
$$
\end{Satz}
\proof
Die Aussage ist lokal, kann also mittels Karten auf die entsprechende
Situation im $\R^n$ zur\"uckgef\"uhrt werden. Somit sei o.B.d.A.
$M= \R^n$ mit den Standardkoordinaten $x=(x_1,\ldots, x_n)$, weiter
sei $p=0$ und $X_0 = \left. \frac{\partial}{\partial x_1}\right|_0 = e_1$.
Dies letzte Eigenschaft kann man immer durch eine lineare Koordinatentransformation
erreichen.

\medskip

Man nutzt folgende Idee f\"ur die Konstruktion der geeigneten Karten: durch den
Punkt $(0,a_2,\ldots, a_n)$ existiert eine eindeutig bestimmte Integralkurve von
$X$. Jeder Punkt auf dieser Kurve ist eindeutig festgelegt durch die ben\"otigte
Zeit.

\medskip

Sei $\varphi_t$ der lokale Flu\ss{} von $X$, dann definiert man durch
$$
\psi(a_1,\ldots, a_n) = \varphi_{a_1}(0,a_2,\ldots, a_n)
$$
eine Abbildung $\psi$ auf einer hinreichend kleinen Umgebung von $0\in \R^n$. Man
m\"ochte nun zeigen, dass $\psi$ ein lokaler Diffeomorphismus ist.
Dazu berechnet man $d\psi$  im Punkt $a=(a_1,\ldots, a_n)$:
$$
\begin{array}{rl}
d\psi \left( \left. \frac{\partial}{\partial x_1}\right|_a \right)(f)
& = \left. \frac{\partial}{\partial x_1}\right|_0 (f\circ \psi) \\[1ex]
& = \lim_{t\rightarrow 0} \frac1t \left[ f(\psi(a_1+t,a_2,\ldots, a_n)) -f(\psi(a))\right]\\[1ex]
& = \lim_{t\rightarrow 0} \frac1t \left[ f(\varphi_{a_1+t}(0,a_2,\ldots, a_n) -f(\psi(a)) ] \right)\\[1ex]
& = \lim_{t\rightarrow 0} \frac1t \left[ f(\varphi_t(\psi(a))) - f(\psi(a))\right] \\[1ex]
& = X(f)_{\psi(a)}
\end{array}
$$

Analog berechnet man nun im Punkt $0$ und f\"ur $i\ge 2$  das Differential von $\psi$ auf den Basisvektoren
$\left. \frac{\partial}{\partial x_i}\right|_0$:
$$
\begin{array}{rl}
d\psi \left( \left. \frac{\partial}{\partial x_i}\right|_0 \right)(f)
& = \left. \frac{\partial}{\partial x_i}\right|_0 (f\circ \psi) \\[1ex]
& = \lim_{t\rightarrow 0} \frac1t \left[ f(\psi(0,\ldots,t, \ldots, 0)) -f(0)\right]\\[1ex]
& = \lim_{t\rightarrow 0} \frac1t \left[ f(0\ldots,t,\ldots, 0) -f(0)  \right]\\[1ex]
& = \left. \frac{\partial f}{\partial x_i}\right|_0
\end{array}
$$

Da $X_0 = \left. \frac{\partial}{\partial x_1}\right|_0$ folgt $d\psi_0= \Id$. Damit ist $\psi$
ein lokaler Diffeomorphismus und man kann $y:=\psi^{-1}$ als neue Koordinaten um die $0 \in \R^n$
einf\"uhren. In diesen Karten gilt dann: $X = \left. \frac{\partial}{\partial y_1}\right.$.
Denn wie schon gezeigt ist
$$
X_b(f) = d\psi \left( \left. \frac{\partial}{\partial x_1}\right|_{\psi^{-1}(b)}\right)(f) =
\left. \frac{\partial f \circ \psi }{\partial x_1}\right|_{\psi^{-1}(b)} \ .
$$
Nach Definition der partiellen Ableitung in einer Karte $(U,x)$ gilt andererseits:
$$
 \left. \frac{\partial}{\partial y_1}\right|_b(f) =
 \left. \frac{\partial f \circ \psi }{\partial x_1}\right|_{\psi^{-1}(b)} \ .
$$

\qed

\bigskip

Im Folgenden soll $L_XX=0$ gezeigt werden. Das folgt zwar direkt aus der Definition der Lie-Klammer,
l\"a\ss t sich aber auch mit Hilfe des lokalen Flu\ss es $\varphi_t$ von $X$ zeigen.  Tats\"achlich
gilt $(\varphi_{t*}X)_p=X_p$ f\"ur alle $p\in M$ und alle $t$, f\"ur die $\varphi_t$ definiert ist:
Der Vektor $X_{\varphi_{-t}(p)}$ ist der Tangentialvektor an die Kurve $s\mapsto \varphi_s(p)$
f\"ur $s=-t$ bzw. der Tangentialvektor an die Kurve $c(s):=\varphi_{s-t}(p)$ f\"ur $s=0$.
Daher folgt:
$$
(\varphi_{t*}X)_p=d\varphi_t(X_{\varphi_{-t}(p)}) = d\phi_t \left(\left.\frac{d}{ds}\right|_{s=0} c(s) \right)
=\left.\frac{d}{ds}\right|_{s=0} \varphi_t\circ\varphi_{s-t}(p)
=\left.\frac{d}{ds}\right|_{s=0}\varphi_s(p) = X_p
$$
Aus der Definition der Lie-Ableitung als Ableitung $L_XX$ der Kurve $(\varphi_{t*}X)_p$ erh\"alt man
damit nochmal $L_XX=0$.

\bigskip

\begin{Lemma}
Sei $\psi$ ein Diffeomorphismus und $X$ ein Vektorfeld auf $M$ mit dem lokalem Flu\ss{}
$\varphi_t$. Dann hat das Vektorfeld $\psi_*X$ den Flu\ss{} $\psi\circ \varphi_t \circ \psi^{-1}$.
\end{Lemma}
\proof
Sei $f$ ein beliebiger Funktionskeim um $q\in M$. Dann ergibt sich die Aussage des Lemmas
aus folgender Rechnung:
$$
\begin{array}{rl}
(\psi_* X)_q(f) &= d\psi (X_{\psi^{-1}(q)})(f)
=
X_{\psi^{-1}} (f\circ \psi)
=
\left.\frac{d}{dt}\right|_{t=0} (f\circ \psi)(\varphi_t(\psi^{-1}(q)))\\[1ex]
& =
\left.\frac{d}{dt}\right|_{t=0} f(\psi \circ \varphi_t \circ \psi^{-1}(q)) \ .
\end{array}
$$
Man sieht, dass das Vektorfeld $\psi_* X$ in einem Punkt $q$ tangential an die
Kurve $\psi \circ \varphi_t \circ \psi^{-1}(q))$. Die Menge dieser Kurven bilden
damit den lokalen Flu\ss{} von $\psi_* X$.
\qed

\bigskip

\begin{Folgerung}\label{folgerung}
Sei $\psi : M \rightarrow M$ ein Diffeomorphismus und $X$ ein Vektorfeld auf $M$ mit
dem lokalen Flu\ss{} $\varphi_t$. Dann gilt:
$$
\psi_* X = X \qquad \mbox{genau dann, wenn} \qquad
\varphi_t \circ \psi = \psi \circ \varphi_t \ .
$$
\end{Folgerung}
\proof
Die Vektorfelder $X$ und $\psi_*X$ stimmen genau dann \"uberein, wenn ihre lokalen
Fl\"usse gleich sind. Nach dem  obigen Lemma bedeutet das:
$\varphi_t = \psi \circ \varphi_t \circ \psi^{-1}$.
\qed


\bigskip

Das folgende Lemma gibt eine naheliegende Charakterisierung der Bedingung $[X,Y]=0$:
zwei Vektorfelder kommutieren genau dann, wenn ihre Fl\"usse kommutieren.

\begin{Lemma}\label{kommutator}
Sei $X$ ein Vektorfeld mit dem lokalem Flu\ss{} $\varphi_t$ und sei $Y$ ein Vektorfeld mit
dem lokalem Flu\ss{} $\psi_t$. Dann gilt:
$$
[X, Y] = 0 \qquad \mbox{genau dann, wenn} \qquad \varphi_t \circ \psi_s = \psi_s\circ \varphi_t \ ,
$$
wobei die rechte Gleichung f\"ur alle $s$ und $t$ erf\"ullt sein soll, f\"ur die die entsprechenden
Fl\"usse definiert sind.
\end{Lemma}
\proof
Sei zun\"achst vorausgesetzt, dass die Fl\"usse kommutieren dann folgt nach Folgerung~\ref{folgerung}
$\varphi_{t*} Y = Y$ f\"ur alle $t$, f\"ur die der Flu\ss{} definiert ist. Nach Ableiten erh\"alt man
daraus $L_XY=0$.

\medskip

F\"ur den Beweis der umgekehrten Richtung hat man in allen Punkte $q\in M$ die Voraussetzung:
$\;\lim_{t\rightarrow 0} \frac1t(Y_q - (\varphi_{t*}Y)_q) = 0 \; (\ast)$. Man definiert nun eine Kurve
$c : (-\varepsilon, \varepsilon)\rightarrow T_pM$ durch $c(t) = (\varphi_{t*}Y)_p$.
Die Berechnung des Tangentialvektors an $c$ im Punkt $t=0$ ergibt:
$$
\begin{array}{rl}
\dot c(t) & = \lim_{t\rightarrow 0} \frac1s [c(t+s)-c(t)]\\[1ex]
& = \lim_{t\rightarrow 0} \frac1s [\varphi_{(t+s)*}Y - \varphi_{t*}Y]_p\\[1ex]
& = \lim_{t\rightarrow 0} \frac1s [d\varphi_t (\varphi_{s*}Y)_{\varphi_{-t}(p)} -  d\varphi_t Y_{\varphi_{-t}(p)}]\\[1ex]
& = d\varphi_t \left(  \lim_{t\rightarrow 0} \frac1s   [(\varphi_{s*}Y)_{\varphi_{-t}(p)} - Y_{\varphi_{-t}(p)} ]\right)\\[1ex]
& = 0
\end{array} \ .
$$
F\"ur die letzte Gleichung nutzt man die Voraussetzung (*) im Punkte $\varphi_{-t}(p)$. Au\ss erdem
verwendet man die offensichtliche Beziehung $(f\circ g)_* X = f_* (g_*X)$. Damit ist $c(t)$
konstant also $c(t) = c(0)$, d.h. $\varphi_{t\ast}Y = Y$ und die Behauptung ergibt sich aus
Folgerung~\ref{folgerung}.
\qed

\bigskip

In Satz~\ref{karte} wurde gezeigt, dass man f\"ur ein Vektorfeld $X$ mit $X_p\neq 0$
Karten $(U,y)$ um $p$ finden kann, f\"ur die $X$ ein Koordinatenvektorfeld auf $U$ ist.
Man betrachtet nun folgendes Problem: gegeben sind zwei Vektorfelder $X$ und $Y$, die in
einem Punkt $p\in M$, und damit in einer kleinen Umgebung von $p$, linear unabh\"angig sind.
Ist es dann m\"oglich Karten zu finden, in denen $X, Y$ Koordinatenvektorfelder sind?
Aus dem Satz von Schwarz folgt, dass der Kommutator von je zwei Koordinatenvektorfeldern
verschwindet, d.h. $[X,Y]=0$ ist eine notwendige Bedingung daf\"ur, dass es die gesuchten Koordinaten gibt.
Tats\"achlich ist diese Bedingung auch hinreichend.

\begin{Satz}\label{karte2}
Seien $X_1,\ldots, X_k$ Vektorfelder auf einer Umgebung $V$ von $p\in M$, die in jedem Punkt von
$U$ linear unabh\"angig sind und es gelte $[X_a, X_b]=0$ f\"ur $1\le a,b \le k$. Dann existiert
eine Karte $(U,x)$ um $p$ mit:
$$
X_a = \frac{\partial}{\partial x_a} \qquad \mbox{auf} \quad U \quad \mbox{f\"ur} \quad a = 1,\ldots, k \ .
$$
\end{Satz}
\proof
O.B.d.A. kann man wieder annehmen: $M=\R^n, p=0$ und $\left.X_a = \frac{\partial}{\partial x_a}\right|_{0}$
f\"ur $a=1,\ldots,k$. Sei $\varphi^a_t$ der lokale Flu\ss{} zum Vektorfeld $X_a$, dann definiert man
$$
\psi(a_1,\ldots, a_n) := \varphi^1_{a_1}(\varphi^2_{a_2}(\ldots (\varphi^k_{a_k}(0,\ldots,0,a_{k+1},\ldots,a_k))\ldots) \ ,
$$
Wie im Beweis von Satz~\ref{karte} berechnet man das Differential von $\psi$ und erh\"alt:
$$
d\psi \left( \left. \frac{\partial}{\partial x_i}\right|_0 \right)
\;=\;
\left\{
\begin{array}{ll}
X_i(0) =\left. \frac{\partial}{\partial x_i} \right|_0\qquad\qquad  & i = 1,\ldots, k \\[2ex]
\left.\frac{\partial}{\partial x_i}\right|_0    \qquad\qquad & i = k+1,\ldots, n
\end{array}
\right.
$$
Somit ist $\psi$ wieder auf einer Umgebung der Null ein Diffeomorphismus und man kann eine
neue Karte durch $y=\psi^{-1}$ definieren. Analog zum Beweis von Satz~\ref{karte} findet
man auf dieser Kartenumgebung
$$
X_1 = \frac{\partial}{\partial y_1} \ .
$$
Aus Lemma~\ref{kommutator} folgt nun, dass man die Reihenfolge der $\phi^i_{a_i}$'s in der
Definition von $\psi$ vertauschen kann. Schreibt man nun $\psi = \phi^i_{a_i}(\ldots)$, so
folgt f\"ur $i=1,\ldots, k$
$$
X_i = \frac{\partial}{\partial y_i} \ .
$$
\qed






\section{Integral-Mannigfaltigkeiten}

Ziel dieses Abschnitts ist eine Verallgemeinerung des Begriffs der Integralkurven, sowie eine
geometrische Interpretation von gewissen Systemen partieller Differentialgleichungen. Die
Darstellung folgt dem entsprechenden Abschnitt im Buch von M. Spivak.

\bigskip

\begin{Definition}
Eine $k$-dimensionale {\em Distribution} auf $M$ ist eine Zuordnung $p\mapsto \Delta_p$ mit
\begin{enumerate}
\item
$\Delta_p \subset T_p M$ ist ein $k$-dimensionaler Unterraum
\item
F\"ur alle $p\in M$ existiert eine Umgebung $\,U$ von $p$ und auf $\,U$ definierte Vektorfelder
$X_1,\ldots, X_k $ mit:
$$
\Delta_q = \mathrm{span} \{X_1(q),\ldots, X_k(q)\} \qquad \mbox{f\"ur alle }\quad q \in U \ .
$$
\end{enumerate}

Sei $X$ ein Vektorfeld auf $M$. Man sagt, dass $X$ ein Vektorfeld in $\Delta$ ist, falls
$X_p \in \Delta_p$ f\"ur alle $p\in M$ gilt.
\end{Definition}

{\bf Bemerkung:} \"Aquivalent kann man Distributionen auch als Unterb\"undel  $\Delta \subset TM$
definieren. Vektorfelder in $\Delta$ sind dann genau die Schnitte von $\Delta$.

\medskip

Sei $E\rightarrow M$ ein Vektorb\"undel \"uber $M$. Eine Teilmenge $F\subset E$ hei\ss t
{\it Unterb\"undel} von $E$, falls $F\rightarrow M$ ein Vektorb\"undel ist und
$F_p$ f\"ur alle $p \in M$ ein Unterraum von $E_p$ ist. Es folgt, dass $F$ eine Untermannigfaltigkeit
von $E$ ist.

\bigskip

\begin{Definition}
Eine {\em Integral-Mannigfaltigkeit} einer Distribution $\Delta$ ist eine Untermannigfaltigkeit
$i: N \rightarrow M$ mit:
$$
di (T_pN) \;=\; \Delta_p \qquad \mbox{f\"ur alle} \quad p \in N \ .
$$
\end{Definition}

\bigskip

{\bf Bemerkung:}
Der Begriff der Untermannigfaltigkeit ist hier in dem schw\"acheren Sinne zu benutzen, d.h. $N$ ist
eine Untermannigfaltigkeit, falls $N$ eine Mannigfaltigkeit ist und eine injektive Immersion
$i:N \rightarrow M$ existiert.
Im Allgemeinen existieren keine Integral-Mannigfaltigkeiten (noch nicht
einmal lokal). Als Beispiel betrachtet man folgende $2$-dimensionale Distribution auf $\R^3$.

\bigskip


{\bf Beispiel:} In $p=(a,b,c)$ sei die Distribution $\Delta$ gegeben als:
$$
\Delta_p := \mathrm{span} \{
\left. \tfrac{\partial }{\partial x}\right|_p + \, b \left. \tfrac{\partial }{\partial z}\right|_p,\;
\left. \tfrac{\partial }{\partial y}\right|_p    \}
=
\{r \left. \tfrac{\partial }{\partial x}\right|_p   + s\left. \tfrac{\partial }{\partial y}\right|_p
+ br \left. \tfrac{\partial }{\partial z}\right|_p  \;|\; r,s \in \R   \} \ .
$$

Es soll nun gezeigt werden, dass zu $\Delta$ keine Integral-Mannigfaltigkeit existiert. Der Beweis
l\"a\ss t sich direkt geometrisch erbringen oder auch, indem man die folgende allgemeinere Situation
betrachtet.

\bigskip

Sei $\Delta \subset \R^3$ eine $2$-dimensionale Distribution definiert in $p=(a,b,c)$ durch
$$
\begin{array}{rl}
\Delta_p
= &
\{
r\left. \tfrac{\partial }{\partial x}\right|_p   + s \left. \tfrac{\partial }{\partial y}\right|_p
+ [rf(a,b) + s g(a,b)] \left. \tfrac{\partial }{\partial z}\right|_p
\}\\[1.5ex]
= &
\mathrm{span} \{
\left. \tfrac{\partial }{\partial x}\right|_p  + f(a,b) \left. \tfrac{\partial }{\partial z}\right|_p,
\left. \tfrac{\partial }{\partial y}\right|_p  + g(a,b) \left. \tfrac{\partial }{\partial z}\right|_p
\}
\end{array}
$$


In $p=(a,b,c)$ ist $\Delta_p$ die Ebene mit der Gleichung
$
z-c = f(a,b) (x-a) + g(a,b) (y-b)
$.
Da $f$ und $g$ nicht von $c$ abh\"angen ist die Ebene $\Delta_{(a,b,c)}$ parallel zur Ebene
$\Delta_{(a,b,0)}$.

\medskip

Sei $N$ lokal um $p$ eine Integral-Mannigfaltigkeit von $\Delta$. Da kein Vektor in $\Delta$,
also kein Tangentialvektor an $N$, senkrecht zur $x\,y-$Ebene ist, l\"a\ss t sich $N$ lokal als
Graph einer Funktion schreiben. Denn als Kodimension-1 Untermannigfaltigkeit ist $N$ lokal
von der Form $h^{-1}(0)$ f\"ur eine Funktion $h:\R^3\rightarrow \R$. Der Tangentialraum an
$N$ in einem Punkt $p$ ist dann gegeben als das orthogonale Komplement von $(\grad h)_p$.
Die Bedingung $e_3 \not\in T_pN$ liefert also
$$
\la e_3, \grad h\ra = \tfrac{\partial h}{\partial z} \neq 0
$$
und aus dem Satz \"uber implizite Funktionen folgt die Existenz einer Funktion $\alpha$
mit $h(x,y,\alpha(x,y))= 0$ in einer kleinen Umgebung von $p$. Somit l\"a\ss t sich in
dieser Umgebung die Integral-Mannigfaltigkeit als Graph der Funtion $\alpha$ schreiben:
$$
N = \{(x,y,z) \in \R^3 \,|\, z = \alpha(x,y)\} \ .
$$
Damit ist $\Phi(x,y) := (x,y,\alpha(x,y))$ eine Parametrisierung von $N$ und der Tangentialraum an
$N$ im Punkt $p$ wird  aufgespannt von den Vektoren $D_1\Phi = \frac{\partial \Phi}{\partial x}$
und $D_2\Phi= \frac{\partial \Phi}{\partial y}$, also den beiden Spalten der Jacobi-Matrix von $\Phi$. Es folgt
in $p=(a,b,\alpha(a,b))$:
$$
T_pN
=
\mathrm{span} \{   \left. \tfrac{\partial }{\partial x}\right|_p
+ \tfrac{\partial \alpha}{\partial x}{\small (a,b)} \left. \tfrac{\partial }{\partial z}\right|_p,
\left. \tfrac{\partial }{\partial y}\right|_p
+ \tfrac{\partial \alpha}{\partial y}(a,b)\left. \tfrac{\partial }{\partial z}\right|_p\}
\ .
$$
Somit gilt
$T_pN = \Delta_p $ genau dann, wenn
\begin{equation}\label{einstern}
f(a,b) = \tfrac{\partial \alpha}{\partial x}(a,b),\qquad g(a,b) = \tfrac{\partial \alpha}{\partial y}(a,b) \ .
\end{equation}

Sei nun $\alpha$ eine L\"osung von $(\ref{einstern})$, dann folgt aus dem Satz von Schwartz:
$$
\tfrac{\partial^2 \alpha}{\partial x \, \partial y}
=
\tfrac{\partial^2 \alpha}{\partial y \, \partial x}
\qquad\mbox{und damit}\qquad
\tfrac{\partial f}{\partial y} = \tfrac{\partial g}{\partial x} \ .
$$

Tats\"achlich gilt auch die Umkehrung:

\begin{Satz}\label{bedingung1}
Eine L\"osung von $(\ref{einstern})$ existiert genau dann, wenn
 $\tfrac{\partial f}{\partial y} = \tfrac{\partial g}{\partial x}$ erf\"ullt ist.
\end{Satz}
\proof
Sei $\omega := f dx + g dy$ definiert als $1$-Form auf $\R^3$, dann erh\"alt man f\"ur das Differential von $\omega$
die Gleichung
$d\omega = (\tfrac{\partial f}{\partial y} - \tfrac{\partial g}{\partial x}) dx \wedge dy$. Weiter gilt
$d\alpha = \tfrac{\partial \alpha}{\partial x}dx + \tfrac{\partial \alpha}{\partial y}dy$.

\medskip

Offensichtlich ist
$\alpha$ eine L\"osung von $(\ref{einstern})$ genau dann, wenn $\omega = d\alpha$. Aus $\omega = d\alpha$ folgt
$d\omega = d^2\alpha = 0$ und damit $\tfrac{\partial f}{\partial y} - \tfrac{\partial g}{\partial x}=0$.
Umgekehrt folgt aus dieser Gleichung $d\omega=0$. Das Lemma von Poincare liefert nun die Existenz einer
Funktion $\alpha$ mit $\omega = d\alpha$.
(Genauere Erkl\"arungen der benutzten Begriffe folgen sp\"ater.)
\qed

\bigskip

In dem ersten Beispiel einer $2$-dimensionalen Distribution $\Delta$ in $\R^3$ hatte man
$f(a,b)=b$ und $g(a,b)=0$, also $\tfrac{\partial f}{\partial y}=1$ und $\tfrac{\partial g}{\partial x}=0$.
Damit ist die Bedingung aus Satz~\ref{bedingung1} nicht erf\"ult, das System $(\ref{einstern})$ hat also keine
L\"osung und es existiert keine (lokale) Integral-Mannigfaltigkeit von $\Delta$.

\bigskip

Das allgemeinste Beispiel einer $2$-dimensionalen Distribution in $\R^3$ ergibt sich, wenn man $f, g$ als Funktionen
 auf ganz $\R^3$ betrachtet, also in $p\in \R^3$ definiert:
$$
\Delta_p
=
\{
r\left. \tfrac{\partial }{\partial x}\right|_p   + s \left. \tfrac{\partial }{\partial y}\right|_p
+ [rf(p) + s g(p)] \left. \tfrac{\partial }{\partial z}\right|_p
\}
= \;
\mathrm{span} \{
\left. \tfrac{\partial }{\partial x}\right|_p  + f(p) \left. \tfrac{\partial }{\partial z}\right|_p,
\left. \tfrac{\partial }{\partial y}\right|_p  + g(p) \left. \tfrac{\partial }{\partial z}\right|_p
\} \ .
$$

Falls eine Integral-Mannigfaltigkeit $N$ von $\Delta$ existiert, kann man sie wieder lokal als
Graph einer Funktion realisieren, d.h.
$
N = \{ (x,y,z) \,|\, z = \alpha(x,y)\}
$.
Wie oben wird $T_pN$ aufgespannt von
$
\left. \tfrac{\partial }{\partial x}\right|_p + \tfrac{\partial \alpha}{\partial x} \left. \tfrac{\partial }{\partial z}\right|_p
$
und
$
\left. \tfrac{\partial }{\partial y}\right|_p
+ \tfrac{\partial \alpha}{\partial y}\left. \tfrac{\partial }{\partial z}\right|_p \ .
$
Daraus folgt, dass die Bedingung $T_pN = \Delta_p$ \"aquivalent ist zu folgendem System partieller
Differentialgleichungen:
\begin{equation}\label{zweistern}
f(a,b,\alpha(a,b)) = \tfrac{\partial \alpha}{\partial x}(a,b),\qquad
g(a,b,\alpha(a,b)) = \tfrac{\partial \alpha}{\partial y}(a,b) \ .
\end{equation}

Wieder liefert der Satz von Schwartz eine notwendige Bedingung f\"ur die L\"osbarkeit des Systems $(\ref{zweistern})$
\begin{equation}\label{vorstrich}
\tfrac{\partial f}{\partial y} + \tfrac{\partial f}{\partial z}\cdot \tfrac{\partial \alpha}{\partial y}
=
\tfrac{\partial^2 \alpha}{\partial y\partial x}
=
\tfrac{\partial^2 \alpha }{\partial x \partial y}
=
\tfrac{\partial g}{\partial x} + \tfrac{\partial g}{\partial z}\cdot \tfrac{\partial \alpha}{\partial x} \ .
\end{equation}
Diese Formeln folgen aus der Kettenregel: Sei wie oben $\Phi:\R^2 \rightarrow \R^3$ die Abbildung
$(x,y)\mapsto (x,y,\alpha(x,y))$. Dann berechnet sich die Jacobi-Matrix der Verkn\"upfung $f\circ \Phi$ durch
$D(f\circ \Phi) = Df \circ D\Phi = \grad (f) \cdot (D_1\Phi, D_2\Phi)$. F\"ur die Ableitung von $f\circ \Phi$
nach der zweiten Variablen findet man dann
$$
\tfrac{\partial f(x,y,\alpha(x,y))}{\partial y} = \la \grad f, D_2\Phi \ra
=
\la(\tfrac{\partial f}{\partial x},\tfrac{\partial f}{\partial y},\tfrac{\partial f}{\partial z}),
(0,1,\tfrac{\partial \alpha}{\partial y}\ra
=
\tfrac{\partial f}{\partial y} + \tfrac{\partial f}{\partial z}\cdot \tfrac{\partial \alpha}{\partial y} \ .
$$

Mit Hilfe von Gleichung $(\ref{zweistern})$ kann man in (\ref{vorstrich}) noch $\alpha\,$ eliminieren und erh\"alt
so die Bedingung:
\begin{equation}\label{strich}
\tfrac{\partial f}{\partial y} + \tfrac{\partial f}{\partial z}\cdot g
=
\tfrac{\partial g}{\partial x} + \tfrac{\partial g}{\partial z}\cdot f  \ .
\end{equation}
Wie schon im vorigen Beispiel stellt sich erneut heraus, dass diese notwendige Bedingung auch schon hinreichend
ist f\"ur die L\"osbarkeit von $(\ref{zweistern})$.

\bigskip

Die Bedingung (\ref{strich}) kann man nun noch geometrischer interpretieren. Seien daf\"ur
$X:=\tfrac{\partial }{\partial x} + f \tfrac{\partial }{\partial z} $ und
$Y:=\tfrac{\partial }{\partial y} + g \tfrac{\partial }{\partial z} $ die beiden Vektorfelder, die
die Distribution $\Delta$ aufspannen. Unter der Voraussetzung, dass eine Integral-Mannigfaltigkeit
von $\Delta$ existiert zeigt sich  (siehe unten), dass auch der Kommutator $[X,Y]$ zu $\Delta$
geh\"oren mu\ss{}. Durch eine kurze Rechnung findet man
$$
[X, Y] = (\tfrac{\partial g}{\partial x} + f\,\tfrac{\partial g}{\partial z}
-
\tfrac{\partial f}{\partial y} - g\,\tfrac{\partial f}{\partial z}
)  \tfrac{\partial }{\partial z} \ .
$$
Wie schon bemerkt, sind die Tangentialvektoren an $N$ also die Vektoren in $\Delta$ niemals
senkrecht zur $xy$-Ebene, d.h. $a \left.\tfrac{\partial }{\partial z} \right|_p \in \Delta_p$ ist
nur f\"ur $a=0$ m\"oglich. Es folgt also $[X,Y]_p \in \Delta_p$ genau dann, wenn die
Bedingung (\ref{strich}) erf\"ullt ist.

\bigskip

\subsection{Der Satz von Frobenius in der Analysis}


\begin{Satz}
Sei $U \times V \subset \R^m \times \R^n$ offen und $U \subset \R^m$ eine Umgebung der Null. Seien
weiter $f_j: U \times V \rightarrow \R^n$ f\"ur $j=1,\ldots,m$ glatte Funktionen. Dann existiert
f\"ur jedes $x\in V$ h\"ochstens eine Funktion $\alpha : W\rightarrow V$, definiert auf einer Umbebung
$W \subset \R^m$ der Null mit:
\begin{enumerate}
\item
$\alpha(0) = x$
\item
$ \tfrac{\partial \alpha}{\partial t_j}(t) = f_j(t, \alpha(t)) $   f\"ur alle $t\in W, j=1,\ldots, m$ \ .
\end{enumerate}
Eine L\"osung $\alpha$ existiert genau dann, wenn
$$
\tfrac{\partial f_j}{\partial t_i} + \sum^n_{k=1} \tfrac{\partial f_j}{\partial x_k} \cdot f^k_i
=
\tfrac{\partial f_i}{\partial t_j} + \sum^n_{k=1} \tfrac{\partial f_i}{\partial x_k} \cdot f^k_j
$$
auf einer Umgebung von $(0,x) \in U \times V$ f\"ur alle $1\le i,j \le m$ erf\"ullt ist.
\end{Satz}

\bigskip

{\bf Notation:}
\begin{enumerate}
\item
$x=(x_1,\ldots, x_n)$ bezeichnet Punkte im $\R^n$ und $t=(t_1,\ldots, t_m)$ bezeichnet Punkte in $\R^m$
\item
$\tfrac{\partial f}{\partial t_i} $ bezeichnet die $i$te Spalte der Jacobi-Matrix von $f$
\item
$\tfrac{\partial f}{\partial x_i} $ bezeichnet die $(m+i)$te Spalte der Jacobi-Matrix von $f$
\item
$f_j = (f^1_j,\ldots,f^n_j)$
\end{enumerate}

\bigskip

Im Beispiel war $m=2, n=1$ und $f_1=f, f_2=g$.


\subsection{Integrable Distributionen}

Sei $N \subset M$ eine Untermannigfaltigkeit und seien $X, Y$ zwei Vektorfelder auf $M$. Durch Einschr\"ankung
auf $N$ erh\"alt man zwei Vektorfelder auf $N$. Es soll zun\"achst gezeigt werden, dass der Kommutator dieser
eingeschr\"ankten Vektorfelder gleich der Einschr\"ankung von $[X,Y]$ auf $N$ ist. Das ist nicht offensichtlich,
da der Kommutator auf $M$ mehr Ableitungen enth\"alt.

\bigskip

\begin{Definition}
Sei $f:N\rightarrow M$ eine differenzierbare Abbildung.  Zwei Vektorfelder $X\in \chi(N)$ und $Y\in \chi(M)$
nennt man {\em $f$-verkn\"upft}, falls
$$
Y_{f(p)} = df (X_p)
$$
f\"ur alle $p\in M$ erf\"ullt ist.
\end{Definition}

\bigskip

{\bf Bemerkung:} Zwei Vektorfelder $X \in \chi(N),  Y\in \chi(M)$ sind genau dann $f$-verkn\"upft, wenn
$$
Y(\phi) \circ f = X(\phi \circ f)
$$
f\"ur alle Funktionen (bzw. Funktionskeime) $\phi$ auf $M$ gilt.

\bigskip

\begin{Lemma}
Sei $f: N \rightarrow M $ eine Immersion und $Y\in \chi(M)$ ein Vektorfeld mit
$$
Y_{f(p)} \in \Im (df_p) \ .
$$
Dann existiert ein eindeutig bestimmtes Vektorfeld $X$ auf $N$, so dass $X$ und $Y$
$f$-verkn\"upft sind.
\end{Lemma}
\proof
Man definiert $X_p$ durch die Gleichung $Y_{f(p)}=df (X_p)$. Der Vektor $X_p$ definiert nach
Voraussetzung und ist eindeutig bestimmt, da $f$ eine Immersion, d.h. $df$ injektiv ist.

\medskip

Da $f$ eine Immersion ist existieren Karten $(U,x)$ bzw. $(V,y)$ um $p$ bzw. $f(p)$ mit
$$
y\circ f \circ x^{-1}(a_1,\ldots,a_n) = (a_1,\ldots, a_n,0,\ldots, 0) \ .
$$
In diesen Karten gilt dann
$$
df (\left. \tfrac{\partial}{\partial x_i}\right|_p) = \left. \tfrac{\partial}{\partial y_i}\right|_{f(p)} \ .
$$

Sei nun $Y = \sum a_i \tfrac{\partial}{\partial y_i}$ mit glatten, auf $V$ definierten, Funktionen $a_i$. Dann ist
notwendigerweise $X = \sum b_i \tfrac{\partial}{\partial x_i}$, wobei $b_i = a_i \circ f$ wieder glatte Funktionen sind
und $X$ damit ein glattes Vektorfeld.
\qed

\bigskip

{\bf Beispiel:}
Sei $N$ eine Untermannigfaltigkeit von $M$ und $i:N\rightarrow M$ die Inklusionsabbildung, weiter sei $Y \in \chi(M)$ ein
Vektorfeld mit $Y_p \in di (T_pN)$ f\"ur alle $p \in N$. Das wie oben beschriebene, eindeutig bestimmte Vektorfeld
$X\in \chi(N)$ mit $di(X)=Y$, ist dann genau die Einschr\"ankung von $Y$ auf $N$.

\bigskip

\begin{Lemma}\label{kommutator1}
Sei $f: N\rightarrow M$ eine differenzierbare  Abbildung und seien  $X_i \in \chi(N)$ bzw. $Y_i \in \chi(M), i=1,2$
zwei Paare von $f$-verkn\"upften Vektorfeldern. Dann sind auch die Kommutatoren $[X_1, X_2]$ und $[Y_1,Y_2]$
$f$-verkn\"upft.
\end{Lemma}
\proof
Sei $g$ eine Funktion (bzw. Funktionskeim) auf $M$. Dann gilt nach Voraussetzung f\"ur $i=1,2$:
$$
Y_i(g)\circ f = X_i (g\circ f)
$$
Damit berechnet man
$$
\begin{array}{rl}
[Y_1, Y_2](g) \circ f & = Y_1(Y_2(g))\circ f  - Y_2(Y_1(g))\circ f \\[1ex]
& = X_1(Y_2(g) \circ f) - X_2(Y_1(g) \circ f) \\[1ex]
& = X_1(X_2(g\circ f)) - X_2(X_1(g\circ f)) \\[1ex]
& = [X_1, X_2](g\circ f)
\end{array}
$$
\qed

\bigskip

{\bf Anwendung:}
Sei $N$ eine Integral-Mannigfaltigkeit einer Distribution $\Delta$ und seien $X,Y$
zwei Vektorfelder von $\Delta$. Wie oben beschrieben existieren eindeutig bestimmte
Vektorfelder $\bar X, \bar Y \in \chi(N)$ mit $X_p= di(\bar X_p), Y_p = di(\bar Y_p)$,
d.h. $\bar X = \left. X\right|_N, \bar Y = \left. Y \right|_N$. Damit folgt
$
di([\bar X, \bar Y]_p) = [di(\bar X), di (\bar Y)] = [X, Y]_p
$,
d.h.
$$
[\left. X\right|_N, \left. Y \right|_N] = \left. [X, Y] \right|_N \ .
$$


Da $\bar X$ und $\bar Y$ Vektorfelder auf $N$ sind, gilt dies auch f\"ur
ihren Kommutator $[\bar X, \bar Y]$. Nach Voraussetzung, da $N$ eine
Integral-Mannigfaltigkeit ist, folgt daraus $di ([\bar X, \bar Y])\in \Delta_p$.
Also, wie oben gezeigt, auch $[X, Y]_p \in \Delta_p$. Damit ist gezeigt:
existiert eine Integral-Mannigfaltigkeit f\"ur die Distribution $\Delta$,
dann liegt mit zwei Vektorfeldern $X, Y$ in $\Delta$ auch deren Kommutator
in $\Delta$.

\bigskip

\begin{Definition}
Eine Distribution $\Delta$ hei\ss t {\em integrabel} (oder auch involutiv), falls
f\"ur je zwei Vektorfelder $X,Y$ in $\Delta$ auch $[X, Y]$ ein Vektorfeld in $\Delta$
ist.
\end{Definition}

\bigskip

\begin{Satz}
Seien $X_1, \ldots, X_k$ Vektorfelder definiert in einer Umgebung $U$ von $p$
 mit $\Delta_q = \mathrm{span} \{X_1(q), \ldots, X_k(q)\}$ f\"ur alle $q\in U$.
 Dann ist $\Delta$ integrabel genau dann, wenn Funktionen
 $c_{ij}^r \in \mathcal C^\infty(U)$ existeren mit
 $$
 [X_i, X_j] = \sum_r c_{ij}^r X_r \ .
 $$
\end{Satz}
\proof
Ist die Distribution integrabel, dann ist die obige Bedingung trivialerweise erf\"ullt. Sei nun die
Bedingung erf\"ullt und $X, Y$ zwei beliebige Vektorfelder in $\Delta$, dann existieren lokal
Funktionen $a_i$ und $b_i$ mit: $X=\sum a_i X_i$ und $Y=\sum b_i X_i$. Berechnet man nun den
Kommutator, so folgt:
$$
[X, Y] = \sum_{i,j} [a_i X_i, b_j X_j] = \sum_{ij} ( a_i b_j [X_i, X_j] + a_i X_i(b_j) X_j - b_j X_j(a_i) X_i )\ .
$$
Unter der Voraussetzung, dass sich der Kommutator $[X_i, X_j]$ f\"ur alle $1\le i,j \le k$ als Linearkombination
der Verktorfelder $X_i$ schreiben l\"a\ss t, folgt damit, dass dies auch f\"ur $[X, Y]$ gilt, d.h. der
Kommutator $[X, Y]$ liegt wieder in der Distribution $\Delta$.
\qed

\bigskip

\subsection{Satz von Frobenius f\"ur Distributionen}

\begin{Satz}
Sei $\Delta$ eine $k$-dimensionale integrable Distribution auf $M$. F\"ur jeden Punkt $p\in M$ existieren
Karten $(U,x)$ mit
$$
x(p) = 0, \qquad x(U) = (-\varepsilon, \varepsilon) \times \ldots \times  (-\varepsilon, \varepsilon) \ ,
$$
so dass f\"ur alle $a_{k+1}, \ldots, a_n$ mit $|a_i| < \varepsilon$ f\"ur $i=k+1,\ldots, n $ die Menge
$$
N_a := \{q\in U \,|\, x_{k+1}(q)= a_{k+1}, \ldots, x_n(q)=a_n\}
$$
eine Integral-Mannigfaltigkeit von $\Delta$ auf $U$ ist und jede Integral-Mannigfaltigkeit auf $U$ sich in
dieser Form beschreiben l\"a\ss t.
\end{Satz}
\proof
Da es um eine lokale Aussage geht, kann man o.B.d.A. annehmen, dass $M=\R^n, p=0$ und
$\Delta_0 = \mathrm{span} \{ \left.\tfrac{\partial }{\partial t_1}\right|_0, \ldots, \left.\tfrac{\partial }{\partial t_k}\right|_0\}
\subset T_0\R^k \cong \R^k$. Sei $\pi : \R^n \rightarrow \R^k$ die Projektion
$\pi(x_1,\ldots, x_n) = (x_1, \ldots, x_k)$. Dann ist nach Voraussetzung
$d\pi : \Delta_0 \rightarrow T_0\R^k$ ein Isomorphismus. Aus Stetigkeitsgr\"unden ist dann auch
$d\pi : \Delta_q \rightarrow T_q\R^k$ ein Isomorphismus f\"ur alle $q$ in einer Umgebung $U$ der Null. Damit
existieren eindeutig bestimmte Vektorfelder $X_1,\ldots, X_k \in \chi(U)$ mit
$X_1(q), \ldots, X_k(q) \in \Delta_q$ und
$$
d\pi (X_i(q)) = \left. \tfrac{\partial}{\partial t_i}\right|_{\pi(q)}
\qquad  \mbox{f\"ur} \quad i = 1, \ldots, k \ .
$$
d.h. die Vektorfelder $X_i$ und $\frac{\partial}{\partial t_i}$ f\"ur $i=1,\ldots, k$ sind
$\pi$-verkn\"upft auf der Umgebung $U$. Da $\frac{\partial}{\partial t_i}$ Koordinatenvektorfelder sind und
mit Lemma~\ref{kommutator} folgt
$$
d\pi ([X_i, X_j]_q) = [\frac{\partial}{\partial t_i}, \frac{\partial}{\partial t_j}] = 0 \ .
$$
Da nun $d\pi$ ein Isomorphismus auf $\Delta_q$ ist (f\"ur alle $q\in U$) und da aufgrund der
Integrabilit\"aat $[X_i, X_j]_q\in \Delta_q$ gilt folgt  $[X_i, X_j]=0$. Nach Satz~\ref{karte2}
findet mal also ein Koordinatensystem $(U,x)$ um Null mit
$$
X_i = \tfrac{\partial}{\partial x_i} \qquad \mbox{f\"ur}\quad  i = 1,\ldots,k \ .
$$
Man \"uberzeugt sich nun, dass $(U,x)$ das gesuchte Koordinatensystem ist. Sei dazu
$\Phi : U \rightarrow \R^{n-k}$ die Abbildung $q\mapsto
(x_{k+1}(q),\ldots,x_n(q))$. Als Teil einer Kartenabbildung ist $\Phi$ eine Submersion
und jeder Punkt im Bild von $\Phi$ ist ein regul\"arer Wert. Nach dem Satz vom
regul\"aren Wert ist also $N_a = \Phi^{-1}(a_{k+1}, \ldots, a_n)$ eine Untermannigfaltigkeit
mit
$$
T_q N_a = \ker d\Phi_a = \mathrm{span} \{ \tfrac{\partial}{\partial x_1},\ldots, \tfrac{\partial}{\partial x_k}\}_q
= \mathrm{span} \{ X_1(q),\ldots, X_k(q)\}
= \Delta_q \ ,
$$
d.h. wie behauptet ist $N_a$ eine Integral-Mannigfaltigkeit der Distribution $\Delta$.

\qed

\bigskip

{\bf Anwendung:}
Seien $f^i_j$ f\"ur $i=1,\ldots, n, \, j=1,\ldots,k, \, k<n$ differenzierbare Funktionen auf $\R^n$, so
dass in jedem Punkt die Matrix $(f^i_j)$ den Rang $k$ hat. Man betrachtet folgendes System partieller
Differentialgleichungen auf $\R^n$
\begin{equation}\label{beispiel}
X_1(u) := \sum_i \, f^i_1 \, \tfrac{\partial u}{\partial x_i},
\quad \ldots \quad,
X_k(u)  := \sum_i \, f^i_k \, \tfrac{\partial u}{\partial x_i}
\end{equation}

Man sucht nun die maximale L\"osungsmenge von (\ref{beispiel}), also Funktionen $u_1,\ldots, u_{n-k}$
f\"ur die die Vektofelder $\grad u_1,\ldots , \grad u_{n-k}$ punktweise linear unabh\"angig sind.

\bigskip

Aus dem Satz von Frobenius f\"ur Distributionen folgt, dass das System $(\ref{beispiel})$ genau dann eine
L\"osung hat, wenn Funktionen $c^r_{ij}$ existieren mit
\begin{equation}\label{beispiel2}
X_i (X_j(u)) - X_j (X_i(u)) = \sum_r c^r_{ij} X_r(u) \qquad \mbox{f\"ur alle} \quad 1 \le i,j \le k \ .
\end{equation}

Gleichung $(\ref{beispiel2})$ bedeutet genau, dass die von $X_1,\ldots, X_k$ aufgespannte Distribution
$\Delta$ integrabel ist. Nach dem Satz von Frobenius folgt die Existenz von Integral-Mannigfaltigkeiten
und von speziellen Karten $(U,x)$. In diesen Karten definiert man $u_i(q) := x_{k+i}(q)$ f\"ur
$i = 1,\ldots, n-k$ und erh\"alt:
$$
(X_i)_q (u_j) = \left. \tfrac{d}{dt}\right|_{t=0} u_j(c_q(t)) = 0 \ ,
$$
dabei ist $c_q$ eine Kurve in der Integral-Mannigfaltigkeit durch $q$, d.h. die Funktion
$u_j$ ist konstant auf $c_q$.

\bigskip

Definiert man $u: \R^n \rightarrow \R^{n-k}$ durch $u(q)=(u_1(q), \ldots, u_{n-k}(q))$ dann sind nach
Voraussetzung die Vektorfelder $\grad u_1,\ldots, \grad u_{n-k}$ linear unabh\"angig, d.h. $u$ ist
eine Submersion und $u^{-1}(a_{k+1}, \ldots, a_n)$ eine Untermannigfaltigkeit mit
$$
T_p u^{-1}(a_{k+1}, \ldots, a_n) = \ker du = \mathrm{span} \{X_1, \ldots, X_k \}
$$
Somit sind die Niveaufl\"achen der L\"osungen genau die Integral-Mannigfaltigkeiten der
Distribution. Die L\"osungen der Gleichung~\ref{beispiel} sind also Funktionen auf den
Niveaufl\"achen. Verschiedene L\"osungen entsprechen verschiedenen Integral-Mannigfaltigkeiten.
Dies enspricht dem Fixieren verschiedener Integrationskonstanten.

\bigskip

\subsection{Bl\"atterungen}

\begin{Definition}
Eine $k$-dimensionale Untermannigfaltigkeit $N$ einer Mannigfaltigkeit $M$
hei\ss t {\em Bl\"atterung} von $M$, falls jeder Punkt von $M$ in einer Zusammenhangskomponente von $N$
liegt und es um jeden Punkt $p\in M$ Karten $(U,x)$ gibt mit:
\begin{enumerate}
\item
$\ x(U) = (-\varepsilon, \varepsilon) \times \ldots \times  (-\varepsilon, \varepsilon)  $
\item
Die Zusammenhangskomponenten von $N\cap U$ sind alles Mengen der Form
$$
\{q\in U \,|\, x_{k+1}(q)= a_{k+1}, \ldots, x_n(q)=a_n\}
$$
f\"ur gewisse $a_i \in \R$ mit $|a_i| <\varepsilon$.
\end{enumerate}
Im Allgemeinen ist die Bl\"atterungs-Mannigfaltigkeit
 $N$ nicht zusammenh\"angend. Die Zusammenhangskomponenten von $N$ nennt man
auch {\em Bl\"atter} der Bl\"atterung. Man sagt $M$ ist gebl\"attert durch $N$.
\end{Definition}

\bigskip

{\bf Bemerkung:}
Unterschiedliche Komponenten von $N\cap U$ k\"onnen zum gleichen Blatt einer Bl\"atterung geh\"oren.

\bigskip

\begin{Satz}\label{blaetter}
Sei $\Delta$ eine $k$-dimensionale integrable Bl\"atterung auf $M$. Dann ist $M$ gebl\"attert durch
eine Integral-Mannigfaltigkeit von $\Delta$. Jede Zusammenhangskomponente nennt man maximale
Integralmannigfaltigkeit von $\Delta$.
\end{Satz}
\proof siehe Spivak \qed

\bigskip

{\bf Bemerkung:} Aus diesem Satz folgt also, dass durch jeden Punkt $p\in M$ eine eindeutig
bestimmte maximale Integral-Mannigfaltigkeit geht. Diese sei mit $N(p)$ bezeichnet und
es gilt
$$
N(p) = \{q \in M \,|\, \exists c:[0,1]\rightarrow M \;\mbox{glatt}, c(0)=p, c(1)=q, \dot c(t) \in \Delta_{c(t)}\} \ .
$$
Aus dieser Beschreibung folgt auch, dass eine Abildung $f:M\rightarrow M$ mit
der Eigenschaft
$df(\Delta_p) = \Delta_{f(p)}$ f\"ur alle $p\in M$, die maximalen Integral-Mannigfaltigkeiten permutiert, d.h.
es gilt $f(N(p)) = N(f(p))$.

\bigskip

\begin{Satz}
Sei $N\subset M$ ein Blatt einer Bl\"atterung auf $M$ und sei $f: B \rightarrow M$ eine glatte
Abbildung mit $f(B) \subset N$. Dann ist auch die  Abbildung $f:B \rightarrow N$ glatt.
\end{Satz}
\proof siehe Spivak \qed


\section{Lie-Gruppen}

\begin{Definition}
Ein Gruppe $G$ hei\ss t {\em Lie-Gruppe}, falls $G$ eine Mannigfaltigkeit ist und die Abbildung
$G\times G \rightarrow G, (g,h)\mapsto gh^{-1}$ differenzierbar ist.
\end{Definition}

{\bf Bemerkung:} Die zweite Bedingung ist dazu \"aquivalent, dass die Abbildung $g\mapsto g^{-1}$
und die Gruppenoperationen $(g,h)\mapsto gh$ differenzierbar sind.

\bigskip

{\bf Beispiele:} Die klassischen Matrizengruppen sind Lie-Gruppen, z.B. $GL(n,\R), O(n)$ oder
$U(n)$. Auch $\R^n$ mit der Vektoraddition ist eine Lie-Gruppe.

\bigskip

\begin{Definition}
Die {\em Links-} bzw. {\em Rechtstranslation} auf $G$ sind f\"ur jedes fixiertes $g\in G$
definiert durch:
$$
l_g(h) = g\cdot h,\qquad r_g(h) = h\cdot g \ .
$$
\end{Definition}


{\bf Bemerkung:} Die Abbildungen $l_g, r_g : G\rightarrow G$ sind Diffeomorphismen. Tats\"achlich
gilt $l_g^{-1} = l_{g^{-1}}$ und analog f\"ur $r_g$.

\begin{Definition}
Ein Vektorfeld $X\in \chi(G)$ hei\ss t {\em links-invariant}, falls f\"ur alle $g\in G$:
$$
l_{g*} X = X \ ,
$$
d.h. $X$ ist $l_g$-verkn\"upft zu sich selbst.
\end{Definition}

\bigskip

{\bf Bemerkung:}~\begin{enumerate}
\item
Ein Vektorfeld $X$ ist links-invariant genau dann, wenn $dl_g (X_h) = X_{gh}$
f\"ur alle $g, h \in G$. Denn nach der Definition
des push-forwards folgt
$$
(l_{g*}X)_h = (dl_g)_{l_g^{-1}(h)} X_{l_g^{-1}(h)} = (dl_g)_{g^{-1}h} X_{g^{-1}h} \ .
$$
\item
Links-invariante Vektorfelder sind bestimmt durch den Wert in einem Punkt:
$$
X_g = dl_g \, (X_e) \ .
$$
dabei ist $e$ das Einselement der Gruppe $G$. Die Abbildung
$X \in T_eG \mapsto \tilde X \in \chi(G)$ mit $\tilde X_g = dl_g (X)$ definierte eine Bijektion
zwischen $T_eG$ und dem Raum der links-invarianten Vektorfelder. Die Umkehrabbildung ist
gegeben durch $\tilde X \mapsto \tilde X_e$.
\item
Links-invariante Vektorfelder sind automatisch differenzierbar.
\item
Lie-Gruppen $G$ sind parallelisierbar, d.h. es existiert die maximal m\"ogliche Anzahl von $\dim G$
punktweise linear unabh\"angigen
Vektorfeldern: Man setzt eine Basis in $T_eG$ fort zu links-invarianten Vektorfeldern auf $G$.
Insbesondere sind Lie-Gruppen orientierbar.
\item
Analog zu links-invarianten kann man auch rechts-invariante Vektorfelder definieren.
\item
Sind $G,H$ Lie-Gruppen, so auch $G\times H$.
\end{enumerate}

Es war schon gezeigt worden, dass der Raum $\chi(M)$ der Vektorfelder auf einer Mannigfaltigkeit $M$
eine Lie-Algebra ist. Neben dieser unendlich-dimensionalen Lie-Algebra gibt es auch endlich-dimensionale
Beispiele, z.B. $\End (V)$ mit $[A, B]:= A \circ B - B \circ A$ oder $\R^3$ mit
$[v, w]:= v \times w$, wobei $\times$ das Vektorkreuzprodukt bezeichnet.  Es soll nun gezeigt
werden, dass zu jeder Lie-Gruppe eine eindeutig bestimmte Lie-Algebra geh\"ort. Die letzten beiden
Beispiele sind Spezialf\"alle davon.

\begin{Satz}\label{LA}
Sei $G$ eine Lie-Gruppe und bezeichne $\g$ die Menge aller links-invarianten Vektorfelder auf $G$.
Dann gilt:
\begin{enumerate}
\item
Die Menge $\g$ ist eine reeller Vektorraum, der isomorph zu $T_eG$ ist. Insbesondere gilt
$\dim G = \dim \g$.

\item
Der Kommutator zweier links-invarianter Vektorfelder ist links-invariant, d.h. $\g$ ist eine
Lie-Algebra. F\"ur $X, Y \in T_eG$ gilt:
$$
[X, Y] := [\tilde X, \tilde Y]_e = \widetilde{[X,Y]}_e \ .
$$
\end{enumerate}
\end{Satz}
\proof
Die erste Aussage, also die Isomorphie zwischen dem Raum der links-invarianten Vektorfelder und
$T_eG$ wurde schon gezeigt. Zum Beweis der zweiten Aussage bemerkt man zuerst, dass sich jedes
links-invariante Vektorfeld als $\tilde X$ f\"ur ein $X\in T_e G$ schreiben l\"a\ss t. Wie schon
bemerkt sind links-invariante Vektorfelder $l_g$-verkn\"upft zu sich selbst. Damit sind auch die
entsprechenden Kommutatoren $l_g$-verkn\"upft und es folgt
$$
dl_g ([\tilde X, \tilde Y]_e) = [\tilde X, \tilde Y ]_g = \widetilde{[ X,  Y]}_g \ .
$$
Der Kommutator von $\tilde X$ und $\tilde Y$ ist also wieder links-invariant und es gilt die behauptete
Gleichung, denn nach Definition ist $\; dl_g (Z) = \tilde Z_g$ f\"ur ein $Z\in \g$.

\qed

\begin{Definition}
Die {\em Lie-Algebra} $\g$ einer Lie-Gruppe $G$ ist definiert als der Raum der links-invarianten
Vektorfelder bzw. der Tangentialraum an $G$ in $e$, mit dem Kommutator von Vektorfeldern als
Lie-Klammer. Man schreibt auch $\g = \mathrm{Lie}(G)$.
\end{Definition}

{\bf Beispiel:}
Sei $G= \GL(n,\R)$, dann ist $\g = \gl (n,\R)= M(n,\R)$. Die Lie-Klammer
auf $\g$ stimmt mit dem \"ublichen Kommutator von Matrizen \"uberein:
$[A, B] = A\cdot B - B\cdot A$.

\bigskip

\subsection{Die Lie-Algebra - Lie-Gruppe Korrespondenz}

Sei $H\subset G$ eine Lie-Untergruppe, d.h. eine Untergruppe von $G$, die
auch eine Untermannigfaltigkeit von $G$ ist. Sei $i:H \rightarrow G$ die
Inklusionsabbildung. Dann identifiziert man $T_eH$ mit dem Unterraum
$di (T_eH)$ von $T_eG$. Jeder Vektor $X \in T_eH$ definiert ein links-invariantes
Vektorfeld $X_G$ auf $G$ und ebenso ein links-invariantes Vektorfeld
$X_H$ auf $H$. Diese beiden Vektorfelder sind $i$-verkn\"upft
und da $i \circ l^H_a = l^G_a \circ i$ folgt:
$$
di X_H(a) = di \circ dl^H_a(X) = dl^G_a \circ di (X) = X_G(a) \ .
$$
Sind nun $X, Y \in T_eH$, dann sind auch die Kommutatoren $i$-verkn\"upft,
d.h. es gilt
$$
[X_G, Y_G]_e = di [X_H, Y_H]_e \ .
$$
Damit ist $T_eH \subset T_eG$ eine Lie-Unteralgebra, d.h. ein Untervektorraum,
der abgeschlossen unter der Lie-Klammer ist. Au\ss erdem ist $T_eH$ mit der
induzierten Lie-Klammer von $G$ genau die Lie-Algebra von $H$. Insbesondere
ist die Lie-Klammer auf der Lie-Algebra einer beliebigen Lie-Untergruppe
von $\GL(n,\R)$ immer die Einschr\"ankung des \"ublichen Kommutators von
Matrizen.

\medskip

Zu jeder Untergruppe $H\subset G$ geh\"ort also eine Unteralgebra $\h \subset \g$
Es zeigt sich nun, dass auch die Umkehrung gilt.

\begin{Satz}\label{unter}
Sei $G$ eine Lie-Gruppe und $\h \subset \g$ eine Lie-Unteralgebra. Dann existiert
eine eindeutig bestimmte zusammenh\"angende Lie-Untergruppe $H\subset G$ mit Lie-Algebra
$\Lie(H)=\h$. D.h. man hat eine bijektive Beziehung zwischen zusammenh\"angenden
Lie-Untergruppen von $G$ und Lie-Unteralgebren von $\g$.
\end{Satz}
\proof
Man definiert durch $\Delta_a = \{ \tilde X_a = dl_a(X) \,|\, X \in \h \} \subset T_aG$ eine Distribution
auf $G$. Die Schnitte in $\Delta$ sind also genau die links-invarianten Vektorfelder $\tilde X$ zu
Vektoren $X\in \h$. Nach Satz~\ref{LA} gilt aber
$
[\tilde X, \tilde Y] = \widetilde{[X,Y]}
$. Da $\h \subset \g$ eine Lie-Unteralgebra ist, mu\ss{} mit $X,Y\in \h$ auch der Kommutator
$[X,Y]$ in $\h$ liegen und damit ist $[\tilde X, \tilde Y]$ wieder ein Schnitt in $\Delta$.
Die Distribution $\Delta$ ist also integrabel. Man definiert $H$ als die eindeutig bestimmte
maximale Integral-Mannigfaltigkeit von $\Delta$ durch $e$, die nach dem Satz von Frobenius
bzw. Satz~\ref{blaetter} existiert. Es bleibt nun zu zeigen, dass $H\subset G$ eine Untergruppe
ist und dass die Gruppenoperationen glatt sind.

\medskip

Sei $b\in G$ dann folgt $dl_b(\Delta_a) = \Delta_{ba}$ aus der Definition von $\Delta$. Daher
permutiert die Links-Translation $l_b$ die verschiedenen maximalen Integral-Mannigfaltigkeiten
von $\Delta$. Ist zum Beispiel $c$ eine Kurve durch $a$ in der Integral-Mannigfaltigkeit durch $a$. Dann
ist $\dot c(t) \in \Delta_{c(t)}$  und $l_b(c)$ ist eine Kurve durch $ba$ mit Tangentialvektoren
$ dl_b(\dot c(t)) \in dl_b(\Delta_{c(t)})=\Delta_{bc(t)}$. Damit mu\ss{} die Kurve $bc(t)$ aber in
der Integral-Mannigfaltigkeit durch $ba$ liegen.

\medskip

Sei $b\in H$ dann ist $l_{b^{-1}}(H)$ die maximale Integral-Mannigfaltigkeit durch $l_{b^{-1}}(b)=e$,
also $l_{b^{-1}}(H) = H$. Daraus folgt $b^{-1}\in H$, aber auch $a\cdot b\in H$, f\"ur $a, b\in H$.
Somit ist $H\subset G$ eine Untergruppe.

\medskip

Es bleibt zu zeigen, dass die Abbildung $(a, b)\mapsto a\cdot b^{-1}$ eine glatte Abbildung von
$H\times H$ nach $H$ ist. Das ergibt sich daraus, dass sie glatt als Abbildung nach $G$ ist
und dass $H$ definiert ist als Blatt einer Bl\"atterung auf $G$ (siehe Spivak).
\qed

\bigskip

{\bf Bemerkung:}
\begin{enumerate}
\item
Nach dem Theorem von Ado ist jede Lie-Algebra isomorph zu einer Lie-Unteralgebra
der Lie-Algebra von $\GL(n,\R)$. Nach dem gerade gezeigtem ist also jede
Lie-Algebra isomorph zur Lie-Algebra einer Lie-Untergruppe von $\GL(n,\R)$.
F\"ur Lie-Algebren mit trivialem Zentrum folgt die Aussage trivialerweise
aus der Existenz der adjungierten Darstellung (siehe unten).
\item
Nach dem {\it Cartan-Kriterium} ist eine abgeschlossene Untergruppe $H\subset G$
automatisch eine Lie-Untergruppe von $G$, ist also eine Unter-Mannigfaltigkeit
von $G$.
\end{enumerate}


\bigskip

\subsection{Homomorphismen}


\begin{Lemma}
Sei $\varphi : G\rightarrow H$ ein Homomorphismus von Lie-Gruppen, dann ist $d\varphi$ ein
Lie-Algebren-Homomorphismus, d.h. es gilt
$$
d\varphi ([X, Y]) = [d\varphi (X), d\varphi (Y)]
$$
f\"ur alle $X, Y \in \g = T_eG$.
\end{Lemma}
\proof
Nach Definition der Lie-Klammer auf den Lie-Algebren von $G$ bzw. $H$ hat man f\"ur
beliebige  $X,Y \in \g= T_eG$ die folgende Gleichung zu zeigen
$$
d\varphi ([\tilde X,\tilde Y]_e) = [\widetilde{d\varphi(X)},\widetilde{d\varphi(Y)}]_e \ ,
$$
dabei bezeichnet $\tilde X$ das links-invariante Vektorfeld zum Vektor $X\in T_e G $,
d.h. das Vektorfeld $\tilde X$ ist definiert durch $\tilde X_g = dl_g (X)$. Da $\varphi$ ein Gruppenhomomorphismus ist, gilt
$\varphi \circ l_g = l_{\varphi(g)} \circ \varphi$.  Nach \"Ubergang zum Differential
folgt daraus
$$
d\varphi \circ dl_g = dl_{\varphi(g)} \circ d\varphi \ .
$$
Diese Gleichung wendet man auf $X\in T_eG$ an und erh\"alt
$$
d\varphi (\tilde X_g) = \widetilde{d\varphi(X)}_{\varphi(g)} \ ,
$$
d.h. die links-invarianten Vektorfelder $\tilde X$ und $\widetilde{d\varphi(X)}$
sind $\varphi$-verkn\"upft. Seien $X,Y$ beliebige Vektoren in $T_eG$ dann sind auch die entsprechenden
Kommutatoren $\varphi$-verkn\"upft und es folgt
$$
d\varphi ([\tilde X, \tilde Y]_e) = [\widetilde{d\varphi(X)}, \widetilde{d\varphi(Y)}]_e \ ,
$$
Das war aber genau die Gleichung, die zu zeigen war und somit ist $d\varphi$
ein Lie-Algebren-Homomorphismus.
\qed

\bigskip

{\bf Beispiel:}
Sei $G=H=\R$ dann existieren \"uberabz\"ahlbar viele Homomorphismen $\varphi : \R \rightarrow \R$,
da $\R$ ein unendlich-dimensionaler Vektorraum \"uber $\Q$ ist und jede $\Q$-lineare Abbildung
ist ein Homomorphismus. Sei $\varphi$ ein differenzierbarer Homomorphismus (tats\"achlich ist
stetig ausreichend) dann folgt $\varphi(t) = c t$ f\"ur ein geeignetes $c$. Denn leitet man
$\varphi(s+t)=\varphi(s) + \varphi(t)$ nach $s$ in $s=0$ ab, dann folgt $\dot \varphi(t) = \dot \varphi(0)$,
also $\varphi(t) = ct$ mit $c= \dot \varphi(0)$.

\medskip

Homomorphismen $\phi: \R \rightarrow S^1 = \R / \Z$ sind alle von der Form $\phi(t) = e^{ict}$.

\bigskip

{\bf Bemerkung:} Es existiert kein nicht-trivialer Lie-Gruppen-Homomorphismus $\varphi : S^1 \rightarrow \R$.
Denn $\varphi(S^1)$ w\"are eine kompakte Untergruppe in $\R$, was nicht m\"oglich ist: Sei $G\subset \R$
eine kompakte Untergruppe, dann ist $G$ beschr\"ankt, aber mit jedem $g\in G$ liegt auch $n g \in G$ f\"ur
alle $n \in \N$, was aber der Beschr\"anktheit widerspricht.

\medskip

Man sieht also, dass es Lie-Algebren-Homomorphismen $\g \rightarrow \h$ gibt, die sich nicht als
Differential eines Lie-Gruppen-Homomorphismus $G\rightarrow H$ ergeben. Mit dem folgenden Satz
hat man aber wenigstens ein lokales Resultat.

\bigskip

\begin{Satz}\label{hom}
Seien $G, H$ Lie-Gruppen und sei $\,\Phi : \g \rightarrow \h$ ein fixierter Lie-Algebren-Homomorphismus. Dann
existiert eine Umgebung $U$ von $e\in G$ und eine glatte Abbildung $\varphi : U \rightarrow H$ mit
\begin{enumerate}
\item\label{11}
$
\varphi (a \cdot b) = \varphi(a) \cdot \varphi(b)
\qquad \mbox{falls}\quad a, b, a\cdot b \in U
$
\item\label{22}
$
d\varphi(X) = \Phi(X)
\qquad \mbox{f\"ur alle}\quad X\in \g= T_eG
$
\item
Sind $\varphi, \psi : G\rightarrow H$ zwei Lie-Gruppen-Homomorphismen mit $d\varphi = d\psi = \Phi$
und ist $G$ zusammenh\"angend, dann folgt $\varphi = \psi$.
\end{enumerate}
\end{Satz}
\proof (nach Spivak)
Die Idee des Beweises ist, dass man eine Abbildung durch ihren Graphen beschreiben kann. Dieser ist
f\"ur Homomorphismen eine Untergruppe bzw. Unteralgebra.

\medskip


Man definiert $\k := \{ (X, \Phi(X)) \,|\, X \in \g \} \subset \g \times \h = \Lie(G \times H)$, d.h.
$\k$ ist der Graph von $\Phi$ und die Voraussetzung, dass $\Phi$ ein Lie-Algebren-Homomorphismus ist,
ist \"aquivalent dazu, dass $\k \subset \g \times \h$ eine Unteralgebra ist:
$$
[(X_1,\Phi(X_1)),(X_2,\Phi(X_2))]
=
([X_1,X_2], [\Phi(X_1), \Phi(X_2)])
=
([X_1,X_2], \Phi([X_1,X_2]))\in \k \ .
$$
Nach Satz~\ref{unter} existiert eine eindeutig bestimmte Lie-Untergruppe $K\subset G\times H$ mit
$\k = \Lie(K)$.

\medskip

Sei $\pi_1: G \times H \rightarrow G$ die Projektion auf den 1.~Faktor. Dann ist
$\omega := \left. \pi_1\right|_K : K \rightarrow G$ ein Lie-Gruppen-Homomorphismus und
$$
d\omega (X, \Phi(X)) = X \ .
$$
Damit ist $d\omega : T_{e,e}K \rightarrow T_eG$ ein Isomorphismus. Es folgt, dass eine Umgebung
$V$ von $(e,e)$ existiert, auf der $\omega$ ein Diffeomorphismus ist. Sei $U:=\omega(V)$, dann
ist $U$ eine Umgebung von $e\in G$.

\medskip

Sei $\pi_2 : G\times H \rightarrow H$ die Projektion auf den 2.~Faktor. Dann definiert man auf $U$:
$$
\varphi := \pi_2 \circ \omega^{-1} \ .
$$
Die Bedingung \ref{11} ist erf\"ullt, da $\pi_2$ und $\omega^{-1}$ Gruppen-Homomorphismen sind. Auch
die Bedingung \ref{22} ist erf\"ullt, da
$$
d\varphi (X) = \pi_2 (X, \Phi(X)) = \Phi(X) \ .
$$

\medskip

Bleibt noch die Eindeutigkeit zu zeigen. Seien also $\varphi, \psi : G\rightarrow H$ zwei Homomorphismen
mit $d\varphi = d\psi =\Phi $. Man definiert den injektiven Gruppen-Homomorphismus

$$
\theta : G \rightarrow G\times H, \quad a\mapsto (a, \psi(a)) \ .
$$
Sei $\hat G = \theta(G) \subset G\times H$ das Bild von $\theta$, also der Graph von $\psi$.
Dann ist $\hat G$ eine zusammenh\"angende
Lie-Untergruppe von $G\times H$ mit Lie-Algebra $\Lie(\hat G) = \k$. Es gilt n\"amlich
$d\theta(X) = (X, \Phi(X))$. Aus Satz~\ref{unter} folgt nun $\hat G = K$. Die Untergruppe $K$ ist
eindeutig durch $\Phi$ bestimmt. Somit stimmen die Graphen von $\varphi$ und $\psi$
\"uberein, d.h. $\varphi = \psi$.
\qed

\bigskip

{\bf Bemerkung:}
Der Gruppen-Homomorphismus aus Satz~\ref{hom} ist auf ganz $G$ definiert, falls
$G$ einfach-zusammenh\"angend ist. Unter dieser Voraussetzung hat man also ein
bijektive Beziehung zwischen Lie-Gruppen-Homomorphismen $G\rightarrow H$
und Lie-Algebren-Homomorphismen $\g \rightarrow \h$.

\bigskip

\begin{Folgerung}\label{hom1}
Zwei Lie-Gruppen mit isomorphen Lie-Algebren sind lokal isomorph. Die Isomorphie
ist global, falls beide Gruppen einfach-zusammenh\"angend sind.
\end{Folgerung}



\begin{Folgerung}
Eine zusammenh\"angende Lie-Gruppe mit abelscher Lie-Algebra ist abelsch. Dabei hei\ss t
eine Lie-Algebra abelsch, falls alle Lie-Klammern Null sind.
\end{Folgerung}
\proof
Eine abelsche Lie-Algebra $\g = \Lie(G)$ ist isomorph zu $\R^n$. Nach Folgerung~\ref{hom1}
ist $G$ dann lokal isomorph zu $\R^n$. Es gilt also $a\cdot b = b \cdot a$ f\"ur alle
$a,b$ in einer hinreichend kleinen Umgebung von $e\in G$. Jede solcher Umgebungen erzeugt
aber die ganze Gruppe, d.h. jedes Gruppenelement l\"a\ss t sich als Produkt von Elementen
aus dieser Umgebung schreiben. Damit ist die ganze Gruppe $G$ abelsch.
\qed

\bigskip

{\bf Bemerkung:}
\begin{enumerate}
\item
Alle zusammenh\"angenden Lie-Gruppen mit der gleichen Lie-Algebra werden von der gleichen
einfach-zusammenh\"angenden Lie-Gruppe \"uberlagert.
\item
Die Lie-Algebra einer Lie-Gruppe $G$ ist gleich der Lie-Algebra der Zusammenhangskomponente
der Eins in $G$. Zum Beispiel ist $\Lie(O(n)) = \Lie(SO(n))$.
\end{enumerate}


\bigskip

\subsection{Die Exponential-Abbildung}

\bigskip

\begin{Satz}\label{satz1}
Sei $G$ eine Lie-Gruppe mit Lie-Algebra $\g$. Sei $X\in \g$ ein links-invariantes Vektorfeld mit
maximaler Integralkurve $c_X$ durch $e\in G$. Dann gilt
\begin{enumerate}
\item
Links-invariante Vektorfelder sind vollst\"andig, d.h. $c_X$ ist definiert auf ganz $\R$.
\item
Die Integralkurve $c_X : \R \rightarrow G$ ist ein Lie-Gruppen-Homomorphismus, dh.
$$
c_X(0) = e, \quad c_X(s+t) = c_X(s) \cdot c_X(t) \qquad \mbox{f\"ur alle }\quad s, t \in \R
$$
\item
F\"ur alle $s, t \in \R$ gilt: $c_{sX}(t) = c_X(s t)$.
\end{enumerate}
\end{Satz}
\proof
Sei $c_X : I = (t_{\min}, t_{\max})\rightarrow G$ die maximale Integralkurve von $X$, d.h.
$c_X(0)=e, \; \dot c_X(t)= X_{c_X(t)}$.

\medskip

Als erstes soll die zweite Aussage bewiesen werden. Sei dazu ein Parameter $s\in I$ fixiert  und $g:=c_X(s)\in G$.
Man betrachtet nun die Kurven
$$
\begin{array}{rll}
a(\tau) &:= g \cdot c_X(\tau ) &\qquad \qquad  \tau \in I \\[1.5ex]
b(\tau) &:= c_X(\tau + s)      &\qquad \qquad\tau \in (t_{\min}-s, t_{\max}+s)
\end{array}
$$


xxxxxxxxxxxxxxxxxxxxxxxxxxxxxxxxxxxxxxxxxxxxxxxxxxxxxxxxxxxx


\qed

\bigskip

\begin{Definition}
Sei $G$ eine Lie-Gruppe mit Lie-Algebra $\g$. Die Abbildung
$$
\exp : \g \rightarrow G, \quad \exp(X) : = c_X(1) \ ,
$$
hei\ss t {\em Exponentialabbildung} der Lie-Gruppe $G$. Hierbei ist
$c_X$ wieder die maximale Integralkurve von $X$ mit $c_X(0)=e$.
\end{Definition}


{\bf Bemerkung:}
\begin{enumerate}
\item
Es gilt f\"ur alle $t$ die Gleichung $\exp(tX) = c_{tX}(1) = c_X(t) =  \varphi_t(e)$.
\item
Homomorphismen $\R \rightarrow G$ nennt man auch 1-parametrige Untergruppen.
Die Abbildung $\R\rightarrow G, t\mapsto \exp tX$ ist der eindeutig bestimmte
glatte Gruppenhomomorphismus mit
$$
\left.\tfrac{d}{dt}\right|_{t=0} \exp tX = X
$$
\item
Sei $X$ ein links-invariantes Vektorfeld mit lokalem Flu\ss{} $\varphi_t$
Dann gilt:
$$
\varphi_t(g) = g \cdot \varphi_t(e) \ ,
$$
d.h. die Kurve $a(t):= g\cdot \exp (tX)$ ist die maximale Integralkurve von
$X$ durch $g$.
Denn sei $X$ links-invariant, dann gilt $X= l_{g\ast}X$ und das bedeutet f\"ur
die Fl\"usse $\varphi_t = l_g \circ \varphi_t \circ l_{g}^{-1}$. Angewandt
auf $g$ folgt die Behauptung.
\end{enumerate}

\bigskip

{\bf Beispiel:} Die Exponential-Abbildung f\"ur $G= \GL(n,\R)$

Man definiert f\"ur $A \in \gl(n,\R)$:
$$
\exp t A = e^{tA} = \sum^\infty_{k=0} \frac{(tA)^k}{k!} \ .
$$

Diese Reihe konvergiert absolut und gleichm\"a\ss ig auf jeder beschr\"ankten Menge
in $\gl(n, \R)$. Die Exponential-Abbildung f\"ur $\GL(n,\R)$ hat die folgenden
Eigenschaften

\begin{enumerate}
\item
Gilt $A \, B = B \, A $ so auch $e^{A+B} =e^A \cdot e^B$.
\item
$\det(e^A) = e^{\tr (A)}$
\item
$B \, e^A \, B^{-1} = e^{B A B^{-1}}$
\end{enumerate}

\bigskip

\begin{Satz}
Sei $\varphi : G\rightarrow H$ ein Lie-Gruppen-Homomorphismus. Dann gilt
$$
\varphi (\exp X) = \exp (d\varphi(X))
$$
f\"ur alle $X\in \g = \Lie(G)$, d.h. man hat das folgende kommutative Diagramm

\begin{equation*}
\begin{CD}
\mathfrak{g} @> d\varphi>> \mathfrak{h}\\
@V\exp VV @VV {\exp}V \\
G @>\varphi>> H
\end{CD}
\end{equation*}





\end{Satz}
\proof
Sei $c(t)= \varphi (\exp tX)$. Dann ist $c(0) = \varphi(e_G)=e_H$ und
$$
\begin{array}{rl}
\dot c(t) &= d\varphi (X_{\exp tX}) = d\varphi \circ dl_{\exp t X} (X_{e_G})\\[1.5ex]
&= dl_{\varphi(\exp tX)} \circ d\varphi (X_{e_G})\\[1.5ex]
&= \widetilde{d\varphi (X_{e_G})}_{c(t)} \ ,
\end{array}
$$
d.h. $c(t)$ ist eine Integralkurve von $\widetilde{d\varphi (X_{e_G})}$ durch $e_H$ und
somit folgt
$$
c(1) = \varphi(\exp X) = \exp (d\varphi (X)) \ .
$$
\qed

\begin{Satz}
Die Exponential-Abbildung $\exp : \g \rightarrow G$ ist beliebig oft differenzierbar
und ein lokaler Diffeomorphismus um $0\in \g$. Weiter gilt:
\begin{enumerate}
\item
$\exp(0) = e$
\item
$\exp (-X)= \exp(X)^{-1}$  f\"ur alle $X\in \g$
\item
$
\exp((t+s)X) = \exp(sX) \exp(tX)
$
\end{enumerate}
\end{Satz}
\proof
Die Glattheit der Abbildung $X \mapsto \exp (X)$ folgt aus den Standards\"atzen \"uber die glatte
Abh\"angigkeit der L\"osungen linearer DGLs von den Anfangswerten.

\medskip

Die drei Aussagen folgen direkt aus der Definition der Exponential-Abbildung  und aus Satz~\ref{satz1}.
Zum Beispiel zeigt man die dritte Aussage durch
$$
\exp(t+s)A = c_{(t+s)A}(1)= c_A(t+s) =c_A(t)\cdot c_A(s) = c_{tA}(1)\cdot c_{sA}(1) = \exp tA \cdot \exp sA \ .
$$

\medskip

Die Exponential-Abbildung ist ein lokaler Diffeomorphismus um $0 \in \g$, da $d \exp_0 = \Id_{T_eG}$. Denn
sei $X\in \g$ ein links-invariantes Vektorfeld, dann folgt
$$
d \exp_0 (X) =\left. \tfrac{d}{dt}\right|_{t=0} \exp(tX) = X \ .
$$
\qed

\bigskip

{\bf Bemerkung:}
Mit Hilfe der Exponential-Abbildung lassen sich spezielle Koordinaten definieren. Sei
$V(0) \subset \g$ eine bzgl. $0$ sternf\"ormige Umgebung, auf der $\exp$ ein
Diffeomorphismus ist. Dann definiert man
$$
W(e) := \exp (V(0)), \quad W(g) := l_g(W(e)), \quad x_g:= \exp^{-1}\circ l_{g^{-1}}:W(g) \rightarrow V(0)\subset \g \cong \R^n
$$
Offensichtlich ist dann $(W(g), x_g)$ eine Karte um $g$. Man nennt $W(g)$ eine {\it Normalenumgebung} von $g$ und $x_g$ die
{\it Normalkoordinaten} um $g$.

\bigskip

{\bf Bemerkung:}
Im Allgemeinen ist die Exponential-Abbildung nicht surjektiv.

\medskip

Sei $G= \SL(2,\R)$ die Menge
der Matrizen mit Determinante Eins. Die Lie-Algebra ist die Menge der spurfreien $2\times 2$-Matrizen.
F\"ur diese Gruppe ist die Exponential-Abbildung nicht surjektiv.
Zum Beispiel liegt $B = \diag(-\frac12,-2)$ nicht im Bild der Exponential-Abbildung. Denn w\"are $B = \exp X$,
so w\"urde folgen
$$
B = \exp X =  \exp (\tfrac12 X + \tfrac12 X) = \exp\tfrac12X \cdot \exp\tfrac12X = (\exp\tfrac12X)^2 \ .
$$
Es existiert
aber keine Matrix $A \in \SL(2,\R)$  mit $B=A^2$: aus dem Satz von Cayley-Hamilton
$$
A^2 - \tr A \cdot A + E = 0 \ .
$$
Nimmt man die Spur von dieser Gleichung erh\"alt man $\tr A^2 \ge -2$. Die Spur von $B$ ist aber kleiner als $-2$.

\medskip

Sp\"ater wird gezeigt, dass die Exponential-Abbildung auf kompakten Lie-Gruppen surjektiv ist.

\bigskip

\begin{Folgerung}\label{xxx}
Jeder injektive glatte Homomorphismus $\varphi : G \rightarrow H$ ist eine Immersion und $\varphi(G)\subset H$
ist damit eine Lie-Untergruppe von $G$.
\end{Folgerung}
\proof
Sei $X \in \g = T_eG$ und sei $\tilde X$ das dazugeh\"orige Vektorfeld, d.h. $\tilde X_g = dl_g (X)$. Ist nun
f\"ur ein $g$ das Differential von $\varphi$ nicht injektiv, also z.B. $d\varphi(\tilde X_g)=0$, dann folgt
$$
0  =  d\varphi \circ dl_g(X) = dl_{\varphi(g)}\,d\varphi(X)
$$
und da $dl_g$ ein Isomorphismus ist erh\"alt man $d\varphi(X)=0$. Wendet man darauf die Exponential-Abbildung an,
so ergibt sich
$$
e = \exp d\varphi(tX) = \varphi(\exp tX) \ ,
$$
was aber im Widerspruch steht zur Annahme, dass $\varphi$ injektiv ist. Damit muss also $d\varphi$ auch injektiv sein,
d.h. $\varphi$ ist eine Immersion.
\qed

\bigskip

\begin{Folgerung}\label{stetig}
Sei $G$ eine Lie-Gruppe mit Lie-Algebra $\g$ und sei $\varphi: \R \rightarrow G$ ein stetiger Gruppen-Homomorphismus.
Dann existiert ein $X\in \g$ mit $\varphi(t) = \exp (tX)$ f\"ur alle $t\in \R$. Insbesondere ist jeder stetige
Gruppen-Homomorphismus $\R \rightarrow G$ schon glatt.
\end{Folgerung}
\proof
Sei $W = \exp V(0)$ eine Normalenumgebung um $e\in G$. Da $\varphi : \R \rightarrow G$ stetig ist, mit $\varphi(0)=e$,
existiert ein $\varepsilon >0$ mit $\varphi(t) \in W$ f\"ur alle $t$ mit $|t| \le 2 \varepsilon$. Sei nun $Y \in V(0)$
der Vektor mit $\exp(Y) = \varphi(\varepsilon)$. Man definiert $X:= \frac1\varepsilon Y$. Man m\"ochte nun zeigen
$$
\varphi(t) = \exp (tX) =: f(t) \qquad \mbox{f\"ur alle} \quad t \in \R \ .
$$
Sei $K \subset \R$ definiert durch $K:= \{ t \in \R \, |\, f(t) = \varphi(t) \}$. Nach Definition gilt
$ 0, \varepsilon \in K$. Da $f$ und $\varphi$ stetige  Gruppen-Homomorphismen sind, ist $K$ eine abgeschlossene
Untergruppe von $\R$. F\"ur nicht-triviale abgeschlossene Untergruppen  $K\subset \R$ gibt es zwei M\"oglichkeiten:
$K = \R$ oder $K = K_d = \{ n \cdot d \,|\, n \in \Z\}$, wobei $d$ die kleinste positive Zahl in $K$ ist. Es
soll nun $K= \R$ gezeigt werden. Annahme es gilt $K= K_d$. Da $\varepsilon \in K$ und $\varepsilon >0$ folgt
%%% ??? $\varepsilon \ge k$ und
$2\varepsilon > d$, d.h. $\frac{d}{2} <\varepsilon$. Da $\varphi$ und $f$ Homomorphismen sind
erh\"alt man
$$
(f(\tfrac{d}{2}))^2 = \exp (\tfrac{d}{2} X)^2 = \exp (dX)=f(d) = \varphi(d) = (\varphi(\tfrac{d}{2}))^2 \ .
$$
Wendet man auf diese Gleichung $\exp^{-1}$ an, so ergibt sich
$$
2 \exp^{-1}f(\tfrac{d}{2}) = \exp^{-1} f(\tfrac{d}{2})^2 = \exp^{-1} \varphi(\tfrac{d}{2})^2   =
2\exp^{-1}\varphi(\tfrac{d}{2})
$$
und damit $f(\tfrac{d}{2}) = \varphi(\tfrac{d}{2})$. D.h. $d$ war nicht das kleinste Element in $K$, im Widerspruch
zur Annahme, also muss $K=\R$ gelten, was zu zeigen war.
\qed

\bigskip

\begin{Satz}\label{Phi}
Sei $G$ eine Lie-Gruppe mit Lie-Algebra $\g = V_1 \oplus \ldots \oplus V_r$. Dann ist die Abbildung
$$
\Phi : V_1 \oplus \ldots \oplus V_r \rightarrow G, \quad v_1 + \ldots + v_r \mapsto
\exp(v_1)\cdot \ldots \cdot \exp(v_r)
$$
ein lokaler Diffeomorphismus um $0 \in \g$.
\end{Satz}
\proof
Man zeigt wieder, dass $d\Phi_0 = \Id_{\g}$ gilt, womit $\Phi$ nach dem Umkehrsatz ein lokaler
Diffeomorphismus um $0 \in \g$ ist. F\"ur $i=1,\ldots, r$ sei $X_i \in V_i $, dann gilt
$$
d\Phi_0(X_i) = \left.\tfrac{d}{dt}\right|_{t=0} \Phi(tX_i)
=  \left.\tfrac{d}{dt}\right|_{t=0} \exp(0) \ldots \exp(tX_i) \ldots \exp(0)
=  \left.\tfrac{d}{dt}\right|_{t=0} \exp(tX_i) = X_i \ .
$$
\qed

\bigskip

\begin{Folgerung}
Jeder stetige Homomorphismus $\psi : G \rightarrow H$ ist schon glatt.
\end{Folgerung}
\proof
Sei $X_1, \ldots, X_n$ eine Basis in $T_eG$. F\"ur $i=1, \ldots, n$ definiert man
stetige Gruppen-Homomorphismen
$$
\psi_i : \R \rightarrow H,\quad t \mapsto \psi(\exp tX_i) \ .
$$
Nach Folgerung~\ref{stetig} existieren $Y_i \in T_eH$ mit
$\psi(\exp tX_i) = \exp tY_i$ f\"ur $i=1,\ldots, n$. Es folgt
\begin{equation}\label{nprodukt}
\psi(\exp t_1X_1 \cdot \ldots \cdot \exp t_n X_n)
=
\exp t_1 Y_1 \cdot \ldots \cdot \exp t_n Y_n \ .
\end{equation}
Nach Satz~\ref{Phi} ist die Abbildung $\Phi : \g \rightarrow G$ definiert durch
$$
\Phi(t_1X_1 + \ldots + t_nX_n) = \exp t_1X_1 \cdot \ldots \cdot \exp t_n X_n
$$
auf einer Umgebung $U$ von $0\in \g$ ein Diffeomorphismus. Sei $V= \Phi(U)$
die entsprechende Umgebung von $e\in G$. Auf $V$ schreibt man $\psi $ als
$$
\psi = (\psi \circ \Phi) \circ \Phi^{-1} \ .
$$
Mit Gleichung~(\ref{nprodukt}) schreibt sich $\psi \circ \Phi$ als
$$
\psi \circ \Phi : (t_1,\ldots, t_n)
\mapsto \exp t_1 Y_1 \cdot \ldots \cdot \exp t_n Y_n  \ ,
$$
d.h. $\psi \circ \Phi$ ist glatt auf $U$ und $\psi$ ist damit glatt
auf $V$. Der Homomorphismus $\psi$ ist also glatt auf einer Normalenumgebung
$W(e) \subset G$ von $e\in G$, doch damit ist $\psi$ glatt auf ganz $G$. Denn
auf $W(g) = l_g(W(e))$ schreibt man
$$
\left. \psi\right|_{W(g)} = l_{\psi(g)} \circ \left.\psi\right|_{W(e)} \circ l_{g^{-1}} \ .
$$
\qed

\bigskip

\begin{Folgerung}
Eine lokal-euklidische topologische Gruppe mit abz\"ahlbarer Topologie besitzt h\"ochstens
eine differenzierbare Struktur bzgl. der sie eine Lie-Gruppe wird.
\end{Folgerung}
\proof
Man betrachtet die Identit\"at, diese ist stetig und damit auch differenzierbar und w\"urde
dann einen Diffeomorphismus zwischen zwei fixierten differenzierbaren Strukturen definieren.
\qed

\bigskip

{\bf Bemerkung:} Das 5. der Hilbert-Probleme war die Frage, ob jede lokal-euklidische Gruppe eine
Lie-Gruppe ist. Die positive Antwort wurde 1952 von Montgomery und Zippen erbracht

\bigskip

Mit Hilfe der Exponential-Abbildung lassen sich nun noch weitere Aussagen \"uber Untergruppen und
Homomorphismen beweisen.

\bigskip

\begin{Lemma}
Sei $i: H \rightarrow G$ eine Lie-Untergruppe und sei $X\in \g$. Dann gilt
\begin{enumerate}
\item
$ X \in di(\h)\;$ dann folgt $\;\exp tX \in i(H)\subset G\;$ f\"ur alle $\;t\in
\R$.
\item
Gilt $\;\exp tX \in i(H)$ f\"ur alle $t$ in einem Intervall $I$, dann folgt $X \in di(\h)$.
\end{enumerate}
\end{Lemma}
\proof
Sei zun\"achst $X\in di(\h)$, es existiere also ein $X_0\in \h$ mit $X =di(X_0)$. Dann folgt
$\exp tX = \exp di(tX_0) = i (\exp tX_0)$ und somit $\exp tX \in i(H)$ f\"ur alle $t\in \R$.

\medskip

F\"ur den Beweis der umgekehrten Richtung betrachtet man die glatte Abbildung $I \subset \R \rightarrow G,
t\mapsto \exp tX$. Nach Voraussetzung liegt das Bild dieser Abbildung in $i(H)$. Daher existiert
eine glatte Abbildung $\alpha : I \rightarrow H$ mit $\exp tX = i(\alpha(t))$.
(Diese Aussage bleibt noch zu zeigen.) Sei nun $\tilde X$ das links-invariante
Vektorfeld auf $H$ zu $\dot \alpha (0)$. Dann gilt $di (\tilde X) = X$ und somit
$X \in di (\h)$.
\qed

\bigskip

\begin{Lemma}\label{kern}
Sei $\varphi : G \rightarrow K$ ein Homomorphismus von Lie-Gruppen. Dann ist $\ker \varphi$ eine
abgeschlossene Untergruppe von $G$ und es gilt:
$$
\Lie(\ker \varphi) = \ker (d\varphi) \ .
$$
\end{Lemma}
\proof
Sei $X\in \g$ dann ist, nach dem vorherigem Lemma, $X\in \Lie(\ker \varphi) $ genau dann, wenn
$\exp tX \in \ker \varphi$ f\"ur alle $t\in \R$, d.h. $\varphi (\exp tX) = e$ bzw.
$\exp d\varphi (tX)=e$ f\"ur alle $t\in \R$. Da $\exp$ aber in einer Umgebung der Null
ein lokaler Diffeomorphismus ist, ist das nur m\"oglich, wenn $d\varphi (X)=0$, also
$X\in \ker (d\varphi)$, was zu zeigen war.
\qed

\bigskip













\subsection{Die adjungierte Darstellung}

\bigskip

\begin{Definition}
Sei $G$ eine Lie-Gruppe mit Lie-Algebra $\g$. Eine {\em Darstellung von $G$} auf einem Vektorraum
$V$ ist ein Gruppen-Homomorphismus
$$
\rho : G \rightarrow \GL(V) = \Aut(V)
$$
d.h. es gilt $\rho (g_1 \cdot g_2) = \rho(g_1)\cdot \rho(g_2)$ f\"ur alle $g_1, g_2 \in G.$

\medskip

Eine {\em Darstellung von $\g$} auf $V$ ist ein Lie-Algebren-Homomorphismus
$$
\varphi : \g \rightarrow \gl(V) = \End(V)
$$
d.h. es gilt $\varphi([X,Y]) = \varphi(X) \circ \varphi(Y) - \varphi(Y) \circ \varphi(X) $.

\medskip

Eine Darstellung hei\ss t {\em treu}, falls $\rho$ bzw. $\varphi$ injektive Abbildungen sind.
\end{Definition}

\bigskip

{\bf Bemerkungen:}
\begin{enumerate}
\item
Sei $\rho : G \rightarrow \GL(V)$ eine Darstellung von $G$, dann ist $\varphi = d\rho :
\g  \rightarrow \gl(V)$ eine Darstellung von $\g$.
\item
Eine Darstellung $\rho : G \rightarrow \GL(V)$ hei\ss t {\it orthogonal} bzw. {\it unit\"ar},
falls $(V, \la \cdot, \cdot \ra)$ ein euklidischer bzw. unit\"arer Vektorraum ist und
$$
\la \rho(g)v, \rho(g)w\ra = \la v, w\ra
$$
f\"ur alle $g\in G$ und $v, w \in V$, d.h. das Skalarprodukt $\la \cdot, \cdot \ra$ auf $V$
 ist $G$-invariant und damit
$\rho(G) \subset \O(n)$ bzw. $\rho(G) \subset \U(n)$, falls $n = \dim V$.
\item
Sei $\rho : G \rightarrow \GL(V)$ eine Darstellung einer kompakten Lie-Gruppe $G$. Dann existiert
auf $V$ ein $G$-invariantes Skalarprodukt. Man definiert
$$
(v,w): = \int_G \la \rho(g)v, \rho(g)w\ra \omega_g \ .
$$
Dann ist $(\cdot, \cdot)$ ein $G$-invariantes Skalarprodukt, falls $\omega_g$ eine rechts-invariante
Volumenform und $\la \cdot, \cdot \ra$ ein beliebiges Skalarprodukt ist.
\end{enumerate}

\bigskip

\begin{Satz}\label{skalar}
Sei $\rho : G \rightarrow \GL(V)$ eine Darstellung einer Lie-Gruppe $G$ und sei
$\la \cdot, \cdot \ra$  ein $G$-invariantes Skalarprodukt auf $V$, dh.
\begin{equation}\label{Ginv}
\la \rho(g)v, \rho(g)w\ra = \la v, w\ra
\end{equation}
f\"ur alle $g\in G$ und $v, w \in V$. Dann gilt
\begin{equation}\label{ginv}
\la d\rho(X)v, w\ra +  \la v, d\rho(X)w \ra = 0
\end{equation}
f\"ur alle $X\in \g$ und $v, w \in V$. Ist $G$ zusammenh\"angend, dann sind
(\ref{Ginv}) und (\ref{ginv}) \"aquivalent.
\end{Satz}
\proof
Sei $X\in \g$ und $v, w \in V$, dann folgt aus Gleichung~\ref{Ginv}
$$
\la v, w \ra = \la \rho(\exp tX)v, \rho(\exp tX )w \ra
=
\la e^{td\rho(X)}v, e^{td\rho(X)} w \ra \ .
$$
Die Ableitung nach $t$ in $t=0$ liefert dann Gleichung~\ref{ginv}.

\medskip

Sei $G$ zusammenh\"angend, dann wird $G$ erzeugt von einer Normalenumgebung von $e$.
Die Gleichung~\ref{Ginv} folgt also aus
$$
 \la \rho(\exp tX)v, \rho(\exp tX)w \ra  = \la v, w \ra
$$
f\"ur alle $X\in \g$ und $v, w \in V$. Man definiert eine Funktion $F$ durch
$$
F(t) :=  \la \rho(\exp tX)v, \rho(\exp tX )w \ra = \la e^{td\rho(X)}v, e^{td\rho(X)}w  \ra \ .
$$
Die Ableitung von $F$ nach $t$ berechnet sich als
$$
\dot F(t) = \la d\rho(X) e^{td\rho(X)}v, e^{td\rho(X)}w  \ra + \la e^{td\rho(X)}v, d\rho(X) e^{td\rho(X)}w  \ra \ .
$$
Nach Gleichung~\ref{ginv} ist also $\dot F(t) = 0$ f\"ur alle $t\in \R$, d.h. $F$ ist konstant und
somit $F(t) = F(0) = \la v, w \ra $, was genau die zu beweisende Gleichung~\ref{Ginv} ist.
\qed


\bigskip

Sei $g\in G$,  als {\it Konjugation} mit $g\in G$ bezeichnet man den Gruppen-Homomorphismus
$\alpha_g : G \rightarrow G, h \mapsto g\cdot h \cdot g^{-1}$, d.h. $\alpha_g = l_g \circ r_{g^{-1}}$.
Tats\"achlich gilt
$$
\alpha_{gh}(a) = gh \cdot a \cdot (gh)^{-1} = g h \cdot a \cdot h^{-1} g^{-1} = \alpha_g \circ \alpha_h (a) \ .
$$
Die Konjugation ist ein {\it Automorphismus} von $G$, d.h. ein Gruppen-Isomorphismus auf $G$. Man
sagt $\alpha_g$ ist ein {\it innerer} Automorphismus. Das Differential $\Ad(g)=d\alpha_g : \g \rightarrow \g$
ist dann ein Automorphismus von $\g$, d.h. ein Lie-Algebren-Isomorphismus auf $\g$:
$$
\Ad(g) [X, Y] = [\Ad(g)X, \Ad(g)Y], \qquad \Ad(g)^{-1} = \Ad(g^{-1})
$$
f\"ur alle $X,Y\in \g$ und alle $g\in G$. Aus der Definition folgt auch direkt:
$$
\Ad(gh) = d\alpha_{gh} = d(\alpha_g \circ \alpha_h) = d\alpha_g \circ d\alpha_h = \Ad(g) \circ \Ad(h) \ ,
$$
d.h. $\Ad : G \rightarrow \GL(\g)$ ist ein Homomorphismus. Sei $X\in \g$ und $t$ hinreichend klein, dann gilt
$$
\Ad(g)X = \tfrac1t \,\exp^{-1} (\alpha_g(\exp tX)) = \tfrac1t \, \exp^{-1}(g \cdot \exp tX \cdot g^{-1}) \ .
$$
Das folgt aus der Gleichung $\varphi(\exp tX)= \exp(d\varphi tX)$ f\"ur den Homomorphismus $\varphi = \alpha_g$.
Man erh\"alt, dass die Abbildung $\Ad$ ein stetiger und damit auch differenzierbarer Lie-Gruppen Homomorphismus
ist. Damit hat man den folgenden Satz bewiesen:

\bigskip

\begin{Satz}\label{Ad}
Die Abbildung $\Ad : G \rightarrow \GL(\g), g \mapsto \Ad(g)=d\alpha_g$ ist eine Darstellung von $G$, d.h.
ein Lie-Gruppen-Homomorphismus. Man nennt $\Ad$ die {\em adjungierte Darstellung} von $G$.
\end{Satz}

\begin{Satz}\label{ad}
F\"ur das Differential $\ad = d \Ad : \g \rightarrow \gl(\g) $ gilt : $\ad(X)Y = [X,Y]$ f\"ur alle
$X,Y\in \g$. Man nennt $\ad$ die {\em adjungierte Darstellung} von $\g$.
\end{Satz}
\proof
Sei $X$ ein links-invariantes Vektorfeld, dann sind seine Integralkurven durch $g\in G$  gegeben als
$c_g(t) = g \cdot \exp t X$. Somit schreibt sich der lokale Flu\ss{} von $X$ als
$\varphi_t(g) = r_{\exp tX}(g)$.  Es gilt nun
$$
\begin{array}{ll}
\ad(X)Y &= d\Ad(X) Y \; = \;
\left( \left. \tfrac{d}{dt}\right|_{t=0} \Ad(\exp tX) \right) Y\\[1.5ex]
&=
\left. \tfrac{d}{dt}\right|_{t=0}  \left(dr_{\exp -tX} \, dl_{\exp tX}\, Y\right) \\[1.5ex]
&=
\left. \tfrac{d}{dt}\right|_{t=0}  \left(  dr_{\exp -tX} \tilde Y_{\exp tX} \right) \\[1.5ex]
&=
\left. \tfrac{d}{dt}\right|_{t=0} \left( d\varphi_{-t} \tilde Y_{\varphi_t(e)} \right)\\[1.5ex]
&=
[\tilde X, \tilde Y]_e = [X,Y]
\end{array}
$$
\qed


\bigskip

{\bf Bemerkungen:}
\begin{enumerate}
\item
F\"ur Matrizengruppen $G\subset \GL(n,\R)$ gilt:
$$
\Ad(g)A = g \, A \, g^{-1} \qquad \mbox{f\"ur alle }\quad g\in G, A\in \g
$$
\item
Es gilt $g \exp X g^{-1} = \exp \Ad(g)X$, d.h. man hat das kommutative Diagramm:

\begin{equation*}
\begin{CD}
\mathfrak{g} @> \Ad(g)=d\alpha_g >> \mathfrak{g}\\
@V\exp VV @VV {\exp}V \\
G @>\alpha_g >> G
\end{CD}
\end{equation*}

Speziell f\"ur Matrizengruppen erh\"alt man die schon erw\"ahnte Gleichung:
$$
B e^A B^{-1}= e^{BAB^{-1}} \ .
$$

\item
Es gilt $\Ad(\exp X)= e^{\ad(X)}$, d.h. man hat das kommutative Diagramm:


\begin{equation*}
\begin{CD}
\mathfrak{g} @> \ad =d \Ad >> \End (\g) \cong \gl(n,\R)\\
@V\exp VV @VV {\exp}V \\
G @>\Ad >> \GL(\g) \cong \GL(n,\R)
\end{CD}
\end{equation*}



\item
Die Jacobi-Identit\"at \"ubersetzt sich in folgende Gleichung
$$
\ad(Z)[X,Y] = [\ad(Z)X, Y] + [X,\ad(Z)Y] \ .
$$
Man sagt $\ad(Z)$ ist eine {\it Derivation} von $\g$, d.h.
$$
\ad(Z) \in \Der(\g) = \{\psi : \g \rightarrow \g \; \mbox{linear und } \; \psi([X,Y]) = [\psi(X), Y] + [X,\psi(Y)]  \}
$$
\item
$\Der(\g) = \Lie(\Aut(\g))$
\item
Man hat das folgende kommutative Diagramm

\begin{equation*}
\begin{CD}
\mathfrak{g} @> \ad  >> \Der (\g) \subset \End(\g) \\
@V\exp VV @VV {\exp}V \\
G @>\Ad >> \Aut (\g) \subset \mathrm{GL}(\g)
\end{CD}
\end{equation*}


\end{enumerate}

\bigskip

Das {\it Zentrum} $Z(G) \subset G $ von $G$ bzw. $Z(\g) \subset \g$ ist definiert als
$$
Z(G) = \{ g \in G \,|\, gh = hg \; \mbox{f\"ur alle} \,  h \in G\}
\quad
Z(\g) = \{ X \in \g \,|\, [X,Y]=0 \; \mbox{f\"ur alle}  \,Y \in \g\}
$$

\bigskip

{\bf Beispiel:}
$
Z(O(n)) = \Z_2,\quad Z(SU(n))= \Z_n, \quad Z(\GL(n,\R)) = \{ c E \,|\, c \neq 0\}
$

\bigskip

\begin{Satz}
Das Zentrum $Z(G)\subset G$ ist eine abgeschlossene Lie-Untergruppe und es gilt
$Z(G) \subset \ker \Ad$. F\"ur zusammenh\"angende Gruppen $G$ gilt
$Z(G) = \ker \Ad$ und au\ss erdem $\Lie \, Z(G) = Z(\g)$.
\end{Satz}
\proof
Offensichtlich gilt $g\in Z(G)$ genau dann, wenn $\alpha_g = \Id_G$. Somit
folgt unmittelbar $Z(G) \subset \ker \Ad$, denn $\Ad(g) = d\alpha_g$ und
$d\Id_G= \Id_\g$.

\medskip

Sei nun $G$ zusammenh\"angend und $g\in \ker \Ad$. Da die Gruppe $G$ von einer
Normalenumgebung der Eins erzeugt wird, gen\"ugt es, die folgende Aussage zu
beweisen
$$
\alpha_g(\exp tX) = \exp tX \qquad \mbox{f\"ur alle}\quad X\in \g, t \in \R \ .
$$
Betrachtet man nun den Homomorphismus $\R\rightarrow G, \;t\mapsto \alpha_g(\exp tX)$,
dann existiert nach Folgerung~\ref{stetig} ein $Y\in \g$ mit
$\exp tY = \alpha_g(\exp tX)$ f\"ur alle $t\in \R$. Bildet man die Ableitung
in $t=0$ und nutzt $g \in \ker \Ad$, also $\Ad(g) = \Id_\g$, so findet man
$$
Y_e = \left. \tfrac{d}{dt}\right|_{t=0} \alpha_g(\exp tX) = d\alpha_g(X_e) = \Ad(g)(X_e)=X_e \ .
$$
Also $X=Y$ und $\alpha_g= \Id$, d.h. $g\in Z(G)$.

\medskip

Die Lie-Algebra des Zentrums $Z(G)$ berechnet man mit Hilfe von Lemma~\ref{kern}:
$$
\Lie(Z(G))=\Lie(\ker \Ad) = \ker(d\Ad) = \ker(\ad) = Z(\g) \ .
$$
\qed

\bigskip

\begin{Folgerung}
Eine zusammenh\"angende Lie-Gruppe ist genau dann abelsch, wenn ihre Lie-Algebra abelsch ist.
\end{Folgerung}

\begin{Folgerung}
Seien $X,Y \in \g$  mit $[X,Y]=0$, dann gilt
$$
\exp(X+Y)= \exp X \cdot \exp Y \ .
$$
\end{Folgerung}
\proof
Sei $\mathfrak{a} := \mathrm{span} \{X, Y\} \subset \g$, dann ist $\mathfrak{a}$ eine
abelsche Unteralgebra von $\g$. Sei $A\subset G$ die zusammenh\"angende Untergruppe mit
Lie-Algebra $\mathfrak{a}$. Dann ist nach der letzten Folgerung $A$ eine abelsche
Lie-Gruppe. Man definiert $\alpha(t):= \exp tX \cdot \exp tY$, dann ist, da $A$ abelsch ist,
$\alpha : \R \rightarrow G$ ein glatter Homomorphismus. Nach Folgerung~\ref{stetig}
existiert ein $Z\in \g$ mit $\alpha(t) = \exp tZ$, wobei $Z= \dot \alpha(0)$. Berechnet
man die Ableitung von $\alpha(t)= \exp tX \cdot \exp tY$ in $t=0$, so erh\"alt man
andererseits $\dot \alpha(0) = X + Y$. Insgesamt folgt
$$
\exp tX \cdot \exp tY = \alpha (t) = \exp t(X+Y) \ .
$$
\qed

\bigskip

{\bf Bemerkung:}
Ist $\g$ eine Lie-Algebra mit $Z(\g)= \{0\}$, dann besitzt $\g$ eine treue Darstellung
$$
\ad : \g \rightarrow \End(\g) \ ,
$$
d.h. $\g$ ist isomorph zu einer Lie-Algebra einer Lie-Gruppe. Das beweist das Theorem
von Ado im Spezialfall von Lie-Algebren mit trivialem Zentrum.

\bigskip

Eine weitere Anwendung der adjungierten Darstellung betrifft das Verh\"altnis von Normalteilern
und Idealen in Lie-Gruppen bzw. Lie-Algebren.

\medskip

\begin{Satz}
Sei $H \subset G$ eine zusammenh\"angende Lie-Untergruppe einer zusammenh\"angenden Lie-Gruppe
$G$. Dann ist $H$ genau dann ein Normalteiler in $G$, wenn $\h= \Lie(H)$ ein Ideal in
$\g = \Lie(G)$ ist.
\end{Satz}
\proof
Sei zuerst $\h\subset \g$ ein Ideal von $\g$, d.h. $[\h, \g] \subset \h$. Sei $Y\in \h, X\in \g$ und
$g= \exp X$. Dann gilt nach den oben angegebenen Vertauschungsregeln
$$
\begin{array}{rl}
g \cdot \exp Y \cdot g^{-1} &= \exp \Ad(g) Y \\[1.5ex]
&=
\exp e^{\ad X} Y \\[1.5ex]
&=
\exp (Y + [X, Y] + \tfrac12 (\ad X)^2Y + \ldots ) \\[1.5ex]
&=
\exp Z
\end{array}
$$
Da nach Voraussetzung alle Kommutatoren in $\h$ liegen, konvergiert die Reihe gegen
einen Vektor $Z\in \h$. Damit folgt $g \cdot \exp Y \cdot g^{-1}\in H$. Aber $H$ und
$G$ werden von einer Normalenumgebung erzeugt und es folgt, dass $H\subset G$
ein Normalteiler ist.

\medskip

F\"ur die umgekehrte Richtung sei $H\subset G$ ein Normalteiler und wieder
$Y \in \h, X \in \g$ und $g= \exp tX$. Wie oben folgt
$$
g\cdot \exp sY \cdot g^{-1} = \exp \Ad(g) s Y = \exp s e^{t\ad X} Y \ .
$$
Da $H$ ein Normalteiler ist, gilt f\"ur alle $s,t \in \R$ die Inklusion
$g\cdot \exp sY \cdot g^{-1} \subset H$. Das bedeutet wiederum
$e^{t\ad X}Y \in \h$ f\"ur alle $t\in \R$. Damit ist
$t \mapsto e^{t\ad X}Y = Y + t[X, Y] + \ldots $ eine glatte Kurve in
$\h$ mit Tangentialvektor $[X,Y]$ in $t=0$, also $[X,Y]\in \h$, womit
gezeigt ist, dass $\h \subset \g$ ein Ideal ist.
\qed


\bigskip


\subsection{Die Killing-Form}

\bigskip

\begin{Definition}
Sei $\g$ eine Lie-Algebra. Die Bilinearform
$$
B_\g : \g \times \g \rightarrow \R, \quad (X,Y) \mapsto \Tr(\ad(X) \circ  \ad(Y))
$$
hei\ss t {\em Killing-Form} von $\g$. Falls $\g$ die Lie-Algebra einer Lie-Gruppe
$G$ ist, dann nennt man $B_\g$ auch Killing-Form von $G$.
\end{Definition}

\bigskip

\begin{Satz}
Sei $G$ eine Lie-Gruppe mit Lie-Algebra $\g$ und Killing-Form $B_\g$. Sei
$\sigma : \g \rightarrow \g$ ein Lie-Algebren Isomorphismus. Dann gilt
$$
B_\g (\sigma(X), \sigma(Y)) = B_\g(X,Y)
$$
f\"ur alle $X,Y \in \g$. Insbesondere gilt f\"ur die adjungierten Darstellungen
$$
\begin{array}{rl}
& B_\g (\Ad(g)X, \Ad(g)Y) \;=\; B_\g(X,Y)\\[1.5ex]
& B_\g(\ad(X)Y, Z) + B_\g(Y, \ad(X)Z) \;=\; 0
\end{array}
$$
f\"ur alle $g\in G$ und alle $X,Y,Z\in \g$.
\end{Satz}
\proof
Sei $\sigma$ ein Automorphismus von $\g$, d.h. $\sigma ([X,Y])= [\sigma(X), \sigma(Y)]$,
oder anders geschrieben: $\sigma \circ \ad(X) = \ad(\sigma(X))\circ \sigma $, also
$$
\ad(\sigma X ) = \sigma \circ \ad(X) \circ \sigma^{-1} \ .
$$
Diese Beziehung setzt man nun in die Definition der Killing-Form ein und erh\"alt
$$
\begin{array}{rl}
B_\g(\sigma X, \sigma Y) &= \Tr (\ad(\sigma X) \circ \ad (\sigma Y))
= \Tr (\sigma \circ \ad(X) \circ \ad(Y) \circ \sigma^{-1}) \\[1.5ex]
&= \Tr (\ad(X) \circ \ad(Y)) \\[1.5ex]
&= B_\g(X,Y) \ .
\end{array}
$$
Die Killing-Form ist also invariant unter Automorphismen von $\g$ und die
beiden restlichen Aussagen ergeben sich aus dem Spezialfall $\sigma = \Ad$
bzw. Satz~\ref{skalar}.
\qed

\bigskip

{\bf Beispiele:}
\begin{enumerate}
\item
Ist $\g$ abelsch, dann gilt $B_\g \equiv 0$.
\item
F\"ur $\g = \gl(n,\R)$ ist $B_\g(X,Y) = 2n \Tr(X\cdot Y) - 2\Tr(X) \, \Tr(Y)$.
\item
Ist $\h \subset \g$ ein Ideal, d.h. gilt $[\h,\g]\subset \h$ dann ist die Killing-Form von
$\h$ gleich der Einschr\"ankung der Killing-Form von $\g$ auf $\h$:
$$
B_h = \left.B_\g\right|_{\h \times \h} \ .
$$
\item
Die Lie-Algebra $\sl(n,\R) = \{ X \in \gl(n,\R) \, | \, \Tr(X)=0 \}$ ist ein Ideal in $\gl(n,\R)$.
Es folgt f\"ur die Killing-Form von $\sl(n,\R)$:
$$
B_{\sl(n,\R)}(X,Y) = 2n \, \Tr (X\, Y) \ .
$$
\item
Die Lie-Algebra $\so(n) = \{ X \in \gl(n,\R) \, |\, X + X^T = 0\}$ ist kein Ideal in $\gl(n,\R)$.
F\"ur die Killing-Form von $\so(n)$ findet man:
$$
B_{\so(n)} (X, Y) = (n-2)\, \Tr(X\,Y ) \ .
$$
\end{enumerate}


\bigskip

{\bf Bemerkung:} Lie-Algebren mit nicht-ausgearteter Killing-Form, z.B. $\sl(n,R)$ oder $\so(n)$
nennt man {\it halb-einfach}. \"Aquivalent dazu ist, dass halb-einfache Algebren direkte Summe
von einfachen Lie-Algebren sind, also Algebren, die nur triviale Ideale besitzen. Halbeinfache
Lie-Algebren sind vollst\"andig klassifiziert.

\bigskip

\begin{Satz}
Sei $G$ eine kompakte Lie-Gruppe. Dann ist ihre Killing-Form negativ semi-definit.
\end{Satz}
\proof
Da $G$ eine kompakte Lie-Gruppe ist, existiert, wie schon bemerkt, auf der Lie-Algebra
$\g = \Lie(G)$ ein $\Ad$-invariantes Skalarprodukt $\la \cdot, \cdot\ra$ und nach
Satz~\ref{skalar} folgt
$$
\la \ad (X)Y, Z \ra +  \la Y, \ad (X)Z \ra = 0 \ .
$$
Sei $e_1, \ldots, e_n$ eine Orthonormal-Basis bzgl. $\la \cdot, \cdot \ra$
in $\g$ und sei $X_{ij}:= \la \ad(X)e_i, e_j \ra$. Die Matrix $(X_{ij})$
ist schiefsymmetrisch und man erh\"alt
$$
B_\g(X,X) = \Tr(\ad(X) \circ \ad(X)) = \Tr((X_{ij})\cdot (X_{kl}))
= \sum_{k,j} X_{kj}\cdot X_{jk} = - \sum_{k,j} X_{kj}^2 \le 0 \ .
$$

\qed

\section{Transformationsgruppen und homogene Mannigfaltigkeiten}

\begin{Definition}
Eine Lie-Gruppe $G$ {\em wirkt} (oder auch {\em operiert})) von links
auf einer Mannigfaltigkeit $M$, falls eine glatte Abbildung
$$
\Phi : G \times M \rightarrow M, \qquad (g,p) \mapsto g\cdot p := \Phi(g,p)
$$
existiert, die die folgenden Eigenschaften hat:
\begin{enumerate}
\item
F\"ur alle $p \in M$ und $g, h \in G$ gilt: \qquad $(g\cdot h)\cdot p = g \cdot (h \cdot p)$
\item
F\"ur alle $p \in M$ gilt: \qquad $ e \cdot p = p$
\end{enumerate}
\end{Definition}

\bigskip

{\bf Bemerkung:} Es folgt sofort, dass die Links-Translation $l_g: M \rightarrow M, p\mapsto g\cdot p$
ein Diffeomorphismus ist: $l_g^{-1}= l_{g^{-1}}$. Manchmal ist dies auch Teil der Definition einer
Gruppenwirkung von $G$ auf $M$.

\bigskip

\begin{Definition}
Sei $G$ eine Lie-Gruppe, die von links auf einer Mannigfaltigkeit $M$ wirkt. Dann definiert man:
\begin{enumerate}
\item
Die {\em Bahn} (auch {\em Orbit}) von $G$ durch $p\in M$:   \qquad $\mathcal O_p := \{g\cdot p \,|\, g\in G\}$.
\item
Die {\em Standgruppe} (auch Isotropiegruppe oder Stabilisator) von $p\in M$:
\quad $G_p := \{ g \in G \,|\, g\cdot p = p\}$.
\item
$G$ wirkt {\em transitiv} auf $M$, wenn f\"ur alle Punkte $p, q \in M$ ein Gruppenelement $g\in G$ existiert
mit $q = g \cdot p$. \"Aquivalent dazu ist, dass $M$ aus einer Bahn besteht: $M = \mathcal O_p$ f\"ur ein und
damit f\"ur alle $p\in M$.
\item
$G$ wirkt {\em frei} auf $M$, wenn f\"ur alle $g\in G, g\neq e$ die Links-Translation $l_g$ fixpunktfrei ist.
\item
$G$ wirkt {\em effektiv} auf $M$, wenn $l_g= \Id_M$ nur f\"ur $g=e$ erf\"ullt ist.
\end{enumerate}
\end{Definition}


\bigskip

{\bf Bemerkungen:}
\begin{enumerate}
\item
Eine Lie-Gruppe $G$ wirkt durch Gruppen-Multiplikation auf sich. Diese Wirkung ist transitiv.
\item
Eine Gruppenwirkung auf $M$ ist eine Darstellung
$
\Phi : G \rightarrow \Diff(M), \; g \mapsto l_g \ ,
$
wobei $\Diff(M)$ die Gruppe der Diffeomorphismen von $M$ bezeichnet. Eine effektive Wirkung entspricht dann einer
treuen Darstellung.
\end{enumerate}

\bigskip

\begin{Definition}
Eine Mannigfaltigkeit $M$, auf der eine Lie-Gruppe $G$ transitiv wirkt nennt man {\em homogen}. Man sagt auch
$M$ ist ein homogener Raum.
\end{Definition}

\bigskip

{\bf Beispiele:}
\begin{enumerate}
\item
Jede $G$-Darstellung auf einem Vektorraum $V$ definiert eine Wirkung von $G$ auf $V$.
\item
$\GL(n,\R)$ wirkt auf $\R^n$: $(A,v)\mapsto A\cdot v$. Diese Wirkung ist transitiv auf
$\R^n \setminus \{0\}$.
\item
$\O(n)$ wirkt transitiv auf $S^{n-1} \subset \R^n$ mit der Standgruppe $\O(n)_{e_1}\cong \O(n-1)$,
dabei ist $e_1$ der erste Vektor der kanonischen Basis von $\R^n$.
\end{enumerate}


\bigskip

Im Weiteren soll gezeigt werden, dass der Raum der Nebenklassen $G/H$ eine homogene Mannigfaltigkeit ist
und das jeder homogene Raum sich so darstellen l\"a\ss t. Sei $G$ eine Lie-Gruppe und $H\subset G$
eine abgeschlossene Untergruppe. Auf $G$ definiert man die \"Aquivalenzrelation: $g_1 \sim g_2
\leftrightarrow g_1^{-1} g_2\in H$. Die \"Aquivalenzklassen der Relation $\sim$ sind  die Nebenklassen
$$
[g] = \{ gh \,|\, h \in H \} = gH \subset G \ .
$$
Der {\it Faktorraum} $G/H$ ist dann der Raum der Nebenklassen: $G/H = \cup_{g\in G}[g] = G/_\sim$. Die
Topologie auf $G/H$ wird so definiert, dass genau die Mengen offen sind, deren Urbild unter $\pi$
offen in $G$ sind. Die nat\"urliche Projektion
$$
\pi : G \rightarrow G/H,\quad g \mapsto gH
$$
wird damit eine (stetige) offene Abbildung wird.

\begin{Satz}\label{faktor}
Sei $H \subset G$ eine abgeschlossene Untergruppe einer Lie-Gruppe $G$. Dann existiert auf dem
Faktorraum $G/H$ die Struktur einer differenzierbaren Mannigfaltigkeit mit
\begin{enumerate}
\item
Die nat\"urliche Projektion $\pi: G \rightarrow G/H$ ist glatt.
\item
Der Faktorraum $G/H$ ist ein homogener Raum bzgl. der $G$-Wirkung
$$
G \times G/H \rightarrow G/H, \quad (g, [a]) \mapsto [ga] \ .
$$
\item
Zu jeder \"Aquivalenzklasse $[a] \in G/H$ existiert eine Umgebung $W([a])\subset G/H$ und eine
glatte Abbildung $s_{[a]}: W([a])\rightarrow G$ mit
$$
\pi \circ s_{[a]} = \Id_{W([a])} \ .
$$
Man sagt die Abbildungen $s_{[a]}$  sind lokale Schnitte der Faserung $\pi: G\rightarrow G/H$.
\end{enumerate}
\end{Satz}
{\bf zum Beweis:} Hier sollen nur einige Anmerkungen zum Beweis gemacht werden. Einen
vollst\"andigen Beweis findet man bei H. Baum oder F. Warner.

\medskip

Auf $G/H$ betrachtet man die Faktorraum-Topologie, damit wird $G/H$ ein Hausdorff-Raum mit
abz\"ahlbarer Topologie. Es soll nun erkl\"art werden, wie man Karten um jeden Punkt
$[g]\in G/H$ konstruiert. Dazu sei $\g= \h \oplus \m$, wobei $\m$ ein algebraisches
Komplement von $\h= \Lie(H)$ ist. Sei $\Phi$ definiert durch
$$
\Phi : \g = \h \oplus \m \rightarrow G, \quad X + Y \mapsto \exp X \cdot \exp Y \ .
$$
Wie schon gezeigt, ist $\Phi$ ein Diffeomorphismus auf einer Umgebung
$W = W_\m \times W_\h$ von $0\in \g$, dabei ist $W_\m \subset \m$ und
$W_h \subset \h$ jeweils eine Umgebung der Null in $\m$ bzw. $\h$. Man kann nun zeigen,
dass man nach geeigneter Verkleinerung von $W$ folgende Bedingungen erf\"ullen kann:
$$
\exp(W_\m \setminus \{0\}) \cap H = \emptyset,
\qquad
\exp W \cdot (\exp W)^{-1} \subset \Phi(W) \ .
$$
Damit zeigt man, dass die Abbildung
$$
\varphi_{[g]}:= \pi \circ l_g \circ \exp: W_\m \rightarrow G \rightarrow G \rightarrow G/H
$$
ein Hom\"oomorphismus auf das Bild ist. Als Karte um $[g]\in G/H$ w\"ahlt man nun
$(\varphi_{[g]}(W_\m), \varphi_{[g]}^{-1})$. Die Kartenwechsel sind glatt, denn man
kann leicht \"uberpr\"ufen, dass
$$
\varphi^{-1}_{[a]} \circ \varphi_{[b]}(X) = \pr_\m \circ \Phi^{-1}\circ l_{a^{-1}b}\circ \exp X
$$
f\"ur alle $X\in W_\m$ gilt.

\medskip

Um die Glattheit von $\pi : G \rightarrow G/H$ zu zeigen, betrachtet man Karten
$(l_g\Phi(W), (l_g \circ \Phi)^{-1})$ um $g\in G$ und $(\varphi_{[g]}(W_\m), \varphi^{-1}_{[g]})$
um $[g]\in G/H$. Damit schreibt sich $\pi$ in lokalen Koordinaten als
$$
\varphi^{-1}_{[g]}\circ \pi \circ l_g \circ \Phi(X)
=
\varphi^{-1}_{[g]} ([g\cdot \exp(X_\m)\cdot \exp(X_\h)])
=
\varphi^{-1}_{[g]}([g\cdot \exp (X_\m)])
=
X_\m = \pr_\m(X) \ ,
$$
f\"ur alle $X \in l_g \, \Phi(W)$. In lokalen Koordinaten ist $\pi $ also genau die
Projektion auf $\m$ und damit glatt. Insbesondere sieht man auch, dass $\pi$ eine
Submersion ist.

\medskip

Zum Schlu\ss{} m\"ussen noch die lokalen Schnitte $s_{[g]}$ auf $W([g])= \varphi_{[g]}(W_\m)$
definiert werden. Man setzt
$$
s_{[g]} := l_g \circ \exp \circ \varphi^{-1}_{[g]} \ .
$$
In lokalen Koordinaten ist $s_{[g]}$ gegeben als $X \mapsto (X,0)$ bzgl. der Zerlegung
$\g = \m \oplus \h$, d.h. die Abbildung $s_{[g]}$ ist glatt. Nun bleibt noch zu zeigen,
dass $s_{[g]}$ ein lokaler Schnitt ist, also die Beziehung  $\pi \circ s_{[g]} = \Id$
erf\"ullt ist. Dazu sei $[a]\in W([g])$, dann existiert ein eindeutig bestimmtes
$X_\m \in W_\m$ mit
$$
[a] = \varphi_{[g]}(X_\m)= \pi \, l_g \, \exp (X_\m)
=
\pi \, l_g \, \exp \, \varphi^{-1}_{[g]} [a]
=
\pi \circ s_{[g]}([a]) \ .
$$

\qed

\bigskip

{\bf Bemerkungen:}
\begin{enumerate}
\item
Die nat\"urliche Projektion $\pi : G \rightarrow G/H$ ist eine Submersion.
\item
$\dim G/H = \dim G - \dim H$
\item
$\ker d\pi = \h \qquad \mbox{und} \qquad T_{[e]}G/H = d\varphi_{[e]}(\m)\cong \m$
\item
Eine Abbildung $f:G/H\rightarrow N$ ist genau dann differenzierbar, wenn $f\circ \pi : G \rightarrow N$
differenzierbar ist. Zur Begr\"undung bemerkt man, dass sich $f$ eingeschr\"ankt auf die Umgebung $W([g])$
schreibt als
$$
\left. f \right|_{W([g])} = f \circ \pi \circ s_{[g]} \ .
$$
\item
Sei $\tau_g: G/H \rightarrow G/H$ die $G$-Wirkung $\tau_g([a])= g[a]= [ga]$. Dann gilt
$\pi \circ l_g = \tau_g \circ \pi$, d.h. man hat das folgende kommutative Diagramm
\begin{equation*}
\begin{CD}
G @> l_g >> G \\
@V \pi  VV @VV \pi V \\
G/H @> \tau_g >> G/H
\end{CD}
\end{equation*}
\end{enumerate}

\bigskip

Es gilt nun auch die Umkehrung von Satz~\ref{faktor}, d.h. jede homogene Mannigfaltigkeit
ist von der Form $G/H$ f\"ur geeignete Lie-Gruppen $G$ und $H$.

\bigskip

\begin{Satz}
Sei $G$ eine Lie-Gruppe, die transitiv von links auf einer Mannigfaltigkeit $M$ wirkt. F\"ur jedes
$p\in M$ ist die Abbildung
$$
\Psi : G/G_p \rightarrow M, \quad [a] \mapsto a \cdot p
$$
ein $G$-\"aquivarianter Diffeomorphismus, d.h. es gilt f\"ur alle $g\in G$ und $[a]\in G/G_p$
die Gleichung
$$
\Psi(g \cdot [a]) = g \cdot \Psi([a]) \ .
$$
Insbesondere ist die Bahn $\mathcal O_p$ diffeomorph zu $G/G_p$ und jeder $G$-homogene Raum $M$
schreibt sich als $M=G/H$ f\"ur eine abgeschlossene Lie-Untergruppe $H \subset G$.
\end{Satz}
\proof
Zun\"achst zeigt man, dass $\Psi$ wohldefiniert ist. Denn aus $[a] = [b]$ folgt $a^{-1} b \in G_p$
und somit $a\cdot p = b \cdot p $. Die \"Aquivarianz folgt direkt aus der Definition von $\Psi$
bzw. der $G$-Wirkung auf $G/G_p$.

\medskip

Sei $\beta : = \Psi \circ \pi$, d.h. $\beta: G \rightarrow M, \beta(g) = g \cdot p$. Die Abbildung
$\beta$ ist offensichtlich glatt, als Teil der $G$-Wirkung auf $M$ und nach Bemerkung 4 ist damit
auch $\Phi$ glatt.

\medskip

Es bleibt zu zeigen, dass $d\Psi : T_{[a]}G/G_p \rightarrow  T_{\Psi([a])}M$ ein Isomorphismus ist.
Daf\"ur gen\"ugt es, $d\Psi$ in $[e]$ zu berechnen, denn
$d\Psi \circ dl_a = dl_a \circ d\Psi$. Der Tangentialraum an $G/G_p$ in $[e]$ ist gegeben als
$$
T_{[e]}G/G_p = d\varphi_{[e]}(\m) \ .
$$
Sei $X\in \m$ dann berechnet man
$$
d\Psi (d\varphi_{[e]}(X)) = d(\Psi \circ \pi \circ \exp)(X) = d(\beta \circ \exp)X = d\beta X
$$
Der Tangentialvektor $d\varphi_{[e]}(X)$ liegt also genau dann im Kern von $d\Psi$, wenn
$X \in \ker d\beta$ also, wie man leicht sieht, wenn $X\in \Lie(G_p) = \h$. Damit liegt aber
$X$ im Durchschnitt $\h \cap \m$ und mu\ss{} damit verschwinden. Man hat also gezeigt, dass
$d\Psi$ injektiv ist. Da aber $\dim G/G_p = \dim \m$ gilt, ist $d\Psi$ auch surjektiv und
somit ein Isomorphismus.
\qed


\bigskip

\subsection{Beispiele homogener Mannigfaltigkeiten}

\bigskip

\begin{enumerate}
\item
Jede Lie-Gruppe $G$ ist ein homogener Raum: $G=G \times G/G = G/\{e\}$.
\item
Die Gruppe $\SO(n+1)$ wirkt durch
Matrizenmultiplikation transitiv auf der Einheitssph\"are
$M = S^n \subset \R^{n+1}$. Die Standgruppe von $e_1$
(und von jedem anderen Punkt auf der Sph\"are) ist isomorph
zu $\SO(n)$. Damit hat man folgende Darstellung von
$S^n$ als homogenen Raum:
$$
S^n = \SO(n+1)/\SO(n) = \O(n+1)/\O(n) \ .
$$
\item
Auf der Einheitssph\"are $S^{2n+1}\subset \C^{n+1}$ wirkt $\SU(n+1)$ transitiv. Die
Standgruppe von $e_1$ (und von jedem anderen Punkt) ist isomorph zu
$\SU(n)$. Damit hat die Sph\"are $S^{2n+1}$ eine weitere M\"oglichkeit
als homogener Raum geschrieben zu werden:
$$
S^{2n+1} = \SU(n+1)/\SU(n) = \U(n+1)/\U(n) \ .
$$
\item
Auf der Einheitssph\"are $S^{4n+3}\subset \H^{n+1}$ wirkt $\Sp(n+1)$ transitiv. Die
Standgruppe von $e_1$ (und von jedem anderen Punkt) ist isomorph zu
$\Sp(n)$. Damit hat die Sph\"are $S^{4n+3}$ eine weitere M\"oglichkeit
als homogener Raum geschrieben zu werden:
$$
S^{4n+3} = \Sp(n+1)/\Sp(n)  \ .
$$
\item
Die (effektiven) transitiven Wirkungen von Lie-Gruppen auf den Sph\"aren wurden von
Montgomery/Samelson bzw. Borel klassifiziert. Es sind die oben genannten Beispiele
und
$$
S^6 = \G_2/\SU(3),\qquad S^7 = \Spin(7)/\G_2,\qquad S^{15} = \Spin(9)/\Spin(7) \ .
$$
Dabei ist $\G_2$ die 14-dimensionale exzeptionelle einfache Lie-Gruppe und $\Spin(n)$
bezeichnet die Spin-Gruppe, eine zweifache \"Uberlagerung der speziellen orthogonalen
Gruppe.
\item
Die einzigen Sph\"aren mit der Struktur einer Lie-Gruppe sind
$$
S^1 = \SO(2) = \U(1), \qquad S^3 = \Sp(1) = \SU(2) = \Spin(3) \ .
$$
\item
Auf dem komplex projektiven Raum $\C P^n = (\C^{n+1}\setminus \{0\})/_\sim$, wobei $\sim$
definiert ist durch $z \sim \lambda z,
\lambda \in \C \setminus \{0\}, z \in \C^{n+1}$ wirkt die Gruppe $\SU(n+1)$ transitiv,
durch $A [z] = [Az]$. Die
Standgruppe von $[1:0:\ldots :0]$ ist isomorph zu der Gruppe
$S(\U(1) \times \U(n)) = \{(\lambda, A)\,|\, \lambda \in \U(1), A \in \U(n), \lambda \det A = 1\}$.
Der komplex projektive Raum hat daher folgende Darstellung als homogener Raum
$$
\C P^n = \SU(n+1)/S(\U(1) \times \U(n)) = \U(n+1)/ \U(1) \times \U(n) \ .
$$
Die Gruppe $\SU(n+1)$ wirkt nicht effektiv. Der Nichteffektivit\"atskern, also die Menge der
Gruppenelemente, die als Identit\"at wirken, berechnet sich als
$$
\Z_{n+1} = Z(\SU(n+1)) = \{\diag(\lambda, \ldots, \lambda) \,|\, \lambda^{n+1} = 1\} \ .
$$
Damit wirkt die projektive unit\"are Gruppe $PSU(n+1) = \SU(n+1)/\Z_{n+1}$ effektiv auf
dem komplex projektiven Raum.
\item
Folgende $S^1$- bzw. $S^3$-Faserungen werden als Hopf-Faserungen bezeichnet:
$$
\begin{array}{lll}
S^{2n+1} & =  \U(n+1)/\U(n)  & \longrightarrow \quad \C P^n = \U(n+1)/\U(1) \times  \U(n) \ , \\[1.5ex]
S^{4n+3} & =  \Sp(n+1)/\Sp(n) & \longrightarrow \quad \H P^n = \Sp(n+1)/\Sp(1) \times  \Sp(n) \ .
\end{array}
$$
\end{enumerate}

\bigskip

Sei $G= \U(n)$ die Gruppe der unit\"aren Matrizen mit der Lie-Algebra
$\g = \u(n) = \{ A \in M(n, \C)\, |\, A + \bar A^T =0\} \cong \R^{n^2}$. Die Gruppe
$G$ wirkt durch die adjungierte Darstellung auf ihrer Lie-Algebra, d.h. f\"ur
Matrizen-Gruppen durch
$$
G \times \g \rightarrow \g,\quad (g, A) \mapsto \Ad(g)A = g\,A\,g^{-1} \ .
$$
Matrizen in $\u(n)$ sind schief-hermitesch und insbesondere normal, lassen sich also
diagonalisieren. Sei $A \in \u(n)$, dann existiert ein $g\in \U(n)$ mit
$g\,A\,g^{-1}= \diag(\lambda_1,\ldots,\lambda_n)$, d.h. der Orbit $\mathcal O_A$
durch $A$ enth\"alt die Matrix $D:=\diag(\lambda_1,\ldots,\lambda_n)$ und besteht aus der Menge
aller schief-hermiteschen Matrizen mit den Eigenwerten $\lambda_1,\ldots,\lambda_n$.
Sei $G_D$ die Standgruppe der Diagonalmatrix $D$, dann findet man
$$
G_D = \U(r_1) \times \ldots \times \U(r_k)
$$
mit $r_1 + \ldots + r_k = n$. Damit hat man bewiesen

\begin{Satz}
Sei $A \in \u(n)$ mit den Eigenwerten $\lambda_1,\ldots, \lambda_k$ mit den geometrischen
Vielfachheiten $r_1,\ldots, r_k$. Dann gilt f\"ur den Orbit durch $A$:
$$
\mathcal O_A = \U(n)/\U(r_1) \times \ldots \times \U(r_k) \ .
$$
\end{Satz}

\bigskip

\section{Differentialformen}

Sei $V$ ein $n$-dimensionaler reeller Vektorraum und sei $V^* = \Hom(V, \R)$
der Dualraum von $V$, d.h. der Raum der linearen Abbildungen $V\rightarrow \R$.
Sp\"ater wird $V$ ersetzt durch $T_pM$.

\begin{Definition}
Eine {\em $k$-Form} auf $\;V$ ist eine multilineare, alternierende Abbildung
$$
\omega : V \times \ldots_{\mbox{k mal}}\ldots \times V \longrightarrow \R \ .
$$
Die Zahl $k$ nennt man die Grad der $k$-Form $\omega$.
\end{Definition}

\bigskip

Dabei ist $\omega$ {\it alternierend} genau dann, wenn eine der folgenden vier \"aquivalenten
Bedingungen erf\"ullt ist:
\begin{enumerate}
\item
$\omega(X_{\sigma(1)}, \ldots, X_{\sigma(k)}) = \sgn(\sigma) \,\omega (X_1,
\ldots, X_k) \qquad \forall \; \sigma \in S_k$
\item
$\omega( \ldots, X_i, \ldots, X_j, \ldots) = - \, \omega (\ldots, X_j, \ldots, X_i, \ldots)$
\item
$\omega (X_1, \ldots, X_k ) = 0 \qquad \mbox{falls es i, j gibt mit }\quad i \neq j \;\mbox{und}\; X_i = X_j$
\item
$\omega (X_1, \ldots, X_k ) = 0 \qquad \mbox{falls} \quad X_1, \ldots, X_k \quad \mbox{linear abh\"angig sind.}$
\end{enumerate}

\bigskip

{\bf Bemerkung:} Die Menge der $k$-Formen auf $V$ bildet einen reellen Vektorraum, der bezeichnet wird
mit
$$
\Lambda^k V^* \ .
$$
Man setzt $\Lambda^0 V^* = \R$ und f\"ur $k >n$ gilt offensichtlich $\Lambda^k V^* = \{0\}$, falls $V$ ein
$n$-dimensionalen Vektorraum ist.
Denn in $V$ sind dann $k$ Vektoren f\"ur $k>n$ immer linear abh\"angig.
Addition und skalare Multiplikation sind definiert durch
$$
\begin{array}{rl}
(\omega_1 + \omega_2)(X_1, \ldots, X_k) &= \quad \omega_1(X_1, \ldots, X_k) + \omega_2(X_1, \ldots, X_k) \ ,
\\[1.5ex]
(\lambda \omega)(X_1, \ldots, X_k) & = \quad \lambda \cdot \omega(X_1, \ldots, X_k) \ .
\end{array}
$$

\bigskip

{\bf Beispiele:}

\begin{enumerate}
\item
{\it 1-Formen:}
$\Lambda^1 V^* = V^* = \Hom(V, \R)$.
Insbesondere gilt $\dim \Lambda^1 V^* = n$, falls $V$ ein $n$-dimensionaler Vektorraum ist.
\item
{\it 2-Formen:}
$\Lambda^2 V^* =$ Menge der schief-symmetrischen bilinearen Abbildungen $\omega : V \times V \rightarrow \R$,
d.h. es gilt $\omega(X, Y) = -\omega (Y,X)$ f\"ur alle Vektoren $X,Y\in V$.

\begin{Lemma}
$
\qquad \Lambda^2 V^* \cong \so(V) \cong \{ A \in M(n,\R) \,|\, A^T = -A\}
$
\end{Lemma}
Beweis:
Sei $\la \cdot , \cdot \ra$ ein Skalarprodukt auf $V$. Man definiert die Identifikation durch
$\Lambda^2V^* \rightarrow \End (V), \omega \mapsto \hat \omega$ mit
$ \la \hat \omega(X), Y \ra = \omega(X, Y) $.
\qed

\begin{Folgerung}
\qquad $ \dim \Lambda^2 V^* = { n \choose 2}$
\end{Folgerung}

\item

Sei $V$ ein $n$-dimensionaler Vektorraum und sei $\omega$ eine $n$-Form auf $V$. Sei
$e_1, \ldots, e_n$ eine fixierte Basis in $V$, bzgl. der man Vektoren $v_1, \ldots, v_n \in V$
als Spaltenvektoren betrachtet. Dann gilt:
$$
\omega (v_1, \ldots, v_n) = \det (v_1, \ldots, v_n)\, \omega(e_1, \ldots, e_n)
$$
Damit folgt $\Lambda^n V^* = \R \cdot \det \cong \R$.
\end{enumerate}

\bigskip

\begin{Definition}
Die {\em \"au\ss ere Algebra} von $V$ ist definiert als $\; \Lambda(V^*) := \oplus^n_{k=0} \, \Lambda^k V^*$, d.h. als
die Menge der Formen von beliebigem Grad auf $V$. Die Vektorraumstruktur kommt von den einzelnen Summanden
$\Lambda^k V^*$, die Produktstruktur ist definiert durch das {\em Dachprodukt}:
$$
\wedge : \Lambda^k V^* \times \Lambda^l V^* \longrightarrow \Lambda^{k+l} V^*, \quad
(\alpha, \beta) \mapsto \alpha \wedge \beta
$$
dabei ist $\wedge $ definiert durch
$$
(\alpha \wedge \beta)(X_1, \ldots, X_{k+l})
=
\frac{1}{k!\,l!}\,\sum_{\;\sigma \in S_{k+l}} \sgn(\sigma)\, \alpha(X_{\sigma(1)}, \ldots, X_{\sigma(k)})
\cdot \beta(X_{\sigma(k+1)}, \ldots, X_{\sigma(k+l)})
$$
\end{Definition}

\bigskip

{\bf Bemerkung:}
Der Raum der $(0,k)$-Tensoren auf $V$ ist definiert als der Vektorraum der $k$-fach multilinearen Abbildungen:
$$
T^{(0,k)} V^* := \{\lambda : V \times  \ldots_{\mbox{\;k mal\;}} \ldots\times  V \rightarrow \R \quad \mbox{multilinear}\} \ .
$$
Das {\it Tensorprodukt} auf der Tensoralgebra $T(V^*) = \oplus_k T^{(0,k)}(V^*)$ ist definiert durch
$$
(\lambda \otimes \mu)(X_1, \ldots, X_{k+l}) = \lambda(X_1, \ldots, X_k) \cdot \mu (X_{k+1}, \ldots, X_{k+l}) \ .
$$
Das Dach-Produkt ergibt sich als Projektion des Tensorproduktes auf den Unterraum
$\Lambda^k V^* \subset T^{(0,k)} V^*$. Man definiert die Abbildung
$\;\mathrm{Alt}: T^{(0,k)}V^* \rightarrow \Lambda^k V^* $ durch
$$
\mathrm{Alt}(\lambda)(X_1, \ldots, X_k )
=
\sum_{\sigma \in S_k} \sgn(\sigma) \, \lambda(X_{\sigma(1)}, \ldots, X_{\sigma(k)}) \ .
$$
F\"ur eine Bilinearform $\lambda \in T^{(0,2)}V^*$ gilt also:
$\quad
\mathrm{Alt}(\lambda)(X, Y) = \lambda(X, Y) - \lambda(Y,X) \ .
$
In dieser Notation kann man f\"ur das Dachprodukt von $\alpha \in \Lambda^k V^*$ und $\beta \in \Lambda^l V^*$
auch schreiben:
$$
\alpha \wedge \beta = \frac{1}{k!\,l!}\, \mathrm{Alt}(\alpha \otimes \beta) \ .
$$

\bigskip

{\bf Beispiele:}
\begin{enumerate}
\item
Seien $\alpha$ und $\beta$ zwei 1-Formen auf $V$ dann gilt:
$$
(\alpha \wedge \beta)(X, Y) = \alpha(X)\,\beta(Y) - \alpha(Y)\,\beta(X)
$$
\item
Sei $\alpha$ eine 1-Form und $\beta $ eine 2-Form auf $V$, dann gilt:
$$
\begin{array}{rl}
(\alpha \wedge \beta)(X,Y,Z) & = \,
\alpha(X)\,\beta (Y,Z) \,  -  \, \alpha(Y)\,\beta(X,Z)  \, +  \, \alpha(Z)\,\beta(X,Y)\\[1.5ex]
& =
\alpha(X)\,\beta(Y,Z)  \, + \,  \alpha(Y)\,\beta(Z,X) \,  +  \, \alpha(Z)\, \beta(X,Y) \ .
\end{array}
$$
Allgemeiner sei $\alpha$ eine 1-Form und $\beta$ eine k-Form, dann gilt
$$
(\alpha \wedge \beta)(X_0, \ldots, X_k) = \sum_{i=0}^k \, (-1)^i \,\alpha(X_i)\,\beta(X_0,\ldots, \widehat X_i, \ldots, X_k) \ .
$$
Dabei bedeutet $\widehat X_i$, dass der entsprechende Eintrag weggelassen wird.
\end{enumerate}

\bigskip

\begin{Lemma}
Das Dachprodukt von Formen hat folgende Eigenschaften:
\begin{enumerate}
\item\;
$
\alpha \wedge \beta = (-1)^{k\cdot l} \beta \wedge \alpha \qquad \mbox{f\"ur alle } \quad
\alpha \in \Lambda^k V^*, \beta \in \Lambda^l V^* \ ,
$
\item\;
$
\omega \wedge \omega = 0 \qquad \mbox{f\"ur alle Formen } \quad \omega \quad \mbox{von ungeradem Grad} ,
$
\item\;
$
(\lambda \alpha + \mu \beta) \wedge \gamma = \lambda \alpha \wedge \gamma + \mu \beta \wedge \gamma
\qquad \mbox{f\"ur alle} \quad \lambda, \mu \in \R, \;\alpha, \beta, \gamma \in \Lambda(V^*) \ ,
$
\item\;
$
(\alpha \wedge \beta ) \wedge \gamma = \alpha \wedge (\beta \wedge \gamma)
\qquad \mbox{f\"ur alle} \quad \alpha, \beta, \gamma \in \Lambda(V^*)
$
\end{enumerate}
\end{Lemma}

\bigskip

\subsection{Das Verhalten unter Abbildungen}

\bigskip

Sei $f: V \rightarrow W$ eine lineare Abbildung zwischen Vektorr\"aumen $V$ und $W$. Man definiert
$$
f^* = \Lambda^k(f) : \Lambda^k W^* \longrightarrow \Lambda^k  V^*,\quad
(f^*\omega)(v_1, \ldots, v_k) := \omega(f v_1, \ldots, f v_k) \ .
$$
f\"ur Vektoren $v_1, \ldots, v_k \in V$ und $\omega \in \Lambda^k W^*$. Man nennt die von
$f$ induzierte Abbildung $f^*$ das {\it Zur\"uckziehen} von Formen. Die Zuordnung
$V\mapsto \Lambda^k V^*, f\mapsto f^*$ definiert also einen kontravarianten Funktor
auf der Kategorie der Vektorr\"aume.

\begin{Lemma}
Das Zur\"uckziehen hat folgende Eigenschaften:
\begin{enumerate}
\item
$ (g\circ f)^* = f^* \circ g^* $
\item
$
f^*(\alpha \wedge \beta) = (f^*\alpha) \wedge (f^*\beta)
$
\item
$
\det(f) = f^* : \Lambda^n V^* \longrightarrow \Lambda^n V^*,
\qquad \mbox{falls}\quad \dim V = n
$
\end{enumerate}
\end{Lemma}

\bigskip

\subsection{Eine Basis im Raum der Formen}

\bigskip

Sei $e_1, \ldots, e_n$ eine Basis in $V$ und sei $e^1, \ldots, e^n$ die dazugeh\"orige duale
Basis in $V^*$, d.h. es gilt $e^i (e_j) = \delta_{ij}$, bzw. $e^i(v) = v_i $ falls
$v = \sum v_i e_i$. Dann ist
$$
e^{i_1} \wedge \ldots \wedge e^{j_k} \qquad \qquad 1 \le i_1 < \ldots < i_k \le n
$$
eine Basis in $\Lambda^k V^*$, d.h. zu jeder $k$-Form $\omega \in \Lambda^k V^*$ existieren
reelle Zahlen $\omega_{i_1, \ldots, i_k}$ mit
$$
\omega = \sum \omega_{i_1, \ldots, i_k} \,e^{i_1} \wedge \ldots \wedge e^{j_k} \ .
$$
Insbesondere gilt $\quad \dim \Lambda^k V^* = {n \choose k}$ und
$\quad \Lambda^k V^* \cong \Lambda^{n-k} V^*$.

\bigskip

\begin{Lemma}
Seien $\omega_1, \ldots, \omega_k \in V^*$ und $v_1, \ldots, v_k \in V$, dann gilt
$$
(\omega_1 \wedge \ldots \wedge \omega_k)(v_1, \ldots, v_k)
=
\det
\left(
\begin{array}{ccc}
\omega_1(v_1)  & \dots & \omega_k(v_1) \\
\vdots         & & \vdots\\
\omega_1(v_k)  & \dots & \omega_k(v_k)
\end{array}
\right) \ .
$$
Insbesondere gilt $\omega_1 \wedge \ldots \wedge \omega_k= 0$ falls die Linearformen
$\omega_1, \ldots, \omega_k$ in $V^*$ linear abh\"angig sind.
\end{Lemma}

\bigskip

{\bf Beispiel:}
Seien $\omega_1, \omega_2$ zwei 1-Formen auf $V$ und $X,Y\in V$. Dann gilt
$$
(\omega_1 \wedge \omega_2)(X,Y) =
\det
\left(
\begin{array}{cc}
\omega_1(X)  & \omega_2(X) \\
\omega_1(Y)  & \omega_2(Y)
\end{array}
\right)
=
\omega_1(X)\,\omega_2(Y) - \omega_1(Y)\,\omega_2(X)
\ .
$$

\bigskip

\subsection{Differentialformen auf Mannigfaltigkeiten}

\bigskip

Man \"ubertr\"agt nun die lineare Algebra der $k$-Formen punktweise auf Mannigfaltigkeiten.

\begin{Definition}
Sei $M$ eine Mannigfaltigkeit. Das B\"undel der $k$-Formen auf $M$ ist definiert als
$$
\Lambda^k (T^*M) \;=\; \bigcup_{p\in M} \, \Lambda^k T^*_p M
$$
wobei $\;T^*_p M = (T_pM)^*$. Die kanonische Projektion ist definiert als
$$
\pi : \Lambda^k (T^*M) \rightarrow M, \quad \pi(\omega) = p \qquad \mbox{falls}\quad \omega \in \Lambda^k T^*_pM \ ,
$$
d.h. die Faser \"uber $p \in M$ ist der Raum der $k$-Formen auf $\,T_pM$, also $\pi^{-1}(p) = \Lambda^k T^*_pM$.
\end{Definition}

\bigskip

{\bf Bemerkungen:}
Die Menge $\Lambda^k (T^*M)$ tr\"agt die Struktur einer differenzierbaren Mannigfaltigkeit (analog zu $TM$):
Sei $(U,x)$ eine Karte um $p\in M$. Dann definiert man
$$
\Phi: \pi^{-1}(U) \rightarrow x(U) \times \Lambda^k \R^n, \quad \omega \mapsto (x(\pi(\omega)),(dx^{-1})^*\omega) .
$$
Mit den Karten $(\pi^{-1}(U), \Phi)$ wird $\Lambda^k (T^*M)$ eine differenzierbare Mannigfaltigkeit der
Dimension $n + {n \choose k}$. Mit der kanonischen Projektion $\pi : \Lambda^k T^*M\rightarrow M$ erh\"alt man
ein reelles Vektorb\"undel vom Rang ${n \choose k}$ \"uber $M$.

\bigskip

\begin{Definition}
Eine Differentialform vom Grad $k$ auf $M$ ist eine differenzierbare Abbildung
$$
\omega : M \longrightarrow \Lambda^k T^*M \qquad \mbox{mit} \quad \pi \circ \omega = \Id_M \ ,
$$
d.h. $\omega$ ist ein Schnitt im Vektorb\"undel $\Lambda^k T^*M$. Den unendlich-dimensionalen
Vektorraum der $k$-Formen auf $M$ bezeichnet man mit $\Omega^k(M) = \Gamma(\Lambda^k T^*M)$.
Insbesondere gilt f\"ur den Raum der 0-Formen: $\;\Omega^0(M) = \mathcal C^\infty(M)$.
\end{Definition}

\bigskip

{\bf Beispiel:} Jede Funktion $f \in \mathcal C^\infty(M)$ definiert eine 1-Form $df \in \Omega^1(M)$.

\bigskip

%{\bf Bemerkung:}
%Das Dachprodukt \"ubertr\"agt sich punktweise auf den Raum der Differentialformen. F\"ur zwei Differentialformen
%$\alpha, \beta$ auf $M$ definiert man das Dachprodukt in $p\in M$ durch
%$$
%(\alpha \wedge \beta)(p) := \alpha(p) \wedge \beta(p)  \ .
%$$


\bigskip

Der Raum $\Omega^*(M)$ aller Differentialformen auf $M$ ist eine reelle Algebra. Man definiert
f\"ur Differentialformen $\omega_1, \omega_2 \in \Omega^*(M)$ und Skalare $\lambda \in \R$ punktweise in
$p \in M$:
$$
(\omega_1+\omega_2)(p):= \omega_1(p) + \omega_2(p),\quad (\lambda \omega)(p) := \lambda \omega(p),\quad
(\omega_1 \wedge \omega_2)(p) := \omega_1(p) \wedge \omega_2(p)
$$

\bigskip

Offensichtlich ist aber der Raum  $\Omega^*(M)$ auch ein Modul \"uber dem Ring der Funktionen
$\mathcal C^\infty(M)$, d.h. f\"ur eine glatte Funktion $f$ und eine Differentialform $\omega$
definiert man das Produkt $f\cdot \omega := f \wedge \omega$, also
$$
(f\cdot \omega)(p) := f(p) \cdot \omega(p) \ .
$$

\bigskip

Jede Differentialform $\hat\omega \in \Omega^k(M)$ induziert eine $\mathcal C^\infty(M)$-multilineare,
alternierende Abbildung
$$
\omega : \chi(M) \times \ldots \times \chi(M) \rightarrow \mathcal C^\infty(M)
$$
man definiert f\"ur Vektorfelder $X_1, \ldots, X_k$ auf $M$ die entsprechende Funktion in $p \in M$ durch
$\omega(X_1, \ldots, X_k)(p) :=\hat\omega(p)(X_1(p), \ldots, X_k(p)) $. Es stellt sich heraus, dass auch die
Umkehrung gilt und man eine \"aquivalente Definition von Differentialformen erh\"alt:

\begin{Lemma}
Jede $\mathcal C^\infty(M)$-multilineare,
alternierende Abbildung
$$
\omega : \chi(M) \times \ldots_{\mbox{k-mal}} \ldots \times \chi(M) \rightarrow \mathcal C^\infty(M)
$$
definiert eine $k$-Form $\hat\omega$ auf $M$.
\end{Lemma}
\proof
Seien $X_1, \ldots, X_k \in T_pM$ mit Fortsetzungen $\tilde X_1, \ldots, \tilde X_k$ zu lokal um $p$
definierten Vektorfeldern. Dann definiert man die $k$-Form $\hat\omega$ durch
$$
\hat \omega(p)(X_1, \ldots, X_k) := \omega(\tilde X_1, \ldots, \tilde X_k)(p) \ .
$$
Zu zeigen bleibt jetzt, dass  $\omega(\tilde X_1, \ldots, \tilde X_k)(p)$ nur von dem Wert der
Vektorfelder $\tilde X_i$ im Punkt $p$, also den Tangentialvektoren $X_i$ abh\"angt. Es reicht den
Beweis f\"ur $k=1$ zu f\"uhren. Hier bleibt zu zeigen, dass $\omega(X)(p)=0$ gilt, falls $X$ ein
Vektorfeld ist mit $X(p)=0$.

\medskip

Sei $(U,x)$ eine Karte um $p$, dann schreibt sich das Vektorfeld $X$ auf $U$ als
$
X = \sum a_i \frac{\partial}{\partial x_i}
$
f\"ur gewisse glatte Funktionen $a_i \in \mathcal C^\infty(U)$ mit $a_i(p)=0, i=1, \ldots, n$. Man w\"ahlt
nun eine Abschneidefunktion $\varphi \in \mathcal C^\infty(M)$ mit $\supp (\varphi ) \subset U$ und
$\varphi \equiv 1$ auf einer kleinen Umgebung von $p$. Dann folgt
$$
\varphi^2\, X \;=\; \sum \varphi a_i \cdot \varphi \tfrac{\partial}{\partial x_i}
$$
dabei sind $ \varphi a_i, i = 1, \ldots, n$ global definierte Funktionen und $ \varphi \tfrac{\partial}{\partial x_i},
i = 1, \ldots, n$ global definierte Vektorfelder auf $M$. Das Vektorfeld $X$ schreibt sich also als
$$
X \;=\; X - \varphi^2 X + \sum \varphi a_i \cdot \varphi \tfrac{\partial}{\partial x_i}
\;=\; (1 - \varphi^2) X +  \sum \varphi a_i \cdot \varphi \tfrac{\partial}{\partial x_i} \ .
$$
Wendet man hierauf die $\mathcal C^\infty(M)$-lineare Abbildung $\omega$ an und benutzt die
Voraussetzungen $\varphi(p) = 1$ und $a_i(p)=0$, so ergibt sich
$$
\omega(X)(p) \;=\; (1-\varphi^2)(p) \omega(X)(p) + \sum \varphi a_i \cdot \omega(\varphi \tfrac{\partial}{\partial x_i}) \;=\; 0 \ .
$$
\qed

\bigskip

\subsection{Differentialformen in lokalen Koordinaten}

Sei $(U,x)$ eine Karte um $p \in M$. Dann sind f\"ur jeden Punkt $p\in M$ die Vektoren
$
\left. \tfrac{\partial}{\partial x_i} \right|_{p}, \; i = 1, \ldots, n \;
$
eine Basis in $T_pM$. Die dazu duale Basis in $T^*_pM$ bezeichnet man mit $dx_i(p)$. \"Uber $U$
hat man also
$$
\tfrac{\partial}{\partial x_i} \in \chi(U),\qquad dx_i \in \Omega^1(U) \qquad \mbox{mit }
\quad dx_i (\tfrac{\partial}{\partial x_j}) = \delta_{ij} \ .
$$
Damit schreibt sich jede Differentialform $\omega \in \Omega^k(M)$ lokal \"uber $U$ als:
$$
\left. \omega \right|_U = \sum_{i_1 < \ldots i_k} \omega_{i_1, \ldots, i_k} \, dx_{i_1} \wedge \ldots \wedge dx_{i_k} \ ,
$$
wobei $\omega_{i_1, \ldots, i_k} \in \mathcal C^\infty(U)$ glatte Funktionen sind, die sich berechnen lassen
durch
$$
\omega_{i_1, \ldots, i_k} = \omega(\tfrac{\partial}{\partial x_{i_1}}, \ldots, \tfrac{\partial}{\partial x_{i_k}}) \ .
$$


\bigskip

\subsection{Das Zur\"uckziehen von Differentialformen}

Sei $f: M \rightarrow N$ eine differenzierbare  Abbildung und sei $\omega \in \Omega^k(N)$. Dann erh\"alt man
mittels Zur\"uckziehen eine $k$-Form $f^*\omega$ auf $M$, die definiert ist durch
$$
(f^*\omega)_p (X_1, \ldots , X_k)
=
\omega_{f(p)} (df (X_1), \ldots . df(X_k))
$$
f\"ur Tangentialvektoren $X_i \in T_pM$, d.h. $(f^*\omega)_p =
((df)^*\omega)_p$. Es \"ubertragen sich nun die Eigenschaften, die f\"ur das Zur\"uckziehen mittels linearer Abbildungen von Formen auf Vektorr\"aumen gelten. Insbesondere:

\begin{Lemma}
Seien $\alpha, \beta$ Differentialformen auf $M$ und seien $f:M\rightarrow N, g: N \rightarrow P$ differenzierbare
Abbildungen, dann gilt
\begin{enumerate}
\item
$ \qquad (g \circ f)^* = f^* \circ g^*$
\item
$ \qquad f^*(\alpha \wedge \beta) = (f^*\alpha) \wedge (f^*\beta) $
\end{enumerate}
\end{Lemma}


\bigskip

\subsection{Das Differential}

Jede Funktion $f \in \mathcal C^\infty(M)$, also jede 0-Form auf $M$, definiert eine 1-Form $df$,
die sich in lokalen  Koordinaten schreibt als
$$
df = \sum \tfrac{\partial f}{\partial x_i}\, dx_i \ .
$$
Es soll nun gezeigt werden, dass sich das Differential fortsetzt zu einer Folge linearer
Abbildungen
$$
\Omega^0(M)   \stackrel{d}{\rightarrow}  \Omega^1(M)    \stackrel{d}{\rightarrow} \ldots
$$


\bigskip

\begin{Satz}
Auf jeder differenzierbaren Mannigfaltigkeit $M$ existiert genau eine Abbildung, das {\em Differential}
$d: \Omega^*(M) \rightarrow \Omega^*(M)$ mit den folgenden Eigenschaften:
\begin{enumerate}
\item
$d$ ist $\R$-linear
\item
F\"ur $f\in \Omega^0(M)$ und $X\in \chi(M)$ gilt $df(X) := X(f)$.
\item
$
d: \Omega^k(M) \rightarrow \Omega^{k+1}(M)
$
\item
$d^2 = 0$
\item
Die Abbildung $d$ ist eine Anti-Derivation, d.h.
$$
d(\alpha \wedge \beta) = d\alpha \wedge \beta +(-1)^k \alpha \wedge d\beta
$$
f\"ur $\alpha\in \Omega^k(M), \beta \in \Omega^*(M)$
\end{enumerate}
\end{Satz}
\proof
Man zeigt zun\"achst die Eindeutigkeit und dann die Existenz durch eine lokale Formel. Als erstes soll gezeigt
werden, dass das Differential durch die Eigenschaften 1. - 5. auf 1-Formen eindeutig festgelegt ist. Genauer
gilt f\"ur eine 1-Form $\omega$ und beliebige Vektorfelder $X,Y$ die wichtige Formel
\begin{equation}\label{fundamental}
d\omega(X,Y) = L_X(\omega (Y)) - L_Y(\omega(X)) - \omega([X, Y]) \ .
\end{equation}
Dabei bezeichnet $L_X$ die Lie-Ableitung einer Funktion in Richtung eines Vektorfeldes $X$, d.h.
$L_X(f)= df(X) = X(f)$. Die Funktion $f$ ist hier $\omega(Y)$. Mit dieser Formel ist dann $d$
eindeutig auf 0- und 1-Formen eindeutig festgelegt.

\medskip

Nun zum Beweis von Gleichung~(\ref{fundamental}). Zun\"achst gilt die Gleichung f\"ur 1-Formen
$\omega = df$. Linke Seite: $ d\omega = d(df)=0$. Rechte Seite:
$$
 L_X(df (Y)) - L_Y(df(X)) - df([X, Y]) = L_X(L_Y(f)) - L_Y(L_X(f)) - L_{[X, Y]}(f) = 0 \ ,
$$
nach Definition des Kommutators $[X,Y]$. Die Gleichung~(\ref{fundamental}) gilt auch f\"ur alle
1-Formen $\omega = g \alpha$, wobei $g$ eine beliebige Funktion und $\alpha$ eine 1-Form ist, f\"ur
die die Gleichung erf\"ullt ist. Linke Seite:
$$
\begin{array}{rl}
d(g\alpha)(X,Y)
& = (dg \wedge \alpha + g \wedge d\alpha)(X,Y)
= dg(X)\alpha(Y) - dg(Y)\alpha(X) + gd\alpha(X,Y)\\[1.5ex]
& = L_X(g)\alpha(Y) - L_Y(g)\alpha(X) + gd\alpha(X,Y) \ .
\end{array}
$$
Rechte Seite:
$$
\begin{array}{rl}
L_X(g\alpha(Y)) & - L_Y(g\alpha(X)) - g\alpha([X,Y])
 =
L_X(g) \alpha(Y) + gL_X(\alpha(Y)) - L_Y(g)\alpha(X) \\[1ex]
& \phantom{xxxxxxxxxxxxxxxxxxxxxxxxxxxxxxxxxx} - g L_Y(\alpha(X)) -g \alpha([X,Y]) \\[1.5ex]
& =
g( L_X(\alpha(Y)) - L_Y(\alpha(X)) - \alpha([X,Y])) + L_X(g) \alpha(Y) - L_Y(g)\alpha(X)\\[1.5ex]
& =
g d\alpha(X,Y) + L_X(g) \alpha(Y) - L_Y(g)\alpha(X) \ .
\end{array}
$$
Damit gilt Gleichung~(\ref{fundamental}) f\"ur alle 1-Formen. Denn lokal schreibt sich jede
1-Form $\omega$ als $\omega = \sum \omega_i dx_i$ und f\"ur die einzelnen Summanden wurde
die Gleichung gerade bewiesen.

\medskip

Sei $d: \Omega^*(M) \rightarrow \Omega^*(M)$ eine lineare Abbildung, definiert auf 0- bzw. 1-Formen
durch  $df(X)=X(f)$ f\"ur $f\in \Omega^0(M)$ und $d\omega$ wie in Gleichung~(\ref{fundamental}) f\"ur
$\omega \in \Omega^1(M)$. Dann erf\"ullt $d$ die Eigenschaften 1. - 5. auf 0-bzw. 1-Formen.

Die Eigenschaften 1. - 3. sind nach Voraussetzung erf\"ullt. F\"ur die Eigenschaft~5. hat man zwei
F\"alle zu beweisen. Zun\"achst seien $f, g \in \mathcal C^\infty(M)$. Dann berechnet man
$$
d(f\cdot g)(X) = L_X(f\cdot g) = L_X(f)\cdot g + f\cdot L_X(g)
=
df(X)\cdot g + f \cdot dg(X)
=
(df \wedge g + f \wedge dg)(X) \ ,
$$
wobei nat\"urlich normalerweise $f \wedge dg$ kurz als  $ f \cdot dg$ geschrieben wird.

\medskip

Als n\"achstes sei $f\in \mathcal C^\infty(M)$ und $\omega \in \Omega^1(M)$ dann gilt
$$
\begin{array}{rl}
d(f \wedge \omega)(X,Y) & = d(f\cdot \omega)(X,Y)
=
L_X(f\cdot \omega(Y) ) - L_Y(f\cdot \omega(X)) -f \, \omega([X,Y])\\[1.5ex]
& =
L_X(f)\cdot \omega(Y) + f\cdot L_X(\omega(Y)) - (L_Y(f))\cdot \omega(X) -
f\cdot L_Y(\omega(X)) \\[1ex]
& \hfill -f \, \omega([X,Y])\\[1.5ex]
& =
df(X) \cdot \omega(Y) - df(Y) \cdot \omega(X) + f\,d\omega(X,Y)\\[1.5ex]
& =
(df \wedge \omega  + f \wedge d\omega)(X,Y)
\end{array}
$$

\medskip

Schlie\ss lich bleibt noch die Eigenschaft 4., also $d^2=0$ auf Funktionen
zu \"uberpr\"ufen. Man rechnet hier
$$
d(df)(X,Y) = L_X(df(Y)) - L_Y(df(X)) - df ([X,Y])
=
L_X L_Y(f) - L_Y L_X(f) - L_{[X,Y]}(f) = 0
$$
nach der Definition des Kommutators $[X,Y]$ zweier Vektorfelder $X,Y$.

\medskip

Sei $(U,x)$ eine beliebige Karte auf $M$ und $\omega \in \Omega^k(M)$. Dann schreibt sich
$\omega$ lokal auf $U$ als
$$
\omega =    \sum \omega_{i_1,\ldots, i_k} \, dx_{i_1} \wedge \ldots \wedge dx_{i_k}
$$
f\"ur gewisse glatte Funktionen $\omega_{i_1,\ldots, i_k}\in \mathcal C^\infty(U)$.
Aus den Eigenschaften 1. - 5. folgt nun, dass $d\omega$ notwendigerweise auf $U$ gegeben ist
durch
\begin{equation}\label{differential}
d\omega = \sum  d\omega_{i_1,\ldots, i_k} \wedge dx_{i_1} \wedge \ldots \wedge dx_{i_k} \ .
\end{equation}
Umgekehrt kann man diese Gleichung auch f\"ur die Definition des Differentials $d$ benutzen.
Damit ist die Existenz und Eindeutigkeit des Differentials gezeigt. Es bleibt noch zu zeigen:
(i) $\left.(d\omega)\right|_U= d(\left. \omega\right|_U)$, d.h. man kann das Differential
wirklich lokal definieren. (ii) Definiert man das Differential $d$ durch Gleichung~(\ref{differential})
dann ist $d$ unabh\"angig von der Wahl der Koordinaten und erf\"ullt die Eigenschaften 1. - 5.

\medskip

Zum Beweis der Koordinatenunabh\"angigkeit im Fall von 1-Formen. Der Fall bliebigen Grades geht
analog. Seien $(U,x)$ und $(V,y)$ zwei Karten um $p\in M$. Eine 1-Form $\omega$ schreibe sich
lokal als $\omega = \sum f_i dx_i = \sum h_j dy_j$. Dann gilt $dy_j = \sum
\frac{\partial y_j}{\partial x_i}dx_i$ und daraus folgt $f_i = \sum \frac{\partial y_j}{\partial x_i} h_j$.
Setzt man dies in die Definition von $d\omega$ ein, so erh\"alt man
$$
\begin{array}{rl}
\sum df_i \wedge dx_i & = \sum \tfrac{\partial f_i}{\partial x_k } dx_k \wedge dx_i =
\sum \tfrac{\partial^2 y_j}{\partial x_i \, \partial x_k} \, h_j \, dx_k \wedge dx_i
+
\sum \tfrac{\partial y_j }{\partial x_i}\,\tfrac{\partial h_j }{\partial x_k}\,dx_k \wedge dx_i \\[1.5ex]
& =
\sum dh_j \wedge dy_j \ .
\end{array}
$$
Hierbei verschwindet der Summand mit den partiellen Ableitungen zweiter Ordnung, da
$\tfrac{\partial^2 y_j}{\partial x_i \, \partial x_k}$ nach dem Satz von Schwarz
symmetrisch in $i$ und $k$ ist, w\"ahrend $dx_k \wedge dx_i$ schiefsymmetrisch ist.
\qed






\bigskip

\subsection{Beispiel: Die Maxwell Gleichungen}

\bigskip

Sei $F$ eine 2-Form auf $\R^4$, d.h. $F= \sum_{i<j } F_{ij} \, dx_i \wedge dx_j$, wobei $0\le i,j \le 3$.
Die 2-Form $F$ beschreibt das elektro-magnetische Feld. Ausgehend von $F$ erh\"alt man zwei Vektorfelder
auf $\R^3$: das elektrische Feld $E = (E_1, E_2, E_3)$ mit $E_i :=-F_{0\,a},\; a=1,2,3$ und das magnetische
Feld $H =(H_1, H_2, H_3)$ mit $H_1 = F_{2\,3}, H_2 = -F_{1\,3}, H_3 = F_{1\,2}$. Es gilt also
$$
F \;=\; - \sum_a E_a \, dx_0 \wedge dx_a \;+\; H_1 \, dx_2 \wedge dx_3 \;-\; H_2 \, dx_1 \wedge dx_3 \;+\; H_3 \, dx_1 \wedge dx_2 \ .
$$

\medskip

Die Gleichung $dF = 0$ entspricht nun einem System partieller Differentialgleichungen. Man berechnet

$$
\begin{array}{rl}
d F  & =\; \sum_{i<j} \, dF_{ij} \, \wedge dx_i \wedge dx_j
\;=\; \sum_{i<j,\, k} \, \frac{\partial F_{ij}}{\partial x_k} \; dx_k \wedge dx_i \wedge dx_j \\[2ex]
& =
(\frac{\partial F_{1\,2}}{\partial x_3} - \frac{\partial F_{1\,3}}{\partial x_2} + \frac{\partial F_{2\,3}}{\partial x_1})\,
dx_1 \wedge dx_2 \wedge dx_3
\;+\;
\sum_{i<j} (\frac{\partial F_{ij}}{\partial x_k})\, dx_0 \wedge dx_i \wedge dx_j \\[2ex]
& =
(\frac{\partial H_3}{\partial x_3} + \frac{\partial H_2}{\partial x_2} + \frac{\partial H_1}{\partial x_1})\,
dx_1 \wedge dx_2 \wedge dx_3 \\[1.5ex]
& \phantom{xxxxxxxxxxx} + \quad
(\frac{\partial F_{1\,2}}{\partial x_0} + \frac{\partial F_{0\,1}}{\partial x_2} - \frac{\partial F_{0\,2}}{\partial x_1})\,
dx_0 \wedge dx_1 \wedge dx_2 \\[1.5ex]
& \phantom{xxxxxxxxxxx} + \quad
(\frac{\partial F_{1\,3}}{\partial x_0} - \frac{\partial F_{0\,3}}{\partial x_1} + \frac{\partial F_{0\,1}}{\partial x_3})\,
dx_0 \wedge dx_1 \wedge dx_3 \\[1.5ex]
& \phantom{xxxxxxxxxxx}  + \quad
(\frac{\partial F_{2\,3}}{\partial x_0} + \frac{\partial F_{0\,2}}{\partial x_3} - \frac{\partial F_{0\,3}}{\partial x_2})\,
dx_0 \wedge dx_2 \wedge dx_3 \\[2ex]
& =
(\frac{\partial H_3}{\partial x_3} + \frac{\partial H_2}{\partial x_2} + \frac{\partial H_1}{\partial x_1}) \,
dx_1 \wedge dx_2 \wedge dx_3 \\[1.5ex]
& \phantom{xxxxxxxxxxx} + \quad
(\frac{\partial H_3}{\partial x_0} + \frac{\partial E_1}{\partial x_2} - \frac{\partial E_2}{\partial x_1})\,
dx_0 \wedge dx_1 \wedge dx_2 \\[1.5ex]
& \phantom{xxxxxxxxxxx} - \quad
( \frac{\partial H_2}{\partial x_0} + \frac{\partial E_3}{\partial x_1} - \frac{\partial E_1}{\partial x_3})\,
dx_0 \wedge dx_1 \wedge dx_3 \\[1.5ex]
& \phantom{xxxxxxxxxxx}  + \quad
(\frac{\partial H_1}{\partial x_0} + \frac{\partial E_2}{\partial x_3} - \frac{\partial E_3}{\partial x_2})\,
dx_0 \wedge dx_2 \wedge dx_3
\end{array}
$$

Substituiert man hier die Definition der Divergenz eines Vektorfeldes $H$ als
$$
\mathrm{div}(H) = \tfrac{\partial H_3}{\partial x_3} + \tfrac{\partial H_2}{\partial x_2} + \tfrac{\partial H_1}{\partial x_1}
$$
und der Definition der Rotation eines Vektorfeldes $E$ als
$$
\mathrm{rot} (E) =
(\tfrac{\partial E_3}{\partial x_2} - \tfrac{\partial E_2}{\partial x_3}) \, \tfrac{\partial}{\partial x_1}
\,+\,
(\tfrac{\partial E_1}{\partial x_3} - \tfrac{\partial E_3}{\partial x_1}) \, \tfrac{\partial}{\partial x_2}
\,+\,
(\tfrac{\partial E_2}{\partial x_1} - \tfrac{\partial E_1}{\partial x_2}) \, \tfrac{\partial}{\partial x_3}
$$
folgt, dass die Gleichung $dF = 0$ \"aquivalent ist zum 1. Teil der Maxwell-Gleichungen:
$$
(i) \quad \mathrm{div} (H) = 0, \qquad \mbox{und} \qquad (ii) \quad \mathrm{rot} (E) = - \tfrac{\partial H}{\partial x_0} \ .
$$

\bigskip





\subsection{Symplektische Mannigfaltigkeiten}

\bigskip

\begin{Definition}
Eine {\em symplektische Mannigfaltigkeit} ist eine differenzierbare Mannigfaltigkeit $M$
zusammen mit einer 2-Form $\omega \in \Omega^2(M)$, die geschlossenen
und nicht ausgeartet ist.
\end{Definition}

\medskip

{\bf Bemerkung:}
\begin{enumerate}
\item
Eine Differentialform $\omega$ hei\ss t {\it geschlossen}, falls $d\omega = 0$ gilt.
Sie hei\ss t {\it exakt}, falls eine Form $\eta $ existiert mit $\omega = d\eta$. Exakte
Formen sind geschlossen, ob die Umkehrung gilt h\"angt von der Topologie der Mannigfaltigkeit ab.
\item
Eine Form $\omega \in \Omega^2(M)$ hei\ss t {\it nicht ausgeartet} genau dann, wenn $\omega_p \in
\Lambda^2T_p^*M$ f\"ur alle $p \in M$ nicht ausgeartet ist.
\item
Existiert auf $M$ eine nicht ausgeartete 2-Form, dann mu\ss{} die Dimension von $M$
gerade sein.
\end{enumerate}

\bigskip

\begin{Lemma}
Sei $\; V$ ein reeller Vektorraum, dann sind folgende Aussagen \"aquivalent:
\begin{enumerate}
\item
$\omega \in \Lambda^2 V^*$ ist nicht ausgeartet.
\item
Ist $\omega (X,Y)=0$ f\"ur alle $Y\in V$, dann folgt $X=0$.
\item
$\omega^n := \omega \wedge \ldots \wedge \omega \neq 0$.
\item
$\omega$ definiert einen Isomorphismus $I: V \rightarrow V^*$, durch
$I(X)(Y) = \omega(Y,X)$.
\item
Die schief-symmetrische Matrix $A = (A_{ij})$ mit $ A_{ij} = \omega(e_i,e_j)$,
f\"ur eine Basis $\{e_i \}$ von $V$, ist invertierbar.
\end{enumerate}
\end{Lemma}

\bigskip

{\bf Beispiele:}
\begin{enumerate}
\item
Auf dem $\R^{2n}$ betrachtet man die Koordinaten $(q_1, \ldots, q_n, p_1, \ldots, p_n)$. Die kanonische
symplektische Form auf $\R^{2n}$ ist definiert durch
$$
\omega_0 = \sum^n_{i=1}\, dp_i \wedge dq_i \ .
$$
Die 2-Form $\omega_0$ ist offensichtlich geschlossen und nicht ausgeartet.
\item
Sei $M = T^*X$ der Totalraum des Kotangentialb\"undels einer differenzierbaren Mannigfaltigkeit $X$,
mit kanonischer Projektion $\pi : T^*X \rightarrow X$. Sei $p \in T^*_qX  $, d.h. $\pi(p) = q$ und $
p:T_qX\rightarrow \R$. Dann definiert man die Liouville-Form $\theta \in \Omega^1(M)$ durch:
$$
\theta (Z) := p(d\pi (Z))
$$
f\"ur $Z\in T_p M = T_p(T^*X)$. Die symplektische Form ist dann definiert als $\omega = d\theta$.
Offensichtlich ist $\omega$ geschlossen, dass $\omega$ nicht ausgeartet ist ergibt sich aus dem
folgendem Lemma.
\end{enumerate}

\bigskip

\begin{Lemma}\label{darboux}
Seien $(q_1, \ldots, q_n)$ lokale Koordinaten auf $X$, dann sind $(q_1, \ldots, q_n, p_1,\ldots, p_n)$
lokale Koordinaten auf $M= T^*X$. Dann gilt in diesen Koordinaten:
\begin{equation*}
\theta \;=\; \sum_{i=1}^n\, p_i \, dq_i
\qquad \qquad \mbox{und}\qquad \qquad
\omega \;=\; \sum_{i=1}^n\, dp_i \wedge dq_i \ .
\end{equation*}
\end{Lemma}

\bigskip

{\bf Bemerkungen:}
\begin{enumerate}
\item
Das Theorem von Darboux sagt, dass es f\"ur jede symplektisch Mannigfaltigkeit $(M, \omega)$
lokale Koordinaten $(q_i, p_i), i=1, \ldots, n$ gibt, so dass $\omega$ lokal von der Form aus Lemma~\ref{darboux}
ist.
\item
Weitere Beispiele symplektischer Mannigfaltigkeiten sind die Bahnen der adjungierten Darstellung von $G$ auf
$\g = \Lie(G)$. Dies sind homogene symplektische Mannigfaltigkeiten.
\end{enumerate}

\bigskip

\begin{Definition}
Das {\em Hamilton-Vektorfeld} $X_H$ zu einer Funktion $H \in \mathcal C^\infty(M)$
ist definiert durch die Gleichung
$$
\omega(Y, X_H) = dH(Y) = Y(H) \ ,
$$
die f\"ur beliebige Vektorfelder $Y \in \chi(M)$ erf\"ullt sein soll. Unter dem Isomorphismus
$I:TM\rightarrow T^*M$ gilt $I(X_H) = dH$.
\end{Definition}

\bigskip

\begin{Satz}
Sei $(M^{2n}, \omega)$ eine symplektische Mannigfaltigkeit und sei $H \in \mathcal C^\infty(M)$,
dann gilt in Darboux-Koordinaten $(U,x)$ mit $x=(q,p)$:
$$
X_H  = \sum^n_{i=1} \left(
 \frac{\partial H}{\partial p_i} \, \frac{\partial }{\partial q_i} \, -
 \,\frac{\partial H}{\partial q_i} \,\frac{\partial }{\partial p_i}
\right)
$$
Sei $\gamma : \R \rightarrow M$ eine glatte Kurve mit
$x(\gamma(t))=(q_1(t), \ldots, q_n(t),p_1(t),\ldots, p_n(t))$. Dann ist $\gamma$
genau dann eine Integralkurve von $X_H$, wenn
$$
\dot p_j = - \frac{\partial H}{\partial q_j }, \qquad
\dot q_j = \frac{\partial H}{\partial p_j }
\qquad \mbox{f\"ur} \qquad j = 1, \ldots, n \qquad \mbox{erf\"ullt ist}\ .
$$
\end{Satz}
\proof
In den Darboux-Koordinaten $(U,x)$ mit $x=(q,p)$ gilt $\omega = \sum dp_i \wedge dq_i$ und damit
$$
\omega( \tfrac{\partial }{\partial p_i },  \tfrac{\partial }{\partial q_j }) \;= \;
- \omega( \tfrac{\partial }{\partial q_i },  \tfrac{\partial }{\partial p_j })
\; = \;
\delta_{ij} \ .
$$
Auf $U$ existieren glatte Funktionen $a_i$ und $b_i, i= 1,\ldots, n$ mit
$
X = \sum a_i \tfrac{\partial }{\partial p_i } + b_j \tfrac{\partial }{\partial q_j }
$
und aus der Definition des Hamilton-Vektorfeldes $X_H$ folgt
$$
\omega(\tfrac{\partial }{\partial p_i }, X_H) \;=\; \tfrac{\partial H}{\partial p_i } \;=\; b_i
\qquad
\omega(\tfrac{\partial }{\partial q_i }, X_H) \;=\; \tfrac{\partial H}{\partial q_i } \;=\; - a_i \ .
$$

In den Darboux-Koordinaten $(U,x)$ gilt
$
\dot \gamma (t) = \sum \dot q_i \, \tfrac{\partial }{\partial q_i } + \sum \dot p_i\, \tfrac{\partial }{\partial p_i }.
$
Vergleicht man das mit der gerade bewiesenen Formel f\"ur $X_H$, dann sieht man, dass  $\dot \gamma (t) = X_H$ genau dann
gilt, wenn die beiden angegebenen Differentialgleichungen erf\"ullt sind.

\qed

\bigskip

{\bf Bemerkung:}
Die kanonischen Bewegungsgleichungen der Hamiltonschen Mechanik sind genau die Gleichungen der
Integralkurven des Hamilton-Vektorfeldes $X_H$ (Satz von Hamilton).

\bigskip

Auf symplektischen Mannigfaltigkeiten besitzt der Raum der glatten Funktionen die Struktur
einer Lie-Algebra. Diese soll im Folgenden beschrieben werden.

\medskip

\begin{Definition}
F\"ur Funktionen $f,g \in \mathcal C^\infty(M)$ auf einer symplektischen Mannigfaltigkeit
$(M, \omega)$ definiert man die {\em Poisson-Klammer} $\{f, g\}$ durch
$$
\{f, g\} = - \omega(X_f, Y_g) \ .
$$
\end{Definition}

\bigskip

\begin{Satz}\label{poisson}
Die Poisson-Klammer hat folgende Eigenschaften:
\begin{enumerate}
\item
$
\{f, g\} \;= \;-\,\{g, f\}
$
\item
$ \{ \cdot, \cdot\}$ ist $\R$-bilinear
\item
$
\{ f, \{g, h\}\} \;+\; \{g, \{h, f\}\} \;+\; \{h, \{f, g\}\} \;=\; 0
$
\item
$
\{f, g \cdot h\} \;=\; g \cdot \{ f, h\} \;+\;  h \cdot \{f, g\}
$
\item
Die Abbildung $H \mapsto X_H$ ist ein Homomorphismus von Lie-Algebren, d.h. es gilt
$$
[X_f, X_g] = X_{\{f,g\}}
$$
\item
In Darboux-Koordinaten gilt:
$$
\{f, g\} \;=\; \sum_{i=1}^n \, \left(
\frac{\partial f}{\partial q_i}\,  \frac{\partial g}{\partial p_i}\,
\,-\,
\frac{\partial f}{\partial p_i}\,  \frac{\partial g}{\partial q_i}\,
\right)
$$
\end{enumerate}
\end{Satz}
\proof
\"Ubungsaufgabe
\qed

\bigskip

\begin{Satz}[Satz von Liouville]
Sei $(M, \omega)$ eine symplektische Mannigfaltigkeit und die Funktion $H \in \mathcal C^\infty(M)$ habe ein
vollst\"andiges Hamilton-Vektorfeld $X_H$, d.h. der Flu\ss{} $\varphi_t$ von $X_H$, sei f\"ur
alle Zeiten $t$ definiert. Dann  gilt
$$
\varphi^*_t \,\omega \;=\; \omega \ ,
$$
d.h. $\varphi_t$ ist f\"ur alle $t$ ein symplektischer Diffeomorphismus. Das Tripel
$(M, \omega, H)$ nennt man {\em Hamilton-System}.
\end{Satz}
\proof
Etwas sp\"ater wird die Lie-Ableitung einer Differentialform $\omega$ nach einem Vektorfeld $X$ definiert. Man hat
$L_X \omega = \frac{d}{dt} \varphi_t^*\,\omega$, wobei $\varphi_t$ der Flu\ss{} von $X$ ist und es gilt die
fundamentale Formel
$$
L_X \omega \;=\; d \, (\omega(X,\cdot)) \;+\; (d\omega) (X, \cdot) \ .
$$
Im ersten Summanden steht das Differential der 1-Form $\omega(X,\cdot)$ und im zweiten Summanden steht
die 2-Form, die man erh\"alt, wenn man $X$ in die 3-Form $d\omega$ einsetzt.

\medskip

Die Gleichung $\varphi^*_t \,\omega \;=\; \omega $ gilt nun f\"ur alle $t$ genau dann, wenn
$L_{X_H}\omega = 0$, also nach der angegebenen Formel und da $\omega$ geschlossen ist, genau dann, wenn
$d (\omega(X_H, \cdot)) = 0$ bzw., nach der Definition von $X_H$, genau dann, wenn $- d (d H) = 0$, was
aber wegen $d^2=0$ immer erf\"ullt ist.
\qed

\bigskip

\begin{Definition}
Eine Funktion $f\in \mathcal C^\infty(M)$ hei\ss t {\em Erhaltungsgr\"o\ss e} des
Hamilton\-systems $(M, \omega, H)$, falls $f$ konstant ist auf den Integralkurven
von $X_H$. Erhaltungs\-gr\"ossen nennt man auch {\em erste Integrale}. Der Raum der
Erhaltungsgr\"o\ss en eines Hamiltonsystems $(M, \omega, H)$  wird mit
$\mathcal E(H)$ bezeichnet.
\end{Definition}

\bigskip

\begin{Satz}
Sei $(M, \omega, H)$ ein Hamiltonsystem. Dann gilt f\"ur den Raum der Erhaltungsgr\"o\ss en:
\begin{enumerate}
\item
 Der Raum der Erhaltungsgr\"o\ss en $\mathcal E(H)$ ist ein reeller Vektorraum.
\item
 $f\in \mathcal E(H)$ genau dann, wenn $\{f, H\} = 0$ gilt.
\item
 $f, g \in \mathcal E(H)$ dann gilt auch $\{f, g\} \in \mathcal E(H)$
        d.h. der Raum der Erhaltungsgr\"o\ss en ist eine Lie-Algebra, genauer eine Unteralgebra
       von $(\mathcal C^\infty(M), \{\cdot, \cdot\})$
\end{enumerate}
\end{Satz}
\proof
Die erste Aussage ist offensichtlich.
Sei $c_p(t) = \varphi_t(p)$ die Integralkurve von $X_H$ durch $p$. Dann gilt $f\in \mathcal E (H)$
genau dann, wenn $f(c_p(t))=f(p)$ f\"ur alle $p$ und alle $t$ gilt. Bildet man die Ableitung, so
findet man, dass diese Gleichung genau dann gilt, wenn $X_H(f)=0$ gilt und damit genau dann,
wenn $\{f, H\}= 0$ erf\"ullt ist. Denn nach Definition hat man
$X_H(f) = df (X_H) = \omega(X_H, X_f) = - \{H, f\}$. Das beweist die zweite Aussage.

\medskip

Nun zur dritten Aussage.
Seien $f, g \in \mathcal E(H)$, d.h. wie gerade gezeigt $\{f, H\} = 0 = \{g, H\}$. Dann folgt
aus Gleichung (4) in Satz~\ref{poisson}
$$
\{f,g\} \in \mathcal E (H) \leftrightarrow \{\{f,g\}, H\} = 0 \leftrightarrow
\{ \{g, H\}, f\} + \{\{H,f\}, g\}=0 \ ,
$$
was aber nach Voraussetzung erf\"ullt ist.
\qed

\bigskip

Der Satz von Noether besagt, dass zu jeder Symmetrie eines Hamiltonsystems eine Erhaltungsgr\"o\ss e
geh\"ort. Genauer gesagt gilt

\begin{Satz}[Satz von Noether]
Sei $(M, \omega, H)$ ein Hamiltonsystem mit $\omega = d\theta$, f\"ur eine geeignete 1-Form $\theta$
und sei $X\in \chi(M)$ ein vollst\"andiges Vektorfeld, dessen Flu\ss{} $\varphi_t$ das Hamiltonsystem
erh\"alt, d.h. es gelte $\varphi_t^*\theta = \theta$ und $\varphi_t^* H = H$. Dann gilt
$$
\theta(X) \in \mathcal E(H) \ .
$$
\end{Satz}
\proof
Aus den Voraussetzungen des Satzes folgt zun\"achst $L_X \theta = 0$ und $H \circ \varphi_t = H$, also
$X(H)=0$. Setzt man das in die Formel f\"ur die Lie-Ableitung $L_X\theta$ ein, so folgt mit $f:= \theta(X)$
$$
0 \;=\; (d\theta)(X, X_H ) + d(\theta(X))(X_H) \;=\; \omega (X, X_H) + df(X_H)
\;=\; X(H) + \omega(X_H, X_f) \;=\; \{H, f\} \ ,
$$
d.h. $\theta(X) \in \mathcal E (H)$.
\qed

\subsection{Tensorfelder auf Mannigfaltigkeiten}

\newcommand{\setd}[1]{\{#1\}}
\newcommand{\CC}{\mathcal{C}}
\newcommand{\ph}{\varphi}
\newcommand{\df}{\mathrm{d}f}
\newcommand{\dx}{\mathrm{d}x}
\newcommand{\dy}{\mathrm{d}y}
\newcommand{\dr}{\mathrm{d}r}
\newcommand{\dt}{\mathrm{d}t}
\newcommand{\diffd}{\mathrm{d}}
\newcommand{\dph}{\mathrm{d}\ph}
\newcommand{\domega}{\mathrm{d}\omega}
\newcommand{\scp}[1]{\left\langle#1\right\rangle}
\newcommand{\setdef}[2]{\left\{#1\;|\; #2\right\}}
\newcommand{\norm}[1]{\left\|#1\right\|}


\begin{Definition}
\begin{enumerate}
  \item Das \emph{B\"undel der $(r,s)$-Tensoren} auf einer Mannigfaltigkeit $M$
  ist definiert als disjunkte Vereinigung:
  \begin{align*}
  T^{(r,s)}(TM) := \bigcup_{p\in M} T^{(r,s)}(T_pM),
  \end{align*}
  wobei
  \begin{align*}
  T^{(r,s)}(T_pM) := &\text{Raum der multi-linearen Abbildungen}\\
  & \underbrace{T_p^*M \times \ldots \times T_p^*M}_{r\text{-mal}} \times
  \underbrace{T_pM\times \ldots \times T_pM}_{s\text{-mal}}
  \to
  \R
  \end{align*}
  \item \emph{Tensoren vom Typ $(r,s)$} sind differenzierbare Abbildungen $R:
  M\to T^{(r,s)}(TM)$,
  \begin{align*}
  \pi\circ R = \Id_M,
  \end{align*}
  d.h. Schnitte im B\"undel $T^{(r,s)}(TM)$. Das sind $\mathcal
  C^\infty$-multilineare Abbildungen,
  \begin{align*}
  \underbrace{\Omega^1(M)\times\ldots\times \Omega^1(M)}_{r\text{-mal}} \times
  \underbrace{\chi(M)\times \ldots \times \chi(M)}_{s\text{-mal}}\to \mathcal
  C^\infty(M).
  \end{align*}
\end{enumerate}
\end{Definition}

\bigskip

{\bf Bemerkungen:}

\begin{enumerate}
  \item Als \emph{tensorielles Verhalten} bezeichnet man folgende Eigenschaft:
  Der Wert eines $(r,s)$-Tensors auf den $r$ 1-Formen und den $s$ Vektorfeldern
  in $p\in M$ h\"angt nur von deren Wert in $p$ ab.
  \item Lokal auf einer Umgebung $U$ von $p$ schreiben sich $(r,s)$-Tensoren als
  \begin{align*}
  R = \sum_{i,j} R_{j_1\ldots j_s}^{i_1\ldots i_r} \frac{\partial}{\partial
  x_{i_1}}\otimes\ldots\otimes\frac{\partial}{\partial
  x_{i_r}}\otimes \dx_{j_1}\otimes\ldots\otimes \dx_{j_s},
  \end{align*}
  wobei $\frac{\partial}{\partial x_i}(\dx_j) =
  \dx_j\left(\frac{\partial}{\partial x_i}\right) = \delta_{ij}$ und
  $R_{j_1\ldots j_s}^{i_1\ldots i_r}\in \mathcal C^\infty(M)$.
\end{enumerate}

\subsection{Lie-Ableitung von Differentialformen}

Sei $X$ ein Vektorfeld mit lokalem Flus $\ph_t$, d.h. $\ph_t: U\subset M\to M$
definiert durch
\begin{align*}
\ph_t(p) = c_p(t),\qquad \ph_0(p) = p,\qquad \dot{\ph_t}(p) = X_{\ph_t(p)}.
\end{align*}

Die Lie-Ableitung einer Funktion $f\in\mathcal C^\infty(M)$ ist gegeben durch
\begin{align*}
L_X(f)_p = \df(X_p) = \frac{\diffd}{\dt}\bigg|_{t=0} \ph_t^*f(p),
\end{align*}
wobei $\ph_t^*f(p) = f\circ\ph_t(p)$.
\medskip

Die Lie-Ableitung eines Vektorfeldes $Y\in\chi(M)$ ist gegeben durch
\begin{align*}
L_X Y_p = [X,Y]_p = \frac{\diffd}{\dt}\bigg|_{t=0} (\ph_t^* Y)_{p},
\end{align*}
wobei $\ph_t^* Y_p = \dph_{-t}(Y_{\ph_t(p)})$.

\bigskip

\begin{Definition}
Die \emph{Lie-Ableitung} einer Differentialform $\omega\in\Omega^k(M)$ nach
einem Vektorfeld $X\in\chi(M)$ ist definiert durch
\begin{align*}
L_X\omega = \frac{\diffd}{\dt}\bigg|_{t=0} \ph_t^*\omega,
\end{align*}
wobei $\ph_t^*\omega (X_1,\ldots,X_k) 
=\omega_{\ph_t(\cdot)}(\dph_{t}X_1,\ldots,\dph_t X_k)$.
\end{Definition}

\bigskip

\begin{Lemma}
\begin{enumerate}
  \item Sei $\omega\in\Omega^1(M)$ und $X,Y\in\chi(M)$, dann gilt
\begin{align*}
(L_X\omega)Y = L_X(\omega(Y)) - \omega(L_XY).
\end{align*}
\item Sei $\omega\in\Omega^k(M)$ und $X,X_1,\ldots,X_k\in\chi(M)$, dann gilt
\begin{align*}
(L_X\omega)(X_1,\ldots,X_k) = L_X(\omega(X_1,\ldots,X_k))-\sum_i
\omega(\ldots,L_X X_i,\ldots)
\end{align*}
\item $L_X(\alpha\wedge\beta) = (L_X\alpha) \wedge \beta + \alpha\wedge
(L_X\beta)$.
\item $L_X(\domega) = \diffd L_X\omega$.
\end{enumerate}
\end{Lemma}

\proof
Zu 1. 
\begin{align*}
L_X\omega(Y) &= \frac{\diffd}{\dt}\bigg|_{t=0} \ph_t^*(\omega(Y)) =
 \frac{\diffd}{\dt}\bigg|_{t=0}
 \omega(Y)\circ\ph_t\\
 %\omega_{\ph_t(\cdot)}(Y_{\ph_t(\cdot)})\\
 &= \frac{\diffd}{\dt}\bigg|_{t=0}
 \omega(\dph_t\circ\dph_{-t} Y)\circ\ph_t\\
 %\omega_{\ph_t(\cdot)}(\dph_{t}\circ\dph_{-t}Y_{\ph_t(\cdot)})\\
 &= \frac{\diffd}{\dt}\bigg|_{t=0}
 %\omega_{\ph_t(\cdot)}(\dph_{t}(\ph_t^* Y))\\
 \omega(\dph_{t}(\ph_t^* Y))\circ\ph_t\\
 &= \frac{\diffd}{\dt}\bigg|_{t=0}
 (\ph_t^*\omega)(\ph_t^* Y)\\
 &= \frac{\diffd}{\dt}\bigg|_{t=0}
 \ph_t^*\omega(Y) + 
 \omega\left(\frac{\diffd}{\dt}\bigg|_{t=0}\ph_t^* Y\right)\\
 &= (L_X\omega)Y + \omega(L_X Y).
\end{align*}
Zu 3.
\begin{align*}
L_X(\alpha\wedge \beta) &= \frac{\diffd}{\dt}\bigg|_{t=0}
\ph_t^*(\alpha\wedge\beta)
 = \frac{\diffd}{\dt}\bigg|_{t=0}
\ph_t^*(\alpha)\wedge\ph_t^*(\beta)\\
&= \left(\frac{\diffd}{\dt}\bigg|_{t=0}
\ph_t^*(\alpha)\right)\wedge \beta +
\alpha\wedge\left( 
\frac{\diffd}{\dt}\bigg|_{t=0}
\ph_t^*(\beta)\right)\\
&= (L_X\alpha) \wedge\beta + \alpha\wedge(L_X\beta).
\end{align*}
Zu 4.
\begin{align*}
L_X \domega = \frac{\diffd}{\dt}\bigg|_{t=0} \ph_t^* \domega =
\frac{\diffd}{\dt}\bigg|_{t=0} \diffd(\ph_t^* \omega) =
\diffd\left(\frac{\diffd}{\dt}\bigg|_{t=0} \ph_t^* \omega\right) =
\diffd L_X\omega.
\end{align*}
\qed

\bigskip

\begin{Definition}
Sei $\omega\in\Omega^k(M)$, $X\in\chi(M)$, dann ist die \emph{Kontraktion}
(oder \emph{inneres Produkt}) von $\omega$ mit $X$ folgenderma\ss{}en definiert
\begin{align*}
(i_X\omega)(X_1,\ldots,X_k) = \omega(X,X_1,\ldots,X_k).
\end{align*}
Man schreibt auch $i_X\omega = X\lrcorner\ \omega$,
\begin{align*}
i_X: \Omega^k(M)\to \Omega^{k-1}(M).
\end{align*}
\end{Definition}


{\bf Bemerkung:}
Sei $V$ ein Vektorraum und $v\in V$, dann ist
\begin{align*}
i_v : \Lambda^kV^*\to \Lambda^{k-1}V^*
\end{align*}
die zu $u\mapsto v\wedge u$ duale Abbildung, d.h.
\begin{align*}
\scp{i_v\alpha,\beta} = \scp{\alpha,v^*\wedge\beta},\qquad v ^*(u) = \scp{v,u}.
\end{align*}
Das Skalarprodukt auf $\Lambda^kV^*$ ist dadurch festgelegt, dass
\begin{align*}
e^{i_1}\wedge \ldots \wedge e^{i_k}
\end{align*}
eine ONB f\"ur $\Lambda^kV^*$ ist f\"ur eine ONB $\setd{e_i}$ von $V$.

\bigskip

\begin{Lemma}
\begin{enumerate}
  \item $i_X(f\omega) = fi_X\omega$.
  \item $i_X(\alpha\wedge\beta) = (i_X\alpha)\wedge \beta + (-1)^k \alpha\wedge
  i_X\beta$ \qquad f\"ur $\alpha \in \Omega^k(M)$ und $\beta\in\Omega^*(M)$.
  \item $i_X(\df) = \df(X)$, falls $f\in\mathcal C^\infty(M)$.
  \item $i_X^2 = 0$.
\end{enumerate}
\end{Lemma}

\bigskip

\begin{Satz}
Sei $\omega\in\Omega^*(M)$ und $X\in\chi(M)$. Dann gilt
\begin{align*}
L_X\omega = i_X\domega + \diffd i_X\omega.\tag{*}
\end{align*}
\end{Satz}

\proof
\textit{(*) gilt auf 0-Formen}: Sei also $f\in \mathcal C^\infty(M)$, dann ist
$L_X f =\df(X)$ und $i_X f = 0$. Somit gilt
\begin{align*}
i_X\df + \diffd i_X f = i_X \df = \df(X) = L_Xf.
\end{align*}
\textit{(*) gilt auf 1-Formen}: Sei also $\omega\in\Omega^1(M)$, dann ist
\begin{align*}
(i_X\domega + \diffd i_X\omega)(Y) &= 
\domega(X,Y) + (\diffd \omega(X))(Y)\\
&= L_X(\omega(Y)) - L_Y(\omega(X)) -\omega([X,Y]) + L_Y(\omega(X))\\
&= L_X(\omega(Y)) - \omega([X,Y]) = (L_X\omega)(Y).
\end{align*}

Sei nun $P_X = i_X\domega + \diffd i_X\omega$. Wir zeigen, dass $P_X$ eine
Derivation ist, d.h.
\begin{align*}
P_X(\alpha\wedge \beta) = P_X(\alpha)\wedge \beta + \alpha\wedge P_X(\beta).
\end{align*}
Da $P_X = L_X$ auf $0$- und 1-Formen, folgt somit die Behauptung.

\medskip

Sei also $\alpha\in\Omega^k(M)$, $\beta\in\Omega^*(M)$. Dann gilt
\begin{align*}
P_X(\alpha\wedge\beta) &= i_X\diffd(\alpha\wedge\beta) + \diffd
i_X(\alpha\wedge\beta) \\ &=  
i_X((\diffd\alpha)\wedge\beta + (-1)^k \alpha\wedge(\diffd\beta))
+ \diffd( (i_X\alpha)\wedge\beta + (-1)^k\alpha\wedge i_X\beta )\\
&= (i_X\diffd\alpha)\wedge \beta + 
(-1)^{k-1} \diffd\alpha\wedge i_X\beta
+ (-1)^ki_X\alpha\wedge\diffd\beta + (-1)^{2k}\alpha\wedge i_X\diffd\beta\\
&+ (\diffd i_X\alpha)\wedge \beta + (-1)^{k-1} (i_X\alpha) \wedge \diffd\beta
+ (-1)^k\diffd\alpha\wedge i_X\beta + (-1)^{2k} \alpha\wedge \diffd i_X\beta\\
&= ((i_X\diffd\alpha)+ (\diffd i_X\alpha))\wedge \beta + 
\alpha\wedge (i_X\diffd\beta +  \diffd i_X\beta)\\
&= P_X(\alpha)\wedge\beta + \alpha\wedge P_X(\beta).
\end{align*}
\qed

\bigskip

F\"ur eine $k$-Form $\omega$ haben wir somit nach obigem Satz
\begin{align*}
\domega(X,X_1,\ldots,X_k) = (L_X\omega)(X_1,\ldots,X_k) -
\diffd(i_X\omega)(X_1,\ldots,X_k).
\end{align*}
Iteration dieses Arguments liefert die

\begin{Folgerung}
Sei $\omega\in\Omega^{k}(M)$ und $X_0,\ldots,X_k\in\chi(M)$. Dann gilt
\begin{align*}
\domega(X_0,\ldots,X_k) &= \sum_{i=0}^k (-1)^i
L_{X_i}(\omega(X_0,\ldots,\hat{X}_i,\ldots,X_k))\\
&+ \sum_{i<j} (-1)^{i+j}
\omega([X_i,X_j],\ldots,\hat{X}_i,\ldots,\hat{X}_j,\ldots,X_k),
\end{align*}
wobei $\hat{X}_i$ bedeutet, dass der entsprechende Eintrag ausgelassen wird.
\end{Folgerung}

\bigskip

{\bf Bemerkung:}
Diese Formel h\"atte auch zur Definition von $\diffd$ benutzt werden k\"onnen.
Sie gibt einen koordinatenfreien Ausdruck f\"ur das Differential, ist
allerdings sehr umst\"andlich f\"ur Beweise.

\bigskip

{\bf Beispiele:}
\begin{enumerate}
  \item Sei $\omega\in \Omega^1(M)$, so ist
\begin{align*}
\domega(X,Y) = L_X(\omega(Y))-L_Y(\omega(X))-\omega([X,Y]).
\end{align*}
\item Sei $\omega\in \Omega^2(M)$, so ist
\begin{align*}
\domega(X,Y,Z) &= L_X(\omega(X,Y)) - L_Y(\omega(X,Z)) + L_Z(\omega(Y,Z)) \\ &-
\omega([X,Y],Z) + \omega([X,Z],Y) - \omega([Y,Z],X).
\end{align*}
\end{enumerate}

\subsection{Die de Rahm - Kohomologie}

{\bf Bemerkung:}
$\diffd^2 = 0$ entspricht der Aussage, dass die gemischten partiellen
Ableitungen gleich sind (Satz von Schwarz); eine andere Interpretation f\"uhrt
auf die de Rham Kohomologie.

\bigskip

\begin{Definition}
\begin{enumerate}
  \item $\alpha\in\Omega^*(M)$ hei\ss{}t \emph{geschlossen}, falls $\diffd\alpha =
  0$.
  \item $\alpha\in \Omega^*(M)$ hei\ss{}t \emph{exakt}, falls $\alpha =\diffd\beta$
  f\"ur ein $\beta\in\Omega^*(M)$.
\end{enumerate}
\end{Definition}

\bigskip

{\bf Bemerkung:} Es gilt $\im (\diffd: \Omega^{k-1}\to \Omega^k) \subset \ker
(\diffd : \Omega^{k}\to \Omega^{k+1})$ und somit ist jede exakte Form geschlossen.
D.h. $\domega = 0$ ist eine notwendige Bedingung f\"ur die L\"osbarkeit der Gleichung
\begin{align*}
\omega = \diffd\eta,\qquad \eta\in\Omega^*(M).
\end{align*}

\bigskip

{\bf Beispiel:} Sei $\omega\in\Omega^1(\R^n)$, d.h. $\omega = \sum_{i} \omega_i
\dx_i$ mit $\omega_i\in\mathcal C^\infty(M)$. Gesucht ist nun eine Funktion
$f\in\mathcal C^\infty(\R^n)$ mit
\begin{align*}
\omega = \df = \sum_{i} \frac{\partial f}{\partial x_i}\dx_i.
\end{align*}
Nun ist $\omega = \df$ \"aquivalent zu
\begin{align*}
\omega_i = \frac{\partial f}{\partial x_i},\qquad i=1,\ldots,n.\tag{*}
\end{align*}
Eine notwendige Bedingung daf\"ur ist 
$\domega = \diffd^2f = 0$,
\begin{align*}
\domega = \sum_{i\neq j}\frac{\partial^2 f}{\partial x_i\partial
x_j}\dx_i\wedge\dx_j = 0\qquad\Leftrightarrow\qquad
\frac{\partial\omega_i}{\partial x_j} =
\frac{\partial \omega_j}{\partial x_i}\tag{**}
\end{align*}

\bigskip

{\bf Bemerkung:} Aus dem Satz von Frobenius aus der Analysis folgt, (*)
$\Leftrightarrow$ (**), d.h. auf $\R^n$ ist jede geschlossene 1-Form bereits
exakt.

\bigskip

{\bf Beispiele:}
\begin{enumerate}
\item Sei $\omega\in\Omega^1(\R^2)$ mit $\omega = f\dx + g\dy$ geschlossen.
Es gilt also
\begin{align*}
\domega = 0 \Leftrightarrow \frac{\partial f}{\partial y} = \frac{\partial
g}{\partial x}.
\end{align*}
Gesucht ist nun $\alpha\in\CC^\infty(\R^2)$ mit $\omega = \diffd\alpha$. Setzen
wir
\begin{align*}
\alpha(x,y) = \int_{x_0}^x f(t,y_0)\dt + \int_{y_0}^y g(x,t)\dt,
\end{align*}
dann gilt
\begin{align*}
\diffd\alpha &= \frac{\partial \alpha}{\partial x} \dx + \frac{\partial
\alpha}{\partial y}\dy 
= \left(f(x,y_0) + \int_{y_0}^y \partial_x g(x,t)\dt\right) \dx
+ g(x,y)\dy\\
&=\left(f(x,y_0) + \int_{y_0}^y \partial_y f(x,t)\dt\right) \dx
+ g(x,y)\dy\\
&= f\dx + g\dy = \omega,
\end{align*}
nach dem Hauptsatz der Integral- und Differentialrechnung.

\item Sei $M=\R^2\setminus\setd{0}$ und
\begin{align*}
\omega = \frac{-y}{x^2+y^2}\dx + \frac{x}{x^2+y^2}\dy,
\end{align*}
dann ist $\omega$ geschlossen. Um dies einzusehen f\"uhren wir Polarkoordinaten
ein
\begin{align*}
\dx &= \diffd(r\cos\ph) = \cos\ph \diffd r - r\sin\ph\dph,\\
\dy &= \diffd(r\sin\ph) = \sin\ph \diffd r + r\cos\ph\dph.
\end{align*}
Dann ist $\omega = \dph$ und offensichtlich geschlossen. $\omega$ ist jedoch
nicht exakt, denn
\begin{align*}
\int_\gamma \omega = 2\pi,\qquad \text{f\"ur }\gamma \text{ die Kreislinie}.
\end{align*}
Und das Integral \"uber exakte Formen entlang geschlossener Kurven verschwindet.
\end{enumerate}

\bigskip

\begin{Definition}
Die \emph{de Rahm-Kohomologie} ist definiert durch
\begin{align*}
H_{dR}^k(M) &= \ker(\diffd: \Omega^k\to \Omega^{k+1})/\im(\diffd:
\Omega^{k-1}\to \Omega^k)\\
&=\text{Menge der \"Aquivalenzklassen geschlossener Formen}.
\end{align*}
\end{Definition}

\bigskip

{\bf Bemerkungen:}
\begin{enumerate}
  \item Alle $H_{dR}^k(M)$ sind Vektorr\"aume.
  \item $H_{dR}^k(M) = \setd{0}$, falls $k> \dim M$.
  \item $H_{dR}^k(M)$ sind topologische Invarianten.
  \item $H_{dR}^k(M) = \setd{0}$ genau dann, wenn jede geschlossene $k$-Form
  auch exakt ist.
  \item $H_{dR}^0(M) = \setdef{f}{\df=0} = \setdef{f}{f\text{ lokal konstant}}
  = \R^r$, falls $r$ die Anzahl der Zusammenhangskomponenten von $M$ ist.
  \item $\setd{H_{dR}^k(M)}$ ist ein Spezialfall folgender Situation: Man hat
  eine Folge von Vektorr\"aumen $(V_k)$ mit linearen Abbildungen
  \begin{align*}
  d_k: V_k\to V_{k+1},\qquad d_k\circ d_{k-1} = 0.\\
  \ldots \longrightarrow V_{k-1}\overset{d_{k-1}}{\longrightarrow} V_k
  \overset{d_{k}}{\longrightarrow} V_{k+1}\longrightarrow \ldots\qquad (V)
  \end{align*}
  Man definiert nun die Kohomologie des Komplexes (V) als
  \begin{align*}
  H^k(V) = \ker d_k/\im d_{k-1}.
  \end{align*}
  \item Ist $M$ kompakt, zusammenh\"angend und orientierbar, so ist $H_{dR}^n(M)
  = \R$.
  \item $H_{dR}^k(S^n) = \setd{0}$ f\"ur $0<k<n$.
  \item $H_{dR}^1(T^2) = \R^2$, $H_{dR}^1(S^1) = \setd{0}$ und folglich ist
  $T^2\neq S^2$.
\end{enumerate}

\bigskip

\begin{Definition}
Eine Mannigfaltigkeit $M$ hei\ss{}t \emph{kontrahierbar} auf $p_0\in M$, falls eine
differenzierbare Abbildung 
\begin{align*}
H: M\times I\to M,\qquad I =[0,1]
\end{align*}
existiert mit
\begin{enumerate}
  \item $H(p,1) = p$,\qquad $\forall p\in M$,
  \item $H(p,0) = p_0$,\qquad $\forall p\in M$. 
\end{enumerate}
D.h. es gibt eine differenzierbare Homotopie zwischen der Identit\"at auf $M$ und
der konstanten Abbildung $p\mapsto p_0$.
\end{Definition}

\bigskip

\begin{Lemma}[Lemma von Poincare]
Sei $M$ eine kontrahierbare Mannigfaltigkeit. Dann gilt
\begin{align*}
H_{dR}^k(M) = \setd{0},\qquad k > 0.
\end{align*}
\end{Lemma}


\bigskip

{\bf Beispiele:}
\begin{enumerate}
  \item Sternf\"ormige Mengen sind kontrahierbar.
  \item $\R^n$ ist kontrahierbar.
\end{enumerate}

\newcommand{\rot}{\mathrm{rot}}
\renewcommand{\div}{\mathrm{div}}

\subsection{Bemerkungen zu $\rot$, $\grad$ und $\div$}

Sei $f: U\subset\R ^n\to \R$ eine differenzierbare Funktion.

\begin{Definition}
Das \emph{Gradientenvektorfeld} von $f$ ist definiert als
\begin{align*}
\grad f = \sum_{i=1}^n \frac{\partial f}{\partial x_i}\frac{\partial}{\partial
x_i}.
\end{align*}
\end{Definition}

\bigskip

{\bf Bemerkungen:}
\begin{enumerate}
  \item Sei $\scp{\cdot,\cdot}$ das Standard-Skalarprodukt im $\R^n$. Dieses
  \"ubertr\"agt sich auf $T_p\R^n$ und es ist
\begin{align*}
\scp{\grad f, X} = \df(X).
\end{align*}
Man definiert nun mit Hilfe des Skalarproduktes (sp\"ater werden wir dieses
durch die Metrik ersetzen) folgende Abbildung
\begin{align*}
{}^*: \chi(\R^n) \to \Omega^1(\R^n),\qquad
X = \sum a_i \frac{\partial }{\partial x_i} \mapsto X^* := \sum a_i \dx_i,
\end{align*}
wobei $X^*(Y) = \scp{X,Y}$. Insbesondere ist dann $(\grad f)^* = \df$.
\item Sei nun $n=3$, dann definiert man den \emph{Hodge-Operator}
\begin{align*}
* : \Omega^k(\R^3)\to \Omega^{3-k}(\R^3).
\end{align*}
F\"ur $k=0$ ist $*$ gegeben als
\begin{align*}
* : &\;\mathcal C^\infty(\R^3)\to \Omega^3(\R^3),\\ 
&\quad f\qquad\mapsto * f = f \dx_1\wedge
\dx_2\wedge \dx_3.
\end{align*}
F\"ur $k=1$ ist $*$ gegeben als
\begin{align*}
* : &\; \Omega^1(\R^3)\qquad\quad\to \Omega^2(\R^3), \\
&\omega = \sum a_i \dx_i\;\, \mapsto *\omega = a_1 \dx_2\wedge \dx_3 + a_2
\dx_3 \wedge \dx_1 + a_3 \dx_1\wedge \dx_2
\end{align*}
\item $*^2 = \id$.
\end{enumerate}

\bigskip

F\"ur alles Weitere beschr\"anken wir uns auf den $\R^3$.

\begin{Lemma}
\begin{enumerate}
  \item $(\grad f)^* = \df$.
  \item $\diffd X^* = *((\rot X)^*)$.
  \item $\diffd(*(X^*)) = *(\div X)$.
\end{enumerate}
\end{Lemma}
\proof
1. folgt direkt aus der Definition von $\grad f$ und der Charakterisierung von
$(\grad f)^*$.

\medskip

Zu 2.: Sei $X= \sum a_j \frac{\partial}{\partial x_j}$ mit $a_j\in \mathcal
C^\infty(\R^3)$. Dann ist
\begin{align*}
\diffd X^* &= \diffd\left( \sum a_j \dx_j \right)
= \sum \diffd a_j \wedge \dx_j  = 
\sum_{i,j} \frac{\partial a_j}{\partial x_i} \dx_i\wedge \dx_j\\
&= \left(\frac{\partial a_3}{\partial x_2} - \frac{\partial a_2}{\partial x_3}
\right)\dx_2\wedge \dx_3 +
\left(\frac{\partial a_1}{\partial x_3} - \frac{\partial a_3}{\partial x_1}
\right)\dx_3\wedge \dx_1 + 
\left(\frac{\partial a_2}{\partial x_1} - \frac{\partial a_1}{\partial x_2}
\right)\dx_1\wedge \dx_2.
\end{align*}
Zu 3.: Man rechnet direkt nach
\begin{align*}
\diffd(*(X^*)) &= \diffd( a_1 \dx_2\wedge \dx_3 + a_2\dx_3\wedge\dx_1 + a_3
\dx_1\wedge \dx_2 )\\
&= \frac{\partial a_1}{\partial x_1} \dx_1\wedge\dx_2\wedge \dx_3
+ \frac{\partial a_2}{\partial x_2} \dx_2\wedge\dx_3\wedge \dx_1
+ \frac{\partial a_3}{\partial x_3} \dx_3\wedge\dx_1\wedge \dx_2\\
&=\left( \frac{\partial a_1}{\partial x_1} 
+ \frac{\partial a_2}{\partial x_2} 
+ \frac{\partial a_3}{\partial x_3}  \right)
\dx_1\wedge\dx_2\wedge \dx_3
= *(\div X).
\end{align*}
\qed

\bigskip

\begin{Folgerung}
F\"ur alle Funktionen $f\in\mathcal C^\infty(\R^3)$ und $X\in\chi(\R^3)$ gilt
\begin{enumerate}
  \item $\rot(\grad f) = 0$,
  \item $\div(\rot X) = 0$.
\end{enumerate}
\end{Folgerung}

\proof
1: Es gen\"ugt zu zeigen, dass $*((\rot(\grad f))^*) = 0$, aber
\begin{align*}
*((\rot(\grad f))^*) = \diffd (\grad f)^* = \diffd^2 f = 0.
\end{align*}
2: Es gen\"ugt zu zeigen, dass $*(\div(\rot X)) = 0$, aber
\begin{align*}
*(\div(\rot X)) = \diffd(*((\rot X)^*)) = \diffd^2 X^* =0.
\end{align*}
\qed

\bigskip

{\bf Bemerkung:}
$\div(\grad f )) = \Delta f$ ist der \emph{Laplace-Operator}.

\bigskip

\begin{Folgerung}
Sei $X$ ein Vektorfeld auf einer kontrahierbaren offenen Menge $U\subset\R^3$.
Dann gelten
\begin{enumerate}
  \item Ist $\rot X = 0 $, so existiert eine glatte Funktion $f: U\to R$ mit
  $X= \grad f$.
  \item Ist $\div X = 0$, so existiert ein Vektorfeld $Y\in\chi(U)$ mit $X =
  \rot Y$.
\end{enumerate}
\end{Folgerung}

\proof
1: Ist $\rot X = 0$, so folgt $\diffd X^* = 0$. Nach dem Lemma von Poincare
existiert nun eine glatte Funktion $f: U\to \R$, so dass $X^*= \df$ und folglich
$X = \grad f$.

\medskip

2: Ist $\div X = 0$, so ist $\diffd(*(X^*)) = 0$ und folglich existiert eine
Form $\eta\in\Omega^1$
\begin{align*}
\eta = Y_1\dx_1 + Y_2\dx_2 + Y_3\dx_3,
\end{align*} 
so dass $*(X^*) = \diffd\eta = \diffd Y^* = *(\rot Y)^*$. Somit ist $X= \rot Y$.
\qed 

\section{Riemannsche Metriken}


Man definiert zun\"achst das Vektorb\"undel der symmetrischen Bilinearformen auf $TM$:
$$
\Sym^2 TM \;:=\; \bigcup_{p\in M} \, \Sym^2 T_pM
$$

wobei: \quad
$
\Sym^2 T_pM := \{b : T_p M \times T_pM \rightarrow \R \,  | \,  b \; \mbox{symmetrische Bilinearform}\} \ .
$

\begin{Definition}
Eine \em{Riemannsche Metrik} auf $M$ ist ein glatter Schnitt $g\in \Gamma(\Sym^2 TM)$, f\"ur
den $g_p$ f\"ur alle $p\in M$ positiv definit ist.
\end{Definition}

{\bf Bemerkungen:}
\begin{enumerate}
 \item
Eine Metrik ist ein symmetrischer $(0,2)$-Tensor, d.h. eine symmetrische
$\mathcal C^\infty(M)$-bilineare Abbildung:
$$
g : \chi(M) \times \chi(M) \rightarrow \mathcal C^\infty(M) \ ,
$$
die punktweise positiv-definit ist.
 \item
Pseudo-Riemannsche Metriken sind nicht-entartete, symmetrische Bilinearformen mit beliebiger
Signatur. Lorentz-Metriken sind Metriken der  Signatur $(n,1)$.
 \item
Lokal, also in einer Karte $(U,x)$, schreibt sich eine Metrik als
$$
g = \sum g_{ij} dx_i \otimes dx_j \ ,
$$
wobei sich die Funktionen $g_{ij} \in \mathcal C^\infty(U)$ bestimmen durch:
$
g_{ij} = g \left(\frac{\partial}{\partial x_i}, \frac{\partial }{ \partial
x_j}\right) \ . $
\end{enumerate}

\bigskip

Metriken sind $(0,2)$-Tensoren, haben also folgendes Transformationsverhalten:
Seien $(U, x), (V, y)$  zwei Karten um $p \in M$. Dann schreibt sich eine Metrik
$g$ auf $U \cap V$ als:
$$
g = \sum g_{ij}^x dx_i \otimes dx_j = \sum g_{ij}^y dy_i \otimes dy_j \ .
$$
Setzt man
$
\frac{\partial}{\partial y_i } = \sum_k \frac{\partial x_k }{\partial y_i} \frac{\partial}{\partial x_k}
$
in die definierende Gleichung f\"ur $g^y_{ij}$ ein, erh\"alt man das
Transformationsverhalten $$
g_{ij}^y
=
g\left(\frac{\partial}{\partial y_i}, \frac{\partial}{\partial y_k}\right)
=
\sum_{k, l} \frac{\partial x_k}{\partial y_i}\cdot \frac{\partial x_l }{\partial y_j}\,g_{kl}^x \ .
$$


Eine Riemannsche Metrik l\"a\ss t sich auch beschreiben, als eine Familie von lokalen
Funktionen $\{ g_{ij} \}$, die dieses Transformationsverhalten bei Kartenwechsel haben und punktweise
Skalarprodukte definieren.


\bigskip

\begin{Satz}
 Auf jeder zusammenh\"angenden Mannigfaltigkeit existiert eine Riemannsche Metrik $g$.
\end{Satz}
\proof
Sei $(U_k,f_k)$ ein Atlas von $M$ und $\setd{\ph_k}$ eine
Zerlegung der Eins zu $\setd{U_k}$. Dann ist $\setd{U_k}$ lokal endlich, d.h.
jeder Punkt von $M$ hat eine Umgebung, die nur endlich viele offene Mengen $U_k$
schneidet. Weiterhin sei $q$ ein Skalarprodukt auf $\R^n$, dann l\"asst sich
dieses lokal auf $M$ zur\"uckziehen. Man definiert nun global 
$$
g = \sum_{k} \ph_k \cdot f_k^*q,
$$
wobei $g$ aufgrund der Zerlegung der Eins wohldefiniert ist. Au\ss{}erdem ist $g$ 
eine glatte Abbildung und f\"ur jedes $p\in M$ ist $g_p$ eine symmetrische
Bilinearform. Es verbleibt zu zeigen, dass $g$ tats\"achlich positiv definit ist. Sei $p\in M$, dann existiert ein
$k_0$, so das $\ph_{k_0}(p) \neq 0$ und folglich, gilt f\"ur $0\neq X\in T_pM$
\begin{align*}
g_p(X,X) &= \sum_k \ph_k(p) q(\df_k(X),\df_k(X))\\
&\ge \ph_{k_0}(p) q(\df_{k_0}(X),\df_{k_0}(X)) > 0.
\end{align*}
\qed

\bigskip

{\bf Bemerkung:}
Auf $S^2$ existiert keine Lorentz-Metrik. Aus deren Existenz w\"urde
auch die Existenz eines nullstellenfreien Vektorfeldes auf $S^2$ folgen. Ob eine
Lorentz-Metrik existiert, h\"angt also von der Topologie auf $M$ ab, w\"ahrend
eine Riemannsche-Metrik immer existiert.

\bigskip

\begin{Definition}
Seien $(M,g)$, $(N,h)$ zwei Riemannsche Mannigfaltigkeiten und $f: M\to N$ eine
differenzierbare Abbildung. Man nennt $f$ eine \emph{Isometrie}, falls $f$ ein
Diffeomorphismus ist mit
\begin{align*}
g = f^*h,\qquad \text{d.h.}\qquad
g_p(X,Y) = h_{f(p)}(\df(X),\df(Y)),
\end{align*}
d.h. $\df_p$ ist eine Isometrie der euklidischen Vektorr\"aume $(T_pM,g_p)$ und
$(T_{f(p)}N,h_{f(p)})$.
\end{Definition}

\bigskip

{\bf Bemerkungen:}
\begin{enumerate}
  \item $\Id_M$ ist eine Isometrie.
  \item Sind $f_1$ und $f_2$ Isometrien, dann auch $f_1\cdot f_2$ und
  $f_1^{-1}$. Die Menge der Isometrien bildet also eine Gruppe -- sogar eine
  Lie-Gruppe, die Isometriegruppe $\Iso(M,g)$ von $(M,g)$.
  \item Vektorfelder, deren lokaler Flu\ss{} aus lokalen Isometrien besteht,
  d.h.
  \begin{align*}
  \ph_t^* g = g,\qquad \text{f\"ur alle }t \text{ f\"ur die }\ph_t\text{
  definiert ist},
  \end{align*}
  nennt man {\it Killing Vektorfelder}. Ein Vektorfeld $X$ ist genau dann ein Killing Vektorfeld,
  wenn $L_Xg = 0$ gilt.
\end{enumerate}

\bigskip

{\bf Beispiele:} 
\begin{enumerate}
  \item Auf $\R^n$ induziert das Standard-Skalarprodukt eine
Riemannsche Metrik
$$
g = \sum \dx_i \otimes \dx_i = \sum \dx_i^2.
$$
Im Spezialfall $\R^2$ mit Standardkoordinaten $(x,y)$ ist also $g= \dx^2 +
\dy^2$. F\"uhren wir Polarkoordinaten ein durch die Transformation
$$
x = r\cos \ph,\qquad y = r\sin\ph,
$$
dann gilt
\begin{align*}
\dx &= \frac{\partial x}{\partial r}\dr + \frac{\partial x}{\partial \ph}\dph =
\cos\ph \dr - r\sin\ph \dph,\\
\dy &= \frac{\partial y}{\partial r}\dr + \frac{\partial y}{\partial \ph}\dph =
\sin\ph \dr + r\cos\ph \dph,
\end{align*}
und folglich ist $g = \dr^2 + r^2\dph^2$.

\item Sei $(M,g_M)$ eine Riemannsche Mannigfaltigkeit und durch $i: N\to M$ eine
Untermannigfaltigkeit gegeben. Dann induziert $g_M$ eine Metrik auf $N$ durch
$$
g_N = i^* g_M,
$$
d.h. $g_N = g_M\big|_{TN\times TN}$, wobei $TN\subset TM$ wieder als Unterraum
identifiziert wird.

Betrachte $S^n\subset\R^{n+1}$, so ist $T_p S^n = p^\bot \subset\R^{n+1}$ und
$$
g(X,Y) = \scp{X,Y},\qquad X,Y\in T_pS^n\subset\R^{n+1},
$$
d.h. $g_p$ ist die Einschr\"ankung des euklidischen Standard-Skalarprodukts auf
$\R^{n+1}$.

Der \textit{Satz von Nash} besagt allgemein, dass jede Riemannsche
Mannigfaltigkeit der Dimension $n$ eine Riemannsche Untermannigfaltigkeit des
$\R^{\frac{1}{2}n(n+1)(3n+11)}$ ist.
\item Auf dem $n$-dimensionalen hyperbolischen Raum,
$$
H^n = \setdef{(x_1,\ldots,x_n)\in\R^n}{x_n > 0}.
$$
ist eine Riemannsche Metrik gegeben durch
$$
g = \frac{1}{x_n^2}\sum \dx_i \otimes \dx_i.
$$
Alternativ kann man $H^n$ auch als Poincare-Modell betrachten,
$$
H^n\cong \setdef{(x_1,\ldots,x_n)\in\R^n}{\sum x_i^2 < 1},
$$
dann ist die Riemannsche Metrik gegeben durch
$$
g = \frac{4}{(1-r^2)^2}\sum\dx_i^2,\qquad r^2 = \sum x_i^2.
$$ 
\end{enumerate}

\bigskip

\begin{Definition}
Seien $(M,g)$ und $(N,h)$ Riemannsche Mannigfaltigkeiten, dann ist auch
$M\times N$ eine Riemannsche Mannigfaltigkeit mit der Produktmetrik $g\times h
= g+h$, wobei $T_{(p,q)} (M\times N) = T_pM \oplus T_qN$.
\end{Definition}

\bigskip



{\bf Beispiele:} 

\begin{enumerate}
  \item Sei $G$ eine Lie-Gruppe und $\scp{\cdot,\cdot}$ das
  Standard-Skalarprodukt auf $\g = \Lie(G) = T_eG$. Dann definiert
  $$
  B_g(X,Y) = \scp{dl_{g^{-1}}(X),dl_{g^{-1}}(Y)},
  $$
  eine glatte Metrik auf $G$. Au\ss{}erdem ist $B$ linksinvariant, d.h.
  $$
  l_g^* B = B
  $$
  und folglich $l_g\in \Iso(G,B)$ f\"ur alle $g\in G$.
  \item Sei $G$ kompakt und halbeinfach, dann definiert die Killing-Form ein
  negativ-definites bi-invariantes Skalarprodukt,
  \begin{align*}
  B(X,Y) &\;\,= -\tr(\ad(X)\circ\ad(Y))\\
  &\overset{\text{``oft''}}{=} -\tr(XY).
  \end{align*}
\end{enumerate}

\bigskip

F\"ur alles Weitere sei $(M,g)$ stets eine zusammenh\"angende Riemannsche
Mannigfaltigkeit.

\begin{Definition}
Die \emph{L\"ange} einer Kurve $\gamma: [a,b]\to M$ ist definiert durch
$$
L[\gamma] = \int_a^b \norm{\dot{\gamma}(t)}\dt,
$$
wobei $\norm{\dot{\gamma}(t)} = \sqrt{g(\dot{\gamma}(t),\dot{\gamma}(t))}$.
\end{Definition}

\bigskip

{\bf Bemerkung:} Der \emph{Abstand} zweier Punkte $p,q\in M$ ist definiert als
$$
d(p,q) = \inf_{\gamma} L[\gamma],
$$
wobei das Infimum \"uber alle st\"uckweise glatten Kurven $\gamma$ zu nehmen ist,
die $p$ und $q$ verbinden.

\bigskip

\begin{Lemma}
Die Abbildung $d: M\times M\to \R$ ist eine Metrik, d.h.
\begin{enumerate}
  \item $d$ ist \emph{symmetrisch},
  \item $d$ erf\"ullt die \emph{Dreiecksungleichung},
  \item $d$ ist \emph{positiv definit}, d.h. $d(p,q) = 0$
  genau dann, wenn $p=q$.
\end{enumerate}
\end{Lemma}

\proof
1. und 2. sind klar. Es verbleibt die positive Definitheit zu zeigen. Seien
dazu $p,q\in M$ mit $d(p,q)=0$.
Angenommen $p\neq q$, dann w\"ahlen wir eine Karte $(U,x)$
um $p$ mit $x(p) = 0$ und $O = x^{-1}(B_r(0))$ mit $\bar{O}\subset U$ und
$q\notin O$. Man definiert nun eine Abbildung $$
f: \R^n\times U \to \R,\qquad f(a_1,\ldots,a_n,p) = \norm{\sum_{i=1}^n a_i
\frac{\partial}{\partial x_i}\bigg|_p}_g.
$$
Dann ist $f$ eine stetige, nichtnegative Funktion und folglich existiert ein $k
>0 $, so dass
$$
\frac{1}{k} \le f\big|_{S^{n-1}\times \bar{O}} \le k.
$$
Sei $\norm{\cdot}'$ die Norm auf $\R^n$, dann gilt f\"ur $(a,p)\in S^{n-1}\times
O$,
$$
\norm{\sum_{i=1}^n a_i \frac{\partial}{\partial x_i}\bigg|_p}' =
\left(\sum_{i=1}^n a_i^2\right)^{1/2} = 1.
$$
Somit ist 
\begin{align*}
\frac{1}{k}\norm{\sum_{i=1}^n a_i \frac{\partial}{\partial x_i}\bigg|_p}'
\le \norm{\sum_{i=1}^n a_i \frac{\partial}{\partial x_i}\bigg|_p}
\le k\norm{\sum_{i=1}^n a_i \frac{\partial}{\partial x_i}\bigg|_p}'\tag{*}
\end{align*}
Da (*) homogen in $a_i$, folgt, dass (*) auf ganz $\R^n\times\bar{O}$ gilt.

Sei nun $\gamma$ eine st\"uckweise glatte Kurve von $p$ nach $q$ und $\gamma'$ sei
der Teil von $\gamma$ bis zum ersten Schnittpunkt mit $\partial O$, $\alpha$ sei
der Radius von $O$, dann ist
\begin{align*}
d(p,q) &= \inf_\gamma L(\gamma)
\ge \inf_\gamma L(\gamma')\\
&\ge \frac{1}{k}\inf L(\gamma')' \\ &\ge \frac{1}{k}\alpha > 0,
\end{align*}
wobei $L(\gamma')'$ die L\"ange bez\"uglich der durch das Standardskalarprodukt
induzierten Norm $\norm{\cdot}'$ bezeichnet.
\qed

\bigskip

{\bf Bemerkung:} Sei $(M,g)$ eine Riemannsche Mannigfaltigkeit und $d$ die
induzierte Metrik auf $M$. Dann definiert $d$ eine Topologie auf $M$ mit offenen
Mengen
$$
\tilde{O}_M = \bigcup_{p,q\in M,\;k\in\N} B_{1/k}^d(p,q).
$$

\begin{Satz}
Die durch $d$ induzierte Topologie stimmt mit der urspr\"unglichen \"uberein.
\end{Satz}
\bigskip

\section{Zusammenh\"ange und kovariante Ableitungen}


\subsection{Kovariante Ableitungen auf dem Tangentialb\"undel}

\bigskip

In diesem Abschnitt soll eine Abbildung konstruiert werden, die zwei Vektorfeldern $X, Y$ ein neues Vektorfeld $\nabla_XY$ zuordnet,
die kovariante Ableitung von $Y$ nach $X$. Dieses Vektorfeld beschreibt in jedem Punkt $p \in M$ die infinitesimale \"Anderung von
$Y$ in Richtung $X_p$.

\bigskip

Auf dem $\R^n$ l\"a\ss t sich $\nabla$ folgenderma\ss en definieren: Seien $X = \sum a_i \frac{\partial}{\partial x_i}$
und $Y = \sum b_j \frac{\partial}{\partial x_j}$ zwei glatte Vektorfelder auf $\R^n$, d.h. $a_i$ und $b_i$ sind
glatte Funktionen auf $\R^n$. Dann definiert man:
$$
\nabla_XY = \sum_j X(b_j)\,\frac{\partial}{\partial x_j} = \sum a_i \frac{\partial b_j}{\partial x_i}
 \frac{\partial}{\partial x_j} = X(Y)\ ,
$$
wobei $X(Y)$ als komponentenweise Ableitung von $Y$ nach $X$ zu verstehen ist. Man nennt $\nabla$ den {\it flachen
Zusammenhang } auf $\R^n$.

\bigskip

Diese Definition ist aber  nicht koordinaten-unabh\"angig und kann somit nicht auf beliebige Mannigfaltigkeiten \"ubertragen
werden. Hier nutzt man folgende

\begin{Definition}
Ein {\em Zusammenhang} (auch {\em kovariante Ableitung}) auf $M$ ist eine Abbildung
$$
\nabla : \chi(M) \times \chi (M) \rightarrow \chi(M), \quad (X, Y) \mapsto \nabla_XY
$$
so dass f\"ur beliebige Vektorfelder $X, Y$ und alle Funktionen $f \in \mathcal C^\infty(M)$ die folgenden Eigenschaften erf\"ullt sind:
\begin{enumerate}
\item
\quad $\nabla_{f\cdot X}Y = f\cdot \nabla_XY$
\item
\quad $\nabla_X(f \cdot Y) = X(f)\,Y + f \cdot \nabla_XY$
\end{enumerate}
\end{Definition}


\bigskip

{\bf Bemerkung:}
Aus 1. folgt, dass f\"ur ein festes $Y$ die Abbildung $X \mapsto \nabla_XY$ ein $(1,1)$-Tensor ist.
Daraus folgt, dass man in jedem Punkt $p\in M$ eine wohldefinierte Abbildung
$$
T_pM \rightarrow T_pM, \qquad v \mapsto (\nabla_XY)(p)
$$
hat, wobei $X$ eine Fortsetzung von $v$ zu einem Vektorfeld auf $M$ ist.

\bigskip


{\bf Beispiel:}
Sei $M\subset\R^n$ eine Untermannigfaltigkeit, dann ist f\"ur $X,Y\in\chi(M)$
$$
\nabla_X Y = \pr_M(\tilde{X}(\tilde{Y})),
$$
f\"ur beliebige Fortsetzungen $\tilde{X}$, $\tilde{Y}$ auf $\R^n$.

\bigskip

\begin{Definition}
Die {\em Torsion} $T$  eines Zusammenhangs $\nabla$ auf $TM$ ist definiert als
die Abbildung $$
T: \chi(M) \times \chi(M) \rightarrow \chi(M), \qquad (X, Y) \mapsto \nabla_XY - \nabla_YX - [X,Y] \ .
$$
Ein Zusammenhang hei\ss t {\em torsionsfrei} (oder auch symmetrisch), falls seine Torsion 
verschwindet, d.h. $T \equiv 0$  gilt.
\end{Definition}



\bigskip

\begin{Lemma}
Die Torsion $T$ ist ein schiefsymmetrischer $(1,2)$-Tensor, d.h. $T(X,Y)$ ist 
$\mathcal C^\infty (M)$-linear in $X$ und $Y$ und es gilt f\"ur beliebige 
Vektorfelder $X,Y$:
$$
T(X,Y) \;=\; - \, T(Y,X) \ .
$$
\end{Lemma}
\proof
Die Schiefsymmetrie ist klar. F\"ur $f\in \mathcal C^\infty(M)$ rechnet man
direkt nach,
\begin{align*}
T(f X,Y) &= \nabla_{fX} Y - \nabla_Y (fX) - [fX,Y]\\
&= f\nabla_XY - f\nabla_Y(X) - Y(f)X - f[X,Y] + Y(f)X\\
&= fT(X,Y).
\end{align*}
\qed

\bigskip

\subsection{Die lokale Beschreibung von Zusammenh\"angen}

\bigskip

Sei $\nabla$ ein Zusammenhang auf $TM$. Dann kann man $\nabla$ in lokalen
Karten $(U,x)$ um $p \in M$ beschreiben durch eine Familie von $n^3$ Funktionen.

\begin{Definition}
Die  {\em Christoffel-Symbole} bzgl. $(U,x)$ sind die lokalen Funktionen
$\Gamma^k_{ij} \in \mathcal C^\infty(M), \; i,j,k = 1,\ldots, n$ definiert
durch die Gleichungen
$$
\nabla_{ \tfrac{\partial }{\partial x_i} } \tfrac{\partial }{\partial x_j}
\;=\;
\sum_k \, \Gamma^k_{i j}\, \tfrac{\partial }{\partial x_k} \ .
$$
\end{Definition}

\bigskip

{\bf Bemerkung:}
Der Zusammenhang $\nabla$ ist durch seine Christoffel-Symbole festgelegt, d.h.
seien $X, Y$ Vektorfelder auf $M$, die sich lokal auf $U$ schreiben als
$
X = \sum a_i \tfrac{\partial }{\partial x_i}, \; 
Y = \sum b_i \tfrac{\partial }{\partial x_i} 
$.
Dann gilt auf $U$:
$$
\nabla_XY \;=\; \sum \left( a_i \tfrac{\partial b_k}{\partial x_i}
\,+\,a_i b_j \Gamma^k_{ij} \right) \tfrac{\partial }{\partial x_k}
$$
Unter Ausnutzung der Eigenschaften von $\nabla$ berechnet man auf $U$:
\begin{align*}
\nabla_XY
& = 
\nabla_{ \sum a_i \tfrac{\partial }{\partial x_i} } \left( \sum b_j
\tfrac{\partial }{\partial x_j}\right)\\[1.5ex]
& = \sum_{i,j} a_i
\left(\frac{\partial b_j}{\partial x_i}\frac{\partial}{\partial x_j}+
b_j\nabla_{\tfrac{\partial}{\partial x_i}}\frac{\partial}{\partial x_j}
\right)\\
&= 
\sum_{i,k} 
\left(a_i\frac{\partial b_k}{\partial x_i} + 
\left(\sum_j a_i b_j \Gamma_{ij}^k\right)
\right)\frac{\partial}{\partial x_k}
% 
% 
% \frac{\partial}{\partial x_j}+
% b_j\nabla_{\tfrac{\partial}{\partial x_i}}\frac{\partial}{\partial x_j}
% \right)
\end{align*}

\bigskip

\begin{Lemma}
Der Zusammenhang $\nabla$ ist genau dann torsionsfrei, wenn die Christoffel-Symbole symmetrisch sind,
d.h. wenn $\Gamma^k_{ij} = \Gamma^k_{ji}$ f\"ur alle Karten und alle Indizes $i,j, k$ gilt. 
\end{Lemma}
\proof
Da die Torsion $\mathcal C^\infty$-linear in beiden Eintr\"agen ist, gen\"ugt es zu
zeigen
\begin{align*}
\nabla_{\frac{\partial}{\partial x_i}}\frac{\partial}{\partial x_j} - 
\nabla_{\frac{\partial}{\partial x_j}}\frac{\partial}{\partial x_i} = 0\quad
\Leftrightarrow\quad \Gamma_{ij}^k = \Gamma_{ji}^k,\qquad \forall i,j,k.
\end{align*}
Dies folgt aber direkt aus der Definition der Christoffel-Symbole.
% Bez\"uglich einer Karte $(U,x)$ ist nach der vorangegangenen Bemerkung,
% \begin{align*}
% \nabla_X Y - \nabla_Y X &= 
% \sum_{i,j}\left(a_i \frac{\partial b_j}{\partial x_i} - b_i \frac{\partial
% a_j}{\partial x_i} \right)\frac{\partial}{\partial x_j}
% +\sum_{i,j,k} \left( a_ib_j \Gamma_{ij}^k - a_jb_i\Gamma_{ji}^k
% \right)\frac{\partial}{\partial x_k}\\ &= [X,Y] + 
% \sum_{i,j,k} \left( a_ib_j \Gamma_{ij}^k - b_ja_i\Gamma_{ji}^k
% \right)\frac{\partial}{\partial x_k}.
% \end{align*}
% $\nabla$ ist genau dann torsionsfrei, wenn $\nabla_X Y - \nabla_Y X - [X,Y] =
% 0$, d.h. wenn
% \begin{align*}
% 0 = \sum_{i,j,k} \left( a_ib_j \Gamma_{ij}^k - b_ja_i\Gamma_{ji}^k
% \right)\frac{\partial}{\partial x_k} = 
% \sum_{i,j,k} a_ib_j \left( \Gamma_{ij}^k - \Gamma_{ji}^k
% \right)\frac{\partial}{\partial x_k}.
% \end{align*} 
% Dies ist aber genau dann der Fall, wenn die Christoffel-Symbole symmetrisch
% sind.
\qed

\bigskip

{\bf Beispiel:}  Die Christoffel-Symbole des flachen Zusammenhangs auf $\R^n$ sind alle Null und
damit insbesondere torsionsfrei.

\bigskip

\begin{Definition}
Ein Vektorfeld $V$ auf $M$ hei\ss t {\em parallel}, falls $\nabla_XV= 0$ f\"ur beliebige
Vektorfelder $X$ auf $M$ erf\"ullt ist. 
\end{Definition}

\bigskip

{\bf Bemerkung:} Auf dem $\R^n$ sind alle Koordinatenvektorfelder $ \tfrac{\partial }{\partial x_i} $
parallel. Im Allgemeinen, auf beliebigen Mannigfaltigkeiten, sind die Koordinatenvektorfelder
genau dann parallel, wenn
$$
\nabla_X  \tfrac{\partial }{\partial x_i} = 0 \quad \forall  X
\quad\Leftrightarrow\quad
\nabla_{  \tfrac{\partial }{\partial x_i} } \tfrac{\partial }{\partial x_j} = 0
\quad\forall i,j
\quad\Leftrightarrow\quad
\Gamma^k_{ij} = 0 \quad\forall i,j,k \ .
$$
d.h. die Christoffel-Symbole $\Gamma^k_{ij}$ messen die Abweichung der Koordinatenvektorfelder
$ \tfrac{\partial }{\partial x_i} $  parallel zu sein.

\bigskip

\subsection{Der Levi-Civita-Zusammenhang}

\bigskip

Sei $(M,g)$ eine Riemannsche Mannigfaltigkeit.

\begin{Definition}
Ein Zusammenhang $\nabla$ hei\ss t {\em metrisch} bzgl. $g$. falls
$$
X(g(Y, Z)) \;=\; g(\nabla_XY, Z) \;+\; g(Y, \nabla_XZ)
$$ 
f\"ur alle Vektorfelder $X, Y, Z \in \chi(M)$ erf\"ullt ist.
\end{Definition}

\bigskip

{\bf Beispiel:} Der flache Zusammenhang auf $\R^n$ ist metrisch bzgl.
des Euklidischen Skalarproduktes.

\bigskip

\begin{Satz}[Fundamentallemma der Riemannschen Geometrie]
Auf einer Riemannschen Mannigfaltigkeit $(M, g)$ existiert ein eindeutig bestimmter torsionsfreier und
metrischer Zusammenhang.  Man nennt ihn den {\em Levi-Civita-Zusammenhang } von $(M,g)$.
\end{Satz}
\proof
F\"ur einen metrischen Zusammenhang $\nabla$ auf $(M,g)$ gilt,
\begin{align*}
X(g(Y,Z)) &= g(\nabla_X Y,Z) + g(Y,\nabla_X Z)\\
Y(g(Z,X)) &= g(\nabla_Y Z,X) + g(Z,\nabla_Y X)\\
Z(g(X,Y)) &= g(\nabla_Z X,Y) + g(X,\nabla_Z Y).
\end{align*} 
Ist $\nabla$ weiterhin torsionsfrei, so ergibt sich
\begin{align*}
X(g(Y,Z)) &+ Y(g(Z,X)) - Z(g(X,Y)) \\ &= g(\nabla_Y Z - \nabla_Z Y, X)
+ g(\nabla_X  Z - \nabla_Z X  ,Y) + g(\nabla_X Y +\nabla_Y X,Z)\\
&= g([Y,Z],X) + g([X,Z],Y) + g([Y,X] + 2\nabla_X Y,Z)
\end{align*}
Damit erh\"alt man die Koszul-Formel, die den Levi-Civita-Zusammenhang eindeutig bestimmt:
\begin{align*}
2\,g(\nabla_XY, Z)
& = X(g(Y, Z)) \;+\; Y(g(Z,X)) \;-\; Z(g(X,Y)) \\[1ex]
& \phantom{xx.xxxxxxxx} \;-\; g([Y,Z],X) \;-\; g([X,Z],Y) \;+\;
g([X,Y],Z)\tag{*}
\end{align*}
Es verbleibt zu zeigen, dass durch die rechte Seite von (*) tats\"achlich ein
metrischer, torsionsfreier Zusammenhang bestimmt wird. Wir zeigen zun\"achst, dass
(*) $\mathcal C^\infty$-linear in $X$ ist. Sei also $f\in\mathcal C^\infty(M)$,
\begin{align*}
2\,g(\nabla_{fX}Y, Z) &= f X(g(Y,Z)) + Y(f)g(Z,X) + f Y(g(Z,X)) \\ &- Z(f)g(X,Y)
- fZ(g(X,Y))\\
&- fg([Y,Z],X) - g(f[X,Y] - Z(f)X,Y) + 
g(f[X,Y] - Y(f)X,Z)\\
&= 2f\,g(\nabla_X Y,Z) + Y(f)g(Z,X) - Z(f)g(X,Y) + Z(f)g(X,Y) - Y(f)g(X,Z)\\
&=2f\,g(\nabla_X Y,Z).
\end{align*}
Analog zeigt man, dass
\begin{align*}
g(\nabla_X fY,Z) = fg(\nabla_X Y,Z) + X(f)g(\nabla_X Y,Z).
\end{align*}
Betrachtet man $g(\nabla_X Y,Z)+g(Y,\nabla_X Z)$ und beachtet dass in den
Summanden gerade die Rollen von $Y$ und $Z$ vertauscht sind, dann ergibt sich
unmittelbar, dass $\nabla$ metrisch ist. Analog \"uberzeugt man sich, dass
$\nabla$ torsionsfrei ist.
\qed

\bigskip

{\bf Beispiel:} Der flache Zusammenhang auf $\R^n$ ist der Levi-Civita
Zusammenhang von $(\R^n,\scp{\cdot,\cdot})$.
\begin{align*}
&T(X,Y) = X(Y) - Y(X) - [X,Y] = 0,\\
&X\scp{Y,Z} = X(Y Z^\top) = \scp{\nabla_X Y, Z} + \scp{Y,\nabla_X Z}.
\end{align*}

\bigskip

\subsection{Die Christoffel-Symbole des Levi-Civita-Zusammenhangs}

\bigskip

\begin{Lemma}
Sei $(M^n,g)$ eine Riemannsche Mannigfaltigkeit und seien $\Gamma^k_{ij}$
die Christoffel-Symbole des Levi-Civita-Zusammenhangs von $(M,g)$. Dann gilt
$$
\Gamma^l_{ij} 
\;=\;
\frac12 \, \sum_{k} g^{lk} 
\left(
 \frac{\partial  g_{jk}}{\partial x_i} \;+\; \frac{\partial
 g_{ik}}{\partial x_j} \;-\;
 \frac{\partial g_{ij}}{\partial x_k}   
\right) \ ,
$$ 
wobei $g$ lokal durch die Matrix $(g_{ij})$ gegeben ist, mit 
$g_{ij} = g( \tfrac{\partial }{\partial x_i} ,  \tfrac{\partial }{\partial x_j} )$
und $(g^{lk})$ bezeichnet die zu $(g_{ij})$ inverse Matrix.
\end{Lemma}
\proof
Zun\"achst ist $[\tfrac{\partial }{\partial x_i},\tfrac{\partial}{\partial x_j}] = 0$ 
und somit folgt die Behauptung mit der Koszul-Formel
\begin{align*}
2g\left(\nabla_{\frac{\partial}{\partial x_i}}\frac{\partial}{\partial
x_j},\frac{\partial}{\partial x_k} \right) &= 
\frac{\partial}{\partial x_i} g_{kj} + \frac{\partial}{\partial x_j} g_{ik} -
\frac{\partial}{\partial x_k}g_{ij} \\ &\overset{!}{=}
2 \sum_{l=1}^n \Gamma_{ij}^l g_{lk} = 
2g\left(\sum_{l=1}^n \Gamma_{ij}^l \frac{\partial}{\partial
x_l},\frac{\partial}{\partial x_k} \right).
\end{align*}
\qed

\bigskip

%%%%%%%%%%%%%%%%%%%%%%%%%%%%%%%%%%%%%%%%%%%%%%%%%%%%%%%%%%

\bigskip

\subsection{Kovariante Ableitung auf Vektorb\"undeln}


\bigskip


Sei $\pi : E \rightarrow M$ ein Vektorb\"undel \"uber $M$.

\begin{Definition}
Eine {\em kovariante Ableitung} auf $E$ ist eine $\R$-bilineare
Abbildung
$$
\nabla : \chi(M) \times \Gamma(E) \rightarrow \Gamma(E),\quad
(X, s) \mapsto \nabla_Xs \ ,
$$
so dass f\"ur alle $f\in \mathcal C^\infty(M)$ und alle $X\in \chi(M)$
folgende Eigenschaften erf\"ullt sind:
\begin{enumerate}
 \item
  $\qquad \nabla_{fX} s = f \nabla_Xs$
 \item
  $ \qquad \nabla_X (f \cdot s) = X(f) \cdot s + f \, \nabla_Xs $
\end{enumerate}
\end{Definition}

\bigskip

{\bf Bemerkung:}
Kovariante Ableitungen auf $TM$ setzen sich fort zu kovarianten
Ableitungen auf Tensor- und Formenb\"undeln. Mit Hilfe von
Hauptfaserb\"undeln kann man daf\"ur eine universelle Beschreibung
geben. Hier bleiben es erstmal ad-hoc Definitionen.

\bigskip

{\bf Beispiel:} Sei $B$ ein $(0,r)$-Tensor, dann ist $\nabla_XB$
ein $(0,r)$-Tensor, der definiert ist durch:
$$
(\nabla_X B)(X_1, \ldots, X_r) \;=\; X(B(X_1,\ldots, X_r))
\;-\;
\sum^r_{i=1} B(X_1, \ldots, \nabla_XX_i, \ldots, X_r) \ .
$$

\bigskip

Im Folgenden sollen drei Anwendungen f\"ur diese Formel gegeben werden.

\medskip

\begin{Lemma}
Eine kovariante Ableitung auf $TM$ ist metrisch bzgl. einer Riemannschen
Metrik genau dann, wenn $g$ parallel bzgl. $\nabla$ ist, d.h. $\nabla g = 0$
gilt.
\end{Lemma}
\proof
Seien $X,Y,Z$ Vektorfelder auf $M$, dann gilt nach Definition von $\nabla g$:
$$
(\nabla_X g)(Y, Z) \;=\; X(g(Y,Z)) \;-\; g(\nabla_XY, Z) \;-\; g(Y, \nabla_XZ) \ ,
$$
woraus sich unmittelbar die Behauptung ergibt.
\qed

\bigskip

\begin{Lemma}
Ein Vektorfeld $X$ auf $M$ ist genau dann ein Killing-Vektorfeld, wenn
f\"ur den Levi-Civita-Zusammenhang zu $g$ die Gleichung
$$
g(\nabla_YX, Z) \;+\; g(Y, \nabla_ZX) \;=\; 0
$$
f\"ur beliebige Vektorfelder $Y, Z$ erf\"ullt ist.
\end{Lemma}
\proof
Nach Definition ist $X \in \chi(M)$ genau dann ein Killing-Vektorfeld, wenn
der lokale Flu\ss{} von $X$ aus lokalen Isometrien besteht, also genau dann,
wenn $L_Xg=0$ gilt. Nach der Definition der Lie-Ableitung von Bilinearformen
und der Voraussetzung, dass $\nabla$ metrisch und torsionsfrei ist, erh\"alt man
$$
\begin{array}{rl}
(L_X g)(Y,Z) & = L_X(g(Y,Z)) - g(L_XY, Z) - g(Y, L_XZ)\\[1.5ex]
& =
g(\nabla_XY, Z) + g(Y, \nabla_XZ) - g(\nabla_XY - \nabla_YX, Z) - g(Y, \nabla_XZ - \nabla_ZX)\\[1.5ex]
& =
g(\nabla_YX, Z) \;+\; g(Y, \nabla_ZX) \ ,
\end{array}
$$
woraus unmittelbar die Behauptung folgt.
\qed

\bigskip

{\bf Bemerkung:}
Killing-Vektorfelder $X$ sind also dadurch charakterisiert, dass $\nabla X$ ein schiefsymmetrischer
Endomorphismus von $TM$ ist.

\bigskip

\begin{Lemma}
Sei $(M^n, g)$ eine Riemannsche Mannigfaltigkeit mit einem torsions-freien Zusammenhang
$\nabla$. Dann gilt f\"ur eine beliebige Differentialform $\omega$ auf $M$:
$$
d\omega \;=\; \sum^n_{i=1} e^i \wedge \nabla_{e_i} \omega
\ ,
$$
wobei $\{e_i\}$ eine lokale ONB von $TM$ ist, mit dualer Basis $\{e^i\}$  von $T^*M$.
\end{Lemma}
\proof
Sei zun\"achst $\omega$ eine 1-Form. Dann gilt f\"ur die linke Seite der zu beweisenden
Gleichung:
$$
d\omega (X,Y) = X(\omega(Y)) - Y(\omega(X)) - \omega([X,Y])
$$
Auf der rechten Seite erh\"alt man:
$$
\begin{array}{rl}
 \sum (e^i \wedge \nabla_{e_i} \omega)(X, Y)
& =
\sum [e^i(X)(\nabla_{e_i} \omega)(Y) - e^i(Y)(\nabla_{e_i} \omega)(X)] \\[1.5ex]
& =
(\nabla_X \omega)(Y) - (\nabla_Y \omega)(X) \\[1.5ex]
& =
X(\omega(Y)) - \omega(\nabla_XY) - Y(\omega(X)) + \omega(\nabla_YX) \\[1.5ex]
& =
 X(\omega(Y)) - Y(\omega(X)) - \omega([X,Y])
\end{array}
$$
In der letzten Gleichung nutzte man die Torsionsfreiheit des Zusammenhangs $\nabla$ durch
die Gleichung $[X, Y] = \nabla_XY - \nabla_YX$.

\medskip

F\"ur Formen beliebigen Grades nutzt man die Formel:
$$
\begin{array}{rl}
d\omega (X_0, \ldots, X_k)
& =\;
\sum_j (-1)^j L_{X_j}(\omega(X_0 \ldots, \hat{X_j},\ldots, X_k) )\\[1.5ex]
& \phantom{xxxxxxxxxxxxxx}\;+\;\;
\sum_{i<j}(-1)^{i+j} \omega([X_i, X_j],\ldots, \hat X_i, \ldots, \hat X_j, \ldots)
\end{array}
$$

und 

$$
\begin{array}{rl}
\sum_i (e^i \wedge \nabla_{e_i} \omega)(X_0, \ldots, X_k)
& = \;
\sum_{i, j} (-1)^j e^i(X_j)(\nabla_{e_i}\omega)(\ldots, \hat X_j, \ldots) \\[1.5ex]
& = \;
\sum_{i, j}(-1)^j (\nabla_{X_j}\omega)(\ldots, \hat X_j, \ldots)\\[1.5ex]
& = \;
\sum_{j} (-1)^j X_j(\omega(\ldots, \hat X_j, \ldots)) \\[1ex] 
& \phantom{xxxxxxxx} \;-\; \sum_{j,k} (-1)^j \omega(\ldots, \nabla_{X_j}X_k, \ldots, \hat X_j, \ldots)
\end{array}
$$


\qed

\bigskip

\begin{Folgerung}
 Parallele Formen sind geschlossen.
\end{Folgerung}


\bigskip

{\bf Bemerkung:}
Die kovariante Ableitung $\nabla \omega$ einer $k$-Form $\omega$ kann man auffassen
als Schnitt im Vektorb\"undel $T^*M \otimes \Lambda^k T^*M$. Man hat
$\nabla \omega = \sum e^i \otimes \nabla_{e_i} \omega$, f\"ur eine lokale ONB $\{e_i\}$.
Tats\"achlich gilt nach Definition des Tensorproduktes
$$
\sum (e^i \otimes \nabla_{e_i} \omega )(X, X_1,\ldots, X_k)
=
\sum e^i(X ) (\nabla_{e_i} \omega) (X_1,\ldots, X_k)
=
(\nabla_X \omega)(X_1,\ldots, X_k) \ .
$$
Man hat folgende Isomorphie von Vektorb\"undeln, die einer punktweisen Isomorphie
von Vektorr\"aumen entspricht:
$$
T^*M \otimes \Lambda^k T^*M \cong \Lambda^{k+1}T^*M \otimes \Lambda^{k-1}T^*M \otimes E \ ,
$$
dabei ist $E$ ein zus\"atzliches Vektorb\"undel, das hier nicht genauer beschrieben
werden soll. Man kann explizit Projektionen auf die ersten beiden Summanden angeben:
$$
\pr_1 :T^*M \otimes \Lambda^k T^*M \rightarrow \Lambda^{k+1}T^*M,
\lambda \otimes \omega \mapsto \lambda \wedge \omega
$$
und
$$
\pr_2 :T^*M \otimes \Lambda^k T^*M \rightarrow \Lambda^{k-1}T^*M,
X^* \otimes \omega \mapsto i_X \omega
$$
wobei $X$ ein Vektorfeld ist mit der dualen 1-Form $X^* := g(X, \cdot)$. Mit diesen
Bezeichungen gilt dann:
$$
d \omega \;=\; \pr_1(\nabla \omega)
$$
Die Projektion auf den zweiten Summanden liefert  (bis auf ein Vorzeichen) das sogenannte
Kodifferential
$d^* : \Omega^k(M) \rightarrow \Omega^{k-1}(M)$. Man schreibt oft auch $d^* = \delta$. Es gilt also
$$
d^*\omega = - \pr_2(\nabla \omega) = - \sum i_{e_k} \nabla_{e_k} \omega
$$
Das Kodifferential $d^*$ ist der zu $d$ adjungierte Operator. Dh. bzgl. des
$L^2$-Skalarproduktes $(\alpha, \beta)  = \int \la \alpha, \beta\ra dvol_g$ gilt
$(d\alpha, \beta) = (\alpha, d^*\beta)$, wobei $\alpha \in \Omega^k(M)$ und
$\beta \in \Omega^{k+1}(M)$.


% Etwas Analysis
\newcommand{\ep}{\varepsilon}
\newcommand{\abs}[1]{|#1|}
\newcommand{\scal}{\mathrm{scal}}

\section{Kr\"ummung Riemannscher Mannigfaltigkeiten} e

Sei $(M, g)$ eine Riemannsche Mannigfaltigkeit mit Levi-Civita Zusammenhang $\nabla$.

\begin{Definition}
Die {\em Riemannsche Kr\"ummung }  von $(M,g)$ ist die Abbildung
$$
R: \chi(M) \times \chi(M) \times \chi(M) \rightarrow \chi(M), \qquad
(X,Y,Z) \mapsto R_{X, Y}Z
$$
die f\"ur Vektorfelder $X, Y, Z$ definiert ist durch
$$
R_{X Y}Z \;=\; \nabla_X \nabla_Y Z \;-\; \nabla_Y\nabla_XZ \;-\; \nabla_{[X,Y]} \ .
$$
\end{Definition}

\bigskip

\begin{Lemma}
Die Abbildung $R$ ist ein $(1,3)$-Tensor, d.h. $\mathcal C^\infty(M)$-linear
in allen Eintr\"agen.
\end{Lemma}
\proof
Es ist zum Beispiel,
\begin{align*}
R_{fX,Y}Z &= f\nabla_X \nabla_Y Z - \nabla_Y f\nabla_X Z -
\nabla_{f[X,Y]-Y(f)X}Z\\
&= f\nabla_X \nabla_Y Z - f\nabla_Y \nabla_X Z
- Y(f)\nabla_X Z - f\nabla_{[X,Y]}Z+ Y(f)\nabla_XZ\\
&= f\nabla_X \nabla_Y Z - f\nabla_Y \nabla_X Z
-f\nabla_{[X,Y]}.
\end{align*}
Analog zeigt man die $\CC^\infty(M)$-Linearit\"at in den \"ubrigen Eintr\"agen.
\qed

\bigskip

{\bf Bemerkungen:}
\begin{enumerate}
 \item
Die Abbildung $R$ definiert auch einen $(0,4)$-Tensor:
$
R(X, Y, Z, U) \;=\;  g(R_{X,Y}Z, U) \ .
$
 \item
Der Riemannsche Kr\"ummungstensor definiert punktweise $\R$-lineare Abbildungen.
F\"ur fixierte Vektoren $X,Y \in T_pM$ hat man
$$
R_{X Y} : T_pM \rightarrow T_pM, \qquad R_{X Y}Z := (R_{\tilde X \tilde Y} \tilde Z)(p) \ ,
$$
wobei $\tilde X, \tilde Y, \tilde Z$ Fortsetzungen der Vektoren $X, Y, Z \in T_p M$ zu
Vektorfeldern auf $M$ sind.
\item
F\"ur die Riemannsche Kr\"ummung gibt es auch die Definition mit dem umgekehrten Vorzeichen.
\item
Die Riemannsche Kr\"ummung ist vollst\"andig durch den Levi-Civita-Zusammenhang bestimmt und
kann also lokal durch die Christoffel-Symbole bzw. in Abh\"angigkeit von der Metrik und
ihren partiellen Ableitungen ausgedr\"uckt werden. Definiert man
$$
R_{ijkl} \;= \;
g(R_{\frac{\partial}{\partial x_i}, \frac{\partial}{\partial x_j} }
\tfrac{\partial}{\partial x_k}, \tfrac{\partial}{\partial x_l}) \ ,
$$
dann ist die genaue Formel:
$$
R_{ijkl} \;= \;
\tfrac{\partial \Gamma^i_{k j}}{\partial x_l}
\;-\;
\tfrac{\partial \Gamma^i_{lj}}{\partial x_k}
\;+\;
\sum_m \, \Gamma^i_{lm} \, \Gamma^m_{k j}
\;-\;
\sum_m \Gamma^i_{km} \, \Gamma^m_{l j} \ .
$$

\end{enumerate}


\bigskip

Der Riemannsche Kr\"ummungstensor hat eine Reihe von Symmetrien
und speziellen Eigenschaften.

\begin{Satz}
\label{Charakterisierung-der-Kruemmung}
 Seien dabei $X,Y,Z,U,V$ beliebige Vektorfelder auf $M$, dann gilt:
\begin{enumerate}
 \item
$\quad
R(X,Y,Z,U) \;=\; -\, R(Y,X, Z,U) \;=\; -\, R(X,Y,U,Z)
$
 \item
$ \quad
R(X,Y,Z,U) \;=\; R(Z,U,X,Y)
$
\item  1. Bianchi-Identit\"at: \\[1ex]
$ \qquad
R(X,Y,Z,U) \;+\; R(Y,Z,X,U) \;+\; R(Z,X,Y,U) \;=\; 0
$
\item  2. Bianchi-Identit\"at: \\[1ex]
$ \qquad
(\nabla_XR)(Y , Z ,U  ,V ) \;+\; (\nabla_YR)( Z, X , U , V) \;+\; (\nabla_ZR)(X ,Y  ,U  ,V ) \;=\; 0
$
\end{enumerate}
\end{Satz}
\proof
Zu 1.: Die Schiefsymmetrie in $X, Y$ folgt direkt aus der Definition. Der Kr\"ummungstensor ist
schiefsymmetrisch in $Z,U$ genau dann, wenn $R(X,Y,Z,Z)= 0$ f\"ur alle Vektorfelder $X,Y,Z$
erf\"ullt ist.  Da $\nabla$ metrisch ist folgt:
$$
\begin{array}{rl}
L_X g(Z,Z) & = g(\nabla_X Z, Z) \;+\; g(Z, \nabla_XZ) \;=\; 2\, g(\nabla_XZ, Z) \\[1.5ex]
L_{[X,Y]} g(Z,Z)&  = 2 \, g(\nabla_{[X,Y]}Z,Z) \\[1.5ex]
L_YL_X g(Z,Z) & = 2 \, g(\nabla_Y\nabla_XZ, Z) \;+\; 2\,g(\nabla_YZ, \nabla_XZ) \\[1.5ex]
L_XL_Y g(Z,Z) & = 2 \, g(\nabla_X\nabla_YZ, Z) \;+\; 2\,g(\nabla_XZ, \nabla_YZ)
\end{array}
$$
Subtrahiert man die letzten drei Gleichungen von einander, so erh\"alt man:
$$
\begin{array}{rl}
0 & = (L_XL_Y \,-\, L_YL_X \,-\, L_{[X,Y]}) g(Z,Z) \\[1.5ex]
& =
2\,g(\nabla_X \nabla_YZ - \nabla_Y\nabla_X Z - \nabla_{[X,Y]}Z,Z) \\[1.5ex]
& =
2\,g(R_{X Y}Z,Z)
\end{array}
$$

Zu 3.: Wir k\"onnen durch eine geeignete Wahl ohne Einschr\"ankung annehmen, dass
$[X,Y] = [Z,X] = [Y,Z] = 0$ in einem Punkt. Dann ist
\begin{align*}
&R_{X,Y}Z + R_{Z,X}Y + R_{Y,Z}X \\&\qquad= 
\nabla_X\nabla_Y Z - \nabla_Y\nabla_X Z + \nabla_Z\nabla_X Y - \nabla_X\nabla_Z
Y + \nabla_Y\nabla_Z X - \nabla_Z\nabla_Y X\\
&\qquad= \nabla_X(\nabla_Y Z - \nabla_Z Y) + \nabla_Y(\nabla_Z X - \nabla_X Z)
+ \nabla_Z(\nabla_X Y - \nabla_Y X)\\
&\qquad= \nabla_X([Y,Z]) + \nabla_Y([Z,X])
+ \nabla_Z([X,Y]) = 0.
\end{align*}

Zu 2.: Nach 3. gelten
\begin{align*}
&R(X,Y,Z,W) + R(Y,Z,X,W) + R(Z,X,Y,W) = 0,\\
&R(Y,Z,W,X) + R(Z,W,Y,X) + R(W,Y,Z,X) = 0,\\
&R(Z,W,X,Y) + R(W,X,Z,Y) + R(X,Z,W,Y) = 0,\\
&R(W,X,Y,Z) + R(X,Y,W,Z) + R(Y,W,X,Z) = 0.
\end{align*}
Addieren dieser Terme ergibt,
\begin{align*}
2R(X,Z,W,Y) - 2R(W,Y,X,Z) = 0.
\end{align*}
\qed

\bigskip

{\bf Bemerkung:} Die ersten drei Eigenschaften sind algebraischer Natur. Punktweise sind sie
\"aquivalent dazu, dass die Riemannsche Kr\"ummung in
$
\Sym^2(\Lambda^2 T^*_pM) \cap \ker b
$
liegt, also eine symmetrische Bilinearform auf $\Lambda^2T^*_pM$ ist, die im Kern der
Bianchi-Abbildung $b: \Lambda^2T^*_pM \otimes \Lambda^2T^*_pM \rightarrow \Lambda^4T^*_pM$
liegt, die definiert ist durch $b(\alpha \otimes \beta ) = \alpha \wedge \beta$.

\bigskip

Man kann $\nabla Z$ auch als Endomorphismus auf $TM$ betrachten, durch
\begin{align*}
\nabla Z : TM\to TM,\qquad X\mapsto \nabla_XZ.
\end{align*}
$\nabla$ induziert nun eine kovariante Ableitung auf $\End(TM)$.
Sei $L\in \End(TM)$, so definiert man
\begin{align*}
(\nabla_X L)(Y) = \nabla_X (L(Y)) - L(\nabla_X Y).
\end{align*}
Dies erm\"oglicht eine sinnvolle Definition von
\begin{align*}
\nabla^2 Z = \nabla (\nabla Z),\qquad \nabla_{XY}^2 Z = (\nabla_X(\nabla Z))(Y).
\end{align*}

\bigskip

\begin{Lemma}
$R_{X,Y}Z = \nabla_{XY}^2 Z - \nabla_{YX}^2 Z$.
\end{Lemma}

\proof
Nach Defintion von $\nabla$ auf $\End(TM)$ ist
\begin{align*}
\nabla_{XY}^2 Z = \nabla_X(\nabla Z(Y)) - \nabla Z(\nabla_X Y) = 
\nabla_X\nabla_Y Z - \nabla_{\nabla_X Y}Z
\end{align*}
Folglich ist
\begin{align*}
\nabla_{XY}^2 Z - \nabla_{YX}^2 Z = 
\nabla_X\nabla_Y Z - \nabla_Y\nabla_X Z - \nabla_{\nabla_X Y - \nabla_Y X} Z =
R_{X,Y}Z.
\end{align*}
\qed

\subsection{Die Riemannsche Kr\"ummung in lokalen Koordinaten}

\begin{Lemma}
Definiert man $R_{ijkl} := R(\frac{\partial}{\partial
x_i},\frac{\partial}{\partial x_j},\frac{\partial}{\partial
x_k},\frac{\partial}{\partial x_l})$, dann gilt
\begin{align*}
R_{ijkl} \;= \;
\tfrac{\partial \Gamma^i_{k j}}{\partial x_l}
\;-\;
\tfrac{\partial \Gamma^i_{lj}}{\partial x_k}
\;+\;
\sum_m \, \Gamma^i_{lm} \, \Gamma^m_{k j}
\;-\;
\sum_m \Gamma^i_{km} \, \Gamma^m_{l j} \ . 
\end{align*}
\end{Lemma}

\bigskip

{\bf Bemerkung:}
Auf einer Riemannschen Mannigfaltigkeit existieren spezielle Koordinaten $(U,x)$
mit $x(p) = 0$ (\emph{geod\"atische Normalkoordinaten}) in denen sich die Metrik
schreiben l\"asst als
\begin{align*}
g_{ij}(x) = \delta_{ij} + \frac{1}{3}\sum_{k,l} R_{ikjl}(0)x_kx_l + O(|x|^2)
\end{align*} 


\subsection{Die Schnittkr\"ummung}

\begin{Lemma}
Sei $E\subset T_pM$ ein 2-dimensionaler Unterraum, dann ist die Zahl
\begin{align*}
K(X,Y) = \frac{R(X,Y,Y,X)}{g(X,X)g(Y,Y)-g(X,Y)^2}
\end{align*}
unabh\"angig von der Wahl der Basis $\setd{X,Y}$ von $E$.
\end{Lemma}

\proof
Sei $E=\mathrm {span}\setd{X,Y} = \mathrm {span}\setd{V,W}$ und $V=aX+bY$, $W=
cX+dY$, dann ist
\begin{align*}
\det\begin{pmatrix}
a & b\\
c & d
\end{pmatrix}
= ad-cb \neq 0
\end{align*}
und
\begin{align*}
g(R_{V,W}W,V) = (ad-cb)^2g(R_{X,Y}Y,X) 
\end{align*}
Analog \"uberzeugt man sich, dass
\begin{align*}
g(V,V)g(W,W) - g(V,W)^2  
&= (ad-cb)^2\left(g(X,X)g(Y,Y) -
g(X,Y)^2 \right) \\ &=
\det\left(\begin{smallmatrix}
a & b\\
c & d
\end{smallmatrix}\right)
\left(g(X,X)g(Y,Y) -
g(X,Y)^2 \right).
\end{align*}
\qed

{\bf Bemerkungen:}
\begin{enumerate}
  \item Die Zahl $K(E) = K(X,Y)$ f\"ur $E = \mathrm{span}\setd{X,Y}$ nennt man
  \emph{Schnittkr\"ummung} von $E$ in $p$.
  \item Falls $\dim M=2$, so ist $K$ eine Funktion auf $M$,
  \begin{align*}
  K(p) = K(T_pM).
  \end{align*}
  Sie hei\ss{}t \emph{Gau\ss{}-Kr\"ummung} auf $M$.
  \item Man kann den Nenner der Schnittkr\"ummung mittels der Determinate
  ausdr\"ucken:
  \begin{align*}
  \det
  \left(\begin{smallmatrix}
  g(X,X) & g(X,Y)\\
  g(Y,X) & g(Y,Y)
  \end{smallmatrix}\right)
  =
  g(X,X)g(Y,Y) - g(X,Y)^2 =: g(X\wedge Y,X\wedge Y). 
  \end{align*}
  Insbesondere ist $\|X\|^2\|Y\|^2 - g(X,Y)^2\neq 0$, falls $X,Y$ linear
  unabh\"angig.
\end{enumerate}

\bigskip

\begin{Satz}
Die Riemannsche Kr\"ummung ist vollst\"andig durch die Schnittkr\"ummung bestimmt.
\end{Satz}

\bigskip

\begin{Definition}
Eine Riemannsche Mannigfaltigkeit ist ein \emph{Raum konstanter Kr\"ummung} $c$,
falls $K(E) = c$ f\"ur alle 2-dimensionalen Unterr\"aume $E\subset T_pM$ und
alle $p\in M$.
\end{Definition}

\bigskip

{\bf Bemerkung:}
R\"aume konstanter Schnittkr\"ummung sind Quotienten von $\R^n$, $S^n$, $H^n$.

\bigskip

\begin{Satz}
Sei $(M,g)$ eine Riemannsche Mannigfaltigkeit mit konstanter Kr\"ummung $c$. Dann
gilt
\begin{align*}
R_{X,Y}Z = c\left(g(Y,Z)X - g(X,Z)Y \right).
\end{align*}
\end{Satz}

\proof
Wir setzen $F(X,Y,Z,V) = c\left(g(Y,Z)g(X,V) - g(X,Z)g(Y,V) \right)$. Man
rechnet direkt nach, dass $F$ Eigenschaft 1.-3. von Satz
\ref{Charakterisierung-der-Kruemmung} erf\"ullt und folglich ist $F\in \Sym^2(\Lambda^2T_pM)\cap \ker b$ mit
\begin{align*}
F: \Lambda^2\to \Lambda^2,\qquad F = c\Id.
\end{align*}
Folglich ist $F(X,Y,Z,V) = -cg(X\wedge Y,Z\wedge V)$ und
\begin{align*}
K^F(X,Y) = \frac{F(X,Y,Y,X)}{g(X\wedge Y,X\wedge Y)} = 
c\frac{g(X\wedge Y,X\wedge Y)}{g(X\wedge Y,X\wedge Y)} = c = K^R(X,Y).
\end{align*}
Somit ist $F=R$.
\qed

\bigskip


{\bf Bemerkungen:}
\begin{enumerate}
  \item Neben den R\"aumen konstanter Kr\"ummung bilden auch die R\"aume mit positiver
  Kr\"ummung eine interessante Familie. Diese bilden ein aktives Forschungsgebiet
  mit vielen offenen Fragen\ldots
  \item Ist die Kr\"ummung $c\equiv 1$, so folgt, dass $M=S^n$. In den 60ern wurde
  gezeigt, dass falls $c$ auf $M$ stets nahe $1$ ist, d.h. $c\in
  [1-\ep,1+\ep]$, $M$ hom\"oomorph zu $S^n$ ist. Erst k\"urzlich wurde gezeigt,
  dass $M$ dann tats\"achlich auch diffeomorph zu $S^n$ ist.
\end{enumerate}

\bigskip

{\bf Beispiele:}
\begin{enumerate}
  \item Ist $M=\R^n$, so ist $K\equiv 0$, denn $\nabla_X Y = X(Y)$. Dann gilt
\begin{align*}
R_{X,Y}Z = X(Y(Z)) - Y(X(Z)) - [X,Y](Z) = 0.
\end{align*}
$\R^n$ ist "flach".
\item Sei $M=S^n\subset\R^{n+1}$, dann ist $T_pS^n = p^\bot\subset\R^n$ und
der Levi-Civita-Zusammenhang
\begin{align*}
\nabla_X Y=\pr_{S^n}(X(Y)) = X(Y)-\scp{X(Y),N}N.
\end{align*}
Der Normalenvektor ist dabei gegeben durch
\begin{align*}
N: M\to TM,\qquad p\mapsto N_p = p.
\end{align*}
F\"ur alle $Y\in T_pS^n$ ist $\scp{Y,N} = 0$ und folglich
\begin{align*}
\scp{X(Y),N} + \scp{Y,X(N)} = 0,\qquad X,Y\in\chi(S^n).
\end{align*}
Weiterhin ist $X(N) = X$ und daher
\begin{align*}
\nabla_X Y = X(Y) + \scp{Y,X(N)}n = X(Y) + \scp{Y,X}N
\end{align*}
Der Kr\"ummungstensor berechnet sich nun zu
\begin{align*}
R_{X,Y}Z &= \nabla_X\nabla_Y Z - \nabla_Y\nabla_X Z - \nabla_{[X,Y]}Z\\
&= X(\nabla_Y Z) + \scp{X, \nabla_YZ}N - 
Y(\nabla_X Z) - \scp{Y,\nabla_XZ}N\\
&\qquad- [X,Y](Z) - \scp{[X,Y],Z}N\\
&= X(Y(Z) + \scp{Y,Z}N) + \scp{X,Y(Z)+\scp{Y,Z}N}N\\
&\quad -Y(X(Z) + \scp{X,Z}N) - \scp{Y,X(Z) + \scp{X,Z}N}N\\
&\quad -X(Y(Z)) + Y(X(Z)) - \scp{[X,Y],Z}N\\
&= X(\scp{Y,Z}N) + \scp{X,Y(Z)}N - Y(\scp{X,Z}N) - \scp{Y,X(Z)}N -
\scp{[X,Y],Z}N\\
&=X(\scp{Y,Z}N) - Y(\scp{X,Z}N) + \left(\scp{X,Y(Z)} -
\scp{Y,X(Z)} - \scp{[X,Y],Z}\right)N\\
&= \scp{Y,Z}X  -\scp{X,Z}Y \\ &\qquad+\left(X(\scp{Y,Z}) - 
 Y(\scp{Y,Z}) + \scp{X,Y(Z)} -
\scp{Y,X(Z)} - \scp{[X,Y],Z}\right)N
\end{align*}
Somit ist
\begin{align*}
\scp{R_{X,Y}Y,X} = \scp{Y,Y}\scp{X,X} - \scp{X,Y}\scp{Y,X} = \scp{X\wedge
Y,X\wedge Y}
\end{align*}
und daher ist $K(S^n) \equiv 1$.
\item Wir betrachten den hyperbolischen Raum
\begin{align*}
H^n = \setdef{x\in\R^n}{x_n > 0},
\end{align*}
mit der Metrik
\begin{align*}
g = \frac{1}{x_n^2}\left(\sum_{i=1}^n \dx_i\otimes\dx_i \right).
\end{align*}
Man verifiziert sofort, dass dann
\begin{align*}
g_{ij} = \frac{1}{x_n^2}E_{ij},\qquad g^{ij} = x_n^2 E^{ij}.
\end{align*}
Damit berechnen sich die Christoffel-Symbole wie folgt,
\begin{align*}
&\Gamma_{ii}^k = \frac{1}{x_n},\qquad i < n,\\
&\Gamma_{nn}^k = -\frac{1}{x_n} = \Gamma_{in}^k = \Gamma_{ni}^k,\\
&\Gamma_{ij}^k = 0,\qquad \text{sonst}.
\end{align*}
Es folgt, dass $K(H^n) \equiv -1$. Im Spezialfall
\begin{align*}
H^2 = \setdef{(x,y)}{y> 0},\qquad g = \frac{1}{y^2} (\dx^2 + \dy^2),
\end{align*}
also
\begin{align*}
g_{xx} = \frac{1}{y^2} = g_{yy},\qquad g_{xy} = g_{yx} = 0,
\end{align*}
erh\"alt man dies auch direkt aus der Koszul-Formel, ohne die Christoffel-Symbole
berechnen zu m\"ussen. Es ist n\"amlich,
\begin{align*}
2g(\nabla_{\partial x}\partial_x,\partial x) &= \partial_xg_{xx} +
\partial_xg_{xx} - \partial_xg_{xx}
= \partial_x g_{xx} = 0,\\
 2g(\nabla_{\partial
x}\partial_x,\partial y) &= \partial_{x}g_{xy} + \partial_x g_{yx} - \partial_y
g_{xx} = - \partial_y(g_{xx}) =
\frac{2}{y^3}.
\end{align*}
Damit erh\"alt man 
\begin{align*}
\nabla_{\partial x}\partial_x = \frac{1}{y}\partial_y.
\end{align*}
Analog berechnet man
\begin{align*}
\nabla_{\partial y}\partial_y = -\frac{1}{y}\partial_y,\qquad
\nabla_{\partial y}\partial_x = \nabla_{\partial x}\partial y =
-\frac{1}{y}\partial_x.
\end{align*}
Der Kr\"ummungstensor ergibt sich damit zu
\begin{align*}
R_{\partial x\partial y}\partial_x = \nabla_{\partial x}\nabla_{\partial y}
\partial_x - \nabla_{\partial y}\nabla_{\partial x}
\partial_x =  \frac{1}{y^2}\partial_y.
\end{align*}
Die Schnittkr\"ummung ist dann gerade
\begin{align*}
K = K(\partial_x,\partial_y) =
\frac{R(\partial_x,\partial_y,\partial_y,\partial_x)}{\abs{\partial_x}^2\abs{\partial_y}^2
- g(\partial_x,\partial_y)^2} = 
-\frac{g\left(\frac{1}{y^2}\partial_y,\partial_y
\right)}{\frac{1}{y^2}\frac{1}{y^2}} = -1.
\end{align*}
\end{enumerate}

\bigskip

\subsection{Ricci-Kr\"ummung}

\begin{Definition}
Die \emph{Ricci-Kr\"ummung} von $(M,g)$ ist punktweise definiert durch die
Abbildung
\begin{align*}
\Ric: T_pM\times T_pM\to \R,
\end{align*}
mit
\begin{align*}
\Ric(X,Y) = \tr\left(Z\mapsto R_{ZX}Y\right).
\end{align*}
\end{Definition}

\bigskip

{\bf Bemerkungen:}
\begin{enumerate}
  \item $\Ric$ ist ein $(0,2)$-Tensor.
  \item $\Ric$ definiert einen Endomorphismus von $TM$ durch die Gleichung
\begin{align*}
g\left(\Ric(X),Y \right) = \Ric(X,Y).
\end{align*}
\item Sei $\setd{e_i}$ eine lokale Orthonormalbasis, dann hat $\Ric$ die
Darstellung
\begin{align*}
\Ric(X,Y) = \sum_{i=1}^k R(e_i,X,Y,e_i) = \sum_{i=1}^k R(X,e_i,e_i,Y).
\end{align*}
\end{enumerate}

\bigskip

\begin{Lemma}
$\Ric$ ist ein symmetrischer $(0,2)$-Tensor.
\end{Lemma}

\proof
Die Behauptung folgt direkt aus den Symmetrien des Kr\"ummungstensors,
\begin{align*}
\Ric(X,Y) = \sum_i R(e_i,X,Y,e_i) = 
\sum_i R(Y,e_i,e_i,X) =
\sum_i R(e_i,Y,X,e_i) = R(Y,X). 
\end{align*}
\qed

Aufgrund der Symmetrie ist $\Ric$ bereits durch $\Ric(X,X)$ bestimmt
(Polarisation),
\begin{align*}
\Ric(X,Y) = \frac12\left(\Ric(X,X)+\Ric(Y,Y) - \Ric(X-Y,X-Y) \right).
\end{align*}
 Es zeigt sich nun, dass $\Ric$ daher bereits vollst\"andig durch
die Schnittkr\"ummung beschrieben ist.

\begin{Lemma}
Sei $\setd{e_i}_{i=1}^{n-1}$ eine Orthonormalbasis in $X^\bot \subset T_pM$.
Dann gilt
\begin{align*}
\Ric(X,X) = \sum_{i=1}^{n-1} K(X,e_i)g(X,X).
\end{align*}
\end{Lemma}

\proof
Ohne Einschr\"ankung sei $\abs{X} = 1$ und eine Orthonormalbasis von $T_pM$
gegeben durch $X=e_n$, $X^\bot = \mathrm{span}\setd{e_1,\ldots,e_{n-1}}$.
Es ist nach Definition der Schnittkr\"ummung,
\begin{align*}
R(e_i,X,X,e_i) = K(X,e_i)\left(\abs{e_i}^2\abs{X}^2 - g(e_i,X)^2 \right) 
= K(X,e_i)\left(1 - \delta_{in}^2 \right),
\end{align*}
und folglich
\begin{align*}
\Ric(X,X) = \sum_{i=1}^n R(e_i,X,X,e_i) = \sum_{i=1}^{n-1} K(X,e_i)g(X,X).
\end{align*}
\qed

\begin{Definition}
Die \emph{Skalarkr\"ummung} ist definiert als
\begin{align*}
\scal_g = \tr(\Ric) = \sum_{i} \Ric(e_i,e_i)
= \sum_{i,j} R(e_i,e_j,e_j,e_i).
\end{align*}
\end{Definition}
Die Skalarkr\"ummung ist eine glatte Funktion auf $M$, die von der Metrik $g$
abh\"angt.

\bigskip

{\bf Beispiel:}
$(M,g)$ habe konstante Schnittkr\"ummung $c$. Dann gilt f\"ur die Ricci-Kr\"umung
\begin{align*}
\Ric(X,X) = \sum_{i=1}^{n-1} K(X,e_i)g(X,X) =  c\cdot(n-1)g(X,X),
\end{align*}
und folglich ist die Skalarkr\"ummung ebenfalls konstant und gegeben durch
\begin{align*}
\scal_g = c\cdot n(n-1).
\end{align*}

\bigskip

{\bf Bemerkung:}
Der Kr\"ummungstensor l\"asst sich als Element
\begin{align*}
R \in \Sym^2(\Lambda^2V)\cap\ker b,\qquad V =T_pM,
\end{align*}
auffassen. Aus der Darstellungstheorie der $\SO(n)$ wissen wir, dass
\begin{align*}
\R\oplus \Sym_0^2V \oplus \Lambda^4V \oplus W = \Sym^2(\Lambda^2V) \subset
\Lambda^2 V\otimes \Lambda^2 V.
\end{align*}
Da $\Lambda^4 V$ nicht im Kern der Bianchi-Abbildung
liegt, l\"asst sich $R$ somit darstellen als
\begin{align*}
R = \scal + \Ric + w,
\end{align*}
wobei $w$ f\"ur die Weyl-Kr\"ummung steht.

\subsection{Einsteinmetriken}

\begin{Definition}
Eine \emph{Einsteinmetrik} ist eine Metrik $g$ mit
\begin{align*}
\Ric_g = \lambda g,
\end{align*}
f\"ur eine Konstante $\lambda\in\R$.
\end{Definition}

{\bf Bemerkungen:}
\begin{enumerate}
  \item R\"aume konstanter Schnittkr\"ummung sind Einstein, denn
\begin{align*}
\Ric_g = c\cdot (n-1)g.
\end{align*}
\item Ist $M$ Einstein, dann gilt $\lambda = \frac{\scal_g}{n}$.
\item Ist $\dim M = 1$, dann verschwinden alle Kr\"ummungsgr\"o\ss{}en.
\item Ist $\dim M = 2$, dann gilt
\begin{align*}
\Ric(e_i,e_j) &= \sum_{k=1}^2 R(e_i,e_k,e_k,e_j)
= 
\begin{cases}
0, & i\neq j,\\
R(e_1,e_2,e_2,e_1)=R(e_2,e_1,e_1,e_2), & i=j,
\end{cases}\\
&= K \cdot g(e_i,e_j)
\end{align*}
Somit ist
\begin{align*}
\Ric = K\cdot g,\qquad
\scal = 2K,
\end{align*}
d.h. alle Kr\"ummungsgr\"o\ss{}en fallen zusammen. Insbesondere ist $M$ genau
dann Einstein, wenn $K$ konstant ist.
\item F\"ur $2$-dimensionale Mannigfaltigkeiten existiert immer eine Metrik 
konstanter Kr\"ummung.
\item Die Schnittkr\"ummung bestimmt die Ricci-Kr\"ummung. Ist $\dim M = 3$, so gilt
auch die Umkehrung (dies gilt nicht mehr f\"ur $\dim M\ge 4$). Insbesondere gilt
\begin{align*}
(M^3,g) \text{ Einstein } \Leftrightarrow M\text{ hat konst. Schnittkr\"ummung}.
\end{align*}
Sei dazu $\setd{e_1,e_2,e_3}$ eine Orthonormalbasis, dann gilt
\begin{align*}
\Ric(e_1,e_1) &= K(e_1,e_2) + K(e_1,e_3),\\
\Ric(e_2,e_2) &= K(e_2,e_1) + K(e_2,e_3),\\
\Ric(e_3,e_3) &= K(e_3,e_1) + K(e_3,e_2).
\end{align*}
Es handelt sich hierbei um ein lineares Gleichungssystem f\"ur die drei
Unbekannten $K(e_1,e_2),K(e_1,e_3)$ und $K(e_2,e_3)$. Man erh\"alt daraus eine
explizite Formel f\"ur die Schnittkr\"ummung in Abh\"anigkeit der Ricci-Kr\"ummung.
\item Die Einsteinsche-Feldgleichung ist
\begin{align*}
\Ric_g - \frac{\scal_g}{2}g = T,
\end{align*}
wobei $T$ den Energie-Impuls-Tensor bezeichnet. Falls keine Masse im System
vorhanden ist, so ist $T=0$ also gilt es zu l\"osen
\begin{align*}
\Ric_g -\frac{\scal_g}{2}g = 0\Leftrightarrow
\Ric_g = 0.
\end{align*}
Es sind also flache Ricci-Metriken zu suchen. 
\item Einstein-Metriken sind kritische Punkte des Hilbert-Funktionals. Bezeichne
\begin{align*}
\mathcal M_1 (M) := \setd{\text{Riemannsche Metriken mit Volumen }1},
\end{align*}
dann ist das Hilbert-Funktional gegeben durch
\begin{align*}
S: \mathcal M_1(M)\to \R,\qquad s(g) = \int_M \scal_g\; \mathrm{d}vol_g.
\end{align*}
\item Aufgrund topologischer Obstruktionen existieren 4-dimensionale
Mannigfaltigkeiten ohne Einstein-Metrik.
\item Die Existenz von Mannigfaltigkeiten der Dimension $\ge 5$ ohne
Einstein-Metrik ist v\"ollig ungekl\"art.
\item Ist $(M,g)$ zusammenh\"angend und Einstein, dann ist die Skalarkr\"ummung
konstant. Man kann diese Mannigfaltigkeiten klassifizieren nach
\begin{align*}
\scal_g < 0,\qquad \scal_g = 0,\qquad \scal_g  > 0.
\end{align*}
\end{enumerate}

\bigskip

\begin{Satz}[Schur]
Sei $(M^n,g)$ eine Riemannsche Mannigfaltigkeit mit $n\ge 3$ und
\begin{align*}
\Ric_g = f\cdot g
\end{align*}
mit einer glatten Funktion $f\in \CC^\infty(M)$. Dann ist $f$
konstant.
\end{Satz}

Die Beweisidee liegt darin, die Gleichung $\Ric_g = f\cdot g$ zu differenzieren
und daraus $\df = 0$ zu folgern. Daf\"ur ben\"otigen wir jedoch noch einige
technische Feinheiten.

\begin{Definition}
Sei $B$ ein $(0,r)$-Tensor und $\setd{e_i}$ eine ONB von $T_pM$. Dann ist die
\emph{Divergenz} von $B$ definiert durch
\begin{align*}
(\delta B)(X_1,\ldots,X_{r-1}) := -\sum_{i} (\nabla_{e_i}
B)(e_i,X_1,\ldots,X_{r-1}).
\end{align*}
\end{Definition}

\bigskip

{\bf Bemerkungen:}
\begin{enumerate}
  \item 
Seien $A,B$ zwei $(0,r)$-Tensoren, dann definiert man
\begin{align*}
\scp{A,B} = \sum_{i_1,\ldots,i_r}
A(e_{i_1},\ldots,e_{i_r})B(e_{i_1},\ldots,e_{i_r}).
\end{align*}
\item $\delta B$ ist ein $(0,r-1)$-Tensor.
\item $\delta g = 0$, denn die Metrik ist parallel bez\"uglich dem
Levi-Civita-Zusammenhang.
\item Ist $B$ symmetrisch, so auch $\delta B$.
\item Sei $B$ ein $(0,2)$-Tensor, dann ist
\begin{align*}
\tr_g(B) = \sum_i B(e_i,e_i) = \scp{B,g}.
\end{align*}
\end{enumerate}

\bigskip

\begin{Definition}
Sei $X$ ein Vektorfeld auf $M$, dann ist
\begin{align*}
\div(X) = \sum g(\nabla_{e_i}X,e_i)
\end{align*}
die Divergenz von $X$.
\end{Definition}
% 
% {\bf Beispiel:}
% Ist $M=\R^n$, dann ist 
% \begin{align*}
% \div(X) = \sum_{i=1}^n g(\nabla_{e_i}X,e_i)
% = \sum_{i=1}^n g(\partial_{i} X,e_i) = \sum_{i=1}^n \frac{\partial
% X_i}{\partial x_i}.
% \end{align*}

\bigskip

\begin{Lemma}
Sei $\omega_X$ die 1-Form zum Vektorfeld $X$, d.h.
\begin{align*}
\omega_X(Y) = g(X,Y).
\end{align*}
Dann gilt
\begin{align*}
\delta \omega_X = -\div(X).
\end{align*}
\end{Lemma}

\proof
$g$ ist metrisch und folglich,
\begin{align*}
\delta\omega_X &= -\sum(\nabla_{e_i}\omega_X)(e_i)
= -\sum\left(L_{e_i}(\omega_X(e_i)) - \omega_X(\nabla_{e_i}e_i)\right)\\
&= -\sum\left(L_{e_i}(g(X,e_i)) - g(X,\nabla_{e_i}e_i)\right)
= -\sum g(\nabla_{e_i}X,e_i)\\
& = -\div(X).
\end{align*}
\qed

\bigskip

{\bf Bemerkung:}
Der \emph{Laplace Operator} l\"asst sich nun f\"ur $f\in\CC^\infty(M)$ definieren
durch
\begin{align*}
\Delta f = -\div(\grad(f)) = - \delta(\df).
\end{align*}

\bigskip

{\bf Bemerkung:}
Sei $X\in T_pM$, dann existiert lokal eine sogenannte \textit{parallele}
Fortsetzung von $X$ zu einem lokalen Vektorfeld $\tilde{X}$ mit $(\nabla \tilde{X})_p = 0$.
Sei nun $B$ ein $(0,r)$-Tensor und $X_1,\ldots,X_r,V\in T_pM$ in diesem Sinne
parallel fortgesetzt, dann ist
\begin{align*}
(\nabla_V B)(X_1,\ldots,X_r)_p &= L_V(B(\tilde{X}_1,\ldots,\tilde{X}_r))_p
- \sum_{i=1}^r B(\ldots,\nabla_V \tilde{X}_i,\ldots)_p\\
&= L_V(B(\tilde{X}_1,\ldots,\tilde{X}_r)).
\end{align*}

\bigskip

\begin{Lemma}
$\delta \Ric = - \frac{1}{2}\,\diffd\scal_g$.
\end{Lemma}
\proof
Sei $p\in M$ und $\setd{e_i}$ eine ONB von $T_pM$, $v\in T_pM$ und
$e_1,\ldots,e_n,v$ seien parallel fortgestzt. Dann ist nach der 2.
Bianchi-Identit\"at
\begin{align*}
0 &= (\nabla_v R)(e_i,e_j,e_j,e_i) + 
(\nabla_{e_i} R)(e_j,v,e_j,e_i) + 
(\nabla_{e_j} R)(v,e_i,e_j,e_i)\\
&= L_v R(e_i,e_j,e_j,e_i)
+ L_{e_i} R(e_j,v,e_j,e_i)
+ L_{e_j} R(v,e_i,e_j,e_i).
\end{align*}
Summation \"uber $i$ und $j$ ergibt,
\begin{align*}
0 &= L_v \scal_g - 
\sum_{i} L_{e_i} \Ric(v,e_i)
- \sum_j L_{e_j} \Ric(v,e_j)\\
&= \diffd\scal_g(v) + 2 (\delta\Ric)(v).
\end{align*}
Somit folgt die Behauptung.
\qed

\bigskip

{\bf Beweis des Satzes von Schur:}
Sei also $\Ric = f\cdot g$ f\"ur ein $f\in \CC^\infty(M)$, dann ist
\begin{align*}
f = \frac{\scal_g}{n},\qquad n = \dim M.
\end{align*}
Es gilt nun nach dem vorigen Lemma,
\begin{align*}
-\frac{1}{2}\diffd\scal_g &= \delta\Ric = \delta(f\cdot g)\\
&= -\sum  (\nabla_{e_i} (f\cdot g))(e_i,\cdot)\\
&= -\sum  L_{e_i} (f\cdot g)(e_i,\cdot)\\
&= -\sum  L_{e_i} (f)\cdot g(e_i,\cdot)-\sum  f\cdot
\underbrace{L_{e_i}g(e_i,\cdot)}_{=0}\\ &= -g(\grad f,\cdot) = -\df =
-\frac{1}{n}\diffd\scal_g.
\end{align*}
Daher ist $0 = \left(\frac{1}{2}-\frac{1}{n}\right)\diffd\scal_g$ und
schlie\ss{}lich
\begin{align*}
\diffd\scal_g = 0,\qquad \text{f\"ur }n\neq 2.
\end{align*}
Also ist $f$ konstant.
\qed

\subsection{Lie-Gruppen}

Sei $G$ eine Lie-Gruppe mit Lie-Algebra $\g = T_eG$.

\bigskip

{\bf Bemerkungen:}
\begin{enumerate}
  \item 
Linksinvariante Metriken auf $G$ stehen in einer 1:1 Beziehung zu
Skalarprodukten auf der Lie-Algebra. Betrachte dazu  ein
Skalarprodukt $\scp{\cdot,\cdot}$ auf $\g$, dann definiert man
\begin{align*}
g_h(X,Y) = \scp{\diffd l_h(X),\diffd l_h(Y)} = 
l_{h}^*\scp{X,Y}.
\end{align*}
Analog definiert jede linksinvariante Metrik ein Skalarprodukt auf $\g$.
\item Bi-invariante Metriken sind Metriken, die links- und rechtsinvariant sind,
d.h. sowohl Links- als auch Rechtstranslation sind Isometrien.
\end{enumerate}

\bigskip

\begin{Lemma}
Sei $G$ eine zusammenh\"angende Lie-Gruppe mit einer links-invarianten Metrik
$\scp{\cdot,\cdot}$. Dann sind \"aquivalent:
\begin{enumerate}
  \item $\scp{\cdot,\cdot}$ ist bi-invariant.
  \item $\scp{\cdot,\cdot}$ ist $\Ad$-invariant.
  \item $\scp{\ad(X)Y,Z} + \scp{Y,\ad(X)Z} = 0$.
  \item Der Levi-Civita-Zusammenhang ist gegeben durch
\begin{align*}
\nabla_X Y = \frac{1}{2}[X,Y],\qquad \text{f\"ur }X,Y\in\g. 
\end{align*}
\end{enumerate}
\end{Lemma}

\proof
Bis auf die \"Aquivalenz 3. und 4. haben wir bereits alles in Kapitel 8 gezeigt.
Da $\scp{\cdot,\cdot}$ links-invariant ist, ist $\scp{X,Y}$ konstant f\"ur
$X,Y\in\g$ und somit ist nach der Koszul-Formel,
\begin{align*}
2\scp{\nabla_X Y,Z} &= -\scp{X,[Y,Z]} + \scp{Y,[Z,X]} + \scp{Z,[X,Y]}\\
&=\scp{X,[Z,Y]} + \scp{[Z,X],Y} + \scp{[X,Y],Z}.
\end{align*}
Somit ist
\begin{align*}
\scp{X,[Z,Y]} + \scp{[Z,X],Y} = 0 \qquad\Leftrightarrow\qquad \nabla_X Y =
\frac{1}{2}[X,Y].
\end{align*}
\qed

\begin{Folgerung}
\begin{enumerate}
  \item $R_{X,Y}Z = -\frac{1}{4}[[X,Y],Z]$.
  \item $K(X,Y) = \frac{1}{4}\frac{\norm{[X,Y]}^2}{\norm{X\wedge Y}^2}\ge 0$.
\end{enumerate}
\end{Folgerung}

\proof
Zu 1: Unter Verwendung der Jacobi-Identit\"at ist
\begin{align*}
R_{X,Y}Z &= \nabla_X\nabla_Y Z - \nabla_Y\nabla_X Z - \nabla_{[X,Y]}Z\\
&= \frac{1}{4}\left( [X,[Y,Z]] - [Y,[X,Z]] - 2[[X,Y],Z]\right)\\
&= \frac{1}{4}\left( [X,[Y,Z]] + [Y,[Z,X]] + 2[Z,[X,Y]]\right)\\
&= \frac{1}{4}[Z,[X,Y]].
\end{align*}
Zu 2:
\begin{align*}
K(X,Y) = \frac{R(X,Y,Y,X)}{\norm{X\wedge Y}^2}
= -\frac{1}{4} \frac{\scp{[[X,Y],Y],X}}{\norm{X\wedge Y}^2}
= \frac{1}{4} \frac{\scp{[X,Y],[X,Y]}}{\norm{X\wedge Y}^2}.
\end{align*}
\qed

{\bf Bemerkung:}
Auf Lie-Gruppen kennen wir bereits die Killing Form
\begin{align*}
B_G(X,Y) = \Tr(\ad(X)\circ\ad(Y)).
\end{align*}
$B_G$ ist genau dann nicht ausgeartet, wenn
$G$ halbeinfach ist.
Ist $G$ kompakt und halbeinfach, dann ist $B_G$ negativ definit. In diesem Fall
wird $G$ durch die Killing Form zu einer Riemannschen Mannigfaltigkeit
$(G,-B_G)$.

\bigskip

\begin{Satz}
Sei $G$ eine kompakte, halbeinfache Lie-Gruppe. Dann ist $(G,-B_G)$ eine
Einstein Mannigfaltigkeit.
\end{Satz}
\proof
Man berechnet die Ricci-Kr\"ummung zu
\begin{align*}
\Ric(X,Y) &= \Tr(Z\mapsto R_{Z,X}Y) = 
\Tr(Z\mapsto -\frac{1}{4}[[Z,X],Y]) = 
-\frac{1}{4}
\Tr(Z\mapsto [[Z,X],Y])\\
&=
-\frac{1}{4}
\Tr(Z\mapsto [X,[Y,Z]])\\
&= 
-\frac{1}{4}
\Tr(Z\mapsto \ad(X)\circ\ad(Y)(Z))\\
&=-\frac{1}{4}B_G(X,Y).
\end{align*}
\qed

\subsection{Riemannsche Produkte}

Seien $(M_1,g_1)$ und $(M_2,g_2)$ zwei Riemmansche Mannigfaltigkeiten. Wir
betrachten die Produktmannigfaltigkeit $M=M_1\times M_2$ mit der Metrik $g=
g_1+ g_2$. Dann idenitifiziert man die Tangentialr\"aume wie folgt
\begin{align*}
TM = TM_1\oplus TM_2,\qquad TM\ni X = X_1 + X_2,\qquad X_1\in TM_1,X_2\in TM_2.
\end{align*}
Der Levi-Civita-Zusammenhang auf $M$ ist daher gegeben durch,
\begin{align*}
\nabla_X Y = \nabla_{X_1}^{g_1}Y_1 + \nabla_{X_2}^{g_2}Y_2.
\end{align*}
F\"ur die Kr\"ummungsgr\"o\ss{}en erh\"alt man weiterhin
\begin{align*}
&R_{X,Y}Z = R_{X_1,Y_1}^{g_1}Z_1 + R_{X_2,Y_2}^{g_2}Z_2,\\
&\scal_g = \scal_{g_1} + \scal_{g_2},\\
&K(X,Y) = 0,\qquad \text{falls } X \in TM_i,Y\in TM_j,\quad i\neq j.
\end{align*}

{\bf Bemerkungen:}
\begin{enumerate}
  \item 
Man erh\"alt sofort, dass f\"ur die Kr\"ummung bez\"uglich der Produktmetrik
$K_{S^2\times S^2} \ge 0$.

\textit{Hopf Vermutung}: Es gibt keine Metrik auf $S^2\times S^2$ mit
\begin{align*}
K_g > 0,\qquad \text{\"uberall auf }S^2\times S^2.
\end{align*} 
\item Sind $M_1$ und $M_2$ Einstein, dann ist auch $M_1\times M_2$ Einstein,
falls
\begin{align*}
\frac{\scal_{g_1}}{\dim M_1} = 
\frac{\scal_{g_2}}{\dim M_2}.
\end{align*}
\end{enumerate}

\bigskip

{\bf Beispiele:}
\begin{enumerate}
  \item Die Produktmetrik auf $S^1\times S^{n-1}$ f\"ur $n\ge 3$ ist nicht
  Einstein. Denn Sei $X\in TS^1$, dann ist
  \begin{align*}
  \Ric(X,X) = \Ric_{S^1}(X,X) = 0,
  \end{align*}
  denn $S^1$ ist eine 1-dimensionale Mannigfaltigkeit. Jedoch ist f\"ur $0\neq
  Y\in TS^{n-1}$,
  \begin{align*}
  \Ric(Y,Y) = c(n-2)g(Y,Y)\neq 0.
  \end{align*}
  \item Auf $S^1\times S^2$ existiert \"uberhaupt keine Einsteinmetrik, denn
  \begin{align*}
  g \text{ Einstein } \Leftrightarrow K_g = \text{ konstant}.
  \end{align*}
  Wir werden noch sehen, dass $S^1\times S^2$ eine universelle \"Uberlagerung
  $\tilde{M}$ besitzt. Als universelle \"Uberlagerung ist $\tilde{M}$ einfach
  zusammenh\"angend und hat Dimension $1+2 = 3$. Verlangt man nun, dass die
  Schnittkr\"ummung konstant ist, ist $\tilde{M}$ diffeomorph zu $S^3$ oder
  $\R^3$, jedoch h\"atte dann $S^1\times S^2$ eine triviale
  Fundamentalgruppe $\Pi_2(M) = 0$ was unm\"oglich ist.
\end{enumerate}

\subsection{Kr\"ummung von Untermannigfaltigkeiten}


\renewcommand{\bar}[1]{\overline{#1}}

Sei $(\bar{M},\bar{g})$ eine Riemannsche Mannigfaltigkeit und $i: M\to \bar{M}$
eine $n$-dimensionale Untermannigfaltigkeit. Auf $M$ betrachtet man die
Riemannsche Metrik 
\begin{align*}
g = i^*\bar{g}  =\bar{g}\big|_{TM\times TM},
\end{align*}
wobei man $TM$ wieder als Unterraum von $T\bar{M}$ identifiziert. Dadurch
erh\"alt man punktweise die Zerlegung
\begin{align*}
T_p\bar{M} = T_pM \oplus T_pM^\bot,\qquad p\in M.
\end{align*}
Wir schreiben weiter $T\bar{M} = TM\oplus TM^\bot$ und jedes $X\in T\bar{M}$
l\"asst sich eindeutig darstellen als
\begin{align*}
X = X^\top + X^\bot,
\end{align*}
wobei $X^\top$ tangential und $X^\bot$ normal zu $TM$ ist.

\begin{Lemma}
Sei $X\in\chi(M)$ und $p\in M$, dann existiert eine offene Umgebung $U$ von $p$
in $\bar{M}$ und eine lokale Fortsetzung von $X$ zu einem Vektorfeld auf $U$. 
\end{Lemma}
\begin{proof}
Sei $(U,x)$ eine Karte um $p$ und weiter
\begin{align*}
X= \sum_{i=1}^n X_i
\frac{\partial}{\partial x_i},
\qquad
X_i\in\CC^\infty(M\cap U).
\end{align*}
$M$ ist eine Untermannigfaltigkeit von $\bar{M}$, also
$x(U\cap M) = x(U)\cap (\R^n\times\setd{0}) \subset \R^n\times \R^k$.
Definiert man nun als lokale Fortsetzung
\begin{align*}
\tilde{X}(x^{-1}(a,b)) = \sum_{i=1}^n X_i(x^{-1}(a,0)) \frac{\partial}{\partial
x_i},
\end{align*}
so ist $\tilde{X}\in\chi(\bar{M}\cap U)$ und $\tilde{X}\big|_{M\cap U} =
X\big|_U$.
\qed
\end{proof}

\begin{Satz}
Sei $M\subset\bar{M}$ eine Riemannsche Untermannigfaltigkeit und $\bar{\nabla}$
der Levi-Civita-Zusammenhang von $\bar{g}$. Dann berechnet sich der
Levi-Civita-Zusammenhang von $g$ durch
\begin{align*}
\nabla_XY = \left(\bar{\nabla}_{\tilde{X}}\tilde{Y} \right)^\top = 
\pr_{TM}\left(\bar{\nabla}_{\tilde{X}}\tilde{Y} \right),
\end{align*}
wobei $X,Y\in\chi(M)$ mit lokalen Fortsetzungen $\tilde{X},\tilde{Y}$.
\end{Satz}

\begin{proof}
\textit{Wohldefiniertheit}.
Wir zeigen zun\"achst, dass die Definition unabh\"angig von der Wahl der
Fortsetzung ist. Da $\nabla_X Y$ tensoriell in $X$ ist, ist nur die
$Y$-Komponente zu betrachten. Sei also $\tilde{Y} = \sum_{i=1}^n \tilde{Y}_i
\frac{\partial}{\partial x_i}$, wobei $\tilde{Y}_i\big|_{M\cap U} = Y_i$. Dann
ist
\begin{align*}
\left(\bar{\nabla}_{\tilde{X}}\tilde{Y}\right)_p =
\sum_{i=1}^n \left(\tilde{X}(\tilde{Y}_i)\frac{\partial}{\partial x_i}
+ \tilde{Y}_i \bar{\nabla}_{\tilde{X}}\frac{\partial}{\partial x_i}
\right)
\bigg|_p.
\end{align*}
$\bar{\nabla}$ ist tensoriell im ersten Eintrag und folglich 
\begin{align*}
\left(\bar{\nabla}_{\tilde{X}}\frac{\partial}{\partial x_i}\right)_p = 
\bar{\nabla}_{X_p} \frac{\partial}{\partial x_i}.
\end{align*}
Weiterhin ist $X_p\in T_pM$, also ist auch
\begin{align*}
\tilde{X}(\tilde{Y}_i)_p = X_p(\tilde{Y}_i) = X_p(Y_i).
\end{align*}
Somit h\"angt $\left(\bar{\nabla}_{\tilde{X}}\tilde{Y}\right)_p$ f\"ur $p\in M$ nur
von den Werten von $X$ und $Y$ in $p$ ab, ist also unabh\"angig von den
Fortsetzungen.

$\nabla$ ist damit ein wohldefinierter Zusammenhang auf $M$. Wir zeigen nun,
dass $\nabla$ torsionsfrei und metrisch ist, es folgt dann, dass $\nabla$ der
eindeutige Levi-Civita-Zusammenhang auf $M$ ist.

\textit{$\nabla$ ist torsionsfrei}. Es ist $X = \tilde{X}\big|_M$, d.h. $X$ und
$\tilde{X}$ sind $i$-verkn\"upft, ebenso $Y$ und $\tilde{Y}$. Somit sind auch
$[X,Y]$ und $[\tilde{X},\tilde{Y}]$ $i$-verkn\"upft, insbesondere ist
$[\tilde{X},\tilde{Y}]_p$ tangential an $TM$ f\"ur $p\in M$. Somit ist
\begin{align*}
\nabla_X Y - \nabla_Y X - [X,Y]\bigg|_p &= 
\left(\bar{\nabla}_{\tilde{X}}\tilde{Y} \right)^\top
-
\left(\bar{\nabla}_{\tilde{Y}}\tilde{X} \right)^\top
- [\tilde{X},\tilde{Y}]
\bigg|_p\\
&= 
\pr_{TM}\left(\bar{\nabla}_{\tilde{X}}\tilde{Y} -
\bar{\nabla}_{\tilde{Y}}\tilde{X}
- [\tilde{X},\tilde{Y}] 
  \right) = 0.
\end{align*}
\textit{$\nabla$ ist metrisch}. Man f\"uhrt dies darauf zur\"uck, dass
$\bar{\nabla}$ metrisch ist,
\begin{align*}
X(g(Y,Z))_p &= \tilde{X}(\bar{g}(\tilde{Y},\tilde{Z}))_p = 
\bar{g}(\nabla_{\tilde{X}}\tilde{Y},\tilde{Z})_p
+ \bar{g}(\tilde{Y},\nabla_{\tilde{X}}\tilde{Z})_p\\
&= \bar{g}((\nabla_{\tilde{X}}\tilde{Y})_p,Z_p)
+ \bar{g}(Y_p,(\nabla_{\tilde{X}}\tilde{Z})_p)\\
&= g(\nabla_X Y , Z)_p + g(Y,\nabla_X Z)_p. 
\end{align*}

\qed
\end{proof}

\bigskip

{\bf Beispiel:}
Wir betrachten auf $S^n\subset\R^{n+1}$ zwei Vektorfelder $X,Y\in\chi(M)$. Wie
bereits festgestellt ist das Normalenvektorfeld gegeben durch 
\begin{align*}
N: S^n\subset\R^{n+1} \to TS^n \subset\R^{n+1},\; p\mapsto N_p = p,
\end{align*}
mit $X(N) = X$. Somit gilt nach obiger Formel f\"ur den Levi-Civita-Zusammenhang
auf $S^n$,
\begin{align*}
\nabla_XY &= \left(\nabla_{\tilde{X}}\tilde{Y} \right)^\top
= \tilde{X}(\tilde{Y}) - \scp{\tilde{X}(\tilde{Y}),N}N
= \tilde{X}(\tilde{Y}) + \scp{\tilde{Y},\tilde{X}(N)}N\\
&= \tilde{X}(\tilde{Y}) + \scp{X,Y}N.
\end{align*}

\bigskip

\newcommand{\II}{\mathrm{II}}

\begin{Definition}
Sei $M\subset\bar{M}$ eine Riemannsche Untermannigfaltigkeit. Dann ist die
\emph{2. Fundamentalform} von $M$ in $\bar{M}$ definiert als
\begin{align*}
\II(X,Y) = (\bar{\nabla}_{\tilde{X}} \tilde{Y} )^\bot,
\end{align*}
f\"ur Vektorfelder $X,Y\in\chi(M)$.
\end{Definition}

\bigskip

{\bf Beispiel:}
F\"ur $S^n\subset\R^{n+1}$ ist die 2. Fundamentalform gegeben durch
\begin{align*}
\II(X,Y) = \scp{\bar{\nabla}_{\tilde{X}}\tilde{Y},N}N
= -\scp{X,Y}n.
\end{align*}

\bigskip

\begin{Lemma}
$\II$ ist eine symmetrische $\CC^\infty(M)$-bilineare Abbildung.
\end{Lemma}
\begin{proof}
% Wir zeigen nur die $\CC^\infty(M)$-linearit\"at im 2. Argument.
% \begin{align*}
% \II(X,f\cdot Y)_p &= \left(\nabla_{\tilde{X}} \tilde{f}\tilde{Y} \right)_p^\bot
% = \left( \tilde{X}(\tilde{f})\tilde{Y} + \tilde{f}\nabla_{\tilde{X}}\tilde{Y}
% \right)_p^\bot\\
% &= \left(\tilde{f} \nabla_{\tilde{X}}\tilde{Y} \right)_p^\bot
% = f(p)\cdot  \II(X,Y)_p.
% \end{align*}
Wir zeigen nur die Symmetrie, die $\CC^\infty(M)$-Bilinearit\"at folgt dann
sofort.
\begin{align*}
\II(X,Y)-\II(Y,X) = \left(\bar{\nabla}_{\tilde{X}}\tilde{Y} -
\bar{\nabla}_{\tilde{Y}}\tilde{X} \right)^\bot =
\left([\tilde{X},\tilde{Y}] \right)^\bot = 0,
\end{align*}
da $[\tilde{X},\tilde{Y}]_p \in T_pM$ f\"ur $p\in M$.
\qed
\end{proof}

\bigskip

{\bf Bemerkung:}
$\II$ definiert punktweise eine $\R$-bilineare Abbildung
\begin{align*}
\II_p : T_pM\times T_p M \to T_pM^\bot,\qquad p\in M.
\end{align*}

\bigskip

Als unmittelbare Konsequenz aus den Definitionen ergibt sich die

\begin{Satz}[Gau\ss{}-Formel]
$\bar{\nabla}_X Y = \nabla_X Y + \II(X,Y)$ f\"ur $X,Y\in\chi(M)$.
\end{Satz}

\bigskip

\begin{Definition}
Der \emph{Form-Operator} in Richtung eines Normalenfeldes $\xi\in
\Gamma(TM^\bot)$ ist definiert als
\begin{align*}
A_\xi X = -(\bar{\nabla}_{\tilde{X}}\xi)^\top, 
\end{align*}
d.h. $A_\xi\in \End TM$.
\end{Definition}

\bigskip

\begin{Lemma}
$A_\xi$ ist ein selbstadjungierter Endomorphismus von $TM$ und es gilt die
Formel
\begin{align*}
\bar{g}(\II(X,Y),\xi) = \bar{g}(A_\xi X, Y), \qquad X,Y\in\chi(M).
\end{align*}
\end{Lemma}

\medskip

\begin{proof}
Die Selbstadjungiertheit folgt direkt aus der Formel. Zu dieser betrachte
\begin{align*}
\bar{g}(\II(X,Y),\xi) &= \bar{g}( (\bar{\nabla}_{\tilde{X}}\tilde{Y}),\xi )
= \tilde{X}(\underbrace{\bar{g}(\tilde{Y},\xi)}_{=0}) -
\bar{g}(\tilde{Y},\bar{\nabla}_{\tilde{X}}\xi)\\
&= \bar{g}(\tilde{Y},-\bar{\nabla}_{\tilde{X}}\xi)
= \bar{g}(\tilde{Y},A_\xi X).
\end{align*}
\qed
\end{proof}

\bigskip

\begin{Satz}[Gau\ss{}-Gleichung]
Seien $X,Y,V,W\in\chi(M)$, dann gilt
\begin{align*}
R(X,Y,V,W) = \bar{R}(X,Y,V,W) - \bar{g}(\II(X,V),\II(Y,W))
+ \bar{g}(\II(X,W),\II(Y,V))
\end{align*}
\end{Satz}

\begin{proof}
Wir k\"onnen ohne Einschr\"ankung annehmen, dass $[X,Y] = 0$. Dann ist
\begin{align*}
\bar{R}(X,Y,V,W) = \bar{g}(\bar{\nabla}_{X}\bar{\nabla}_Y V -
\bar{\nabla}_Y\bar{\nabla}_X V,W).
\end{align*}
Verwenden wir $\bar{\nabla}_XY = \nabla_X Y  +\II(X,Y)$, so erhalten wir
\begin{align*}
\bar{g}(\bar{\nabla}_{X}\bar{\nabla}_Y V,W) &= 
\bar{g}(\bar{\nabla}_X\nabla_Y V, W) 
+ \bar{g}(\bar{\nabla}_X \II(Y,V),W)\\
&=
\bar{g}(\nabla_X\nabla_Y V, W) 
+ \underbrace{\bar{g}(\II(X,\nabla_Y V),W)}_{=0} 
+ \bar{g}(\bar{\nabla}_X \II(Y,V),W).
\end{align*}
Weiterhin ist
\begin{align*}
\bar{g}(\bar{\nabla}_X \II(Y,V),W) &= 
X(\underbrace{\bar{g}(\II(V,Y),W)}_{=0}) -
\bar{g}(\II(Y,V),\bar{\nabla}_X W)\\
&= -\underbrace{\bar{g}(\II(Y,V),\nabla_X W)}_{=0}
- \bar{g}(\II(Y,V),\II(X,W)).
\end{align*}
Zusammenfassend ergibt sich
\begin{align*}
\bar{R}(X,Y,V,W) &= \bar{g}(\nabla_X\nabla_Y V, W) - 
\bar{g}(\nabla_Y\nabla_X V, W)\\
&- \bar{g}(\II(Y,V),\II(X,W))
+ \bar{g}(\II(X,V),\II(Y,V))\\
&= R(X,Y,V,W) - \bar{g}(\II(Y,V),\II(X,W))
+ \bar{g}(\II(X,V),\II(Y,W)).
\end{align*}
\qed
\end{proof}

\bigskip

\begin{Folgerung}
Sei $\setd{X,Y}$ eine Basis f\"ur einen 2-dimensionalen Unterraum von $T_pM$. Dann
ist
\begin{align*}
K(X,Y) = \bar{K}(X,Y) - \frac{\norm{\II(X,Y)}^2 - \bar{g}(\II(X,X),\II(Y,Y))}{
\norm{X\wedge Y}^2} 
\end{align*}
\end{Folgerung}

\bigskip

{\bf Beispiel:}
Wir betrachten $S^n\subset\R^{n+1}$. Dann ist $\bar{K} = 0$, denn der $\R^{n+1}$
ist flach und $\II(X,Y) =\scp{X,Y}N$. Somit erh\"alt man
\begin{align*}
K(X,Y) = 0 - \frac{\scp{X,Y}^2 - \abs{X}^2\abs{Y}^2}{\abs{X}^2\abs{Y}^2-
\scp{X,Y}^2} = 1.
\end{align*}
\bigskip

{\bf Bemerkung:}
Den zweiten Summanden, also  $\frac{\norm{\II(X,Y)}^2 - \bar{g}(\II(X,X),\II(Y,Y))}{
\norm{X\wedge Y}^2}$, nennt man auch Gau\ss{}-Kr\"ummung. Er  h\"angt von der Einbettung
$M\subset\bar{M}$ ab. Wie die obige Formel zeigt, stimmt er allerdings im Fall von Hyperfl\"achen 
des $\R^3$ mit der intrinsisch definierten Schnittkr\"ummung \"uberein, ist dann also 
unabh\"angig von der Einbettung. Diese Tatsache ist Inhalt des Theorema Egregium von
Gau\ss{}.

\bigskip

{\bf Beispiel:}
Sei $M^2\subset\R^3$. Wir betrachten das Normalenvektorfeld $N: M\to
S^2\subset\R^3$. Die Weingartenabbildung ist
\begin{align*}
\diffd N : T_pM \to T_{n(p)}S^2 =N(p)^\bot = T_pM
\end{align*}
Wir berechnen $\diffd n$ mit Hilfe der Integralkurve $c$ von $X$, $c(0) =p$
und $\dot{c}(0) = X$. Dann ist
\begin{align*}
\diffd N(X) = \frac{\diffd}{\dt}\bigg|_{t=0} N(c(t))
= X(N) = \bar{\nabla}_X N \overset{!}{=} (\bar{\nabla}_X N)^\top = -A_NX, 
\end{align*}
denn
\begin{align*}
g(\bar{\nabla}_X N,N) = \frac{1}{2}L_X g(N,N) = 0
\end{align*}
und folglich ist $\bar{\nabla}_X N = (\bar{\nabla}_X N)^\top$. Somit entspricht
die Weingarten-Abbildung gerade dem Form-Operator.
\bigskip

{\bf Bemerkungen:}
\begin{enumerate}
\item 
Der Form-Operator ist selbstadjungiert also diagonaliserbar. Seine Eigenwerte
hei\ss{}en Hauptkr\"ummungen, seine Eigenvektoren Hauptkr\"ummungsrichtungen.
\item Das \emph{Hauptkr\"ummungs Vektorfeld} ist gegeben durch
\begin{align*}
H = \frac{1}{\dim M} \Tr\,\II.
\end{align*}
\item Eine spezielle Klasse von Mannigfaltigkeiten sind die \emph{total
geod\"atischen Untermannigfaltigkeiten}. Das sind Untermannigfaltigkeiten mit
$\II \equiv 0$.
\item Eine Abschw\"achung davon sind die \emph{Minimalfl\"achen}, das sind
Untermannigfaltigkeiten mit $H\equiv 0$, d.h. die Hauptkr\"ummungen heben sich in
jedem Punkt auf, die mittlere Kr\"ummung ist Null.
\item CMC-Fl\"achen (constant-mean-curvature) lassen beliebige aber konstante
mittlere Kr\"ummung zu. Dies sind gerade die Untermannigfaltigkeiten f\"ur die das
Hauptkr\"ummungs Vektorfeld parallel ist $\bar{\nabla}H = 0$.
\item \emph{Nabel-Fl\"achen} (total umbilisch) sind Untermannigfaltigkeiten mit
\begin{align*}
\II(X,Y) = g(X,Y)N.
\end{align*}
\end{enumerate}




\clearpage



\section{Der Indexssatz von Poincare-Hopf und der Satz von Gau\ss{}-Bonet}

\newcommand{\ind}{\mathrm{ind}}
\newcommand{\Hess}{\mathrm{Hess}}
\newcommand{\dvol}{\mathrm{d}vol}
\newcommand{\NN}{\mathcal{N}}

\subsection{Integration auf Mannigfaltigkeiten}

Sei $M$ wieder eine differenzierbare Mannigfaltigkeit der Dimension $n$.
Um technische Komplikationen bei der Integration zu vermeiden, beschr\"anken wir
uns auf $n$-Formen auf $M$ mit \emph{kompaktem Tr\"ager}.

\begin{Definition}
Der Tr\"ager einer Form $\omega\in\Omega^*(M)$ ist definiert durch
\begin{align*}
\mathrm{supp}\, \omega := \overline{\setdef{p\in M}{\omega_p\neq 0}}\subset M.
\end{align*}
\end{Definition}
\medskip
Ist $M$ selbst kompakt, so haben alle $n$-Formen auf $M$ kompakten
Tr\"ager.
\medskip
\begin{Definition}
Eine Form mit kompaktem Tr\"ager hei\ss{}t \emph{integrierbar}, wenn f\"ur
jede Karte $(U,x)$ bez\"uglich der Darstellung
\begin{align*}
\omega\big|_U = \sum_I \omega_I \dx_I,
\end{align*} 
die Funktionen $\omega_I$ Lebesgue-integrierbar sind. Den Raum der
integrierbaren $n$-Formen mit kompaktem Tr\"ager bezeichnen wir mit
$\Omega_c^n(M)$.
\end{Definition}
 
Sei nun $\omega\in\Omega_c^n(M^n)$ und $(U_i,\ph_i)$ seien Karten, die $\supp
\omega$ \"uberdecken. Weiterhin sei $(U_i,\rho_i)$ eine den Karten
untergeordnete Zerlegung der Eins, dann definieren wir das \emph{Integral \"uber
$\omega$} auf $M$, indem wir $\omega$ unter den Karten $\ph_i$ auf den $\R^n$
zur\"uckziehen,
\begin{align*}
\int_M \omega := \sum_i \int_{\R^n} \ph_i^*(\rho_i\omega).
\end{align*}

\begin{Satz}[Stokes]
Sei $M$ eine orientierte Mannigfaltigkeit mit glattem Rand $\partial M$, dessen
Orientierung der der von $M$ induzierten entspricht. Sei
$\omega\in\Omega_c^{n-1}(M)$. Dann gilt
\begin{align*}
\int_M \domega = \int_{\partial M} \omega.
\end{align*}
\end{Satz}

\subsection{Der Abbildungsgrad}

Seien $M$ und $N$ kompakte, zusammenh\"angende, orientierte $n$-dimensionale
Mannigfaltigkeiten.

\begin{Definition}
Sei $f: M\to N$ differenzierbar und $\omega\in\Omega_c^n(M)$. Dann definiert man
den \emph{Abbildungsgrad} $\deg(f)$ durch
\begin{align*}
\int_M f^*\omega =  \deg(f)\int_N\omega.
\end{align*}
\end{Definition}

Diese Definition ist tats\"achlich sinnvoll, d.h. h\"angt nicht von der gew\"ahlten
$n$-Form $\omega$ ab. Au\ss{}erdem ist der Abbildungsgrad erstaunlicherweise
stets eine ganze Zahl. Um dies einzusehen ben\"otigen wir zun\"achst folgende
Definition:

\begin{Definition}
Sei $f: M\to N$ differenzierbar und $q$ ein regul\"arer Punkt von $f$. Dann
definiert man
\begin{align*}
\sign(\df_q) = 
\begin{cases}
+1, & \df_q \text{ ist orientierungserhaltend},\\
-1, & \df_q \text{ ist orientierungsumkehrend}. 
\end{cases}
\end{align*}
\end{Definition}

\bigskip

{\bf Bemerkung:} Ist $f: M\to N$ ein Diffeomorphismus, dann ist $\sign(\df_q)$
konstant auf $M$ und weiterhin f\"ur beliebiges $\omega\in\Omega_c^n(N)$,
\begin{align*}
\int_M f^*\omega = \sign(\df)\int_N\omega.
\end{align*}
Insbesondere ist $\deg(f) = \sign(\df_q)$ f\"ur jedes $q\in M$.

\bigskip

\begin{Satz}
Sei $p$ ein regul\"arer Wert von $f$. Dann gilt
\begin{align*}
\deg(f) = \sum_{q\,\in\, f^{-1}(p)} \sign(\df_q).
\end{align*}
\end{Satz}

\begin{proof}
$p$ ist regul\"arer Wert von $f$, somit ist $f^{-1}(p)$ diskret und kompakt,
also endlich,
\begin{align*}
f^{-1}(p) = \setd{q_1,\ldots,q_k}.
\end{align*}
Es existieren daher Umgebungen $U$ von $p$ und $V_i$ von $q_i$, so dass
\begin{align*}
f^{-1}(U) = \bigcup_{i=1}^k V_i,\qquad
f_i := f\big|_{V_i} : V_i\to U\text{ ist Diffeomorphismus}.
\end{align*}
Insbesondere ist jedes $f_i$ eingeschr\"ankt auf $V_i$ entweder \"uberall
orientierungserhaltend oder orientierungsumkehrend. W\"ahle nun
$\omega\in\Omega_c^n(N)$ mit $\supp\omega\subset U$ und $\int_N \omega = 1$.
Dann gilt
\begin{align*}
\deg(f) &= \int_M f^*\omega =
\int_M f^*(\omega\big|_U)
=
\sum_{i=1}^k 
\int_{V_i} f_i^*(\omega\big|_U)\\
 &=
 \sum_{i=1}^k 
 \sign(\df_{q_i})\int_{U}\omega \\ 
 &=
 \sum_{q\, \in\, f^{-1}(p)} 
 \sign(\df_{q}). 
\end{align*}
\qed
\end{proof}

\bigskip

\begin{Satz}
Sei $f:\partial X \to N$, wobei $X$ $n+1$-dimensional, kompakt, orientiert und
$f = F\big|_{\partial X}$ f\"ur ein $F: X\to N$. Dann gilt
\begin{align*}
\deg(f) = 0.
\end{align*}
\end{Satz}
\begin{proof}
Sei $\omega\in\Omega^n(N)$ mit $\int_{N}\omega = 1$. Dann gilt
nach dem Satz von Stokes
\begin{align*}
\deg(f) = \int_{\partial X} f^*\omega = \int_X\diffd F^*\omega
= \int_X F^*\domega = 0.
\end{align*}
Denn $\domega\in\Omega^{n+1}(N) = \setd{0}$.
\qed
\end{proof}

\subsection{Index von Vektorfeldern}

Sei $F\in\CC^\infty(U\to\R^n)$ ein Vektorfeld auf einer offenen Umgebung
$U\subset\R^n$. Sei $0\in U$ eine isolierte Nullstelle von $F$, d.h. es gibt ein
$\ep > 0$, so dass $0$ die einzige Nullstelle ist von $F$ auf der Kugel
\begin{align*}
B_\ep := \setdef{x\in\R^n}{\norm{x}<\ep}\subset U.
\end{align*}

\begin{Definition}
Der \emph{lokale Index} von $F$ in $0$ ist
\begin{align*}
\ind_0(F) = \deg\left(S^{n-1}\cong\partial B_\ep\ni p \mapsto
\frac{F(p)}{\norm{F(p)}}\in S^{n-1} \right).
\end{align*}
\end{Definition}

\bigskip

{\bf Bemerkung:}
Die Definition von $F$ ist invariant unter Homotopien und h\"angt insbesondere
nicht von $\ep$ ab. Somit ist der lokale Index bereits f\"ur stetige Abbildungen
definiert, denn die glatten Abbildungen bilden eine dichte Teilmenge der
stetigen Abbildungen, d.h. es existiert f\"ur jede stetige Abbildung eine
Homotopie zu einer differenzierbaren.

\bigskip

\begin{Definition}
Sei $V$ ein Vektorfeld auf $M$. $\NN(V)$ bezeichne die \emph{Menge der
isolierten Nullstellen} von $V$. Sei weiter $p\in \NN(V)$ und $(U,x)$ eine
Karte um $p$ mit $x(p) = 0$. Dann definiert man den \emph{lokalen Index} von
$V$ in $p$,
\begin{align*}
\ind_p(V) = \ind_0(x_*V).
\end{align*}
\end{Definition}

\bigskip

\begin{Lemma}
Sei $p\in M$ eine nicht entartete Nullstelle von $V$, d.h. $x_* V$ ist ein
lokaler Diffeomorphismus, dann ist
\begin{align*}
\ind_p(V) = \sign(\det \diffd(x_*V)_0).
\end{align*}
\end{Lemma}

\bigskip

\begin{Definition}
Sei $V$ ein Vektorfeld auf $M$, das nur isolierte Nullstellen besitzt.
Der \emph{totale Index} von $V$ ist definiert als
\begin{align*}
I(V) = \sum_{p\,\in\,\NN(V)}\ind_p(V).
\end{align*}
\end{Definition}

\subsection{Der Indexsatz von Poincare-Hopf}

\begin{Satz}[Indexsatz von Poincare-Hopf]
Sei $M$ eine kompakte Mannigfaltigkeit und $V$ ein Vektorfeld mit isolierten
Nullstellen. Dann ist $I(V)$ eine topologische Invariante. Insbesondere ist
$I(V)$ unabh\"angig von $V$ und man setzt $I(M) := I(V)$.
\end{Satz}

{\bf Zur Beweisidee:}
Sei $X\subset\R^m$ eine kompakte berandete Mannigfaltigkeit der Dimension $m$.
Die \emph{Gau\ss{}-Abbildung} bildet einen Punkt auf dem Rand auf die \"au\ss{}ere Normale
ab,
\begin{align*}
g : \partial X\to S^{m-1},\qquad x\mapsto N_x.
\end{align*}
Der Indexsatz von Poincare-Hopf folgt in diesem Spezialfall nun aus folgendem
Lemma:

\begin{Lemma}[Hopf]
Sei $V$ ein Vektorfeld auf $X$ mit isolierten Nullstellen und auf $\partial X$
zeige $V$ "`nach au\ss{}en"'. Dann gilt
\begin{align*}
I(V) = \sum_{p\,\in\,\NN(V)} \ind_p(V) = \deg(g),
\end{align*}
insbesondere h\"angt der totale Index $I(V)$ nicht von $V$ ab.
\end{Lemma}
\begin{proof}
Sei $(p_i)$ eine Abz\"ahlung der endlich vielen Nullstellen $\NN(V)$ von $V$ und
$B_i$ seien Kugeln um $p_i$, so dass alle $B_i$ disjunkt sind und ganz in $X$
liegen. Man setzt dann $\bar{X} := X\setminus B_i$ und
\begin{align*}
\bar{V}(x) = \frac{V(x)}{\norm{V(x)}},\qquad x\in \bar{X},
\end{align*}
dann gilt aufgrund der Homotopie\"aquivalenz des Abbilungsgrads,
\begin{align*}
0 = \deg(\bar{V}\big|_{\partial \bar{X}}) =
\deg(\bar{V}\big|_{\partial X}) - \sum_i \ind\left(\bar{V}\big|_{\partial B_i}
\right),
\end{align*}
und $g$ ist homotop zu $\bar{V}\big|_{\partial X}$
\qed
\end{proof}

\bigskip

Beliebige unberandete Mannigfaltigkeiten besitzen nach dem Satz von
Whitney eine Einbettung in $\R^N$ f\"ur $N$ gro\ss{} genug. Man betrachtet dann
"`$\ep$-Verdickungen"' um die Mannigfaltigkeit,
\begin{align*}
M_\ep = \setdef{x\in\R^N}{d(x,M) \le \ep},
\end{align*}
so dass $M_\ep$ tats\"achlich eine $N$-dimensionale Mannigfaltigkeit ist,
und verf\"ahrt analog zur obigen Situation.

\bigskip

F\"ur berandete Mannigfaltigkeiten ist die Situation jedoch deutlich verwickelter.


\bigskip

{\bf Bemerkungen:}
\begin{enumerate}
  \item Existiert ein Vektorfeld ohne Nullstellen, dann folgt $I(M) = 0$.
  \item Der \textit{Satz von Hopf} besagt, dass unter gewissen
  Voraussetzungen auch die Umkehrung gilt, d.h. ist $I(M) = 0$, dann existiert
  ein Vektorfeld ohne Nullstellen.
\end{enumerate}

\bigskip

{\bf Beispiele:}
\begin{enumerate}
  \item 
Auf $S^n$ betrachten wir das Vektorfeld $V(x) = p_+ - \scp{p_+,x}x$, wobei $p_+$
den Nordpol bezeichnet. Man sieht leicht ein, dass $V$ genau zwei Nullstellen hat,
n\"amlich den Nord $p_+$ und den S\"udpol $p_-$. Weiterhin ist
\begin{align*}
\ind_{p_\pm}(V) = (\mp 1)^n,
\end{align*}
und somit folgt
\begin{align*}
I(S^n) = 
\begin{cases}
2, & \dim S^n\text{ gerade},\\
0, & \text{sonst}.
\end{cases}
\end{align*}
Als Spezialfall erhalten wir somit sofort den Satz vom Igel.
\item Ist $\dim M = n$ ungerade, folgt $I(M) = 0$, denn es gilt f\"ur ein
beliebiges Vektorfeld $V$ auf $M$,
\begin{align*}
I(-V) = (-1)^nI(V).
\end{align*}
\item Sei $M$ eine kompakte Fl\"ache, d.h. eine kompakte, randlose
2-Mannigfaltigkeit. Dann ist $M$ triangulierbar und es gilt
\begin{align*}
I(M) = \#\text{Ecken }  -
\#\text{Kanten }  +
\#\text{Fl\"achen }.
\end{align*}
Dann ist $M$ Hom\"oomorph zu $\Sigma_g$ der Standard-Fl\"ache vom Geschlecht $g$,
d.h. $\Sigma_g$ hat $g$-L\"ocher. Es folgt, dass
\begin{align*}
I(\Sigma_g) = 2-2g.
\end{align*}
Insbesondere ist $T^2$ die einzige orientierte kompakte Fl\"ache mit einem
Vektorfeld ohne Nullstellen.
\end{enumerate}

\subsection{Euler Charakteristik und Index}

\begin{Definition}
Die \emph{Euler Charakteristik} von $M$ ist definiert als
\begin{align*}
\chi(M) := \sum_{k=0}^n (-1)^k \beta_k,\qquad \beta_k = \dim
H_{dR}^k(M).
\end{align*}
$\beta_k$ hei\ss{}t auch \emph{$k$-te Betti-Zahl}.
\end{Definition}

\bigskip

\begin{Satz}
\label{Euler-Index}
$\chi(M) = I(M)$.
\end{Satz}

Diesen Satz kann man mit Hilfe der Morse Theorie beweisen, die sich mit den
Eigenschaften kritischer Punkte von Funktionen befasst.

\bigskip

\begin{Definition}
Sei $f\in\CC^\infty(M)$. Ein Punkt $p\in M$ hei\ss{}t \emph{kritischer Punkt} von
$f$, falls
\begin{align*}
\grad_p(f) = 0.
\end{align*} 
Ein kritischer Punkt $p$ hei\ss{}t \emph{nicht entartet}, falls 
\begin{align*}
\det \Hess(f)_p \neq 0,\qquad \Hess(f) = \nabla \df.
\end{align*}
Der \emph{Index} eines nicht entarteten kritischen Punktes $p$ ist die Dimension
des maximalen Unterraums von $T_pM$ auf dem die Hessische negativ definit ist, d.h.
\begin{align*}
\ind(p) = \max\setdef{\dim V}{V\le T_pM,\quad \Hess(f)_p\big|_V \le 0}.
\end{align*} 
\end{Definition}

\bigskip

\begin{Lemma}[Morse]
Sei $p$ ein nichtentarteter kritischer Punkt von $f$. Dann existieren lokale
Koordinaten $(U,x)$ um $p$ und ein $\lambda\in\N$, so dass
\begin{align*}
f\circ x^{-1} = f(p) - x_1^2 - \ldots - x_\lambda^2 + x_{\lambda+1}^2 + \ldots +
x_n^2.
\end{align*} 
$p$ hat dann den Index $\lambda$. Weiterhin gilt dann
\begin{align*}
\ind(p) = \ind_p(\grad f) = (-1)^\lambda,
\end{align*}
und insbesondere ist
\begin{align*}
I(\grad f ) = \sum_{\lambda = 0}^n (-1)^\lambda c_\lambda,\qquad
c_\lambda = \#\setd{\text{krit. Punkte vom Index }\lambda}.
\end{align*}
\end{Lemma}

\bigskip

Mit dem Lemma von Morse l\"asst sich der Beweis von Satz \ref{Euler-Index}
erbringen, indem man die Gleichheit
\begin{align*}
\chi(M) = \sum_{\lambda\ge 0} (-1)^\lambda c_\lambda,
\end{align*}
zeigt.

\subsection{Der Satz von Gau\ss{}-Bonet}

Sei nun $M^2\subset\R^3$ eine orientierte Fl\"ache, dann ist $\chi(M) = 2-2g$,
wenn $g$ das Geschlecht von $M$ bezeichnet.

\begin{Satz}[Gau\ss{}-Bonet]
Sei $K$ die Gau\ss{}-Kr\"ummung von $M$. Dann gilt
\begin{align*}
2\pi\ \chi(M) = \int_M K\;\dvol_M.
\end{align*} 
\end{Satz}

\begin{Lemma}
Sei $N: M\to S^2$, $p\mapsto N_p$ die Gau\ss{}-Abbildung, dann gilt
\begin{align*}
K(p) = \det \diffd N_p,\qquad p\in M. 
\end{align*}
\end{Lemma}

\proof
$N$ bildet einen Punkt von $M$ auf die \"au\ss{}ere Normale ab. Sei $\setd{X,Y}$
eine Basis von $T_pM$, und $\omega_{Z} = \bar{g}(\cdot,Z)$. Dann ist
\begin{align*}
\bar{g}(\II(X,X),\II(Y,Y))
&= \bar{g}(\II(X,X),N)\bar{g}(\II(Y,Y),N)\\
&= \omega_X(A_N X)\omega_Y(A_NY),
\end{align*}
und analog $\norm{\II(X,Y)}^2 = \omega_X(A_NY)\omega_Y(A_N X)$. Folglich gilt
\begin{align*}
K(X,Y) &= -\frac{\norm{\II(X,Y)}^2-\bar{g}(\II(X,X),\II(Y,Y))}
{\norm{X\wedge Y}^2} \\ &=
\frac{A_N^*(\omega_X\wedge\omega_Y)(X,Y)}{(\omega_X\wedge\omega_Y)(X,Y)}\\
&= \det A_N,
\end{align*}
denn $\omega_X\wedge\omega_Y$ ist eine Volumenform auf $M$. Wie schon bemerkt
ist $\diffd N_p =  -A_N$ also ist $K(p) = K(X,Y) = (-1)^2\det \diffd N_p$.
\qed

\bigskip

{\bf Beweis des Satzes von Gau\ss{}-Bonet:}
Nach dem obigen Lemma und der Morse Theorie ist also zu zeigen,
\begin{align*}
2\pi\ I(\grad h) = \int_M \det\diffd N_p\;\dvol_p, 
\end{align*}
f\"ur ein geeignetes $h: M\to \R$, das keine entarteten kritischen
Punkte besitzt. Wir wollen nun solch ein $h$ w\"ahlen. 
Sei dazu $i:M\to \R^3$ die Einbettung von $M$ und
\begin{align*}
f: \R^3\to \R,\qquad (x,y,z)\mapsto z,
\end{align*}
dann existieren lokal um jeden Punkt $p$ auf $M$ Koordinaten
und eine Parametrisierung $F$ von $M$, so dass
\begin{align*}
F: M\to \R^3,\qquad (u,v) \mapsto (u,v,f\circ i(u,v)).
\end{align*}
Setzen wir $h=f\circ i$, dann gilt
\begin{align*}
g(\grad h,X) = \diffd h(X) = 
\diffd f\circ \diffd i(X)
= g(\grad(f),\diffd i(X)).
\end{align*}
Somit ist $p$ ein kritischer Punkt von $h$ genau dann, wenn $\diffd i(X)\ \bot\
\grad(f) = e_3$ f\" ur alle $X\in T_pM$, d.h. genau dann, wenn $N(p) = (0,0,\pm
1) =: p_{\pm}$. Ohne Einschr\"ankung sind $p_\pm$ regul\"are Punkte von $N$, denn nach dem Satz von
Sard liegen die regul\"aren Punkte dicht in $M$, so dass wir dies immer einrichten
k\"onnen. Dann ist $\diffd N_{p_\pm}$ ist Isomorphismus, d.h. $K(p_{\pm}) = \det
\diffd N_{p_\pm} \neq 0$ und $N$ ist ein lokaler Diffeomorphismus. Insbesondere
gilt
\begin{align*}
\int_M \det \diffd N_p\;\dvol_p = \int_{M} N^* \dvol_{S^2} = 
\deg(N) \int_{S^2} \dvol_{S^2}.
\end{align*}

\medskip

Die Metrik $g = (g_{ij})$ auf $M$ bestimmt sich durch die Parametrisierung
als
\begin{align*}
g_{ij} = \scp{\frac{\partial F}{\partial x_i},\frac{\partial F}{\partial x_j}},
\end{align*}
und folglich ist die Schnittkr\"ummung gegeben durch,
\begin{align*}
K = \frac{\det \Hess(h)}{1 + \left(\frac{\partial h}{\partial
x}\right)^2+\big(\frac{\partial h}{\partial y}\big)^2}.
\end{align*}

$\Hess(h)_p$ ist symmetrisch und folglich diagonalisierbar und $K(p)\neq 0$.
Somit ist $\ind(p)$ gerade die Anzahl der negativen Eigenwerte von $\Hess(h)_p$. Wir machen also die folgende
Fallunterscheidung:
\begin{enumerate}
  \item $K(p)  >0$, dann ist $\det \Hess(h)_p > 0$, und folglich ist
  $\ind(p)  = 0$ oder $2$.
  \item $K(p) < 0$, dann ist $\det \Hess(h)_p < 0$, und folglich ist $\ind(p) =
  1$.
\end{enumerate}
Nach dem Lemma von Morse gilt f\"ur die Gr\"o\ss{}en,
\begin{align*}
\chi(M) = \#\setdef{p}{N(p)=p_{\pm}, K(p) > 0}
-
\#\setdef{p}{N(p)=p_{\pm}, K(p) < 0},\\
\deg(N) = 
\#\setdef{p}{N(p)=p_{+}, K(p) > 0}
-
\#\setdef{p}{N(p)=p_{+}, K(p) < 0},
\end{align*}
also ist $\chi(M) = 2 \deg(N)$. Somit folgt der Satz von Gau\ss{}-Bonet
\begin{align*}
\int_M K\dvol_M = 
\int_M N^*\dvol_{S^2}
= \deg(N)\int_{S^2} \dvol_{S^2}
= 4\pi \deg(N)
= 2\pi\ \chi(M).
\end{align*}
\qed




\end{document}

\begin{equation*}
\begin{CD}
\mathfrak{g} @> d\varphi>> \mathfrak{h}\\
@V\exp VV @VV {\exp}V \\
G @>\varphi>> H
\end{CD}
\end{equation*}
